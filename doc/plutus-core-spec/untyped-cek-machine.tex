\section{The CEK machine}
This section contains a description of an abstract machine for efficiently
executing Plutus Core.  This is based on the CEK machine of Felleisen and
Friedman~\cite{Felleisen-CK-CEK}.

\noindent The machine alternates between two main phases: the
\textit{compute} phase ($\triangleright$), where it recurses down
the AST looking for values, saving surrounding contexts as frames (or
\textit{reduction contexts}) on a stack as it goes; and the
\textit{return} phase ($\triangleleft$), where it has obtained a value and
pops a frame off the stack to tell it how to proceed next.  In
addition there is an error state $\cekerror$ which halts execution
with an error, and a halting state $\cekhalt{}$ which halts execution and
returns a value to the outside world.%
\nomenclature[Ga1]{$\triangleright$}{CEK compute phase}%
\nomenclature[Ga2]{$\triangleleft$}{CEK return phase}%
\nomenclature[Ga3]{$\cekerror$}{CEK error state}%
\nomenclature[Ga4]{$\cekhalt$}{CEK halting state}

To evaluate a program $\texttt{(program}\ v\ M \texttt{)}$, we first check that
the version number $v$ is valid, then start the machine in the state $[];[]
\triangleright M$.  It can be proved that the transitions in
Figure~\ref{fig:untyped-cek-machine} always preserve validity of states, so that
the machine can never enter a state such as $[] \triangleleft M$ or $s,
\texttt{(force \_)} \triangleleft \texttt{(lam}\ x\ A \ M\texttt{)}$ which isn't
covered by the rules.  If such a situation were to occur in an implementation
then it would indicate that the machine was incorrectly implemented or that it
was attempting to evaluate an ill-formed program (for example, one which attempts
to apply a variable to some other term).

\begin{figure}[H]
    \centering
    \[\begin{array}{lrclr}
    \textrm{State} & \Sigma & ::= & s;\rho \compute M \enspace | \enspace s \return V  \enspace |
       \enspace \cekerror{} \enspace | \enspace \cekhalt{V}\\
    \textrm{Stack} & s      & ::= & f^*\\
    \textrm{CEK value} & V &  ::= & \VCon{\tn}{c} \enspace | \enspace \VDelay{M}{\rho}
       \enspace| \enspace \VLamAbs{x}{M}{\rho} \enspace \\
       &&& | \enspace \VConstr{i}{\repetition{V}} \enspace | \enspace \VBuiltin{b}{\repetition{V}}{\eta}\\
    \textrm{Environment} & \rho & ::= & [] \enspace | \enspace \rho[x \mapsto V] \\
    \textrm{Expected builtin arguments} & \eta & ::= & [\iota] \enspace | \enspace \iota \cons \eta\\
    \end{array}\]
    \caption{Grammar of CEK machine states for Plutus Core}
    \label{fig:untyped-cek-states}
\end{figure}%
\nomenclature[Gb1]{$\Sigma$}{CEK machine state}%
\nomenclature[Gb2]{$s$}{CEK machine stack}%
\nomenclature[Gb2a]{$f$}{CEK stack frame: $\inForceFrame, \inAppLeftFrame{(M,\rho)}, \inAppRightFrame{V}, \inConstrFrame{i}{M_0 \ldots M_{i-1}}{V_{i+1} \ldots V_n}, \inCaseFrame{M_0 \ldots M_n}$}
\nomenclature[Gb3]{$V$}{CEK value: $\VCon{\tn}{c}, \VDelay{M}{\rho},\VLamAbs{x}{M}{\rho},\VBuiltin{b}{\repetition{V}}{\eta}$}%
\nomenclature[Gb4]{$\rho$}{CEK environment}%
\nomenclature[Gb5]{$\rho[x]$}{Value bound to variable $x$ in environment $\rho$}%
\nomenclature[Gb6]{$\eta$}{Arguments expected by partial builtin application}

\begin{figure}[H]
    \centering
    \[\begin{array}{lrclr}
        \textrm{Frame} & f  & ::=   & \inForceFrame                                                       & \textrm{force}\\
                       &    &       & \inAppLeftFrame{(M,\rho)}                                           & \textrm{left application to term}\\
                       &    &       & \inAppLeftFrame{V}                                                  & \textrm{left application to value}\\
                       &    &       & \inAppRightFrame{V}                                                 & \textrm{right application of value}\\
                       &    &       & \inConstrFrame{i}{\repetition{V}}{(\repetition{M}, \rho)}           & \textrm{constructor argument}\\
                       &    &       & \inCaseFrame{(\repetition{M}, \rho)}                                & \textrm{case scrutinee}

    \end{array}\]
    \caption{Grammar of CEK stack frames}
    \label{fig:untyped-cek-reduction-frames}
\end{figure}%

\kwxm{$\eta$ is the same as $\alpha$ except that we require it to be
  nonempty syntactically, whereas we put extra conditions on $\alpha$ in the
  definition of arities for builtins that make $\alpha(b)$ nonempty for all $b
  \in \Fun$.  This means that we can never have an empty $\eta$ in
  $\VBuiltin{b}{\repetition{V}}{\eta}$, which isn't entirely obvious.
  We'll have to revisit this if we ever have nullary builtins.}

\kwxm{Do we need to insist that CEK-values are well-formed, for example that
  there are enough variables in the environments to yield closed terms and that
  in $\VBuiltin{b}{\repetition{V}}{\eta}$ $\eta$ is a suffix of $\arity{b}$?
  Presumably the answer is no: you'd hope that a closed (and well-formed?) term
  $M$ will always yield a well-formed CEK value.}

\noindent Figures~\ref{fig:untyped-cek-states} and \ref{fig:untyped-cek-reduction-frames}
define some notation for \textit{states} of the CEK machine: these involve a
modified type of value adapted to the CEK machine, environments which bind names
to values, and a stack which stores partially evaluated terms whose evaluation
cannot proceed until some more computation has been performed (for example,
since Plutus Core is a strict language function arguments have to be reduced to
values before application takes place, and because of this a lambda term may
have to be stored on the stack while its argument is being reduced to a value).
Environments are lists of the form $\rho = [x_1 \mapsto V_1, \ldots, x_n \mapsto
  V_n]$ which grow by having new entries appended on the right; we say that
\textit{$x$ is \textit{bound} in the environment $\rho$} if $\rho$ contains an
entry of the form $x \mapsto V$, and in that case we denote by $\rho[x]$ the
value $V$ in the rightmost (ie, most recent) such entry.\footnote{The
  description of environments we use here is more general than necessary in that
  it permits a given variable to have multiple bindings; however, in what
  follows we never actually retrieve bindings other than the most recent one and
  we never remove bindings to expose earlier ones.  The list-based definition
  has the merit of simplicity and suffices for specification purposes but in an
  implementation it would be safe to use some data structure where existing
  bindings of a given variable are discarded when a new binding is added.}

To make the CEK machine fit into the built-in evaluation mechanism defined in
Section~\ref{sec:specify-builtins} we define $\Inputs = V$ and $\Con{\tn} =
\{\VCon{\tn}{c} : \tn \in \Uni, c \in \denote{\tn}\}$.

The rules in Figure~\ref{fig:untyped-cek-machine} show the transitions of the
machine; if any situation arises which is not included in these transitions (for
example, if a frame $\inAppRightFrame{\VCon{\tn}{c}}$ is encountered or if an
attempt is made to apply \texttt{force} to a partial builtin application which
is expecting a term argument), then the machine stops immediately in an error
state.


% Allow page break for (slightly) better placement
\begin{figure}[H]
  \begin{subfigure}[c]{\linewidth}
    \judgmentdef{$\Sigma \mapsto \Sigma'$}{Machine takes one step from state $\Sigma$ to state $\Sigma'$}

%\hspace{-1cm}
    \begin{minipage}{\linewidth}
\begin{alignat*}{2}
 s;\rho & \compute x                                 &~\mapsto~& s \return  \rho[x] \enskip \text{if $x$ is bound in $\rho$}\\
 s;\rho & \compute \con{\tn}{c}                       &~\mapsto~& s \return \VCon{\tn}{c}\\
 s;\rho & \compute \lamU{x}{M}                       &~\mapsto~& s \return \VLamAbs{x}{M}{\rho}\\
 s;\rho & \compute \delay{M}                         &~\mapsto~& s\return \VDelay{M}{\rho}\\
 s;\rho & \compute \force{M}                         &~\mapsto~& \inForceFrame{} \cons s;\rho \compute M \\
 s;\rho & \compute \appU{M}{N}                       &~\mapsto~& \inAppLeftFrame{(N,\rho)} \cons s ;\rho \compute M\\
 s;\rho & \compute \constr{i}{M \cons \repetition{M}} &~\mapsto~& \inConstrFrame{i}{}{(\repetition{M},\rho)} \cons s ;\rho \compute M\\
 s;\rho & \compute \constr{i}{[]} &~\mapsto~&  s \return \VConstr{i}{[]}\\
 s;\rho & \compute \kase{N}{\repetition{M}} &~\mapsto~&  \inCaseFrame{(\repetition{M},\rho)} \cons s ;\rho \compute N\\
% No nullary builtins (yet)
 s;\rho & \compute \builtin{b}                      &~\mapsto~& s \return \VBuiltin{b}{[]}{\arity{b}}\\
 s;\rho & \compute \errorU                           &~\mapsto~& \cekerror{}\\
\\[-10pt] %% Put some vertical space between compute and return rules, but not a whole line
[] & \return V                                    &~\mapsto~& \cekhalt{V}\\
\inAppLeftFrame{(M,\rho)}  \cons s            & \return V  &~\mapsto~& \inAppRightFrame{V} \cons s;\rho \compute M\\
\inAppRightFrame{\VLamAbs{x}{M}{\rho}} \cons s   & \return V  &~\mapsto~& s;\rho[x \mapsto V] \compute M\\
\inAppLeftFrame{V} \cons s   & \return \VLamAbs{x}{M}{\rho}  &~\mapsto~& s;\rho[x \mapsto V] \compute M\\
\inAppRightFrame{\VBuiltin{b}{\repetition{V}}{(\iota \cons \eta)}} \cons s & \return V &~\mapsto~&
                         s \return \VBuiltin{b}{(\repetition{V} \snoc V)}{\eta} \enskip \text{if $\iota \in \Unihash \cup \Var_*$}\\
\inAppLeftFrame{V} \cons s & \return \VBuiltin{b}{\repetition{V}}{(\iota \cons \eta)} &~\mapsto~&
                         s \return \VBuiltin{b}{(\repetition{V} \snoc V)}{\eta} \enskip \text{if $\iota \in \Unihash \cup \Var_*$}\\
\inAppRightFrame{\VBuiltin{b}{\repetition{V}}{[\iota]}} \cons s  & \return V &~\mapsto~&
                         \EvalCEK\,(s, b, \repetition{V}\snoc V) \enskip \text{if $\iota \in \Unihash \cup \Var_*$}\\
\inAppLeftFrame{V} \cons s & \return \VBuiltin{b}{\repetition{V}}{[\iota]} &~\mapsto~&
                         \EvalCEK\,(s, b, \repetition{V}\snoc V) \enskip \text{if $\iota \in \Unihash \cup \Var_*$}\\
\inForceFrame{} \cons s & \return \VDelay{M}{\rho}         &~\mapsto~& s;\rho \compute M\\
\inForceFrame{} \cons s & \return \VBuiltin{b}{\repetition{V}}{(\iota \cons \eta)} &~\mapsto~&
                         s \return \VBuiltin{b}{\repetition{V}}{\eta} \enskip \text{if $\iota \in \QVar$}\\
\inForceFrame{} \cons s & \return \VBuiltin{b}{\repetition{V}}{[\iota]}   &~\mapsto~&
                         \EvalCEK\,(s, b, \repetition{V}) \enskip \text{if $\iota \in \QVar$}\\
\inConstrFrame{i}{\repetition{V}}{(M \cons \repetition{M}, \rho)} \cons s & \return V   &~\mapsto~&
                         \inConstrFrame{i}{\repetition{V} \cons V}{(\repetition{M}, \rho)} \cons s;\rho \compute M \\
\inConstrFrame{i}{\repetition{V}}{([], \rho)} \cons s & \return V   &~\mapsto~&
                         s \return \VConstr{i}{\repetition{V} \cons V} \\
\inCaseFrame{(M_0 \ldots M_n, \rho)} \cons s & \return \VConstr{i}{V_0 \ldots V_m}   &~\mapsto~&
                         \inAppLeftFrame{V_m} \cons \cdots \cons \inAppLeftFrame{V_0} \cons s ;\rho \compute M_i \enskip \text{if $0 \leq i \leq n$}
\end{alignat*}
\end{minipage}
    \caption{CEK machine transitions for Plutus Core}
    \label{fig:untyped-cek-transitions}
\end{subfigure}

\bigskip
  \begin{subfigure}[c]{\linewidth}
$$ \EvalCEK(s, b, [V_1, \ldots, V_n]) =
   \begin{cases}
      \cekerror  & \text{if $\Eval\,(b,[V_1, \ldots, V_n]) = \errorX$}\\
      s \return \Eval\,(b,[V_1, \ldots, V_n]) & \text{otherwise}
   \end{cases}
$$
    \caption{Evaluation of built-in functions}
    \label{fig:untyped-cek-builtins}
    \end{subfigure}

  \caption{A CEK machine for Plutus Core}
\label{fig:untyped-cek-machine}
\end{figure}

\subsection{Converting CEK evaluation results into Plutus Core terms}
The purpose of the CEK machine is to evaluate Plutus Core terms, but in the
definition in Figure~\ref{fig:untyped-cek-machine} it does not return a Plutus
Core term; instead the machine can halt in two different ways:
\begin{itemize}
\item The machine can halt in the state $\cekhalt{V}$ for some CEK value $V$.
\item The machine can halt in the state $\cekerror{}$ .
\end{itemize}

\noindent To get a complete evaluation strategy for Plutus Core we must convert
these states into Plutus Core terms.  The term corresponding to $\cekerror{}$ is
$\errorU$, and to obtain a term from $\cekhalt{V}$ we perform a process which we
refer to as \textit{discharging} the CEK value $V$ (also known as
\textit{unloading}: see~\cite[pp. 129--130]{Plotkin-cbn-cbv},
\cite[pp. 71ff]{Felleisen-pllc}).  This process substitutes bindings in
environments for variables occurring in the value $V$ to obtain a term
$\unload{V}$: see Figure~\ref{fig:discharge-val}.  Since environments contain
bindings $x \mapsto W$ of variables to further CEK values, we have to
recursively discharge those bindings first before substituting: see
Figure~\ref{fig:discharge-env}, which defines an operation $\Unload{\rho}{}$ which
does this.  As before $[N/x]M$ denotes the usual (capture-avoiding) process of
substituting the term $N$ for all unbound occurrences of the variable $x$ in the
term $M$. Note that in Figure~\ref{fig:discharge-env} we substitute the
rightmost (ie, the most recent) bindings in the environment first.

\begin{figure}[H]
  \centering

  \begin{subfigure}[b]{\textwidth}
    \begin{align*}
      \unload{\VCon{\tn}{c}} &= \con{\tn}{c}\\
      \unload{\VDelay{M}{\rho}}
        &= \Unload{\rho}{\delay{M}}\\
      \unload{\VLamAbs{x}{M}{\rho}} &= \Unload{\rho}{\lamU{x}{M}}\\
      \unload{\VConstr{i}{\repetition{V}}} &= \constr{i}{\repetition{\unload{V}}}\\
      \unload{\VBuiltin{b}{V_1 V_2\ldots V_k}{\eta}} &=
      \appU{\ldots}  % [...[[(builtin b) V1!] V2!] ... Vk!]
           {\appU
             {\appU
               {\builtin{b}}
               {(\unload{V_1})}
             }
             {(\unload{V_2})}
             {\ldots (\unload{V_k})}
           }
    \end{align*}
    \caption{Discharging CEK values}
    \label{fig:discharge-val}
  \end{subfigure}

  \begin{subfigure}[c]{\textwidth}
    \begin{align*}
      \Unload{\rho}{M} &= [(\unload{V_1})/x_1]\cdots[(\unload{V_n})/x_n]M \quad
      \text{if $\rho = [x_1 \mapsto V_1, \ldots, x_n \mapsto V_n]$}
    \end{align*}
    \caption{Iterated substitution/discharging}
    \label{fig:discharge-env}
  \end{subfigure}

  \caption{Discharging CEK values to obtain Plutus Core terms}
  \label{fig:discharge-cek-val}
\end{figure}%
\nomenclature[Gc1]{$\unload{V}$}{Discharge a CEK value $V$ to obtain a Plutus Core term}%
\nomenclature[Gc2]{$\Unload{\rho}{M}$}{Discharge all variables bound by $\rho$ in the term $M$}

\noindent We can prove that if we evaluate a closed Plutus Core term in the CEK
machine and then convert the result back to a term using the above procedure
then we get the result that we should get according to the semantics in
Figure~\ref{fig:untyped-term-reduction}.
