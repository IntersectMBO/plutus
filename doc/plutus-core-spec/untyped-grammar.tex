\section{The Grammar of Plutus Core}
\label{sec:untyped-plc-grammar}
This section presents the grammar of Plutus Core in a Lisp-like form.  This is
intended as a specification of the abstract syntax of the language; it may also
by used by tools as a concrete syntax for working with Plutus Core programs, but
this is a secondary use and we do not make any guarantees of its completeness
when used in this way.  The primary concrete form of Plutus Core programs is the
binary format described in Appendix~\ref{appendix:flat-serialisation}.

\subsection{Lexical grammar}
\label{sec:untyped-plc}
\thispagestyle{plain}
\pagestyle{plain}

\begin{minipage}{\linewidth}
    \centering
    \[\begin{array}{lrclr}

        \textrm{Name}        & n      & ::= & \texttt{[a-zA-Z][a-zA-Z0-9\_\textquotesingle]\textsuperscript{*}}   & \textrm{name}\\

        \textrm{Var}           & x      & ::= & n & \textrm{term variable}\\
        \textrm{BuiltinName}   & bn     & ::= & n & \textrm{built-in function name}\\
        \textrm{Version} & v & ::= & \texttt{[0-9]\textsuperscript{+}.[0-9]\textsuperscript{+}.[0-9]\textsuperscript{+}}& \textrm{version}\\

        \textrm{Constant} & c & ::= & \langle{\textrm{literal constant}}\rangle& \\

    \end{array}\]
    \captionof{figure}{Lexical grammar of Plutus Core}
    \label{fig:lexical-grammar-untyped}
\end{minipage}%
\nomenclature[C]{$L,M,N$}{A term}%
\nomenclature[C]{$n$}{A name}%
\nomenclature[C]{$i$}{An integer}%
\nomenclature[C]{$x$}{A variable name}%
\nomenclature[C]{$bn, b$}{The name of a built-in function}%
\nomenclature[C]{$c$}{A literal constant}%
\nomenclature[C]{$P$}{A Plutus Core program}%
\nomenclature[C]{$v$}{Plutus Core version}



%   @sqs   = '  ( ($printable # ['\\])  | (\\$printable) )* '
%   
%   -- A double quoted string, allowing escaped characters including \".  Similar to @sqs
%   @dqs   = \" ( ($printable # [\"\\]) | (\\$printable) )* \"
%   
%   -- A sequence of printable characters not containing '(' or ')' such that the
%   -- first character is not a space or a single or double quote.  If there are any
%   -- further characters then they must comprise a sequence of printable characters
%   -- possibly including spaces, followed by a non-space character.  If there are
%   -- any leading or trailing spaces they will be consumed by the $white+ token
%   -- below.
%   $nonparen = $printable # [\(\)]
%   @chars = ($nonparen # ['\"$white]) ($nonparen* ($nonparen # $white))?
%   
%       <literalconst> "()" | @sqs | @dqs | @chars { tok (\p s -> alex $ TkLiteralConst p (textOf s)) `andBegin` 0 }

%% "()"
%% @sqs   = '  ( ($printable # ['\\])  | (\\$printable) )* '
%% @dqs   = \" ( ($printable # [\"\\]) | (\\$printable) )* \"
%% @chars = ($nonparen # ['\"$white]) ($nonparen* ($nonparen # $white))?


\subsection{Grammar}
\begin{minipage}{\linewidth}
    \centering
    \[\begin{array}{lrclr}
    \textrm{Term}       & L,M,N  & ::= & x                               & \textrm{variable}\\
                        &        &     & \con{\tn}{c}                    & \textrm{constant}\\
                        &        &     & \builtin{b}                     & \textrm{builtin}\\
                        &        &     & \lamU{x}{M}                     & \textrm{$\lambda$ abstraction}\\
                        &        &     & \appU{M}{N}                     & \textrm{function application}\\
                        &        &     & \delay{M}                       & \textrm{delay execution of a term}\\
                        &        &     & \force{M}                       & \textrm{force execution of a term}\\
                        &        &     & \constr{i}{M_0 \ldots M_{m-1}}  & \textrm{constructor with tag $i$ and $m$ arguments}\\
                        &        &     & \kase{M}{N_0 \ldots N_{m-1}}    & \textrm{case anaylsis with $m$ alternatives}\\
                        &        &     & \errorU                         & \textrm{error}\\
        \textrm{Program}& P      & ::= & \version{v}{M}                  & \textrm{versioned program}

    \end{array}\]
    \captionof{figure}{Grammar of untyped Plutus Core}
    \label{fig:untyped-grammar}
\end{minipage}


\subsection{Notes}
\label{sec:grammar-notes}
\paragraph{Scoping.} For simplicity, \textbf{we assume throughout that the body of a
Plutus Core program is a closed term}, ie, that it contains no free
variables.  Thus \texttt{(program 1.0.0 (lam x x))} is a valid program but
\texttt{(program 1.0.0 (lam x y))} is not, since the variable \texttt{y} is
free. This condition should be checked before execution of any program
commences, and the program should be rejected if its body is not closed.  The
assumption implies that any variable $x$ occurring in the body of a program must
be bound by an occurrence of \texttt{lam} in some enclosing term; in this case,
we always assume that $x$ refers to the \textit{most recent} (ie, innermost)
such binding.

\paragraph{Iterated applications.}
An application of a term $M$ to a term $N$ is represented by
$\appU{M}{N}$. We may occasionally write
$\appU{M}{N_1 \ldots N_k}$ or
$\appU{M}{\repetition{N}}$ as an abbreviation for an iterated application
$\mathtt{[}\ldots\mathtt{[[}M\;N_1\mathtt{]}\;N_2\mathtt{]}\ldots N_k\mathtt{]}$,
and tools may also use this as concrete syntax.

\paragraph{Built-in types and functions.} The language is parameterised by a set $\Uni$ of
\textit{built-in types} (we sometimes refer to $\Uni$ as the
\textit{universe}) and a set $\Fun$ of \textit{built-in functions}
(\textit{builtins} for short), both of which are sets of Names.
Briefly, the built-in types represent sets of constants such as
integers or strings; constant expressions $\con{\tn}{c}$ represent
values of the built-in types (the integer 123 or the string
\texttt{"string"}, for example), and built-in functions are functions
operating on these values, and possibly also general Plutus Core
terms.  Precise details are given in
Section~\ref{sec:specify-builtins}.  Plutus Core comes with a default
universe and a default set of builtins, which are described in
Appendix~\ref{appendix:default-builtins-alonzo}.%
\nomenclature[E1]{$\Fun$}{The set of built-in functions}

\paragraph{De Bruijn indices.}
The grammar defines names to be textual strings, but occasionally (specifically
in Appendix~\ref{appendix:flat-serialisation}) we want to use de Bruijn indices
(\cite{deBruijn}, \cite[C.3]{Barendregt}), and for this we redefine names to be
natural numbers.  In de Bruijn terms, $\lambda$-expressions do not need to bind
a variable, but in order to re-use our existing syntax we arbitrarily use 0 for
the bound variable, so that all $\lambda$-expresssions are of the form
\texttt{(lam 0 $M$)}; other variables (ie, those not appearing immediately after
a \texttt{lam} binder) are represented by natural number greater than zero.

\paragraph{Lists in constructor and case terms}
The grammar defines constructor and case terms to have a variable number of
subterms written in sequence with no delimiters. This corresponds to the
concrete syntax, e.g. we write $\constr{0}{t_1\ t_2\ t_3}$. However, in the
rest of the specification we will abuse notation and treat these terms as
having \emph{lists} of subterms.
