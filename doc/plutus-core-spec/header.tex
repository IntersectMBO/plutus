%% Stuff for index of notation
%% FYI, the nomencl package seems to discard entries containing '@'
\usepackage[intoc,refpage]{nomencl}
\renewcommand{\nomname}{Index of Notation}
\renewcommand{\pagedeclaration}[1]{,\ \ \hyperlink{page.#1}{\nobreakspace#1}}
\setlength{\nomitemsep}{-0.6\parsep} % Adjust vertical spacing between items in index
\renewcommand{\nomgroup}[1]{\medskip%  Grouping for entries in symbol index
  \ifthenelse{\equal{#1}{A}}{\item[\textbf{Sets}]}{%
    \ifthenelse{\equal{#1}{B}}{\item[\textbf{Lists}]}{%
      \ifthenelse{\equal{#1}{C}}{\item[\textbf{Plutus Core grammar}]}{%
        \ifthenelse{\equal{#1}{D}}{\item[\textbf{Built-in types}]}{%
          \ifthenelse{\equal{#1}{E}}{\item[\textbf{Built-in functions}]}{%
            \ifthenelse{\equal{#1}{F}}{\item[\textbf{Term reduction}]}{%
              \ifthenelse{\equal{#1}{G}}{\item[\textbf{CEK machine}]}{%
               \ifthenelse{\equal{#1}{H}}{\item[\textbf{Concrete syntax}]}{%
                 \ifthenelse{\equal{#1}{I}}{\item[\textbf{Serialisation and deserialisation}]}{}
                }
              }
            }
          }
        }
      }
    }
  }
}
\makenomenclature

\renewcommand{\thefootnote}{\fnsymbol{footnote}}

% correct bad hyphenation here
\hyphenation{byte-string}

\usepackage[hyphens]{url}
\usepackage[colorlinks=true,linkcolor=MidnightBlue,citecolor=ForestGreen,urlcolor=Plum,pageanchor]{hyperref}
% ^ pageanchor is for nomencl

\usepackage{tocbibind}  % To get the bibiliograpy in the table of contents
\usepackage{blindtext, graphicx}
%\usepackage[numbers]{natbib}
\usepackage{natbib}

% *** FONTS ***
%
\usepackage{amsmath}
%\usepackage{amssymb}
\usepackage{stmaryrd}
\usepackage{stix}
\usepackage{alltt}
\usepackage{bussproofs}
\usepackage[mathscr]{euscript}
\DeclareMathAlphabet{\mathcal}{OMS}{cmsy}{m}{n}

% *** FIGURES/ALIGNMENT ***
%
\usepackage{array}
\usepackage{float}  %% Try to improve placement of figures.  Doesn't work well with subcaption package.
\usepackage{subcaption}
\usepackage{fancyvrb}
\usepackage[tableposition=below]{caption}
\captionsetup[longtable]{skip=6pt}

\usepackage{subfiles}
\usepackage{geometry}
\usepackage{pdflscape}
\usepackage[title]{appendix}

\usepackage[T1]{fontenc}
\usepackage[dvipsnames]{xcolor}

\usepackage{listings}
\lstset{
  basicstyle=\ttfamily,
  columns=fullflexible,
  mathescape=true,
  escapeinside={|}{|}   %% Inside listings you can say things like |\textit{blah blah}|
}

\usepackage{makecell}
\usepackage{longtable}

% You're supposed to make this the final package
\usepackage[disable]{todonotes}
% \usepackage{todonotes}

\setcounter{tocdepth}{3}
\setcounter{secnumdepth}{3}

\newcommand{\kwxm}[1]{\smallskip\todo[inline,color=yellow!72,author=kwxm,caption={}]{#1}\smallskip}
\renewcommand{\roman}[1]{\smallskip\todo[inline,color=green,author=effectfully,caption={}]{#1}\smallskip}

\newcommand{\TT}[1]{\texttt{#1}}  %% This is used in some tables to make the Latex more compact

\newcommand{\termarg}{\smblkcircle}
\newcommand{\typearg}{\circ}  % stix redefines \bullet to something smaller

% % Stuff for splitting figures over page breaks
% \DeclareCaptionLabelFormat{continued}{#1~#2 (Continued)}
% \captionsetup[ContinuedFloat]{labelformat=continued}


%%% General Misc. Definitions

\newcommand{\red}[1]{\textcolor{red}{#1}}
\newcommand{\redfootnote}[1]{\red{\footnote{\red{#1}}}}
\newcommand{\blue}[1]{\textcolor{blue}{#1}}
\newcommand{\bluefootnote}[1]{\blue{\footnote{\blue{#1}}}}

\newcommand{\diffbox}[1]{\text{\colorbox{lightgray}{\(#1\)}}}
\newcommand{\judgmentdef}[2]{\fbox{#1}

\vspace{0.5em}

#2}

\newcommand{\hyphen}{\operatorname{-}}
\newcommand{\repetition}[1]{\overline{#1}}
\newcommand{\Fomega}{F$^{\omega}$}
\newcommand{\keyword}[1]{\texttt{#1}}
\newcommand{\inparens}[1]{\texttt{(} #1 \texttt{)}}
\newcommand{\construct}[1]{\texttt{(} #1 \texttt{)}}
% \newcommand\discharge[1]{\widehat{#1}}

%%% Term Grammar

\newcommand{\sig}[3]{[#1](#2)#3}
\newcommand{\constsig}[1]{#1}
\newcommand{\con}[2]{\inparens{\keyword{con} ~ #1 ~ #2}}
\newcommand{\abs}[3]{\inparens{\keyword{abs} ~ #1 ~ #2 ~ #3}}
\newcommand{\inst}[2]{\texttt{\{}#1 ~ #2\texttt{\}}}
\newcommand{\lam}[3]{\inparens{\keyword{lam} ~ #1 ~ #2 ~ #3}}
\newcommand{\app}[2]{\texttt{[} #1 ~ #2 \texttt{]}}
\newcommand{\iwrap}[3]{\inparens{\keyword{iwrap} ~ #1 ~ #2 ~ #3}}
\newcommand{\wrap}{\iwrap}
%% ^ Temporary fix to avoid substituting all occurrences of new keyword
\newcommand{\unwrap}[1]{\inparens{\keyword{unwrap} ~ #1}}
\newcommand{\builtin}[1]{\inparens{\keyword{builtin} ~ \mathit{#1}}}
\newcommand{\error}[1]{\inparens{\keyword{error} ~ #1}}

%% Extra untyped terms
\newcommand{\lamU}[2]{\inparens{\keyword{lam} ~ #1 ~ #2        }}
\newcommand{\appU}[2]{\texttt{[} #1 ~ #2 \texttt{]}}
\newcommand{\builtinappU}[3]{\inparens{\keyword{builtin} ~ #1 ~ #2 ~ #3}}
\newcommand{\errorU}{\inparens{\keyword{error}}}
\newcommand{\delay}[1]{\inparens{\keyword{delay} ~ #1}}
\newcommand{\force}[1]{\inparens{\keyword{force} ~ #1}}
\newcommand{\constr}[2]{\inparens{\keyword{constr} ~ #1 ~ #2}}
\newcommand{\kase}[2]{\inparens{\keyword{case} ~ #1 ~ #2}}

\newcommand{\erase}[1]{\llbracket#1\rrbracket}

%%%  Type Grammar

\newcommand{\funT}[2]{\inparens{\keyword{fun} ~ #1 ~ #2}}
\newcommand{\ifixT}[2]{\inparens{\keyword{ifix} ~ #1 ~ #2}}
\newcommand{\fixT}{\ifixT}
%% ^ Temporary fix to avoid substituting all occurrences of new keyword
\newcommand{\allT}[3]{\inparens{\keyword{all} ~ #1 ~ #2 ~ #3}}
\newcommand{\conIntegerType}[1]{\keyword{integer}}
\newcommand{\conBytestringType}[1]{\keyword{bytestring}}
\newcommand{\conT}[1]{\inparens{\keyword{con} ~ #1}}
\newcommand{\lamT}[3]{\inparens{\keyword{lam} ~ #1 ~ #2 ~ #3}}
\newcommand{\appT}[2]{\texttt{[} #1 ~ #2 \texttt{]}}

\newcommand{\typeK}{\inparens{\keyword{type}}}
\newcommand{\funK}[2]{\inparens{\keyword{fun} ~ #1 ~ #2}}



%%% Program Grammar

\newcommand{\version}[2]{\inparens{\keyword{program} ~ #1 ~ #2}}

\newcommand{\evalbuiltin}[2]{\llbracket #1 ~ #2 \rrbracket}

%%% Judgments

\newcommand{\hypJ}[2]{#1 \vdash #2}
\newcommand{\ctxni}[2]{#1 \ni #2}
\newcommand{\validJ}[1]{#1 \ \operatorname{valid}}
\newcommand{\termJ}[2]{#1 : #2}
\newcommand{\typeJ}[2]{#1 :: #2}
\newcommand{\istermJ}[2]{#1 : #2}
\newcommand{\istypeJ}[2]{#1 :: #2}



%%% Contextual Normalization
\newcommand{\pba}{B}  % Possibly ill-formed partial builtin application
\newcommand{\pbas}{\mathsf{\pba}} % Set of all partial builtin applications
\newcommand{\pbasize}[1]{\lVert #1 \rVert}

\newcommand{\ctxsubst}[2]{#1\{#2\}}
\newcommand{\typeStep}[2]{#1 ~ \rightarrow_{ty} ~ #2}
\newcommand{\typeMultistep}[2]{#1 ~ \rightarrow_{ty}^{*} ~ #2}
\newcommand{\typeBoundedMultistep}[3]{#2 ~ \rightarrow_{ty}^{#1} ~ #3}
\newcommand{\step}[2]{#1 ~ \rightarrow ~ #2}
\newcommand{\normalform}[1]{\lfloor #1 \rfloor}
\newcommand{\subst}[3]{[#1/#2]#3}
\newcommand{\kindEqual}[2]{#1 =_{\mathit{k}} #2}
\newcommand{\typeEqual}[2]{#1 =_{\mathit{ty}} #2}
\newcommand{\typeEquiv}[2]{#1 \equiv_{\mathit{ty}} #2}


\newcommand{\inConTFrame}[1]{\conT{#1}}
\newcommand{\inAppTLeftFrame}[1]{\appT{\_}{#1}}
\newcommand{\inAppTRightFrame}[1]{\appT{#1}{\_}}
\newcommand{\inFunTLeftFrame}[1]{\funT{\_}{#1}}
\newcommand{\inFunTRightFrame}[1]{\funT{#1}{\_}}
\newcommand{\inAllTFrame}[2]{\allT{#1}{#2}{\_}}
\newcommand{\inFixTLeftFrame}[1]{\fixT{\_}{#1}}
\newcommand{\inFixTRightFrame}[1]{\fixT{#1}{\_}}
\newcommand{\inLamTFrame}[2]{\lamT{#1}{#2}{\_}}

\newcommand{\inBuiltin}[5]{\builtin{#1}{#2}{#3 #4 #5}}

%% Inputs for the untyped CEK machine
\newcommand{\invalue}[1]{\texttt{\textlangle} #1 \texttt{\textrangle}}

\newcommand{\VCon}[2]{\invalue{\keyword{con} ~ #1 ~ #2}}
\newcommand{\VDelay}[2]{\invalue{\keyword{delay} ~ #1 ~ #2}}
\newcommand{\VLamAbs}[3]{\invalue{\keyword{lam} ~ #1 ~#2 ~ #3}}
\newcommand{\VBuiltin}[3]{\invalue{\keyword{builtin} ~ #1 ~ #2 ~ #3}}
\newcommand{\VConstr}[2]{\invalue{\keyword{constr} ~ #1 ~ #2}}
 %% (builtin bn v_1 ... v_k _ m_{k+1} ... m_n, arity)

%%% CK Machine Normalization

\newcommand{\compute}{\triangleright}
\newcommand{\return}{\triangleleft}
\newcommand{\cekerror}{\mdblkdiamond}
\newcommand{\cekhalt}[1]{\mdwhtsquare\,#1}
\newcommand{\unload}[1]{\mathcal{U}(#1)}
\newcommand{\Unload}[2]{{#2}@{#1}}

\newcommand{\bcompute}{\color{blue}\compute}
\newcommand{\breturn}{\color{blue}\return}
\newcommand{\bmapsto}{\color{blue}\mapsto}
% This is to get blue symbols after an '&' inside an alignat
% environment.  It seems to mess up the spacing if you do anything
% else.



\newcommand{\inInstLeftFrame}[1]{\inst{\_}{#1}}
\newcommand{\inWrapRightFrame}[2]{\iwrap{#1}{#2}{\_}}
\newcommand{\inUnwrapFrame}{\unwrap{\_}}
\newcommand{\inAppLeftFrame}[1]{\app{\_}{#1}}
\newcommand{\inAppRightFrame}[1]{\app{#1}{\_}}
\newcommand{\inConstrFrame}[3]{\inparens{\keyword{constr} ~ #1 ~ #2 ~ \_ ~ #3}}
\newcommand{\inCaseFrame}[1]{\inparens{\keyword{case} ~ \_ ~ #1}}

% Extra frames for untyped term normalisation
\newcommand{\inForceFrame}{\force{\_}}
\newcommand{\inBuiltinU}[4]{\builtin{#1}{#2 #3 #4}}

% These are for use inside listings and $...$.  If you just use
% \textit in listings it uses the italic tt font and the spacing
% inside the words is a bit strange.  Spacing is also bad if you
% just put something like "integer" in math text.
\newcommand\unit{\ensuremath{\mathit{unit}}}
\newcommand\one{\ensuremath{\mathit{one}}}
\renewcommand\boolean{\ensuremath{\mathit{boolean}}}
\newcommand\integer{\ensuremath{\mathit{integer}}}
\newcommand\bytestring{\ensuremath{\mathit{bytestring}}}
\newcommand\str{\ensuremath{\mathit{str}}}
\newcommand\true{\ensuremath{\mathsf{true}}}
\newcommand\false{\ensuremath{\mathsf{false}}}
\newcommand\case{\ensuremath{\mathit{case}}}
\newcommand\signed{\ensuremath{\mathit{signed}}}
\newcommand\txhash{\ensuremath{\mathit{txhash}}}
\newcommand\pubkey{\ensuremath{\mathit{pubkey}}}
\newcommand\blocknum{\ensuremath{\mathit{blocknum}}}
%% \newcommand\uniqmem[1]{#1^{\blacktriangledown}}  %% Unique member of size type

\newcommand\listOf[1]{\mathtt{list}\,\mathtt{(}#1\mathtt{)}}
\newcommand\pairOf[2]{\mathtt{pair}\,\mathtt{(}#1\mathtt{,}\,#2\mathtt{)}}


%% \newcommand\disj{\mathbin{\dot{\cup}}}    % Infix disjoint union
%% \newcommand\bigdisj{\bigcup^{\bullet}}      % Prefix disjoint union
\newcommand\disj{\uplus}                  % Infix disjoint union
\newcommand\bigdisj{\biguplus}            % Prefix disjoint union
\newcommand\TyOp{\mathscr{O}}             % Type operators
\newcommand\Var{\mathscr{V}}              % Type variables
\newcommand\QVar{\mathscr{Q}}             % Quantified type variables
\newcommand\Uni{\mathscr{U}}              % Universe of built-in types
\newcommand\Unihash{{\mathscr{U}_{\#}}}     % Universe of built-in types extended with type variables
\newcommand\UniTop{\Uni^{\top}}      % Universe of built-in types extended with a top element
\newcommand\Unihashn[1]{\Uni_{\#,#1}}
\newcommand\Sext{\hat{S}}
\newcommand\UnihashStar{\Unihash \disj \Var_*} % Built-in type or type name
\newcommand\UnihashTop{\Unihash^{\top}}      % Universe of built-in types extended with a top element
\newcommand\op{\textit{op}}               % Type operator names
\newcommand\valency[1]{\left| #1 \right|} % Valency of a type operator
\newcommand\Fun{\mathscr{B}}              % Built-in functions
\newcommand\Inputs{\mathscr{I}}           % Built-in functions
\newcommand\Con[1]{\mathscr{C}_{#1}}  % Constant terms of a given types
\newcommand\R{\mathscr{R}}           % Things that a builtin can return
\newcommand\bsig[1]{\lvert{#1}\rvert} % Signature |b| of a built-in function.
\newcommand\arity[1]{\alpha({#1})}   % Arity \alpha(b) of a built-in function.
\newcommand\sat[1]{\gamma({#1})} % Interleaving structure of a partial builtin application
\newcommand\fv[1]{\mathsf{FV}_{\#}(#1)}
\newcommand{\N}{\mathbb{N}}
\newcommand{\Nplus}{\N^{+}}
\newcommand{\Z}{\mathbb{Z}}
\newcommand{\B}{\mathbb{B}}
\renewcommand{\b}{\mathbb{b}}
\newcommand{\U}{\mathbb{U}}
\newcommand{\cn}{\textit{cn}}
\newcommand{\tn}{\textit{T}}
\newcommand{\at}{\textit{at}}
\newcommand{\lit}[1]{\mathbf{\mathsf{lit\,_{#1}}}}
\newcommand{\alphabar}{\bar{\alpha}}
\newcommand{\sigmabar}{\bar{\sigma}}
\newcommand{\Eval}{\mathsf{Eval}}
\newcommand{\EvalCEK}{\mathsf{Eval_{\,CEK}}}
%% NOTE!!!  Probably \byte and \Z are only used in V1-builtins

\newcommand{\kindstar}[1]{#1::\text{\textasteriskcentered}}  % "v::*"  (The ordinary * is a bit above the centreline)
\newcommand{\kindhash}[1]{#1::\text{\#}}                     % "v::#"

\newcommand\errorX{\textbf{\texttimes}}
\newcommand\ValueOrError{\Inputs\disj\{\errorX\}}
\newcommand\withError[1]{\ensuremath{#1_{\errorX}}}



\newcommand{\cons}{\mathbin{\!\cdot\!}}
\newcommand{\snoc}{\mathbin{\!\cdot\!}}

\newcommand{\forallty}[1]{\mathbf{\forall}#1}
\newcommand{\denote}[1]{\llbracket\mathit{#1}\rrbracket}
\newcommand{\reify}[1]{\lBrace{#1}\rBrace}
%\newcommand{\args}[1]{\textbf{\textsf{args}}(#1)}
\DeclareMathOperator{\args}{\mathbf{\mathsf{args}}}
\DeclareMathOperator{\type}{\mathbf{\mathsf{type}}}
\DeclareMathOperator{\length}{\ell}
\DeclareMathOperator{\nextArg}{\mathsf{next}}
\DeclareMathOperator{\divfn}{\mathsf{div}}
\DeclareMathOperator{\modfn}{\mathsf{mod}}
\DeclareMathOperator{\quotfn}{\mathsf{quot}}
\DeclareMathOperator{\remfn}{\mathsf{rem}}
\DeclareMathOperator{\inj}{inj}
\DeclareMathOperator{\proj}{proj}
\DeclareMathOperator{\is}{is}
\DeclareMathOperator{\dom}{dom}
\newcommand{\units}[1]{#1^{\times}}  % Group of units of a field

%%% Spacing in tables

\newcommand\sep{4pt}
% The table of abbreviations previously had \\\\ at the end of each line, which
% made it quite long. Lines are now separated by a vertical space of size \sep.
% This makes it a bit more readable than no spacing at all, but not too long

\newcommand{\Strut}{\rule[-2mm]{0mm}{6mm}}

\newcommand{\conUnitType}{\keyword{unit}}
\newcommand{\conBooleanType}{\keyword{bool}}
\newcommand{\ty}[1]{\ensuremath{\mathtt{#1}}}

\newcommand\Nab[2]{\N_{[#1,#2]}}
\newcommand\halfOpenInterval[2]{\N_{[#1#2)}}


%% Macros for CBOR encoders and decoders

\newcommand\e{\mathcal{E}}
\renewcommand\d{\mathcal{D}}    %% This usually puts a dot under its argument

\newcommand\eHead{\e_{\mathsf{head}}}
\newcommand\dHead{\d_{\mathsf{head}}}

\newcommand\eIndef{\e_{\mathsf{indef}}}
\newcommand\dIndef{\d_{\mathsf{indef}}}

\newcommand\ecTag{\e_{\texttt{ctag}}}  % A constructor tag, not a CBOR tag.
\newcommand\dcTag{\d_{\texttt{ctag}}}

\newcommand\hex[1]{$\mathtt{#1_{16}}$}

\newcommand\eZ{\e_{\Z}}
\newcommand\dZ{\d_{\Z}}

\newcommand\eBS{\e_{\B^*}}
\newcommand\dBS{\d_{\B^*}}

\newcommand\itos{\mathsf{itos}}
\newcommand\stoi{\mathsf{stoi}}

\newcommand\byte{\mathsf{b}}

\newcommand\dBytes{\d_{\mathsf{bytes}}}
\newcommand\dBlock{\d_{\mathsf{block}}}
\newcommand\dBlocks{\d_{\mathsf{blocks}}}

\newcommand\intToBS{\mathsf{e}}
\newcommand\bsToInt{\mathsf{d}}  % bytestring -> integer

%% Macros for flat encoders and decoders

\newcommand\T{\mathsf{T}}  % Type tags

\newcommand\E{\mathsf{E}}  % Encoders
\newcommand\Elist{\overrightarrow{\mathsf{E}}}
\newcommand\Eblock{\mathsf{E}_{\mathsf{block}}}

\newcommand\encode[1]{\E_{\mathsf{#1}}}
\newcommand\encodelist[1]{\Elist_{\mathsf{#1}}}

\newcommand\Eprogram{\encode{program}}
\newcommand\Eterm{\encode{term}}
\newcommand\Ebuiltin{\encode{builtin}}
\newcommand\Econstant[1]{\encode{constant}^{#1}}
\newcommand\Ename{\encode{name}}
\newcommand\Ebinder{\encode{\text{$\lambda$}var}}
\newcommand\Etype{\encode{type}}
\newcommand\zigzag{\mathbb{z}}

\newcommand\D{\mathsf{D}}  % Decoders
\newcommand\Dlist{\overrightarrow{\mathsf{D}}}

\newcommand\decode[1]{\D_{\mathsf{#1}}}
\newcommand\decodelist[1]{\Dlist_{\mathsf{#1}}}

\newcommand\Dprogram{\decode{program}}
\newcommand\Dterm{\decode{term}}
\newcommand\Dbuiltin{\decode{builtin}}
\newcommand\dConstant[1]{\decode{constant}^{#1}}
\newcommand\Dname{\decode{name}}
\newcommand\Dbinder{\decode{\text{$\lambda$}var}}
\newcommand\Dtype{\decode{type}}

\newcommand\bit{\mathsf{b}}
\newcommand\Bits{\bit^*}
\newcommand\bits[1]{\mathtt{#1}}

\newcommand\bytesof{\mathsf{bytes}}
\newcommand\bitsof{\mathsf{bits}}

\newcommand\pad{\mathsf{pad}}
% \newcommand\Pad{\mathsf{pad}^{+}}
\newcommand\unpad{\mathsf{unpad}}
% \newcommand\unPad{\mathsf{unpad}^{+}}
\newcommand\pp[1]{\mathsf{p}_{#1}}

\newcommand\plc[2]{\texttt{(#1}\ #2\texttt{)}}

\newcommand\Prog[4]{\plc{program}{#1.#2.#3\ #4}}
\newcommand\Const[2]{\plc{const}{#1\ #2}}
\newcommand\Builtin[1]{\plc{builtin}{#1}}
\newcommand\Lam[2]{\plc{lam}{#1\ #2}}
\newcommand\Apply[2]{\texttt{[}#1\ #2\texttt{]}}
\newcommand\Delay[1]{\plc{delay}{#1}}
\newcommand\Force[1]{\plc{force}{#1}}
\newcommand\Constr[2]{\plc{constr}{#1\ #2}}
\newcommand\Kase[2]{\plc{case}{#1\ #2}}
\newcommand\Error{\texttt{(error)}}

\newcommand{\Tag}{\mathsf{tag}}  %% \renewcommand{\tag} gave odd results inside alignat*
\newcommand{\unTag}{\mathsf{tag}^{-1}}

\newcommand{\LL}[1]{\texttt{#1}}
