\section{Interpretation of built-in types and functions}
\label{sec:specify-builtins}
As mentioned above, Plutus Core is generic over a universe $\Uni$ of types and
a set $\Fun$ of built-in functions.  As the terminology suggests, built-in
functions are interpreted as functions over terms and elements of the built-in
types: in this section we make this interpretation precise by giving a
specification of built-in types and functions in a set-theoretic denotational
style.  We require a considerable amount of extra notation in order to do this,
and we emphasise that nothing in this section is part of the syntax of Plutus
Core: it is meta-notation introduced purely for specification purposes.


\subsection{Built-in types}
\label{sec:built-in-types}
We require some extra syntactic notation for built-in types: see Figure~\ref{fig:type-names-operators}.

\begin{minipage}{\linewidth}
    \centering
    \[\begin{array}{rclr}
    \at    & ::= & n & \textrm{Atomic type}\\
     \op             & ::= & n & \textrm{Type operator}\\
     \tn             & ::= & \at \ | \ \op(\tn,\tn,...,\tn) & \textrm{Built-in type}\\
    \end{array}\]
    \captionof{figure}{Type names and operators}
    \label{fig:type-names-operators}
\end{minipage}

\medskip
\noindent
We assume that we have a set $\Uni_0$ of \textit{atomic type names} and a set $\TyOp$
of \textit{type operator names}.  Each type operator name $\op \in \TyOp$ has an
\textit{argument count} $\valency{\op} \in \Nplus$, and a type name $\op(\tn_1,
\ldots, \tn_n)$ is well-formed if and only if $n = \valency{\op}$.  We define
the \textit{universe} $\Uni$ to be the closure of $\Uni_0$ under repeated
applications of operators in $\TyOp$:
$$
\Uni_{i+1} = \Uni_i \cup \{\op(\tn_1, \ldots, \tn_{\valency{\op}}): \op \in \TyOp, \tn_1, \ldots, \tn_{\valency{op}} \in \Uni_i\}
$$
$$
\Uni = \bigcup\{\Uni_i: i \in \Nplus\}
$$%
\nomenclature[Da]{$\TyOp$}{The set of built-in type operator names}%
\nomenclature[Db]{$\op$}{A built-in type operator (an element of $\TyOp$)}%
\nomenclature[Da]{$\Uni_0$}{The set of atomic type names}%
\nomenclature[Da]{$\Uni$}{The universe of built-in types}%
\nomenclature[Dr2]{$T$}{A built-in type (an element of $\Uni$)}

\kwxm{Maybe we could have a judgment like $\Uni \vdash t\ \textsf{type}$
  and use inference rules instead of sets.  That would amount to the same thing but
would be considerably less compact.}

\kwxm{I'm inconsistently using ``type'' and ``type name'' for the things in
  $\Uni$, and that's further complicated by the introduction of polymorphic types later.}

The universe $\Uni$ consists entirely of \textit{names}, and the semantics of
these names are given by \textit{denotations}. Each built-in type $\tn \in \Uni$
is associated with some mathematical set $\denote{\tn}$, the \textit{denotation}
of $\tn$. For example, we might have $\denote{\texttt{bool}}= \{\mathsf{true},
\mathsf{false}\}$ and $\denote{\texttt{integer}} = \mathbb{Z}$ and
$\denote{\pairOf{a}{b}} = \denote{a} \times \denote{b}$.  We assume that if $T,
T^{\prime} \in \Uni$ and $T \ne T^{\prime}$ then $\denote{T}$ and
$\denote{T^{\prime}}$ are disjoint, and we put

$$\denote{\Uni} = \bigdisj{\{\denote{T}: T \in \Uni\}}.$$%%
\nomenclature[Dr4]{$\denote{T}$}{The denotation of a type name $T$}
\nomenclature[Dr5]{$\denote{\Uni}$}{$\bigdisj{\{\denote{T}: T \in \Uni\}}$}

\noindent See Section~\ref{sec:cardano-builtins}
for a description of the types and functions which have already been deployed on
the Cardano blockchain (or will be in the near future).
% \nomenclature[Dz1]{$\denote{T}$}{Denotation of a type $T \in \Uni$}

For non-atomic type names $\tn = \op(\tn_1, \ldots, \tn_r)$ we would generally
expect the denotation of $\tn$ to be obtained in some uniform way (ie,
parametrically) from the denotations of $\tn_1, \ldots, \tn_r$; we do not insist
on this though.

\newcommand{\tv}{\ensuremath{\textit{tv}}}

\subsubsection{Type variables}
\label{sec:type-variables}
Built-in functions can be polymorphic, and to deal with this we need
\textit{type variables}.  An argument of a polymorphic function can be either
restricted to built-in types or can be an arbitrary term, and we define two
different kinds of type variables to cover these two situations.  See
Figure~\ref{fig:type-variables}.

\begin{minipage}{\linewidth}
  \centering
      \[\begin{array}{lrclr}
        \textrm{TypeVariable}    & \tv & ::= & n_{\#} & \textrm{built-in-polymorphic type variable}\\
                                 &    &      & \star & \textrm{fully-polymorphic type variable}\\
    \end{array}\]
    \captionof{figure}{Type variables}
    \label{fig:type-variables}
\end{minipage}
% Previously we had fully polymorphic type variables like a_*, which allowed us
% to re-use this notation with typed PLC.  This was OK until the experiments
% with caseData and caseList, which took arguments which whose types involved
% arrows: the problem was that (a), we couldn't represent these in the UPLC
% "type system", and (b) caseData and caseList in fact returned a HeadSpine
% object and if you supplied a non-function argument it would be embedded in one
% of these and the type mismatch wouldn't be discovered until the HeadSpine was
% later evaluated.  At some point we will remove the notational machinery
% introduced to deal with these builtins, but it will probably make sense to
% keep the single fully-polymorphic type variable * if we do that.

% If we ever specify TPLC we'll need a richer grammar of types for it, and here
% we will need genuine type variables which are polymorphic over terms; these
% will have to undergo an erasure process (replacing all a_*, b_*, ... with *)
% to obtain the types used here.

\medskip
\noindent
We denote the set of all possible type variables by $\Var$ and the set of all
built-in-polymorphic type variables $v_\#$ by $\Var_\#$.  Note that $\Var \cap
\Uni = \varnothing$ since the symbols $\star$ and ${}_\#$ do not occur in names in $\Uni$.%
\nomenclature[Da]{$\Var$}{The set of all type variables, $\Var_\# \cup \VarStar$}%
\nomenclature[Da]{$\Var_\#$}{The set of builtin-polymorphic type variables}%
\nomenclature[Db]{$v_\#$}{A builtin-polymorphic type variable}%
\nomenclature[Db]{$\star$}{The unique fully-polymorphic type variable}%

The two kinds of type variable are required because we have two different kinds
of polymorphism. Later on we will see that built-in functions can take arguments
which can be of a type which is unknown but must be in $\Uni$, whereas other
arguments can range over a larger set of values such as the set of all Plutus
Core terms. Type variables in $\Var_\#$ are used in the former situation and
$\star$ is used in the latter.

\kwxm{I'm using syntax to represent kinds here.  I haven't introduced actual
  kinds because (a) we don't have a proper type system in Plutus Core (yet), and
  (b) the \texttt{TypeScheme} type in the implementation only has one kind of
  type variable.  I'm only using this in the specification of the signatures of
  built-in functions to characterise some aspects how the builtins machinery
  works in practice: things in $\VarStar\backslash\Var_\#$ can fail with an
  unlifting error at runtime but that's not statically enforced anywhere.}

\subsubsection{Polymorphic types}
\label{sec:polymorphic-types}
We also need to talk about polymorphic types, and to do this we define an
extended universe of polymorphic types $\Unihash$ by adjoining $\Var_\#$ to
$\Uni_0$ and closing under type operators as before:

$$
\Unihashn{0} = \Uni_0 \cup \Var_\#
$$
$$
\Unihashn{i+1} = \Unihashn{i} \cup \{\op(\tn_1, \ldots, \tn_{\valency{\op}}): \op \in \TyOp, \tn_1, \ldots, \tn_{\valency{op}} \in \Unihashn{i}\}
$$
$$
\Unihash = \bigcup\{\Unihashn{i}: i \in \Nplus\}.$$%
\nomenclature[Da]{$\Unihash$}{The set of builtin-polymorphic types}%
\noindent We will denote a typical element of $\Unihash$ by the symbol $P$
(possibly subscripted).%
\nomenclature[Ds1]{$P$}{A builtin-polymorphic type}

\noindent We define the set of \textit{free \#-variables} of an element of $\Unihash$ by
$$
\fv{P} = \varnothing \ \text{if $P \in \Uni_0$}
$$
$$
\fv{v_\#} = \{v_\#\}
$$
$$
\fv{\op(P_1, \ldots, P_k)} = \fv{P_1} \cup \fv{P_2} \cup \cdots \cup \fv{P_r}.
$$%
\nomenclature[Ds2]{$\fv{P}$}{Free \#-variables of a polymorphic type $P \in \Unihash$}

\noindent Thus $\fv{P} \subseteq \Var_\#$ for all $P \in \Uni$.  We say that a
type name $P \in \Unihash$ is \textit{monomorphic} if $\fv{P} = \varnothing$ (in
which case we actually have $P \in \Uni$); otherwise $P$ is
\textit{polymorphic}.  The fact that type variables in $\Unihash$ are only
allowed to come from $\Var_\#$ will ensure that values of polymorphic types such
as lists and pairs can only contain values of built-in types: in particular, we
will not be able to construct types representing things such as lists of Plutus
Core terms.


\subsubsection{Type assignments}
\label{sec:type-assignments}
A \textit{type assignment} is a function $S: D \rightarrow \Uni$ where $D$ is
some subset of $\Var_\#$.  As usual we say that $D$ is the \textit{domain} of
$S$ and denote it by $\dom S$.%
\nomenclature[Dr6]{$S$}{A type assignment}

\medskip
\noindent We can extend a type assignment $S$ to a map
$\Sext : \Unihash \disj \VarStar \rightarrow \Unihash \disj \VarStar$ by defining

\begin{align*}
    \Sext(v_\#) &= S(v_\#) \quad \text{if $v_\# \in \dom S$}\\
    \Sext(v_\#) &= v_\# \quad \text{if $v_\# \in \Var_\# \backslash \dom S$}\\
    \Sext(T) &= T \quad\text{if $T \in \Uni_0$}     \\
    \Sext(\op(P_1,\ldots,P_n)) &= \op(\Sext(P_1),\ldots,\Sext(P_n))\\
    \Sext(\star) &= \star.
\end{align*}%
\nomenclature[Dr7]{$\Sext$}{The extension of a type assignment $S$ to $\Unihash \cup \VarStar$}

\noindent If $P \in \Unihash$ and $S$ is a type assignment with $\fv{P}
\subseteq \dom S$ then in fact $\Sext(P) \in \Uni$; in this case we say that
$\Sext(P)$ is an \textit{instance} or a \textit{monomorphisation} of $P$
(\textit{via $S$}).  If $T$ is an instance of $P$ then there is a unique
smallest $S$ (with $\fv{P}=\dom S$) such that $T = \Sext(P)$: we write
$T\preceq_S P$ to indicate that $T$ is an instance of $P$ via $S$ and $S$ is
minimal.%
\nomenclature[Ds3]{$T\preceq_S P$}{$T$ is an instance of $P$ via $S$ and $\dom S = \fv{P}$}

\paragraph{Constructing type assignments.}
We say that a collection $\{S_i: 1 \leq i \leq n\}$ of type assignments is
\textit{consistent} if $S_i|_{D_{ij}} = S_j|_{D_{ij}}$ for all $i$ and $j$,
where $|$ denotes function restriction and $D_{ij} = \dom S_i \ \cap \ \dom
S_j$.  If this is the case then (viewing functions as sets of pairs in the usual
way) $S_1 \cup \cdots \cup S_n$ is also a well-formed type assignment (each
variable in its domain is associated with exactly one type).

\medskip
\noindent Given $T \in \Uni$ and $P \in \Unihash$ it can be shown that $T \preceq_S P$ if
and only if one of the following holds:
\begin{itemize}
  \item $T = P$ and $S =\varnothing$.
  \item $P \in \Var_\#$ and $S = \{(v_\#, T)\}$.
  \item
    \begin{itemize}
    \item $T = \op(T_1, \ldots, T_n)$ with each $T_i \in \Uni$.
    \item $P = \op(P_1, \ldots, P_n)$ with each $P_i \in \Unihash$.
    \item $T_i \preceq_{S_i} P_i$ for $1 \leq i \leq n$.
    \item $\{S_1, \ldots, S_n\}$ is consistent.
    \item $S = S_1 \cup \cdots \cup S_n$.
    \end{itemize}
\end{itemize}

\noindent This allows us to decide whether $T \in \Uni$ is an instance of $P \in
\Unihash$ and, if so, to construct an $S$ with $T \preceq_S P$.

\kwxm{\sf I tried using inference rules for this but the final case
  got kind of out of hand.}

\subsection{Built-in functions}
\label{sec:builtin-functions}

\subsubsection{Inputs to built-in functions}
\label{sec:builtin-inputs}
To treat the typed and untyped versions of Plutus Core uniformly it is necessary
to make the machinery of built-in functions generic over a set $\Inputs$ of
\textit{inputs} which are taken as arguments by built-in functions.  In practice
$\Inputs$ will be the set of Plutus Core values or something very closely
related.%
\nomenclature[Ea]{$\Inputs$}{The set of inputs to built-in functions}

\medskip
\noindent We require $\Inputs$ to have the following two properties:
\begin{itemize}
\item $\Inputs$ is disjoint from $\denote{\tn}$ for all $\tn \in \Uni$
\item There should be disjoint subsets $\Con{\tn} \subseteq \Inputs$ (where $\tn
  \in \Uni$) of \textit{constants of type $\tn$} and maps $\denote{\cdot}_{\tn}:
  \Con{\tn} \rightarrow \denote{\tn}$ (\textit{denotation}) and
  $\reify{\cdot}_{\tn}: \denote{\tn} \rightarrow \Con{\tn}$
  (\textit{reification}) such that $\reify{\denote{c}_{\tn}}_{\tn} = c \text{
    for all } c \in \Con{\tn}$.  We do not require these maps to be bijective
  (for example, there may be multiple inputs with the same denotation), but the
  condition implies that $\denote{\cdot}_{\tn}$ is surjective and
  $\reify{\cdot}_{\tn}$ is injective.
\end{itemize}%
\nomenclature[Da]{$\Con{\tn}$}{Constants of built-in type $T$}%
\nomenclature[Dz2]{$\denote{\cdot}_T$}{Denotation of constants of type $T$}%
\nomenclature[Dz3]{$\reify{\cdot}_T$}{Reification of constants of type $T$}
\kwxm{I've forgotten why I want $\Inputs$ to be disjoint from the
  $\denote{\tn}$s \dots.}

\noindent It is also convenient to let $\denote{\Inputs} = \Inputs$ and define both
  $\denote{\cdot}_{\Inputs}$ and $\reify{\cdot}_{\Inputs}$ to be the identity
function, and we write
$$
\denote{\Uni}_{\Inputs} = \denote{\Uni} \disj \Inputs.
$$

\kwxm{I'm still not sure what to call the things what we feed to builtins.
  Previously there were called ``terms'', but in our setting they're
  \textit{not} actually arbitrary terms.  I tried ``values'' instead, but that
  was confusing.  ``Inputs'' is a bit better, but (a) they're also what builtins
  \textit{output}, and (b) there's a small risk of confusion with UTXO inputs.}

\noindent For example, we could take $\Inputs$ to be the set of all Plutus Core
values (see Section~\ref{sec:uplc-values}), $\Con{\tn}$ to be the set of all
terms of the form $\con{\tn}{c}$, and $\denote{\cdot}_{\tn}$ to be the function
which maps $\con{\tn}{c}$ to $c$.  For simplicity we are assuming that
mathematical entities occurring as members of type denotations $\denote{\tn}$
are embedded directly as values $c$ in Plutus Core constant terms. In reality,
tools which work with Plutus Core will need some concrete syntactic
representation of constants; we do not specify this here, but see
Section~\ref{sec:cardano-builtins}
for suggested syntax for the built-in types currently in use on the Cardano
blockchain.

\subsubsection{Outputs of built-in functions}
\label{sec:builtin-outputs}
All built-in functions either fail or conceptually return a non-empty list whose
entries lie either in the denotation of some built-in type $T$ or in the set of
inputs $\Inputs$, i.e., builtins return elements of the set ${(\R^+)_{\errorX}}$,
where

$$
\R := \bigdisj\left\{\denote{\tn}: \tn \in \Uni \right\} \disj \Inputs.
$$%%
\nomenclature[Eb]{$\R$}{The set of single values returned by built-in functions}%%
\nomenclature[Ec]{$\R^+$}{The set of sequences of values returned by built-in functions}

\noindent We will denote elements of $\R^+$ by expressions of the form $(v|v_1,
\ldots, v_k)$ with $v, v_i \in \R$ and $k \geq 0$, the case $k=0$ indicating a
list $(v|)$ with a single entry.  Currently all builtins return a single
value, and to simplify notation we will identify $\R$ with $\{(v|): v \in \R\}
\subseteq \R^+$.  The intention is that $(v|v_1, \ldots, v_k)$ will
immediately be interpreted as an application $v \;v_1\; \ldots\; v_k$ in the
ambient language (eg, typed or untyped Plutus Core); the number of arguments $k$
may depend on the values of the inputs to the function.

\subsubsection{Signatures and denotations of built-in functions}
\label{sec:signatures}
We will consistently use the symbol $\tau$ and subscripted versions of it to
denote members of $\UnihashStar$ in the rest of the document; these indicate the
types of values consumed by built-in functions.%
\nomenclature[Eg]{$\tau$}{A member of $\UnihashStar$, ie a type or type variable}

\medskip
\noindent We also define a class of \textit{quantifications} which are used to
introduce type variables: a quantification is a symbol of the form
$\forallty{v}$ with $v \in \Var$; the set of of all possible quantifications is
denoted by $\QVar$.%
\nomenclature[Ez3]{$\forallty{v}$}{Type quantification}%
\nomenclature[Eb]{$\QVar$}{The set of all type quantifications}

\medskip
\paragraph{Signatures.}
Every built-in function $b \in \Fun$ has a \textit{signature} $\sigma(b)$ which describes
the types of its arguments and its return value: a signature is of the form
$$[\iota_1, \ldots, \iota_n] \rightarrow \omega$$ with
\begin{itemize}
  \item $\iota_j \in \UnihashStar \disj \QVar \enspace\text{for all $j$}$
  \item $\omega \in \UnihashStarAp$
  \item $\lvert\{j : \iota_j \notin \QVar\}\rvert \geq 1$ (so $n \geq 1$)
  \item If $\iota_j$ involves $v \in \Var$ then $\iota_k = \forallty{v}$ for
    some $k < j$, and similarly for $\omega$; in other words, any type variable
    $v$ must be introduced by a quantification before it is used. (Here $\iota$
    \textit{involves} $v$ if either $\iota = \tn \in \Unihash$ and $v \in \fv{\tn}$
    or $\iota = v$ and $v \in \VarStar$.)
  \item If $\omega$ involves $v \in \Var$ then some $\iota_j$ must involve $v$;
    this implies that $\fv{\omega} \subseteq \bigcup \{\fv{\iota_j}: \iota_j \in
    \Unihash\}$ (where we extend the earlier definition of $\mathsf{FV}_{\#}$ by
    setting $\fv{\ap}=\varnothing$).
  \item If $j \neq k$ and $\iota_j, \iota_k \in \QVar$ then $\iota_j \neq
    \iota_k$; ie, no quantification appears more than once.
  \item If $\iota_i = \forall v \in \QVar$ then some $i_j \notin \QVar$ with $j
    > i$ must involve $v$ (signatures are not allowed to contain phantom type variables).
\end{itemize}%
\nomenclature[Eg]{$\iota$}{Signature item}%
\nomenclature[Eg]{$\sigma$}{Signature of built-in function}%
\nomenclature[Eg]{$\omega$}{Return type of built-in function}


\kwxm{\textsf{@effectfully} says that a builtin could actually return something
  in $\QVar$ too, so the target of the arrow could be an $\iota$. We don't need
  that at the moment though.}

\noindent For example, in our default set of built-in functions we have the
functions \texttt{mkCons} with signature $[\forall a_\#, a_\#,$  %% Allow line break
  $\listOf{a_\#}] \rightarrow \listOf{a_\#}$ and \texttt{ifThenElse} with signature
$[\forallStar, \mathtt{bool}, \star, \star] \rightarrow \star$.  When we use
\texttt{mkCons} its arguments must be of built-in types, but the two final
arguments of \texttt{ifThenElse} can be any Plutus Core values.

\smallskip
\noindent If $b$ has signature $[\iota_1, \ldots, \iota_n] \rightarrow \omega$ then we define
the \textit{arity}  of $b$ to be
$$
\alpha(b) = [\iota_1, \ldots, \iota_n].
$$%
\nomenclature[Eg]{$\alpha$}{Arity of built-in function}

\noindent We also define
$$
  \chi(b) = n.
$$%
\nomenclature[Eg]{$\chi$}{Number of arguments of built-in function}

\noindent We may abuse notation slightly by using the symbol $\sigma$ to denote
a specific signature as well as the function which maps built-in function names
to signatures, and similarly with the symbol $\alpha$.

\medskip
\noindent Given a signature
$\sigma = [\iota_1, \ldots, \iota_n] \rightarrow \omega$,
we define the \textit{reduced signature} $\sigmabar$ to be
$$
\sigmabar = [\iota_j : \iota_j \notin \QVar] \rightarrow \omega
$$%
\nomenclature[Eg]{$\sigmabar$}{Reduced signature}

\noindent Here we have extended the usual set comprehension notation to lists in the
obvious way, so $\sigmabar$ just denotes the signature $\sigma$ with all
quantifications omitted. We will often write a reduced signature in the form
$[\tau_1, \ldots, \tau_m] \rightarrow \omega$ to emphasise that the entries are
\textit{types}, and $\mathbf{\forall}$ does not appear.

\medskip
\noindent Also, given an arity $= [\iota_1, \ldots, \iota_n]$, the \textit{reduced
  arity} is
$$
\alphabar = [\iota_j : \iota_j \notin \QVar].
$$%
\nomenclature[Eg]{$\alphabar$}{Reduced arity}

\paragraph{Commentary.} What is the intended meaning of the notation introduced
above?  In Typed Plutus Core we have to instantiate polymorphic functions (both
built-in functions and polymorphic lambda terms) at concrete types before they
can be applied, and in Untyped Plutus Core instantiation is replaced by an
application of \texttt{force}.  When we are applying a built-in function we
supply its arguments one by one, and we can also apply \texttt{force} (or
perform type instantiation in the typed case) to a partially-applied builtin
``between'' arguments (and also after the final argument); no computation occurs
until all arguments have been supplied and all \texttt{force}s have been
applied. The arity (read from left to right) specifies what types of arguments
are expected and how they should be interleaved with applications of
\texttt{force}, and $\chi(b)$ tells you the total number of arguments and
applications of \texttt{force} that a built-in function $b$ requires. The
fully-polymorphic type variable $\star$ indicates that an arbitrary value from
$\Inputs$ can be provided, whereas a type from $\Unihash$ indicates that a value
of the specified built-in type is expected. Occurrences of quantifications
indicate that \texttt{force} is to be applied to a partially-applied builtin; we
allow this purely so that partially-applied builtins can be treated in the same
way as delayed lambda-abstractions: \texttt{force} has no effect unless it is
the very last item in the signature.  In Plutus Core, partially-applied
builtins are values which can be treated like any others (for example, by being
passed as an argument to a \texttt{lam}-expression): see
Section~\ref{sec:uplc-values}.

In general a builtin returns a sequence $(v|v_1,\ldots,v_k) \in \R^+$, but in
fact the majority of builtins currently deployed on Cardano only return a single
value, and in this case we can specify a signature where $\omega$ is either a
built-in type name $T$ or $\star$, denoting an input (typically a value in the
ambient language), and this tells us exactly what sort of value is returned.
The general case is considerably more complicated: the size of the list
returned, and the types of its entries, may be different for different
inputs. To specify this sort of behaviour precisely in a signature would require
a considerable increase in the complexity of the notation for signatures, so
instead we approximate all return types involving elements of $\R^+\backslash
\R$ by $\ap$.  However, when specifying the semantics of particular builtins
with $\omega = \ap$ we will always give a precise description of the possible
return values.

\subsubsection{Denotations of built-in functions}
\label{sec:builtin-denotations}
The basic idea is that a built-in function $b$ should represent some
mathematical function on the denotations of the types of its inputs.  However,
this is complicated by the presence of polymorphism and we have to require that
there is such a function for every possible monomorphisation of $b$.

More precisely, suppose that we have a builtin $b$ with reduced signature
$[\tau_1, \ldots \tau_n] \rightarrow \omega$.  For every type assignment $S$ with
$\dom S = \fv{\tau_1} \cup \cdots \cup \fv{\tau_n}$ (which contains $\fv{\omega}$ by
the conditions on signatures in Section~\ref{sec:signatures}) we require a
\textit{denotation of $b$ at $S$}, a function
$$
\denote{b}_S: \denote{\Sext(\tau_1)} \times \cdots \times \denote{\Sext(\tau_n)} \rightarrow \withError{\denote{\Sext(\omega)}}
$$%
\nomenclature[Ez4]{$\denote{b}_S$}{The denotation of the built-in function $b$ at the type assignment $S$}
\noindent where
$$
\denote{\star} = \Inputs \quad\text{and}\quad \denote{\ap} = \R^+.
$$
\noindent This makes sense because $\Sext(\tau_i) \in \Uni \disj
\Inputs$ for all $i$, so $\denote{\Sext(\tau_i)}$ is always defined,
and similarly for $\omega$ (extending $\Sext$ by setting $\Sext(\ap) = \ap$).

\medskip
\noindent If $\fv{\sigmabar(b)} = \varnothing$ (in which case we say that $b$ is
\textit{monomorphic}) then the only relevant type assignment will be the empty
one; in this case we have a single denotation
$$
\denote{b}_\varnothing: \denote{\tau_1} \times \cdots \times \denote{\tau_n} \rightarrow \withError{\denote{\omega}}.
$$

\smallskip
\noindent Denotations of builtins are mathematical functions which terminate on
every possible input; the symbol $\errorX$ can be returned by a function to
indicate that something has gone wrong, for example if an argument is out of
range.

\smallskip
\noindent In practice we expect most builtins to be \textit{parametrically
  polymorphic}~\cite{Wadler-theorems-for-free, Reynolds-parametric}, so that the
denotation $\denote{b}_S$ will be the ``same'' for all type assignments $S$; we do
not insist on this though.

\subsubsection{Results of built-in functions.}
\label{sec:builtin-results}
Recall from Section~\ref{sec:builtin-outputs} that the result of the
evaluation of a built-in function lies in the set 
$$
(\R^+)_{\errorX} = \left(\bigdisj\left\{\denote{\tn}: \tn \in \Uni \right\} \disj \Inputs \right)^+ \disj \{\errorX\}.
$$
\noindent Since we have assumed that all denotations $\denote{T}$ with $T \in
\Uni$ are disjoint from each other and from $\Inputs$
(Section~\ref{sec:builtin-inputs}) we can define a function
$$
\reify{\cdot}: \R \rightarrow \withError{\Inputs}
$$
which converts elements $r \in \R$ back into inputs by
$$
\reify{r} = 
\begin{cases}
  \reify{r}_{\tn} \in \Con{\tn} \subseteq \Inputs & \text{if $r \in \denote{\tn}$}\\
  r & \text{if $r \in \Inputs$}
\end{cases}
$$
\noindent(see Section~\ref{sec:builtin-inputs} for the definition of
$\reify{\cdot}_{\tn}$), and we can extend this to a function
$\reify{\cdot}: (\R^+)_{\errorX} \rightarrow \withError{\Inputs}$ by defining

\begin{align*}
  \reify{(r, r_1, \ldots, r_k)} &= (\reify{r}|\reify{r_1}, \ldots, \reify{r_k})\\
  \reify{\errorX} &= \errorX.
\end{align*}%%
\nomenclature[Dz4]{$\reify{\cdot}$}{Reification of the result of a built-in
  function application}

\kwxm{We have to use $\Inputs \disj \{\errorX\}$ to deal with the fact that
  $\errorU$ isn't a value, so we have to defer handling errors until later.}
\kwxm{The notation here is maybe a bit confusing.  The $\Con{\tn}$ live in the
  syntactic world (where they're subsets of $\Inputs$) and the $\denote{\tn}$
  live in the world of sets; however $\Inputs$ lives in \textit{both} worlds,
  and it's disjoint from all of the $\denote{\tn}$ in the world of sets but not
  in the world of syntax.  I think we do need something like this because
  sometimes a builtin argument must be a constant but at other times it can be
  an arbitrary value, which includes all of the constants.}

\subsubsection{Parametricity for fully-polymorphic arguments}
\label{sec:builtin-behaviour}
A built-in function $b$ can only inspect arguments which are values of built-in
types; other arguments (occurring as $\star$ in $\sigmabar(b)$) are treated
opaquely, and can be discarded or returned as (part of) a result, but cannot be
altered or examined (in particular, they cannot be compared for equality): $b$
is \textit{parametrically polymorphic} in such arguments.  This implies that if
the sequence returned by a builtin contains a value $v \in \Inputs$, then $v$
must have been an argument of the builtin.
\roman{This is not quite true, unfortunately. \texttt{toBuiltinMeaning}
  constrains \texttt{term} with \texttt{HasConstantIn uni term}. This means that we can
  check whether an argument whose type is a type variable is a constant or
  not. And if it's a constant, we can obtain the type tag and do all kinds of
  fancy things with it. For example a builtin checking if two values of
  different types are equal constants is representable. This breaks parametricity.
}

\kwxm{I think we should just require that people don't do that. $\uparrow$}

%% When (the meaning of) a built-in function $b$ is applied (perhaps partially) to
%% arguments, the types of constant arguments must correspond to the types in
%% $\sigmabar(b)$, and the function will return $\errorX$ if this is not the
%% case; builtins may also return $\errorX$ in other circumstances, for example if
%% an argument is out of range.

\kwxm{A lot of the complexity we have here is due to the fact that we've got
  explicit \texttt{delay} and \texttt{force} instead of the usual $\lambda().M$
  and $M()$.  We use the explicit version because experiments showed that it was
  noticeably faster (and we have a lot of these due to erasure of type-level
  abstraction/instantiation).  Also, who's to say that the universe contains a
  unit type?
}

%% \begin{minipage}{\linewidth}
%%     \centering \[\begin{array}{llll}

%%     t & ::= & b   &\text{for } b \in \Uni  \\
%%     & & \typearg                 \\
%%     & & \star               \\
%%     & & t \rightarrow t          \\
%%     \end{array}\]
%% \end{minipage}


\subsection{Evaluation of built-in functions}
\label{sec:builtin-evaluation}

\subsubsection{Compatibility of inputs and signature entries}
\label{sec:compatibility}
The previous section describes how a built-in function is interpreted as a
mathematical function.  When a Plutus Core built-in function $b$ is applied to a
sequence of arguments, the arguments must have types which are compatible with
the signature of $b$; for example, if $b$ has signature
$[\forallStar, \forall a_\#, \forall b_\#, a_\#, b_\#, a_\#, \star, \star] \rightarrow \star$ and $b$
is applied to a
sequence of inputs $V_1, V_2, V_3, V_4, V_5$ then $V_1, V_2$, and $V_3$ must all
be constants of some monomorphic built-in types and the types of $V_1$ and $V_3$
must be the same; $V_4$ and $V_5$ can be arbitrary inputs.  This section
describes the conditions for type compatibility.

\kwxm{The stuff about consistent instantiation of built-in-polymorphic type
  variables may in fact be slightly inaccurate.  I think that it's not enforced
  by the builtin application machinery, but must be checked dynamically in the
  implementation of the builtin when it's fully applied (\texttt{mkCons} is an
  example of this). It would be perfectly possible (I think) to implement a
  builtin with signature $[\forall a_\#, a_\#, a_\#] \rightarrow a_\#$ which
  could be successfully applied to arguments of two different built-in types,
  returning a constant of a third type.  However, I think it'll be very hard to
  describe what's actually going on so I'm just assuming that builtins always
  check this stuff.}

\medskip
\noindent In detail, given a reduced arity $\alphabar = [\tau_1, \ldots,
  \tau_n]$, a sequence $\bar{V} = [V_1, \ldots, V_m]$, and a type assignment $S$
we say that $\bar{V}$ is \textit{compatible with} $\alphabar$ (\textit{via} $S$)
if and only if $n=m$ and, letting $I = \{i: 1 \leq i \leq n, \tau_i \in
\Unihash\}$ (so $\tau_j = \star$ if $j \notin I$), there exist type
assignments $S_i$ ($1 \leq i \leq n$) such that all of the following are
satisfied
\begin{itemize}
\item For all $i \in I$ there exists $T_i \in \Uni$ such that $V_i \in \Con{T_i}$ and $T_i \preceq_{S_i} \tau_i$.
\item $\{S_i: i \in I\}$ is consistent (see Section~\ref{sec:type-assignments}).
\item $S = \bigcup\{S_i: i \in I\}$.
\end{itemize}

\noindent If these conditions are all satisfied then we can find suitable $S_i$
using the procedure described in Section~\ref{sec:type-assignments} and this
allows us to construct $S$ explicitly since the $S_i$ are consistent.  Note that
in this case $\dom S = \dom S_1 \cup \ldots \cup \dom S_n = \fv{\tau_1} \cup
\cdots \cup \fv{\tau_n} = \fv{\alpha}$, so $S$ is minimal in the sense that no
$S'$ with $\dom S'$ strictly smaller than $\dom S$ is sufficient
to monomorphise all of the $\tau_i$ simultaneously.  We write
$$
[V_1, \ldots, V_m] \approx_S [\tau_1, \ldots, \tau_n]
$$
in this case.  If $\bar{V}$ is not compatible with $\alphabar$ then we write
$\bar{V} \napprox \alphabar$.%
\nomenclature[Es2]{$\approx_S$}{Compatibility of built-in function arguments with function arity via $S$}

\subsubsection{Evaluation}
\label{sec:eval}
For later use we define a function $\Eval$ which attempts to evaluate an
application of a built-in function $b$ to a sequence of inputs $[V_1, \ldots,
  V_m]$.  This fails if the number of inputs is incorrect or if the inputs are
not compatible with $\alphabar(b)$:
$$
\Eval(b,[V_1, \ldots, V_n]) = \errorX \quad \text{if $[V_1, \ldots, V_n] \napprox \alphabar(b)$}.
$$

\noindent Otherwise, the conditions for the existence of a denotation of $b$ are
met and we can apply that denotation to the denotations of the inputs and then
reify the result. If $[V_1, \ldots, V_n] \approx_S \alphabar(b) = [\tau_1,
  \ldots, \tau_n]$, let $T_i = \Sext(\tau_i)$ for $1 \leq i \leq n$; then we
define

$$
\Eval(b,[V_1, \ldots, V_n]) = \reify{\denote{b}_S (\denote{V_1}_{T_1}, \ldots, \denote{V_n}_{T_n})}.
$$%
\nomenclature[Ez5]{$\Eval$}{Evaluation of built-in functions}%
%
\noindent It can be checked that the compatibility condition guarantees that
this makes sense according to the definition of $\denote{b}_S$ in
Section~\ref{sec:builtin-denotations}.

\paragraph{Notes.}
\begin{itemize}
  \item All of the machinery which we have defined for built-in functions is
    parametric over the set $\Inputs$ of inputs and the sets $\Con{T} \subseteq
    \Inputs$ of constants. This also applies to the $\Eval$ function, so its
    meaning is not fully defined until we have given concrete definitions of the
    sets of inputs and constants.
  \item The error value $\errorX$ can occur in two different ways: either
    because the arguments are not compatible with the signature, or because the
    builtin itself returns $\errorX$ to signal some error condition.
\item The symbol $\errorX$ is not part of Plutus Core; when we define reduction
  rules and evaluators for Plutus Core later some extra translation will be
  required to convert the result of $\Eval$ into something appropriate to the
  context.
\end{itemize}

\kwxm{For type operators, ``polymorphic'' really means ``polymorphic over
  built-in types''.}

\kwxm{$\Uni$ is the set of built-in types and $\Unihash$ is
  that set extended to include polymorphic types as well.  Later on we quite
  often have to look at $\Unihash\backslash \Uni$ to talk about types that really
  are polymorphic. Maybe it would be better to have a separate universe of
  polymorphic types called $\mathscr{P}$ or something, but then we'd also have to
  talk about $\Uni \disj \mathscr{P}$ sometimes.  Maybe we could define $\Unihash =
  \Uni \disj \mathscr{P}$?
  }
