\documentclass[a4paper]{report}

\title{Formal Specification of\\the Plutus Core Language\\
  \vspace{5mm}
  \LARGE{\red{\textsf{DRAFT}}}
}

\date{30th May 2025}
\author{Plutus Core Team}

\renewcommand{\thefootnote}{\fnsymbol{footnote}}

% correct bad hyphenation here
\hyphenation{}

\usepackage[colorlinks=true,linkcolor=MidnightBlue,citecolor=ForestGreen,urlcolor=Plum,pageanchor]{hyperref}
% ^ pageanchor is for nomencl

%% Stuff for index of notation
%% FYI, the nomencl package seems to discard entries containing '@' 
\usepackage[intoc,refpage]{nomencl}
\renewcommand{\nomname}{Index of Notation}
\renewcommand{\pagedeclaration}[1]{,\ \ \hyperlink{page.#1}{\nobreakspace#1}}
\setlength{\nomitemsep}{-0.6\parsep} % Adjust vertical spacing between items in index
\renewcommand{\nomgroup}[1]{\medskip%  Grouping for entries in symbol index
  \ifthenelse{\equal{#1}{A}}{\item[\textbf{Sets}]}{%
    \ifthenelse{\equal{#1}{B}}{\item[\textbf{Lists}]}{%
      \ifthenelse{\equal{#1}{C}}{\item[\textbf{Plutus Core grammar}]}{%
        \ifthenelse{\equal{#1}{D}}{\item[\textbf{Built-in types}]}{%
          \ifthenelse{\equal{#1}{E}}{\item[\textbf{Built-in functions}]}{%
            \ifthenelse{\equal{#1}{F}}{\item[\textbf{Term reduction}]}{%
              \ifthenelse{\equal{#1}{G}}{\item[\textbf{CEK machine}]}{%
                \ifthenelse{\equal{#1}{H}}{\item[\textbf{Serialisation and deserialisation}]}{}
              }
            }
          }
        }
      }
    }
  }
}
\makenomenclature

\usepackage{tocbibind}  % To get the bibiliograpy in the table of contents
\usepackage{blindtext, graphicx}
\usepackage{url}
%\usepackage[numbers]{natbib}
\usepackage{natbib}

% *** FONTS ***
%
\usepackage{amsmath}
%\usepackage{amssymb}
\usepackage{stmaryrd}
\usepackage{stix}
\usepackage{alltt}
\usepackage{bussproofs}
\usepackage[mathscr]{euscript}
\DeclareMathAlphabet{\mathcal}{OMS}{cmsy}{m}{n}

% *** FIGURES/ALIGNMENT ***
%
\usepackage{array}
\usepackage{float}  %% Try to improve placement of figures.  Doesn't work well with subcaption package.
\usepackage{subcaption}
\usepackage{fancyvrb}
%% \usepackage{caption}

\usepackage{subfiles}
\usepackage{geometry}
\usepackage{pdflscape}
\usepackage[title]{appendix}

\usepackage[T1]{fontenc}
\usepackage[dvipsnames]{xcolor}

\usepackage{listings}
\lstset{
  basicstyle=\ttfamily,
  columns=fullflexible,
  mathescape=true,
  escapeinside={|}{|}   %% Inside listings you can say things like |\textit{blah blah}|
}

\usepackage{longtable}

% You're supposed to make this the final package
\usepackage[disable]{todonotes}
% \usepackage{todonotes}


\newcommand{\kwxm}[1]{\smallskip\todo[inline,color=yellow!72,author=kwxm,caption={}]{#1}\smallskip}
\renewcommand{\roman}[1]{\smallskip\todo[inline,color=green,author=effectfully,caption={}]{#1}\smallskip}

\newcommand{\TT}[1]{\texttt{#1}}  %% This is used in some tables to make the Latex more compact

\newcommand{\termarg}{\smblkcircle}
\newcommand{\typearg}{\circ}  % stix redefines \bullet to something smaller

% % Stuff for splitting figures over page breaks
% \DeclareCaptionLabelFormat{continued}{#1~#2 (Continued)}
% \captionsetup[ContinuedFloat]{labelformat=continued}


%%% General Misc. Definitions

\newcommand{\red}[1]{\textcolor{red}{#1}}
\newcommand{\redfootnote}[1]{\red{\footnote{\red{#1}}}}
\newcommand{\blue}[1]{\textcolor{blue}{#1}}
\newcommand{\bluefootnote}[1]{\blue{\footnote{\blue{#1}}}}

\newcommand{\diffbox}[1]{\text{\colorbox{lightgray}{\(#1\)}}}
\newcommand{\judgmentdef}[2]{\fbox{#1}

\vspace{0.5em}

#2}

\newcommand{\hyphen}{\operatorname{-}}
\newcommand{\repetition}[1]{\overline{#1}}
\newcommand{\Fomega}{F$^{\omega}$}
\newcommand{\keyword}[1]{\texttt{#1}}
\newcommand{\inparens}[1]{\texttt{(} #1 \texttt{)}}
\newcommand{\construct}[1]{\texttt{(} #1 \texttt{)}}
% \newcommand\discharge[1]{\widehat{#1}}

%%% Term Grammar

\newcommand{\sig}[3]{[#1](#2)#3}
\newcommand{\constsig}[1]{#1}
\newcommand{\con}[2]{\inparens{\keyword{con} ~ #1 ~ #2}}
\newcommand{\abs}[3]{\inparens{\keyword{abs} ~ #1 ~ #2 ~ #3}}
\newcommand{\inst}[2]{\texttt{\{}#1 ~ #2\texttt{\}}}
\newcommand{\lam}[3]{\inparens{\keyword{lam} ~ #1 ~ #2 ~ #3}}
\newcommand{\app}[2]{\texttt{[} #1 ~ #2 \texttt{]}}
\newcommand{\iwrap}[3]{\inparens{\keyword{iwrap} ~ #1 ~ #2 ~ #3}}
\newcommand{\wrap}{\iwrap}
%% ^ Temporary fix to avoid substituting all occurrences of new keyword
\newcommand{\unwrap}[1]{\inparens{\keyword{unwrap} ~ #1}}
\newcommand{\builtin}[1]{\inparens{\keyword{builtin} ~ \mathit{#1}}}
\newcommand{\error}[1]{\inparens{\keyword{error} ~ #1}}

%% Extra untyped terms
\newcommand{\lamU}[2]{\inparens{\keyword{lam} ~ #1 ~ #2        }}
\newcommand{\appU}[2]{\texttt{[} #1 ~ #2 \texttt{]}}
\newcommand{\builtinappU}[3]{\inparens{\keyword{builtin} ~ #1 ~ #2 ~ #3}}
\newcommand{\errorU}{\inparens{\keyword{error}}}
\newcommand{\delay}[1]{\inparens{\keyword{delay} ~ #1}}
\newcommand{\force}[1]{\inparens{\keyword{force} ~ #1}}
\newcommand{\constr}[2]{\inparens{\keyword{constr} ~ #1 ~ #2}}
\newcommand{\kase}[2]{\inparens{\keyword{case} ~ #1 ~ #2}}

\newcommand{\erase}[1]{\llbracket#1\rrbracket}

%%%  Type Grammar

\newcommand{\funT}[2]{\inparens{\keyword{fun} ~ #1 ~ #2}}
\newcommand{\ifixT}[2]{\inparens{\keyword{ifix} ~ #1 ~ #2}}
\newcommand{\fixT}{\ifixT}
%% ^ Temporary fix to avoid substituting all occurrences of new keyword
\newcommand{\allT}[3]{\inparens{\keyword{all} ~ #1 ~ #2 ~ #3}}
\newcommand{\conIntegerType}[1]{\keyword{integer}}
\newcommand{\conBytestringType}[1]{\keyword{bytestring}}
\newcommand{\conT}[1]{\inparens{\keyword{con} ~ #1}}
\newcommand{\lamT}[3]{\inparens{\keyword{lam} ~ #1 ~ #2 ~ #3}}
\newcommand{\appT}[2]{\texttt{[} #1 ~ #2 \texttt{]}}

\newcommand{\typeK}{\inparens{\keyword{type}}}
\newcommand{\funK}[2]{\inparens{\keyword{fun} ~ #1 ~ #2}}



%%% Program Grammar

\newcommand{\version}[2]{\inparens{\keyword{program} ~ #1 ~ #2}}

\newcommand{\evalbuiltin}[2]{\llbracket #1 ~ #2 \rrbracket}

%%% Judgments

\newcommand{\hypJ}[2]{#1 \vdash #2}
\newcommand{\ctxni}[2]{#1 \ni #2}
\newcommand{\validJ}[1]{#1 \ \operatorname{valid}}
\newcommand{\termJ}[2]{#1 : #2}
\newcommand{\typeJ}[2]{#1 :: #2}
\newcommand{\istermJ}[2]{#1 : #2}
\newcommand{\istypeJ}[2]{#1 :: #2}



%%% Contextual Normalization
\newcommand{\pba}{B}  % Possibly ill-formed partial builtin application
\newcommand{\pbas}{\mathsf{\pba}} % Set of all partial builtin applications
\newcommand{\pbasize}[1]{\lVert #1 \rVert}

\newcommand{\ctxsubst}[2]{#1\{#2\}}
\newcommand{\typeStep}[2]{#1 ~ \rightarrow_{ty} ~ #2}
\newcommand{\typeMultistep}[2]{#1 ~ \rightarrow_{ty}^{*} ~ #2}
\newcommand{\typeBoundedMultistep}[3]{#2 ~ \rightarrow_{ty}^{#1} ~ #3}
\newcommand{\step}[2]{#1 ~ \rightarrow ~ #2}
\newcommand{\normalform}[1]{\lfloor #1 \rfloor}
\newcommand{\subst}[3]{[#1/#2]#3}
\newcommand{\kindEqual}[2]{#1 =_{\mathit{k}} #2}
\newcommand{\typeEqual}[2]{#1 =_{\mathit{ty}} #2}
\newcommand{\typeEquiv}[2]{#1 \equiv_{\mathit{ty}} #2}


\newcommand{\inConTFrame}[1]{\conT{#1}}
\newcommand{\inAppTLeftFrame}[1]{\appT{\_}{#1}}
\newcommand{\inAppTRightFrame}[1]{\appT{#1}{\_}}
\newcommand{\inFunTLeftFrame}[1]{\funT{\_}{#1}}
\newcommand{\inFunTRightFrame}[1]{\funT{#1}{\_}}
\newcommand{\inAllTFrame}[2]{\allT{#1}{#2}{\_}}
\newcommand{\inFixTLeftFrame}[1]{\fixT{\_}{#1}}
\newcommand{\inFixTRightFrame}[1]{\fixT{#1}{\_}}
\newcommand{\inLamTFrame}[2]{\lamT{#1}{#2}{\_}}

\newcommand{\inBuiltin}[5]{\builtin{#1}{#2}{#3 #4 #5}}

%% Inputs for the untyped CEK machine
\newcommand{\invalue}[1]{\texttt{\textlangle} #1 \texttt{\textrangle}}

\newcommand{\VCon}[2]{\invalue{\keyword{con} ~ #1 ~ #2}}
\newcommand{\VDelay}[2]{\invalue{\keyword{delay} ~ #1 ~ #2}}
\newcommand{\VLamAbs}[3]{\invalue{\keyword{lam} ~ #1 ~#2 ~ #3}}
\newcommand{\VBuiltin}[3]{\invalue{\keyword{builtin} ~ #1 ~ #2 ~ #3}}
\newcommand{\VConstr}[2]{\invalue{\keyword{constr} ~ #1 ~ #2}}
 %% (builtin bn v_1 ... v_k _ m_{k+1} ... m_n, arity)

%%% CK Machine Normalization

\newcommand{\compute}{\triangleright}
\newcommand{\return}{\triangleleft}
\newcommand{\cekerror}{\mdblkdiamond}
\newcommand{\cekhalt}[1]{\mdwhtsquare\,#1}
\newcommand{\unload}[1]{\mathcal{U}(#1)}
\newcommand{\Unload}[2]{{#2}@{#1}}

\newcommand{\bcompute}{\color{blue}\compute}  
\newcommand{\breturn}{\color{blue}\return}  
\newcommand{\bmapsto}{\color{blue}\mapsto}
% This is to get blue symbols after an '&' inside an alignat
% environment.  It seems to mess up the spacing if you do anything
% else.



\newcommand{\inInstLeftFrame}[1]{\inst{\_}{#1}}
\newcommand{\inWrapRightFrame}[2]{\iwrap{#1}{#2}{\_}}
\newcommand{\inUnwrapFrame}{\unwrap{\_}}
\newcommand{\inAppLeftFrame}[1]{\app{\_}{#1}}
\newcommand{\inAppLeftFrame}[1]{\app{\_}{#1}}
\newcommand{\inAppRightFrame}[1]{\app{#1}{\_}}
\newcommand{\inConstrFrame}[3]{\inparens{\keyword{constr} ~ #1 ~ #2 ~ \_ ~ #3}}
\newcommand{\inCaseFrame}[1]{\inparens{\keyword{case} ~ \_ ~ #1}}

% Extra frames for untyped term normalisation
\newcommand{\inForceFrame}{\force{\_}}
\newcommand{\inBuiltinU}[4]{\builtin{#1}{#2 #3 #4}}

% These are for use inside listings and $...$.  If you just use
% \textit in listings it uses the italic tt font and the spacing
% inside the words is a bit strange.  Spacing is also bad if you
% just put something like "integer" in math text.
\newcommand\unit{\ensuremath{\mathit{unit}}}
\newcommand\one{\ensuremath{\mathit{one}}}
\renewcommand\boolean{\ensuremath{\mathit{boolean}}}
\newcommand\integer{\ensuremath{\mathit{integer}}}
\newcommand\bytestring{\ensuremath{\mathit{bytestring}}}
\newcommand\str{\ensuremath{\mathit{str}}}
\newcommand\true{\ensuremath{\mathit{true}}}
\newcommand\false{\ensuremath{\mathit{false}}}
\newcommand\case{\ensuremath{\mathit{case}}}
\newcommand\signed{\ensuremath{\mathit{signed}}}
\newcommand\txhash{\ensuremath{\mathit{txhash}}}
\newcommand\pubkey{\ensuremath{\mathit{pubkey}}}
\newcommand\blocknum{\ensuremath{\mathit{blocknum}}}
%% \newcommand\uniqmem[1]{#1^{\blacktriangledown}}  %% Unique member of size type

\newcommand\listOf[1]{\mathtt{list}\,\mathtt{(}#1\mathtt{)}}
\newcommand\pairOf[2]{\mathtt{pair}\,\mathtt{(}#1\mathtt{,}\,#2\mathtt{)}}


%% \newcommand\disj{\mathbin{\dot{\cup}}}    % Infix disjoint union
%% \newcommand\bigdisj{\bigcup^{\bullet}}      % Prefix disjoint union
\newcommand\disj{\uplus}                  % Infix disjoint union
\newcommand\bigdisj{\biguplus}            % Prefix disjoint union
\newcommand\TyOp{\mathscr{O}}             % Type operators
\newcommand\Var{\mathscr{V}}              % Type variables
\newcommand\QVar{\mathscr{Q}}             % Quantified type variables
\newcommand\Uni{\mathscr{U}}              % Universe of built-in types
\newcommand\Unihash{{\mathscr{U}_{\#}}}     % Universe of built-in types extended with type variables
\newcommand\UniTop{\Uni^{\top}}      % Universe of built-in types extended with a top element
\newcommand\Unihashn[1]{\Uni_{\#,#1}}
\newcommand\Sext{\hat{S}}
\newcommand\UnihashStar{\Unihash \disj \Var_*} % Built-in type or type name
\newcommand\UnihashTop{\Unihash^{\top}}      % Universe of built-in types extended with a top element
\newcommand\op{\textit{op}}               % Type operator names
\newcommand\valency[1]{\left| #1 \right|} % Valency of a type operator
\newcommand\Fun{\mathscr{B}}              % Built-in functions
\newcommand\Inputs{\mathscr{I}}           % Built-in functions
\newcommand\Con[1]{\mathscr{C}_{#1}}  % Constant terms of a given types
\newcommand\R{\mathscr{R}}           % Things that a builtin can return
\newcommand\bsig[1]{\lvert{#1}\rvert} % Signature |b| of a built-in function.
\newcommand\arity[1]{\alpha({#1})}   % Arity \alpha(b) of a built-in function.
\newcommand\sat[1]{\gamma({#1})} % Interleaving structure of a partial builtin application
\newcommand\fv[1]{\mathsf{FV}_{\#}(#1)}
\newcommand{\N}{\mathbb{N}}
\newcommand{\Nplus}{\N^{+}}
\newcommand{\Z}{\mathbb{Z}}
\newcommand{\B}{\mathbb{B}}
\newcommand{\U}{\mathbb{U}}
\newcommand{\cn}{\textit{cn}}
\newcommand{\tn}{\textit{T}}
\newcommand{\at}{\textit{at}}
\newcommand{\lit}[1]{\mathbf{\mathsf{lit\,_{#1}}}}
\newcommand{\alphabar}{\bar{\alpha}}
\newcommand{\sigmabar}{\bar{\sigma}}
\newcommand{\Eval}{\mathsf{Eval}}
\newcommand{\EvalCEK}{\mathsf{Eval_{\,CEK}}}
%% NOTE!!!  Probably \byte and \Z are only used in builtins-alonzo

\newcommand{\kindstar}[1]{#1::\text{\textasteriskcentered}}  % "v::*"  (The ordinary * is a bit above the centreline)
\newcommand{\kindhash}[1]{#1::\text{\#}}                     % "v::#" 

\newcommand\errorX{\textbf{\texttimes}}
\newcommand\ValueOrError{\Inputs\disj\{\errorX\}}
\newcommand\withError[1]{\ensuremath{#1_{\errorX}}}



\newcommand{\cons}{\mathbin{\!\cdot\!}}
\newcommand{\snoc}{\mathbin{\!\cdot\!}}

\newcommand{\forallty}[1]{\mathbf{\forall}#1}
\newcommand{\denote}[1]{\llbracket\mathit{#1}\rrbracket}
\newcommand{\reify}[1]{\lBrace{#1}\rBrace}
%\newcommand{\args}[1]{\textbf{\textsf{args}}(#1)}
\DeclareMathOperator{\args}{\mathbf{\mathsf{args}}}
\DeclareMathOperator{\type}{\mathbf{\mathsf{type}}}
\DeclareMathOperator{\length}{\ell}
\DeclareMathOperator{\nextArg}{\mathsf{next}}
\DeclareMathOperator{\divfn}{div}
\DeclareMathOperator{\modfn}{mod}
\DeclareMathOperator{\quotfn}{quot}
\DeclareMathOperator{\remfn}{rem}
\DeclareMathOperator{\inj}{inj}
\DeclareMathOperator{\proj}{proj}
\DeclareMathOperator{\is}{is}
\DeclareMathOperator{\dom}{dom}

%%% Spacing in tables

\newcommand\sep{4pt}
% The table of abbreviations previously had \\\\ at the end of each line, which
% made it quite long. Lines are now separated by a vertical space of size \sep.
% This makes it a bit more readable than no spacing at all, but not too long

\newcommand{\Strut}{\rule[-2mm]{0mm}{6mm}}

\newcommand{\conUnitType}{\keyword{unit}}
\newcommand{\conBooleanType}{\keyword{bool}}
\newcommand{\ty}[1]{\mathtt{#1}}

\newcommand\Nab[2]{\N_{[#1,#2]}}
\newcommand\halfOpenInterval[2]{\N_{[#1#2)}}


%% Macros for CBOR encoders and decoders

\newcommand\e{\mathcal{E}} 
\renewcommand\d{\mathcal{D}}    %% This usually puts a dot under its argument
  
\newcommand\eHead{\e_{\mathsf{head}}}
\newcommand\dHead{\d_{\mathsf{head}}}

\newcommand\eIndef{\e_{\mathsf{indef}}}
\newcommand\dIndef{\d_{\mathsf{indef}}}

\newcommand\ecTag{\e_{\texttt{ctag}}}  % A constructor tag, not a CBOR tag.
\newcommand\dcTag{\d_{\texttt{ctag}}}

\newcommand\hex[1]{$\mathtt{#1_{16}}$}

\newcommand\eZ{\e_{\Z}}
\newcommand\dZ{\d_{\Z}}

\newcommand\eBS{\e_{\B^*}}
\newcommand\dBS{\d_{\B^*}}

\newcommand\itos{\mathsf{itos}}
\newcommand\stoi{\mathsf{stoi}}

\newcommand\byte{\mathsf{b}}

\newcommand\dBytes{\d_{\mathsf{bytes}}}
\newcommand\dBlock{\d_{\mathsf{block}}}
\newcommand\dBlocks{\d_{\mathsf{blocks}}}

\newcommand\intToBS{\mathsf{e}}
\newcommand\bsToInt{\mathsf{d}}  % bytestring -> integer

%% Macros for flat encoders and decoders

\newcommand\T{\mathsf{T}}  % Type tags

\newcommand\E{\mathsf{E}}  % Encoders
\newcommand\Elist{\overrightarrow{\mathsf{E}}}
\newcommand\Eblock{\mathsf{E}_{\mathsf{block}}}

\newcommand\encode[1]{\E_{\mathsf{#1}}}
\newcommand\encodelist[1]{\Elist_{\mathsf{#1}}}

\newcommand\Eprogram{\encode{program}}
\newcommand\Eterm{\encode{term}}
\newcommand\Ebuiltin{\encode{builtin}}
\newcommand\Econstant[1]{\encode{constant}^{#1}}
\newcommand\Ename{\encode{name}}
\newcommand\Ebinder{\encode{\text{$\lambda$}var}}
\newcommand\Etype{\encode{type}}
\newcommand\zigzag{\mathbb{z}}

\newcommand\D{\mathsf{D}}  % Decoders
\newcommand\Dlist{\overrightarrow{\mathsf{D}}}

\newcommand\decode[1]{\D_{\mathsf{#1}}}
\newcommand\decodelist[1]{\Dlist_{\mathsf{#1}}}

\newcommand\Dprogram{\decode{program}}
\newcommand\Dterm{\decode{term}}
\newcommand\Dbuiltin{\decode{builtin}}
\newcommand\dConstant[1]{\decode{constant}^{#1}}
\newcommand\Dname{\decode{name}}
\newcommand\Dbinder{\decode{\text{$\lambda$}var}}
\newcommand\Dtype{\decode{type}}

\newcommand\Bits{\mathbb{S}}
\newcommand\bit{\mathsf{b}}
\newcommand\bits[1]{\mathtt{#1}}

\newcommand\pad{\mathsf{pad}}
% \newcommand\Pad{\mathsf{pad}^{+}}
\newcommand\unpad{\mathsf{unpad}}
% \newcommand\unPad{\mathsf{unpad}^{+}}
\newcommand\pp[1]{\mathsf{p}_{#1}}

\newcommand\plc[2]{\texttt{(#1}\ #2\texttt{)}}

\newcommand\Prog[4]{\plc{program}{#1.#2.#3\ #4}}
\newcommand\Const[2]{\plc{const}{#1\ #2}}
\newcommand\Builtin[1]{\plc{builtin}{#1}}
\newcommand\Lam[2]{\plc{lam}{#1\ #2}}
\newcommand\Apply[2]{\texttt{[}#1\ #2\texttt{]}}
\newcommand\Delay[1]{\plc{delay}{#1}}
\newcommand\Force[1]{\plc{force}{#1}}
\newcommand\Constr[2]{\plc{constr}{#1\ #2}}
\newcommand\Kase[2]{\plc{case}{#1\ #2}}
\newcommand\Error{\texttt{(error)}}

\newcommand{\Tag}{\mathsf{tag}}  %% \renewcommand{\tag} gave odd results inside alignat*
\newcommand{\unTag}{\mathsf{tag}^{-1}}


  


\begin{document}
\maketitle

\begin{abstract}
  This is intended to be a reference guide for developers who want to utilise
  the Plutus Core infrastructure.  We lay out the grammar and syntax of untyped
  Plutus Core terms, and their semantics and evaluation rules.  We also describe
  the built-in types and functions.  The Appendices include a list of supported
  builtins in each era and some aspects of Plutus Core which have been
  mechanically formalised.

  This document only describes untyped Plutus Core: a subsequent version will also
  include the syntax and semantics of Typed Plutus Core and describe its relation to
  untyped Plutus Core.
\end{abstract}

\newpage
\tableofcontents
\newpage


\kwxm{This could still do with some tidying up.  We use a mixture of syntactic
  BNF-style notation, set-theoretic notation, and type theory, and that could
  maybe be rationalised a bit.}

% We are using Title Case for Chapters, and Sentence Case for everything below that
\chapter{Preliminaries}
\section{Introduction}
\label{sec:introduction}
Plutus Core (more correctly, Untyped Plutus Core) is an eagerly-evaluated
version of the untyped lambda calculus extended with some ``built-in'' types and
functions; it is intended for the implementation of validation scripts on the
Cardano blockchain.  This document presents the syntax and semantics of Plutus
Core, a specification of an efficient evaluator, a description of the built-in
types and functions available in various releases of Cardano, and
a specification of the binary serialisation format used by Plutus Core.

Since Plutus Core is intended for use in an environment where
computation is potentially expensive and excessively long computations can be
problematic we have also developed a costing infrastructure for Plutus Core
programs. A description of this will be added in a later version of this
document.

We also have a typed version of Plutus Core which provides extra robustness when
untyped Plutus Core is used as a compilation target, and we will eventually
provide a specification of the type system and semantics of Typed Plutus Core
here as well, together with its relationship to Untyped Plutus Core.


\section{Some basic notation}
We begin with some notation which will be used throughout the document.

\subsection{Sets}
\label{sec:notation-sets}
\begin{itemize}
  \item The symbol $\disj$ denotes a disjoint union of sets;  for emphasis we often use this
    to denote the union of sets which we know to be disjoint.%
    \nomenclature[Azz]{$\disj$}{Disjoint union of sets}%

  \item Given a set $X$, $X^*$ denotes the set of finite sequences of elements of $X$:
    $$
    X^*= \bigdisj{\{X^n: n \in \mathbb{N}\}}.
    $$
    \nomenclature[Azz]{$X^*$}{The set of all finite sequences of elements of a set $X$}
  \item $\N = \{0,1,2,3,\ldots\}$.%
    \nomenclature[An1]{$\N$}{$\{0,1,2,3,\ldots\}$}

  \item $\Nplus = \{1,2,3,\ldots\}$.%
    \nomenclature[An2]{$\Nplus$}{$\{1,2,3,\ldots\}$}

  \item $\Nab{a}{b} = \{n \in \N: a \leq n \leq b\}$.%
    \nomenclature[An3]{$\Nab{a}{b}$}{$\{n \in \N: a \leq n \leq b\}$}

  \item $\B = \Nab{0}{255}$, the set of 8-bit bytes.%
    \nomenclature[Ab1]{$\B$}{$\{n \in \Z: 0 \leq n \leq 255\}$}%

  \item $\B^*$ is the set of all bytestrings.%
    \nomenclature[Ab2]{$\B^*$}{The set of all bytestrings}

  \item $\Z = \{\ldots, -2, -1, 0, 1, 2, \ldots\}$.%
    \nomenclature[Aw]{$\Z$}{$\{\ldots, -2, -1, 0, 1, 2, \ldots\}$}

  \item $\mathbb{F}_q$ denotes a finite field with $q$ elements ($q$ a prime power).%
    \nomenclature[Af]{$\mathbb{F}_q$}{The finite field with $q$ elements}

  \item $\mathbb{F}_q^*$ denotes the multiplicative group of nonzero elements of $\mathbb{F}_q$.%
    \nomenclature[Af]{$\mathbb{F}_q^*$}{The multiplicative group of $\mathbb{F}_q$}%

  \item $\U$ denotes the set of Unicode scalar values, as defined in~\cite[Definition D76]{Unicode-standard}.%
    \nomenclature[Au1]{$\U$}{The set of Unicode scalar values}%

  \item $\U^*$ is the set of all Unicode strings.%
    \nomenclature[Au2]{$\U^*$}{The set of Unicode strings}%

  \item We assume that there is a special symbol $\errorX$ which does not appear
    in any other set we mention.  The symbol $\errorX$ is used to indicate that
    some sort of error condition has occurred, and we will often need to consider
    situations in which a value is either $\errorX$ or a member of some set $S$.
    For brevity, if $S$ is a set then we define
    $$
    \withError{S} := S \disj \{\errorX\}.
    $$%
    \nomenclature[Ax]{$\withError{S}$}{$S \disj \{\errorX\}$ ($S$ a set)}%
\end{itemize}%

\subsection{Lists}
\label{sec:notation-lists}
\begin{itemize}
\item  The symbol $[]$ denotes an empty list.%
  \nomenclature[B1]{$[]$}{The empty list}%

\item The notation $[x_m, \ldots, x_n]$ denotes a list containing the elements
  $x_m, \ldots, x_n$.  If $m>n$ then the list is empty.%

\item The length of a list $L$ is denoted by $\length(L)$.%
\nomenclature[B6]{$\length(\cdot)$}{Length of a list or bytestring}

\item Given two lists $L = [x_1,\ldots, x_m]$ and $L' = [y_1,\ldots, y_n]$, $L\cdot L'$
  denotes their concatenation  $[x_1,\ldots, x_m,$ $y_1, \ldots, y_n]$.  % Broken in the middle to keep it out of the margin.%
  \nomenclature[B2]{$L \cdot L'$}{Concatenation of lists $L$ and $L'$}%

\item Given an object $x$ and a list $L = [x_1,\ldots, x_n]$,
  we denote the list $[x,x_1,\ldots, x_n]$ by $x \cons L$.%
  \nomenclature[B3]{$x \cdot L$}{$[x] \cdot L$}%

\item Given a list $L = [x_1, \ldots, x_n]$ and an object $x$,
  we denote the list $[x_1, \ldots, x_n, x]$ by $L \snoc x$.%
  \nomenclature[B4]{$L \cdot x$}{$L \cdot [x]$}%

%%\item We say that the list $L'$ is a \textit{proper prefix} of the list
%%  $L = [x_1, \ldots, x_n]$, and
%%  write $L' \prec L$,  if $L' = [x_1, \ldots, x_m]$ for some $m<n$.
\item In the special case of bitstrings (ie, lists of elements of $\{0,1\}$) we
  sometimes use notation such as \texttt{101110} to denote the list
  $[1,0,1,1,1,0]$; we use a teletype font to avoid confusion with decimal
  numbers.

\item Given a syntactic category $V$, the symbol $\repetition{V}$ denotes a
  possibly empty list $[V_1,\ldots, V_n]$ of elements $V_i \in V$.%
  \nomenclature[B5]{$\repetition{V}$}{A sequence $[V_1,\ldots, V_n]$}%

\end{itemize}%

\chapter{Untyped Plutus Core}
\section{The grammar of Plutus Core}
\label{sec:untyped-plc-grammar}
This section presents the grammar of Plutus Core in a Lisp-like form.  This is
intended as a specification of the abstract syntax of the language; it may also
by used by tools as a concrete syntax for working with Plutus Core programs, but
this is a secondary use and we do not make any guarantees of its completeness
when used in this way.  The primary concrete form of Plutus Core programs is the
binary format described in Appendix~\ref{appendix:flat-serialisation}.

\subsection{Lexical grammar}
\label{sec:untyped-plc}
\thispagestyle{plain}
\pagestyle{plain}

\begin{minipage}{\linewidth}
    \centering
    \[\begin{array}{lrclr}

        \textrm{Name}          & n      & ::= & \texttt{[a-zA-Z][a-zA-Z0-9\_\textquotesingle]\textsuperscript{*}}   & \textrm{name}\\

        \textrm{Var}           & x      & ::= & n & \textrm{term variable}\\
        \textrm{BuiltinName}   & bn     & ::= & n & \textrm{built-in function name}\\
        \textrm{Version} & v & ::= & \texttt{[0-9]\textsuperscript{+}.[0-9]\textsuperscript{+}.[0-9]\textsuperscript{+}}& \textrm{version}\\
        \textrm{Natural}  & k      & ::= & \texttt{[0-9]+} & \textrm{a natural number}\\
        \textrm{Constant} & c & ::= & \langle{\textrm{literal constant}}\rangle& \\

    \end{array}\]
    \captionof{figure}{Lexical grammar of Plutus Core}
    \label{fig:lexical-grammar-untyped}
\end{minipage}%
\nomenclature[C]{$L,M,N$}{A term}%
\nomenclature[C]{$n$}{A name}%
\nomenclature[C]{$n$}{A natural number}%
\nomenclature[C]{$x$}{A variable name}%
\nomenclature[C]{$bn, b$}{The name of a built-in function}%
\nomenclature[C]{$c$}{A literal constant}%
\nomenclature[C]{$P$}{A Plutus Core program}%
\nomenclature[C]{$v$}{Plutus Core version}



%   @sqs   = '  ( ($printable # ['\\])  | (\\$printable) )* '
%
%   -- A double quoted string, allowing escaped characters including \".  Similar to @sqs
%   @dqs   = \" ( ($printable # [\"\\]) | (\\$printable) )* \"
%
%   -- A sequence of printable characters not containing '(' or ')' such that the
%   -- first character is not a space or a single or double quote.  If there are any
%   -- further characters then they must comprise a sequence of printable characters
%   -- possibly including spaces, followed by a non-space character.  If there are
%   -- any leading or trailing spaces they will be consumed by the $white+ token
%   -- below.
%   $nonparen = $printable # [\(\)]
%   @chars = ($nonparen # ['\"$white]) ($nonparen* ($nonparen # $white))?
%
%       <literalconst> "()" | @sqs | @dqs | @chars { tok (\p s -> alex $ TkLiteralConst p (textOf s)) `andBegin` 0 }

%% "()"
%% @sqs   = '  ( ($printable # ['\\])  | (\\$printable) )* '
%% @dqs   = \" ( ($printable # [\"\\]) | (\\$printable) )* \"
%% @chars = ($nonparen # ['\"$white]) ($nonparen* ($nonparen # $white))?


\subsection{Grammar}
\begin{minipage}{\linewidth}
    \centering
    \[\begin{array}{lrclr}
    \textrm{Term}       & L,M,N  & ::= & x                               & \textrm{variable}\\
                        &        &     & \con{\tn}{c}                    & \textrm{constant}\\
                        &        &     & \builtin{b}                     & \textrm{builtin}\\
                        &        &     & \lamU{x}{M}                     & \textrm{$\lambda$ abstraction}\\
                        &        &     & \appU{M}{N}                     & \textrm{function application}\\
                        &        &     & \delay{M}                       & \textrm{delay execution of a term}\\
                        &        &     & \force{M}                       & \textrm{force execution of a term}\\
                        &        &     & \constr{k}{M_1 \ldots M_m}      & \textrm{constructor with tag $k$ and $m$ arguments ($m \geq 0$)}\\
                        &        &     & \kase{M}{N_1 \ldots N_n}        & \textrm{case analysis with $n$ alternatives ($n \geq 0$)}\\
                        &        &     & \errorU                         & \textrm{error}\\
        \textrm{Program}& P      & ::= & \version{v}{M}                  & \textrm{versioned program}

    \end{array}\]
    \captionof{figure}{Grammar of untyped Plutus Core}
    \label{fig:untyped-grammar}
\end{minipage}


\subsection{Notes}
\label{sec:grammar-notes}
\paragraph{Version numbers.} The version number at the start of a program specifies
the Plutus Core language version used in the program.

A \textit{Plutus Core language version} describes a version of
the basic language with a particular set of features. A language version
consists of three non-negative integers separated by decimal points, for
example \texttt{1.4.2}. Language versions are ordered lexicographically.

The grammar above describes Plutus Core version 1.1.0. Version 1.0.0
is identical, except that \texttt{constr}
and \texttt{case} are not included.  Version 1.0.0 is fully forward-compatible
with version 1.1.0, so any valid version 1.0.0 program is also a valid version
1.1.0 program.  The semantics, evaluator and serialisation formats described
later in this document all apply to both versions, except that it is an error to
use \texttt{constr} or \texttt{case} in any program with a version prior to
1.1.0: a parser, deserialiser, or evaluator should fail immediately if
\texttt{constr} or \texttt{case} is encountered when processing such a program.

\paragraph{Scoping.} For simplicity, \textbf{we assume throughout that the body of a
Plutus Core program is a closed term}, ie, that it contains no free variables.
Thus \texttt{(program 1.0.0 (lam x x))} is a valid program but \texttt{(program
  1.0.0 (lam x y))} is not, since the variable \texttt{y} is free. This
condition should be checked before execution of any program commences, and the
program should be rejected if its body is not closed.  The assumption implies
that any variable $x$ occurring in the body of a program must be bound by an
occurrence of \texttt{lam} in some enclosing term; in this case, we always
assume that $x$ refers to the \textit{most recent} (ie, innermost) such binding.

\paragraph{Iterated applications.}
An application of a term $M$ to a term $N$ is represented by
$\appU{M}{N}$. We may occasionally write
$\appU{M}{N_1 \ldots N_k}$ or
$\appU{M}{\repetition{N}}$ as an abbreviation for an iterated application
$\mathtt{[}\ldots\mathtt{[[}M\;N_1\mathtt{]}\;N_2\mathtt{]}\ldots $  %% Try to avoid a bad line break
  $N_k\mathtt{]}$,
and tools may also use this as concrete syntax.

\paragraph{Constructors and case analysis.}
Plutus Core supports creating structured data using $\keyword{constr}$ and
deconstructing it using $\keyword{case}$. Both of these terms are unusual in
that they have (possibly empty) lists of children: $\keyword{constr}$ has the
(0-based) \emph{tag} and then a list of arguments; $\keyword{case}$ has a
scrutinee and then a list of case branches. The behaviour of $\keyword{constr}$
and $\keyword{case}$ is mostly straightforward: $\keyword{constr}$ evaluates its
arguments and forms a value; $\keyword{case}$ evaluates the scrutinee into a
$\keyword{constr}$ value, selects the branch corresponding to the tag on the
value (if the tag is $k$ and the branches are $N_1, \ldots, N_n$ then it selects
$N_{k+1}$, with an error occurring if $k$ does not lie between 0 and $n-1$), and
then applies that to the arguments in the value. Note that $\keyword{case}$ does
\emph{not} strictly evaluate the case branches, only applying (and hence
evaluating) the one that is eventually selected.  The list of branches in
$\keyword{case}$ is allowed to be empty, but in that case there will be an error
if it is ever applied to a scrutinee.

\paragraph{Constructor tags.}
Constructor tags can in principle be any natural number. In practice, since they
cannot be dynamically constructed, we can limit them to a fixed size without
having to worry about overflow. So we limit them to 64 bits, although this is
currently only enforced in the binary format (see
Section~\ref{sec:flat-term-encodings}).

\paragraph{Built-in types and functions.} The language is parameterised by a set $\Uni$ of
\textit{built-in types} (we sometimes refer to $\Uni$ as the \textit{universe})
and a set $\Fun$ of \textit{built-in functions} (\textit{builtins} for short),
both of which are sets of Names.  Briefly, the built-in types represent sets of
constants such as integers or strings; constant expressions $\con{\tn}{c}$
represent values of the built-in types (the integer 123 or the string
\texttt{"string"}, for example), and built-in functions are functions operating
on these values, and possibly also general Plutus Core terms.  Precise details
are given in Section~\ref{sec:specify-builtins}.

See Section~\ref{sec:cardano-builtins}
for a description of the types and functions which have already been deployed on
the Cardano blockchain (or will be in the near future).

\nomenclature[E1]{$\Fun$}{The set of built-in functions}

\paragraph{De Bruijn indices.}
The grammar defines names to be textual strings, but occasionally (specifically
in Appendix~\ref{appendix:flat-serialisation}) we want to use de Bruijn indices
(\cite{deBruijn}, \cite[C.3]{Barendregt}), and for this we redefine names to be
natural numbers.  In de Bruijn terms, $\lambda$-expressions do not need to bind
a variable, but in order to re-use our existing syntax we arbitrarily use 0 for
the bound variable, so that all $\lambda$-expressions are of the form
\texttt{(lam 0 $M$)}; other variables (ie, those not appearing immediately after
a \texttt{lam} binder) are represented by natural number greater than zero.

\paragraph{Lists in constructor and case terms.}
The grammar defines constructor and case terms to have a variable number of
subterms written in sequence with no delimiters. This corresponds to the
concrete syntax, e.g. we write $\constr{0}{t_1\ t_2\ t_3}$. However, in the
rest of the specification we will abuse notation and treat these terms as
having \emph{lists} of subterms.

\section{Interpretation of built-in types and functions}
\label{sec:specify-builtins}
As mentioned above, Plutus Core is generic over a universe $\Uni$ of types and
a set $\Fun$ of built-in functions.  As the terminology suggests, built-in
functions are interpreted as functions over terms and elements of the built-in
types: in this section we make this interpretation precise by giving a
specification of built-in types and functions in a set-theoretic denotational
style.  We require a considerable amount of extra notation in order to do this,
and we emphasise that nothing in this section is part of the syntax of Plutus
Core: it is meta-notation introduced purely for specification purposes.


\subsection{Built-in types}
\label{sec:built-in-types}
We require some extra syntactic notation for built-in types: see Figure~\ref{fig:type-names-operators}.

\begin{minipage}{\linewidth}
    \centering
    \[\begin{array}{rclr}
    \at    & ::= & n & \textrm{Atomic type}\\
     \op             & ::= & n & \textrm{Type operator}\\
     \tn             & ::= & \at \ | \ \op(\tn,\tn,...,\tn) & \textrm{Built-in type}\\
    \end{array}\]
    \captionof{figure}{Type names and operators}
    \label{fig:type-names-operators}
\end{minipage}

\medskip
\noindent
We assume that we have a set $\Uni_0$ of \textit{atomic type names} and a set $\TyOp$
of \textit{type operator names}.  Each type operator name $\op \in \TyOp$ has an
\textit{argument count} $\valency{\op} \in \Nplus$, and a type name $\op(\tn_1,
\ldots, \tn_n)$ is well-formed if and only if $n = \valency{\op}$.  We define
the \textit{universe} $\Uni$ to be the closure of $\Uni_0$ under repeated
applications of operators in $\TyOp$:
$$
\Uni_{i+1} = \Uni_i \cup \{\op(\tn_1, \ldots, \tn_{\valency{\op}}): \op \in \TyOp, \tn_1, \ldots, \tn_{\valency{op}} \in \Uni_i\}
$$
$$
\Uni = \bigcup\{\Uni_i: i \in \Nplus\}
$$%
\nomenclature[Da]{$\TyOp$}{The set of built-in type operator names}%
\nomenclature[Db]{$\op$}{A built-in type operator (an element of $\TyOp$)}%
\nomenclature[Da]{$\Uni_0$}{The set of atomic type names}%
\nomenclature[Da]{$\Uni$}{The universe of built-in types}%
\nomenclature[Dr2]{$T$}{A built-in type (an element of $\Uni$)}

\kwxm{Maybe we could have a judgment like $\Uni \vdash t\ \textsf{type}$
  and use inference rules instead of sets.  That would amount to the same thing but
would be considerably less compact.}

\kwxm{I'm inconsistently using ``type'' and ``type name'' for the things in
  $\Uni$, and that's further complicated by the introduction of polymorphic types later.}

The universe $\Uni$ consists entirely of \textit{names}, and the semantics of
these names are given by \textit{denotations}. Each built-in type $\tn \in \Uni$
is associated with some mathematical set $\denote{\tn}$, the \textit{denotation}
of $\tn$. For example, we might have $\denote{\texttt{bool}}= \{\mathsf{true},
\mathsf{false}\}$ and $\denote{\texttt{integer}} = \mathbb{Z}$ and
$\denote{\pairOf{a}{b}} = \denote{a} \times \denote{b}$.  We assume that if $T,
T^{\prime} \in \Uni$ and $T \ne T^{\prime}$ then $\denote{T}$ and
$\denote{T^{\prime}}$ are disjoint, and we put

$$\denote{\Uni} = \bigdisj{\{\denote{T}: T \in \Uni\}}.$$%%
\nomenclature[Dr4]{$\denote{T}$}{The denotation of a type name $T$}
\nomenclature[Dr5]{$\denote{\Uni}$}{$\bigdisj{\{\denote{T}: T \in \Uni\}}$}

\noindent See Section~\ref{sec:cardano-builtins}
for a description of the types and functions which have already been deployed on
the Cardano blockchain (or will be in the near future).
% \nomenclature[Dz1]{$\denote{T}$}{Denotation of a type $T \in \Uni$}

For non-atomic type names $\tn = \op(\tn_1, \ldots, \tn_r)$ we would generally
expect the denotation of $\tn$ to be obtained in some uniform way (ie,
parametrically) from the denotations of $\tn_1, \ldots, \tn_r$; we do not insist
on this though.

\newcommand{\tv}{\ensuremath{\textit{tv}}}

\subsubsection{Type variables}
\label{sec:type-variables}
Built-in functions can be polymorphic, and to deal with this we need
\textit{type variables}.  An argument of a polymorphic function can be either
restricted to built-in types or can be an arbitrary term, and we define two
different kinds of type variables to cover these two situations.  See
Figure~\ref{fig:type-variables}.

\begin{minipage}{\linewidth}
  \centering
      \[\begin{array}{lrclr}
        \textrm{TypeVariable}    & \tv & ::= & n_{\#} & \textrm{built-in-polymorphic type variable}\\
                                 &    &      & \star & \textrm{fully-polymorphic type variable}\\
    \end{array}\]
    \captionof{figure}{Type variables}
    \label{fig:type-variables}
\end{minipage}
% Previously we had fully polymorphic type variables like a_*, which allowed us
% to re-use this notation with typed PLC.  This was OK until the advent of
% caseData and caseList, which take term arguments which are expected to be
% whose types involve arrows: the problem is that (a), we can't represent these
% in the UPLC "type system", and (b) caseData and caseList in fact return a
% HeadSpine object and if you supply a non-function argument it'll be embedded
% in one of these and the type mismatch won't be discovered until the HeadSpine
% is later evaluated.

% If we ever specify TPLC we'll need a richer grammar of types for it, and here
% we will need genuine type variables which are polymorphic over terms; these
% will have to undergo an erasure process (replacing all a_*, b_*, ... with *)
% to obtain the types used here.

\medskip
\noindent
We denote the set of all possible type variables by $\Var$ and the set of all
built-in-polymorphic type variables $v_\#$ by $\Var_\#$.  Note that $\Var \cap
\Uni = \varnothing$ since the symbols $\star$ and ${}_\#$ do not occur in names in $\Uni$.%
\nomenclature[Da]{$\Var$}{The set of all type variables, $\Var_\# \cup \VarStar$}%
\nomenclature[Da]{$\Var_\#$}{The set of builtin-polymorphic type variables}%
\nomenclature[Db]{$v_\#$}{A builtin-polymorphic type variable}%
\nomenclature[Db]{$\star$}{The unique fully-polymorphic type variable}%

The two kinds of type variable are required because we have two different kinds
of polymorphism. Later on we will see that built-in functions can take arguments
which can be of a type which is unknown but must be in $\Uni$, whereas other
arguments can range over a larger set of values such as the set of all Plutus
Core terms. Type variables in $\Var_\#$ are used in the former situation and
$\star$ is used in the latter.

\kwxm{I'm using syntax to represent kinds here.  I haven't introduced actual
  kinds because (a) we don't have a proper type system in Plutus Core (yet), and
  (b) the \texttt{TypeScheme} type in the implementation only has one kind of
  type variable.  I'm only using this in the specification of the signatures of
  built-in functions to characterise some aspects how the builtins machinery
  works in practice: things in $\VarStar\backslash\Var_\#$ can fail with an
  unlifting error at runtime but that's not statically enforced anywhere.}

\subsubsection{Polymorphic types}
\label{sec:polymorphic-types}
We also need to talk about polymorphic types, and to do this we define an
extended universe of polymorphic types $\Unihash$ by adjoining $\Var_\#$ to
$\Uni_0$ and closing under type operators as before:

$$
\Unihashn{0} = \Uni_0 \cup \Var_\#
$$
$$
\Unihashn{i+1} = \Unihashn{i} \cup \{\op(\tn_1, \ldots, \tn_{\valency{\op}}): \op \in \TyOp, \tn_1, \ldots, \tn_{\valency{op}} \in \Unihashn{i}\}
$$
$$
\Unihash = \bigcup\{\Unihashn{i}: i \in \Nplus\}.$$%
\nomenclature[Da]{$\Unihash$}{The set of builtin-polymorphic types}%
\noindent We will denote a typical element of $\Unihash$ by the symbol $P$
(possibly subscripted).%
\nomenclature[Ds1]{$P$}{A builtin-polymorphic type}

\noindent We define the set of \textit{free \#-variables} of an element of $\Unihash$ by
$$
\fv{P} = \varnothing \ \text{if $P \in \Uni_0$}
$$
$$
\fv{v_\#} = \{v_\#\}
$$
$$
\fv{\op(P_1, \ldots, P_k)} = \fv{P_1} \cup \fv{P_2} \cup \cdots \cup \fv{P_r}.
$$%
\nomenclature[Ds2]{$\fv{P}$}{Free \#-variables of a polymorphic type $P \in \Unihash$}

\noindent Thus $\fv{P} \subseteq \Var_\#$ for all $P \in \Uni$.  We say that a
type name $P \in \Unihash$ is \textit{monomorphic} if $\fv{P} = \varnothing$ (in
which case we actually have $P \in \Uni$); otherwise $P$ is
\textit{polymorphic}.  The fact that type variables in $\Unihash$ are only
allowed to come from $\Var_\#$ will ensure that values of polymorphic types such
as lists and pairs can only contain values of built-in types: in particular, we
will not be able to construct types representing things such as lists of Plutus
Core terms.


\subsubsection{Type assignments}
\label{sec:type-assignments}
A \textit{type assignment} is a function $S: D \rightarrow \Uni$ where $D$ is
some subset of $\Var_\#$.  As usual we say that $D$ is the \textit{domain} of
$S$ and denote it by $\dom S$.%
\nomenclature[Dr6]{$S$}{A type assignment}

\medskip
\noindent We can extend a type assignment $S$ to a map
$\Sext : \Unihash \disj \VarStar \rightarrow \Unihash \disj \VarStar$ by defining

\begin{align*}
    \Sext(v_\#) &= S(v_\#) \quad \text{if $v_\# \in \dom S$}\\
    \Sext(v_\#) &= v_\# \quad \text{if $v_\# \in \Var_\# \backslash \dom S$}\\
    \Sext(T) &= T \quad\text{if $T \in \Uni_0$}     \\
    \Sext(\op(P_1,\ldots,P_n)) &= \op(\Sext(P_1),\ldots,\Sext(P_n))\\
    \Sext(\star) &= \star.
\end{align*}%
\nomenclature[Dr7]{$\Sext$}{The extension of a type assignment $S$ to $\Unihash \cup \VarStar$}

\noindent If $P \in \Unihash$ and $S$ is a type assignment with $\fv{P}
\subseteq \dom S$ then in fact $\Sext(P) \in \Uni$; in this case we say that
$\Sext(P)$ is an \textit{instance} or a \textit{monomorphisation} of $P$
(\textit{via $S$}).  If $T$ is an instance of $P$ then there is a unique
smallest $S$ (with $\fv{P}=\dom S$) such that $T = \Sext(P)$: we write
$T\preceq_S P$ to indicate that $T$ is an instance of $P$ via $S$ and $S$ is
minimal.%
\nomenclature[Ds3]{$T\preceq_S P$}{$T$ is an instance of $P$ via $S$ and $\dom S = \fv{P}$}

\paragraph{Constructing type assignments.}
We say that a collection $\{S_i: 1 \leq i \leq n\}$ of type assignments is
\textit{consistent} if $S_i|_{D_{ij}} = S_j|_{D_{ij}}$ for all $i$ and $j$,
where $|$ denotes function restriction and $D_{ij} = \dom S_i \ \cap \ \dom
S_j$.  If this is the case then (viewing functions as sets of pairs in the usual
way) $S_1 \cup \cdots \cup S_n$ is also a well-formed type assignment (each
variable in its domain is associated with exactly one type).

\medskip
\noindent Given $T \in \Uni$ and $P \in \Unihash$ it can be shown that $T \preceq_S P$ if
and only if one of the following holds:
\begin{itemize}
  \item $T = P$ and $S =\varnothing$.
  \item $P \in \Var_\#$ and $S = \{(v_\#, T)\}$.
  \item
    \begin{itemize}
    \item $T = \op(T_1, \ldots, T_n)$ with each $T_i \in \Uni$.
    \item $P = \op(P_1, \ldots, P_n)$ with each $P_i \in \Unihash$.
    \item $T_i \preceq_{S_i} P_i$ for $1 \leq i \leq n$.
    \item $\{S_1, \ldots, S_n\}$ is consistent.
    \item $S = S_1 \cup \cdots \cup S_n$.
    \end{itemize}
\end{itemize}

\noindent This allows us to decide whether $T \in \Uni$ is an instance of $P \in
\Unihash$ and, if so, to construct an $S$ with $T \preceq_S P$.

\kwxm{\sf I tried using inference rules for this but the final case
  got kind of out of hand.}

\subsection{Built-in functions}
\label{sec:builtin-functions}

\subsubsection{Inputs to built-in functions}
\label{sec:builtin-inputs}
To treat the typed and untyped versions of Plutus Core uniformly it is necessary
to make the machinery of built-in functions generic over a set $\Inputs$ of
\textit{inputs} which are taken as arguments by built-in functions.  In practice
$\Inputs$ will be the set of Plutus Core values or something very closely
related.%
\nomenclature[Ea]{$\Inputs$}{The set of inputs to built-in functions}

\medskip
\noindent We require $\Inputs$ to have the following two properties:
\begin{itemize}
\item $\Inputs$ is disjoint from $\denote{\tn}$ for all $\tn \in \Uni$
\item There should be disjoint subsets $\Con{\tn} \subseteq \Inputs$ (where $\tn
  \in \Uni$) of \textit{constants of type $\tn$} and maps $\denote{\cdot}_{\tn}:
  \Con{\tn} \rightarrow \denote{\tn}$ (\textit{denotation}) and
  $\reify{\cdot}_{\tn}: \denote{\tn} \rightarrow \Con{\tn}$
  (\textit{reification}) such that $\reify{\denote{c}_{\tn}}_{\tn} = c \text{
    for all } c \in \Con{\tn}$.  We do not require these maps to be bijective
  (for example, there may be multiple inputs with the same denotation), but the
  condition implies that $\denote{\cdot}_{\tn}$ is surjective and
  $\reify{\cdot}_{\tn}$ is injective.
\end{itemize}%
\nomenclature[Da]{$\Con{\tn}$}{Constants of built-in type $T$}%
\nomenclature[Dz2]{$\denote{\cdot}_T$}{Denotation of constants of type $T$}%
\nomenclature[Dz3]{$\reify{\cdot}_T$}{Reification of constants of type $T$}
\kwxm{I've forgotten why I want $\Inputs$ to be disjoint from the
  $\denote{\tn}$s \dots.}

\noindent It is also convenient to let $\denote{\Inputs} = \Inputs$ and define both
  $\denote{\cdot}_{\Inputs}$ and $\reify{\cdot}_{\Inputs}$ to be the identity
function, and we write
$$
\denote{\Uni}_{\Inputs} = \denote{\Uni} \disj \Inputs.
$$

\kwxm{I'm still not sure what to call the things what we feed to builtins.
  Previously there were called ``terms'', but in our setting they're
  \textit{not} actually arbitrary terms.  I tried ``values'' instead, but that
  was confusing.  ``Inputs'' is a bit better, but (a) they're also what builtins
  \textit{output}, and (b) there's a small risk of confusion with UTXO inputs.}

\noindent For example, we could take $\Inputs$ to be the set of all Plutus Core
values (see Section~\ref{sec:uplc-values}), $\Con{\tn}$ to be the set of all
terms of the form $\con{\tn}{c}$, and $\denote{\cdot}_{\tn}$ to be the function
which maps $\con{\tn}{c}$ to $c$.  For simplicity we are assuming that
mathematical entities occurring as members of type denotations $\denote{\tn}$
are embedded directly as values $c$ in Plutus Core constant terms. In reality,
tools which work with Plutus Core will need some concrete syntactic
representation of constants; we do not specify this here, but see
Section~\ref{sec:cardano-builtins}
for suggested syntax for the built-in types currently in use on the Cardano
blockchain.

\subsubsection{Outputs of built-in functions}
\label{sec:builtin-outputs}
All built-in functions either fail or conceptually return a non-empty list whose
entries lie either in the denotation of some built-in type $T$ or in the set of
inputs $\Inputs$, i.e., builtins return elements of the set ${(\R^+)_{\errorX}}$,
where

$$
\R := \bigdisj\left\{\denote{\tn}: \tn \in \Uni \right\} \disj \Inputs.
$$%%
\nomenclature[Eb]{$\R$}{The set of single values returned by built-in functions}%%
\nomenclature[Ec]{$\R^+$}{The set of sequences of values returned by built-in functions}

\noindent We will denote elements of $\R^+$ by expressions of the form $(v|v_1,
\ldots, v_k)$ with $v, v_i \in \R$ and $k \geq 0$, the case $k=0$ indicating a
list $(v|)$ with a single entry.  The majority of builtins return a single
value, and to simplify notation we will identify $\R$ with $\{(v|): v \in \R\}
\subseteq \R^+$.  The intention is that $(v|v_1, \ldots, v_k)$ will
immediately be interpreted as an application $v \;v_1\; \ldots\; v_k$ in the
ambient language (eg, typed or untyped Plutus Core); the number of arguments $k$
may depend on the values of the inputs to the function, as is the case for the
$\texttt{caseList}$ and $\texttt{caseData}$ functions described in
Appendix~\ref{sec:default-builtins-6}.

\subsubsection{Signatures and denotations of built-in functions}
\label{sec:signatures}
We will consistently use the symbol $\tau$ and subscripted versions of it to
denote members of $\UnihashStar$ in the rest of the document; these indicate the
types of values consumed by built-in functions.%
\nomenclature[Eg]{$\tau$}{A member of $\UnihashStar$, ie a type or type variable}

\medskip
\noindent We also define a class of \textit{quantifications} which are used to
introduce type variables: a quantification is a symbol of the form
$\forallty{v}$ with $v \in \Var$; the set of of all possible quantifications is
denoted by $\QVar$.%
\nomenclature[Ez3]{$\forallty{v}$}{Type quantification}%
\nomenclature[Eb]{$\QVar$}{The set of all type quantifications}

\medskip
\paragraph{Signatures.}
Every built-in function $b \in \Fun$ has a \textit{signature} $\sigma(b)$ which describes
the types of its arguments and its return value: a signature is of the form
$$[\iota_1, \ldots, \iota_n] \rightarrow \omega$$ with
\begin{itemize}
  \item $\iota_j \in \UnihashStar \disj \QVar \enspace\text{for all $j$}$
  \item $\omega \in \UnihashStarAp$
  \item $\lvert\{j : \iota_j \notin \QVar\}\rvert \geq 1$ (so $n \geq 1$)
  \item If $\iota_j$ involves $v \in \Var$ then $\iota_k = \forallty{v}$ for
    some $k < j$, and similarly for $\omega$; in other words, any type variable
    $v$ must be introduced by a quantification before it is used. (Here $\iota$
    \textit{involves} $v$ if either $\iota = \tn \in \Unihash$ and $v \in \fv{\tn}$
    or $\iota = v$ and $v \in \VarStar$.)
  \item If $\omega$ involves $v \in \Var$ then some $\iota_j$ must involve $v$;
    this implies that $\fv{\omega} \subseteq \bigcup \{\fv{\iota_j}: \iota_j \in
    \Unihash\}$ (where we extend the earlier definition of $\mathsf{FV}_{\#}$ by
    setting $\fv{\ap}=\varnothing$).
  \item If $j \neq k$ and $\iota_j, \iota_k \in \QVar$ then $\iota_j \neq
    \iota_k$; ie, no quantification appears more than once.
  \item If $\iota_i = \forall v \in \QVar$ then some $i_j \notin \QVar$ with $j
    > i$ must involve $v$ (signatures are not allowed to contain phantom type variables).
\end{itemize}%
\nomenclature[Eg]{$\iota$}{Signature item}%
\nomenclature[Eg]{$\sigma$}{Signature of built-in function}%
\nomenclature[Eg]{$\omega$}{Return type of built-in function}


\kwxm{\textsf{@effectfully} says that a builtin could actually return something
  in $\QVar$ too, so the target of the arrow could be an $\iota$. We don't need
  that at the moment though.}

\noindent For example, in our default set of built-in functions we have the
functions \texttt{mkCons} with signature $[\forall a_\#, a_\#,$  %% Allow line break
  $\listOf{a_\#}] \rightarrow \listOf{a_\#}$ and \texttt{ifThenElse} with signature
$[\forallStar, \mathtt{bool}, \star, \star] \rightarrow \star$.  When we use
\texttt{mkCons} its arguments must be of built-in types, but the two final
arguments of \texttt{ifThenElse} can be any Plutus Core values.

\smallskip
\noindent If $b$ has signature $[\iota_1, \ldots, \iota_n] \rightarrow \omega$ then we define
the \textit{arity}  of $b$ to be
$$
\alpha(b) = [\iota_1, \ldots, \iota_n].
$$%
\nomenclature[Eg]{$\alpha$}{Arity of built-in function}

\noindent We also define
$$
  \chi(b) = n.
$$%
\nomenclature[Eg]{$\chi$}{Number of arguments of built-in function}

\noindent We may abuse notation slightly by using the symbol $\sigma$ to denote
a specific signature as well as the function which maps built-in function names
to signatures, and similarly with the symbol $\alpha$.

\medskip
\noindent Given a signature
$\sigma = [\iota_1, \ldots, \iota_n] \rightarrow \omega$,
we define the \textit{reduced signature} $\sigmabar$ to be
$$
\sigmabar = [\iota_j : \iota_j \notin \QVar] \rightarrow \omega
$$%
\nomenclature[Eg]{$\sigmabar$}{Reduced signature}

\noindent Here we have extended the usual set comprehension notation to lists in the
obvious way, so $\sigmabar$ just denotes the signature $\sigma$ with all
quantifications omitted. We will often write a reduced signature in the form
$[\tau_1, \ldots, \tau_m] \rightarrow \omega$ to emphasise that the entries are
\textit{types}, and $\mathbf{\forall}$ does not appear.

\medskip
\noindent Also, given an arity $= [\iota_1, \ldots, \iota_n]$, the \textit{reduced
  arity} is
$$
\alphabar = [\iota_j : \iota_j \notin \QVar].
$$%
\nomenclature[Eg]{$\alphabar$}{Reduced arity}

\paragraph{Commentary.} What is the intended meaning of the notation introduced
above?  In Typed Plutus Core we have to instantiate polymorphic functions (both
built-in functions and polymorphic lambda terms) at concrete types before they
can be applied, and in Untyped Plutus Core instantiation is replaced by an
application of \texttt{force}.  When we are applying a built-in function we
supply its arguments one by one, and we can also apply \texttt{force} (or
perform type instantiation in the typed case) to a partially-applied builtin
``between'' arguments (and also after the final argument); no computation occurs
until all arguments have been supplied and all \texttt{force}s have been
applied. The arity (read from left to right) specifies what types of arguments
are expected and how they should be interleaved with applications of
\texttt{force}, and $\chi(b)$ tells you the total number of arguments and
applications of \texttt{force} that a built-in function $b$ requires. The
fully-polymorphic type variable $\star$ indicates that an arbitrary value from
$\Inputs$ can be provided, whereas a type from $\Unihash$ indicates that a value
of the specified built-in type is expected. Occurrences of quantifications
indicate that \texttt{force} is to be applied to a partially-applied builtin; we
allow this purely so that partially-applied builtins can be treated in the same
way as delayed lambda-abstractions: \texttt{force} has no effect unless it is
the very last item in the signature.  In Plutus Core, partially-applied
builtins are values which can be treated like any others (for example, by being
passed as an argument to a \texttt{lam}-expression): see
Section~\ref{sec:uplc-values}.

In general a builtin returns a sequence $(v|v_1,\ldots,v_k) \in \R^+$, but in
fact the majority of builtins currently deployed on Cardano only return a single
value, and in this case we can specify a signature where $\omega$ is either a
built-in type name $T$ or $\star$, denoting an input (typically a value in the
ambient language), and this tells us exactly what sort of value is returned.
The general case is considerably more complicated: the size of the list
returned, and the types of its entries, may be different for different
inputs. To specify this sort of behaviour precisely in a signature would require
a considerable increase in the complexity of the notation for signatures, so
instead we approximate all return types involving elements of $\R^+\backslash
\R$ by $\ap$.  However, when specifying the semantics of particular builtins
with $\omega = \ap$ we will always give a precise description of the possible
return values.

\subsubsection{Denotations of built-in functions}
\label{sec:builtin-denotations}
The basic idea is that a built-in function $b$ should represent some
mathematical function on the denotations of the types of its inputs.  However,
this is complicated by the presence of polymorphism and we have to require that
there is such a function for every possible monomorphisation of $b$.

More precisely, suppose that we have a builtin $b$ with reduced signature
$[\tau_1, \ldots \tau_n] \rightarrow \omega$.  For every type assignment $S$ with
$\dom S = \fv{\tau_1} \cup \cdots \cup \fv{\tau_n}$ (which contains $\fv{\omega}$ by
the conditions on signatures in Section~\ref{sec:signatures}) we require a
\textit{denotation of $b$ at $S$}, a function
$$
\denote{b}_S: \denote{\Sext(\tau_1)} \times \cdots \times \denote{\Sext(\tau_n)} \rightarrow \withError{\denote{\Sext(\omega)}}
$$%
\nomenclature[Ez4]{$\denote{b}_S$}{The denotation of the built-in function $b$ at the type assignment $S$}
\noindent where
$$
\denote{\star} = \Inputs \quad\text{and}\quad \denote{\ap} = \R^+.
$$
\noindent This makes sense because $\Sext(\tau_i) \in \Uni \disj
\Inputs$ for all $i$, so $\denote{\Sext(\tau_i)}$ is always defined,
and similarly for $\omega$ (extending $\Sext$ by setting $\Sext(\ap) = \ap$).

\medskip
\noindent If $\fv{\sigmabar(b)} = \varnothing$ (in which case we say that $b$ is
\textit{monomorphic}) then the only relevant type assignment will be the empty
one; in this case we have a single denotation
$$
\denote{b}_\varnothing: \denote{\tau_1} \times \cdots \times \denote{\tau_n} \rightarrow \withError{\denote{\omega}}.
$$

\smallskip
\noindent Denotations of builtins are mathematical functions which terminate on
every possible input; the symbol $\errorX$ can be returned by a function to
indicate that something has gone wrong, for example if an argument is out of
range.

\smallskip
\noindent In practice we expect most builtins to be \textit{parametrically
  polymorphic}~\cite{Wadler-theorems-for-free, Reynolds-parametric}, so that the
denotation $\denote{b}_S$ will be the ``same'' for all type assignments $S$; we do
not insist on this though.

\subsubsection{Results of built-in functions.}
\label{sec:builtin-results}
Recall from Section~\ref{sec:builtin-outputs} that the result of the
evaluation of a built-in function lies in the set 
$$
(\R^+)_{\errorX} = \left(\bigdisj\left\{\denote{\tn}: \tn \in \Uni \right\} \disj \Inputs \right)^+ \disj \{\errorX\}.
$$
\noindent Since we have assumed that all denotations $\denote{T}$ with $T \in
\Uni$ are disjoint from each other and from $\Inputs$
(Section~\ref{sec:builtin-inputs}) we can define a function
$$
\reify{\cdot}: \R \rightarrow \withError{\Inputs}
$$
which converts elements $r \in \R$ back into inputs by
$$
\reify{r} = 
\begin{cases}
  \reify{r}_{\tn} \in \Con{\tn} \subseteq \Inputs & \text{if $r \in \denote{\tn}$}\\
  r & \text{if $r \in \Inputs$}
\end{cases}
$$
\noindent(see Section~\ref{sec:builtin-inputs} for the definition of
$\reify{\cdot}_{\tn}$), and we can extend this to a function
$\reify{\cdot}: (\R^+)_{\errorX} \rightarrow \withError{\Inputs}$ by defining

\begin{align*}
  \reify{(r, r_1, \ldots, r_k)} &= (\reify{r}|\reify{r_1}, \ldots, \reify{r_k})\\
  \reify{\errorX} &= \errorX.
\end{align*}%%
\nomenclature[Dz4]{$\reify{\cdot}$}{Reification of the result of a built-in
  function application}

\kwxm{We have to use $\Inputs \disj \{\errorX\}$ to deal with the fact that
  $\errorU$ isn't a value, so we have to defer handling errors until later.}
\kwxm{The notation here is maybe a bit confusing.  The $\Con{\tn}$ live in the
  syntactic world (where they're subsets of $\Inputs$) and the $\denote{\tn}$
  live in the world of sets; however $\Inputs$ lives in \textit{both} worlds,
  and it's disjoint from all of the $\denote{\tn}$ in the world of sets but not
  in the world of syntax.  I think we do need something like this because
  sometimes a builtin argument must be a constant but at other times it can be
  an arbitrary value, which includes all of the constants.}

\subsubsection{Parametricity for fully-polymorphic arguments}
\label{sec:builtin-behaviour}
A built-in function $b$ can only inspect arguments which are values of built-in
types; other arguments (occurring as $\star$ in $\sigmabar(b)$) are treated
opaquely, and can be discarded or returned as (part of) a result, but cannot be
altered or examined (in particular, they cannot be compared for equality): $b$
is \textit{parametrically polymorphic} in such arguments.  This implies that if
the sequence returned by a builtin contains a value $v \in \Inputs$, then $v$
must have been an argument of the builtin.
\roman{This is not quite true, unfortunately. \texttt{toBuiltinMeaning}
  constrains \texttt{term} with \texttt{HasConstantIn uni term}. This means that we can
  check whether an argument whose type is a type variable is a constant or
  not. And if it's a constant, we can obtain the type tag and do all kinds of
  fancy things with it. For example a builtin checking if two values of
  different types are equal constants is representable. This breaks parametricity.
}

\kwxm{I think we should just require that people don't do that. $\uparrow$}

%% When (the meaning of) a built-in function $b$ is applied (perhaps partially) to
%% arguments, the types of constant arguments must correspond to the types in
%% $\sigmabar(b)$, and the function will return $\errorX$ if this is not the
%% case; builtins may also return $\errorX$ in other circumstances, for example if
%% an argument is out of range.

\kwxm{A lot of the complexity we have here is due to the fact that we've got
  explicit \texttt{delay} and \texttt{force} instead of the usual $\lambda().M$
  and $M()$.  We use the explicit version because experiments showed that it was
  noticeably faster (and we have a lot of these due to erasure of type-level
  abstraction/instantiation).  Also, who's to say that the universe contains a
  unit type?
}

%% \begin{minipage}{\linewidth}
%%     \centering \[\begin{array}{llll}

%%     t & ::= & b   &\text{for } b \in \Uni  \\
%%     & & \typearg                 \\
%%     & & \star               \\
%%     & & t \rightarrow t          \\
%%     \end{array}\]
%% \end{minipage}


\subsection{Evaluation of built-in functions}
\label{sec:builtin-evaluation}

\subsubsection{Compatibility of inputs and signature entries}
\label{sec:compatibility}
The previous section describes how a built-in function is interpreted as a
mathematical function.  When a Plutus Core built-in function $b$ is applied to a
sequence of arguments, the arguments must have types which are compatible with
the signature of $b$; for example, if $b$ has signature
$[\forallStar, \forall a_\#, \forall b_\#, a_\#, b_\#, a_\#, \star, \star] \rightarrow \star$ and $b$
is applied to a
sequence of inputs $V_1, V_2, V_3, V_4, V_5$ then $V_1, V_2$, and $V_3$ must all
be constants of some monomorphic built-in types and the types of $V_1$ and $V_3$
must be the same; $V_4$ and $V_5$ can be arbitrary inputs.  This section
describes the conditions for type compatibility.

\kwxm{The stuff about consistent instantiation of built-in-polymorphic type
  variables may in fact be slightly inaccurate.  I think that it's not enforced
  by the builtin application machinery, but must be checked dynamically in the
  implementation of the builtin when it's fully applied (\texttt{mkCons} is an
  example of this). It would be perfectly possible (I think) to implement a
  builtin with signature $[\forall a_\#, a_\#, a_\#] \rightarrow a_\#$ which
  could be successfully applied to arguments of two different built-in types,
  returning a constant of a third type.  However, I think it'll be very hard to
  describe what's actually going on so I'm just assuming that builtins always
  check this stuff.}

\medskip
\noindent In detail, given a reduced arity $\alphabar = [\tau_1, \ldots,
  \tau_n]$, a sequence $\bar{V} = [V_1, \ldots, V_m]$, and a type assignment $S$
we say that $\bar{V}$ is \textit{compatible with} $\alphabar$ (\textit{via} $S$)
if and only if $n=m$ and, letting $I = \{i: 1 \leq i \leq n, \tau_i \in
\Unihash\}$ (so $\tau_j = \star$ if $j \notin I$), there exist type
assignments $S_i$ ($1 \leq i \leq n$) such that all of the following are
satisfied
\begin{itemize}
\item For all $i \in I$ there exists $T_i \in \Uni$ such that $V_i \in \Con{T_i}$ and $T_i \preceq_{S_i} \tau_i$.
\item $\{S_i: i \in I\}$ is consistent (see Section~\ref{sec:type-assignments}).
\item $S = \bigcup\{S_i: i \in I\}$.
\end{itemize}

\noindent If these conditions are all satisfied then we can find suitable $S_i$
using the procedure described in Section~\ref{sec:type-assignments} and this
allows us to construct $S$ explicitly since the $S_i$ are consistent.  Note that
in this case $\dom S = \dom S_1 \cup \ldots \cup \dom S_n = \fv{\tau_1} \cup
\cdots \cup \fv{\tau_n} = \fv{\alpha}$, so $S$ is minimal in the sense that no
$S'$ with $\dom S'$ strictly smaller than $\dom S$ is sufficient
to monomorphise all of the $\tau_i$ simultaneously.  We write
$$
[V_1, \ldots, V_m] \approx_S [\tau_1, \ldots, \tau_n]
$$
in this case.  If $\bar{V}$ is not compatible with $\alphabar$ then we write
$\bar{V} \napprox \alphabar$.%
\nomenclature[Es2]{$\approx_S$}{Compatibility of built-in function arguments with function arity via $S$}

\subsubsection{Evaluation}
\label{sec:eval}
For later use we define a function $\Eval$ which attempts to evaluate an
application of a built-in function $b$ to a sequence of inputs $[V_1, \ldots,
  V_m]$.  This fails if the number of inputs is incorrect or if the inputs are
not compatible with $\alphabar(b)$:
$$
\Eval(b,[V_1, \ldots, V_n]) = \errorX \quad \text{if $[V_1, \ldots, V_n] \napprox \alphabar(b)$}.
$$

\noindent Otherwise, the conditions for the existence of a denotation of $b$ are
met and we can apply that denotation to the denotations of the inputs and then
reify the result. If $[V_1, \ldots, V_n] \approx_S \alphabar(b) = [\tau_1,
  \ldots, \tau_n]$, let $T_i = \Sext(\tau_i)$ for $1 \leq i \leq n$; then we
define

$$
\Eval(b,[V_1, \ldots, V_n]) = \reify{\denote{b}_S (\denote{V_1}_{T_1}, \ldots, \denote{V_n}_{T_n})}.
$$%
\nomenclature[Ez5]{$\Eval$}{Evaluation of built-in functions}%
%
\noindent It can be checked that the compatibility condition guarantees that
this makes sense according to the definition of $\denote{b}_S$ in
Section~\ref{sec:builtin-denotations}.

\paragraph{Notes.}
\begin{itemize}
  \item All of the machinery which we have defined for built-in functions is
    parametric over the set $\Inputs$ of inputs and the sets $\Con{T} \subseteq
    \Inputs$ of constants. This also applies to the $\Eval$ function, so its
    meaning is not fully defined until we have given concrete definitions of the
    sets of inputs and constants.
  \item The error value $\errorX$ can occur in two different ways: either
    because the arguments are not compatible with the signature, or because the
    builtin itself returns $\errorX$ to signal some error condition.
\item The symbol $\errorX$ is not part of Plutus Core; when we define reduction
  rules and evaluators for Plutus Core later some extra translation will be
  required to convert the result of $\Eval$ into something appropriate to the
  context.
\end{itemize}

\kwxm{For type operators, ``polymorphic'' really means ``polymorphic over
  built-in types''.}

\kwxm{$\Uni$ is the set of built-in types and $\Unihash$ is
  that set extended to include polymorphic types as well.  Later on we quite
  often have to look at $\Unihash\backslash \Uni$ to talk about types that really
  are polymorphic. Maybe it would be better to have a separate universe of
  polymorphic types called $\mathscr{P}$ or something, but then we'd also have to
  talk about $\Uni \disj \mathscr{P}$ sometimes.  Maybe we could define $\Unihash =
  \Uni \disj \mathscr{P}$?
  }

\section{Term reduction}
\label{sec:reduction}

This section defines the semantics of (untyped) Plutus Core.

\subsection{Values in Plutus Core}
\label{sec:uplc-values}
The semantics of built-in functions in Plutus Core are obtained by instantiating
the sets $\Con{\tn}$ of constants of type $\tn$ (see
Section~\ref{sec:builtin-inputs}) to be the expressions of the form
\texttt{(con} $\tn$ $c$\texttt{)} and the set $\Inputs$ to be the set of Plutus
Core \textit{values}, terms which cannot immediately undergo any further
reduction, such as lambda terms and delayed terms.  Values also include partial
applications of built-in functions such as \texttt{[(builtin modInteger) (con
    integer 5)]}, which cannot perform any computation until a second integer
argument is supplied.  However, partial applications must also be
\textit{well-formed}, in the sense that applications of \texttt{force} must be
correctly interleaved with genuine arguments, and the arguments must themselves
be values.

We define syntactic classes $V$ of Plutus Core values and $A$ of partial builtin
applications simultaneously:

\begin{minipage}{\linewidth}
    \centering
    \[\begin{array}{lrcl}
        \textrm{Value}  & V   & ::= & \con{\tn}{c} \\
                        &     &     & \delay{M} \\
                        &     &     & \lamU{x}{M} \\
                        &     &     & A
    \end{array}\]
    \captionof{figure}{Values in Plutus Core}
    \label{fig:untyped-cek-values}
\end{minipage}%
\nomenclature[F4]{$V$}{Plutus Core value}%
\nomenclature[F1]{$A$}{Well-formed partial built-in function application}

\medskip
\noindent Here $A$ is the class of well-formed partial applications, and to define
this we first define a class of possibly ill-formed iterated applications $\pba$ for
each built-in function $b \in \Fun$:

\begin{minipage}{\linewidth}
    \centering
  \[\begin{array}{lrl}
  \pba & ::= & \builtin{b}\\
       &     & \appU{\pba}{V}\\
       &    & \force{\pba}\\
    \end{array}\]
    \captionof{figure}{Partial built-in function application}
    \label{fig:partial-applications}
\end{minipage}%
\nomenclature[F2]{$\pba$}{Partial built-in function application (possibly ill-formed)}%
\nomenclature[F3]{$\pbas$}{Set of all  partial built-in function applications}

\medskip
\noindent We let $\pbas$ denote the set of terms generated by the grammar
in Figure~\ref{fig:partial-applications} and 
we define a function $\beta$ which extracts the name of the built-in
function occurring in a term in $\pbas$:
$$
 \begin{array}{ll}
 \beta(\builtin{b}) &= b\\
 \beta(\appU{\pba}{V}) & =\beta(\pba)\\
 \beta({\force{\pba}}) & =\beta(\pba)\\
\end{array}
$$%
\nomenclature[F5]{$\beta(\pba)$}{Function in partial builtin application $\pba$}

%% $$
%% \begin{array}{ll}
%%   \sat{\builtin{b}} &= []\\
%%   \sat{\appU{P}{V}} &= \sat{P}\snoc\type(V)\\
%%   \sat{\force{P}}   &= \sat{P}\snoc\fforce\\
%% \end{array}
%% $$

\noindent We also define a function $\pbasize{\cdot}$ which measures the size of
a term $\pba \in \pbas$:
$$
\begin{array}{ll}
\pbasize{\texttt{(builtin $b$)}} &= 0\\
\pbasize{\texttt{[$\pba$ $V$]}} &= 1+\pbasize{\pba}\\
\pbasize{\texttt{(force $\pba$)}} & = 1+\pbasize{\pba}
\end{array}
$$%
\nomenclature[F6]{$\pbasize{\pba}$}{Size of partial builtin application $\pba$}


%% \item Our built-in functions can take general members of $\Inputs$ as arguments
%%   as well as elements of the sets $\denote{\tn}$ and the symbol $\top$ is used
%%   to denote the type of elements of $\Inputs$. We use the symbol $\top$
%%   (which we assume does not appear in any other set we mention) to denote the
%%   type of non-constant elements of $\Inputs$ and write $\UniTop = \Uni \disj
%%   \{\top\}$ and $\UnihatTop = \Unihat \disj \{\top\}$.
%% \item We should be able to examine inputs (even during execution) to determine
%%   their types.  More precisely we assume that there is a function $\type:
%%   \Inputs \rightarrow \UniTop$ such that
%%   $$\type(x) =
%%   \begin{cases}
%%     \tn &\ \text{if } x \in \Con{\tn} \text{ for some } \tn \in \Uni\\
%%     \top &\text{otherwise}
%%   \end{cases}
%%   $$
%%   \noindent This is well defined because of our assumption that the sets $\Con{\tn}$ are disjoint.
%% \item We also define a partial order $\preceq$  on the set $\Uni^{\top}$ by
%%   $t_1 \preceq t_2$ if $t_1 = t_2$ or $t_2 = \top$.   


\paragraph{Well-formed partial applications.} A term $\pba \in \pbas$ is
an application of $b = \beta(\pba)$ to a number of values in $S$, interleaved
with applications of $\texttt{force}$.  We now define what it means for $\pba$
to be a \textit{well-formed partial application}.  Suppose that $\alpha(b) =
[\iota_1, \ldots, \iota_n]$. Firstly we require that $\pbasize{\pba} < n$, so
that $b$ is not fully applied; in this case we put
$\iota=\iota_{\pbasize{\pba}}$, the element of $b$'s signature which describes
what kind of ``argument'' $b$ currently expects.  The definition is completed by
induction on the structure of $\pba$:
\begin{enumerate}
\item $\pba=\mathtt{(builtin}\ b \mathtt{)}$ is always well-formed.
\item $\pba=\mathtt{[}\pba'\ V\mathtt{]}$ is well-formed if $\pba'$ is
  well-formed and $\iota \in \Unihash$ or $\iota \in \Var_*$ (equivalently, $\iota \notin \QVar$).
\item $\pba=\mathtt{(force}\ \pba'\mathtt{)}$ is well-formed if $\pba'$ is
  well-formed and $\iota \in \QVar$.
\end{enumerate}

\kwxm{Note that apart from type names all of this stuff is meta-notation that is
  need to describe the builtins machinery but isn't part of the language.}


\medskip
\noindent The definition of values in Figure~\ref{fig:untyped-cek-values} is now
completed by defining $A$ to be the syntactic class of well-formed
\textit{partial} built-in function applications:
$$
A = \{\pba \in \pbas: \pba \text{ is a well-formed partial application} \}.
$$

\noindent Note that this definition does not impose any requirements of type
correctness.  For example, with the types and functions defined in
Appendix~\ref{appendix:default-builtins-alonzo} the term $X =\texttt{[(builtin
    modInteger) (con string "blue")]}$ is a valid value which could be
treated like any other, for instance by being passed as an argument to a
\texttt{lam} expression.  However, the evaluation rules described in the next
section require that when a built-in function $b$ becomes \textit{fully} applied
the types of the arguments are checked against the signature of $b$ using the
relation $\approx$ and the function $\Eval$ defined in
Sections~\ref{sec:compatibility} and \ref{sec:eval}, so an error would arise if
the term $X$ were ever applied to another argument.



\paragraph{More notation.} Suppose that $A$ is a well-formed partial application with
$\alpha(\beta(A)) = [\iota_1,\ldots,\iota_n]$.  We define a function $\nextArg$
which extracts the next argument (or \texttt{force}) expected by $A$:
$$
    \nextArg(A) = \iota_{\pbasize{A}+1}.
$$
\noindent%
\nomenclature[Fr1]{$\nextArg(A)$}{Next argument type (or \texttt{force}) required by a partial builtin application $A$}
This makes sense because in a well-formed partial application $A$ we have
$\pbasize{A} < n$.

\medskip
\noindent We also define a function $\args{}$ which extracts the arguments which
$b$ has received so far in $A$:
$$
\begin{array}{ll}
  \args(\builtin{b}) &= []\\
  \args(\appU{A}{V}) &= \args(A)\snoc V\\
  \args(\force{A})   &= \args(A).\\
\end{array}
$$%
\nomenclature[Fr2]{$\args(A)$}{Term arguments received so far by partial builtin application $A$}

\subsection{Term reduction}

%% ---------------- Grammar of Reduction Frames ---------------- %%
\kwxm{Explain what this stuff means. Remember that when we apply the reduction
  rules we always use the first applicable one.

  I'm somewhat tempted to dump this in favour of SOS.}
\kwxm{Do we need a uniqueness condition on names somewhere?}

We define the semantics of Plutus Core using contextual semantics (or reduction
semantics): see~\cite{Felleisen-Hieb} or~\cite{Felleisen-Semantics-Engineering}
or~\cite[5.3]{Harper:PFPL}, for example.  We use $A$ to denote a partial
application of a built-in function as in Section~\ref{sec:uplc-values} above.
For builtin evaluation, we instantiate the set $\Inputs$ of
Section~\ref{sec:builtin-inputs} to be the set of Plutus Core values.  Thus all
builtins take values as arguments and return a value or $\errorX$.  Since values
are terms here, we can take $\reify{V} = V$.

\medskip
\noindent The notation $[V/x]M$ below denotes substitution of the value $V$ for
the variable $x$ in $M$.  This is \textit{capture-avoiding} in that substitution
is not performed on occurrences of $x$ inside subterms of $M$ of the form
$\lamU{x}{N}$.%
\nomenclature[Fr3]{$[V/x]M$}{Capture-avoiding substitution of value $V$ for variable $x$ in term $M$}

\begin{figure}[H]
\begin{subfigure}[c]{\linewidth}
    \centering
    \[\begin{array}{lrclr}
        \textrm{Frame} & f  & ::=   & \inAppLeftFrame{M}                                       & \textrm{left application}\\
                       &   &     & \inAppRightFrame{V}                                         & \textrm{right application}\\
                       &   &     & \inForceFrame                                               & \textrm{force}\\
                       &   &     & \inConstrFrame{i}{\repetition{V}}{\repetition{M}}           & \textrm{constructor argument}\\
                       &   &     & \inCaseFrame{\repetition{M}}                                & \textrm{case scrutinee}
    \end{array}\]
    \caption{Grammar of reduction frames for Plutus Core}
    \label{fig:untyped-reduction-frames}
\end{subfigure}
%\end{figure}%
\nomenclature[Fr3]{$f$}{Reduction frame for contextual semantics: $\inAppLeftFrame{M}, \inAppRightFrame{V}, \inForceFrame$}



\bigskip
%\begin{figure}[H]
%\ContinuedFloat
%% ---------------- Reduction via Contextual Semantics ---------------- %%
\begin{subfigure}[c]{\linewidth}
  % \def\labelSpacing{20pt}

  \judgmentdef{$\step{M}{M'}$}{Term $M$ reduces in one step to term $M'$. \red{We still have to deal with $\Inputs^{+}$.}}

   % [(lam x M) V] -> [V/x]M
    \begin{prooftree}
        \AxiomC{}
        % \RightLabel{\textsf{apply-lambda}}
        \UnaryInfC{$\step{\app{\lamU{x}{M}}{V}}{\subst{V}{x}{M}}$}
    \end{prooftree}

    % [A V] saturated
    \begin{prooftree}
      \AxiomC{$\length(A) = \chi(\beta(A))-1$}
      \AxiomC{$\nextArg(A) \in \Unihash \cup \VarStar$}
        % \RightLabel{\textsf{final-apply}}
        \BinaryInfC{$\step{\app{A}{V}}{\Eval'(\beta(A), \args(A)\snoc V)}$}
    \end{prooftree}

    % [A V] unsaturated
    \begin{prooftree}
      \AxiomC{$\length(A) < \chi(\beta(A))-1$}
      \AxiomC{$\nextArg(A) \in \Unihash \cup \VarStar$}
        % \RightLabel{\textsf{intermediate-apply}}
        \BinaryInfC{$\step{\app{A}{V}}{\app{A}{V}}$}
    \end{prooftree}

    % force (delay M) -> M
    \begin{prooftree}
        \AxiomC{}
        % \RightLabel{\textsf{force-delay}}
        \UnaryInfC{$\step{\force{\delay{M}}}{M}$}
    \end{prooftree}

    % case (constr i vs) cs -> [c_i vs]
    \begin{prooftree}
        \AxiomC{$0 \leq i \leq m$}
        % \RightLabel{\textsf{case-of-constr}}
        \UnaryInfC{$\step{\kase{\constr{i}{\repetition{V}}}{U_0 \ldots U_m}}{\app{U_i}{\repetition{V}}}$}
    \end{prooftree}

    % Saturated force
    \begin{prooftree}
      \AxiomC{$\length(A) = \chi(\beta(A))-1$}
      \AxiomC{$\nextArg(A) \in \QVar$}
        % \RightLabel{\textsf{final-force}}
        \BinaryInfC{$\step{\force{A}}{\Eval'(\beta(A), \args(A))}$}
    \end{prooftree}


    % Unsaturated force
    \begin{prooftree}
      \AxiomC{$\length(A) < \chi(\beta(A))-1$}
      \AxiomC{$\nextArg(A) \in \QVar$}
        % \RightLabel{\textsf{intermediate-force}}
        \BinaryInfC{$\step{\force{A}}{A}$}
    \end{prooftree}

%    \hfill\begin{minipage}{0.3\linewidth}
      \begin{prooftree}
        \AxiomC{} % If we're putting these side by side we need \strut here to get rules aligned
        % \RightLabel{\textsf{error}}
        \UnaryInfC{$\step{\ctxsubst{f}{\errorU}}{\errorU}$}
      \end{prooftree}
%    \end{minipage}
%    \begin{minipage}{0.3\linewidth}
    \begin{prooftree}
        \AxiomC{$\step{M}{M'}$}  % Need \strut for side-by-side alignment again
        \UnaryInfC{$\step{\ctxsubst{f}{M}}{\ctxsubst{f}{M'}}$}
    \end{prooftree}
% \end{minipage}\hfill\hfill %% Don't know why we need two \hfills here but only one at the start
% \\
    \caption{Reduction via contextual semantics} %% Oops
    \label{fig:untyped-reduction}
\end{subfigure}

\bigskip

\begin{subfigure}[c]{\linewidth}
  $$ \Eval'(b, [V_1, \ldots, V_n]) =
  \begin{cases}
  \errorU  & \text{if $\Eval(b,[V_1, \ldots, V_n]) = \errorX$}\\
  \Eval(b,[V_1, \ldots, V_n]) & \text{otherwise}
  \end{cases}
$$
    \caption{Built-in function application}
    \label{fig:bif-appl}
\end{subfigure}

\caption{Term reduction for Plutus Core}
\label{fig:untyped-term-reduction}
\end{figure}

\bigskip
\noindent It can be shown that any closed Plutus Core term whose evaluation
terminates yields either \texttt{(error)} or a value. Recall from
Section~\ref{sec:grammar-notes} that we require the body of every Plutus Core
program to be closed.

\kwxm{I was worried because we only have rules for eg application of a builtin
  $b$ to a final argument, and when we're applying $b$ to other arguments we
  don't check that a term argument is actually expected (rather than a
  \texttt{force}), and that the argument has the right type.  I think this is OK
  though: for example, if we have a builtin $b$ with $\arity{b} = [\forall a_\#,
    \texttt{int}]$ and we have a term $M = \texttt{[(builtin b) (con 5)]}$ then
  none of the rules apply because $M$ isn't in $A$, so the semantics get stuck.
  This happens in general as well: the definition of $A$ doesn't even let us
  talk about partial builtin applications where the interleaving is wrong.  We
  \textit{do} need special rules for the final argument because if $M \in A$ we
  have to look at $b$ to make sure that the final argument (or force) is the
  right kind of thing.}


  % Also inputs untyped-values.tex
\section{The CEK machine}
This section contains a description of an abstract machine for efficiently
executing Plutus Core.  This is based on the CEK machine of Felleisen and
Friedman~\cite{Felleisen-CK-CEK}.

\noindent The machine alternates between two main phases: the
\textit{compute} phase ($\triangleright$), where it recurses down
the AST looking for values, saving surrounding contexts as frames (or
\textit{reduction contexts}) on a stack as it goes; and the
\textit{return} phase ($\triangleleft$), where it has obtained a value and
pops a frame off the stack to tell it how to proceed next.  In
addition there is an error state $\cekerror$ which halts execution
with an error, and a halting state $\cekhalt{}$ which halts execution and
returns a value to the outside world.%
\nomenclature[Ga1]{$\triangleright$}{CEK compute phase}%
\nomenclature[Ga2]{$\triangleleft$}{CEK return phase}%
\nomenclature[Ga3]{$\cekerror$}{CEK error state}%
\nomenclature[Ga4]{$\cekhalt$}{CEK halting state}

To evaluate a program $\texttt{(program}\ v\ M \texttt{)}$, we first check that
the version number $v$ is valid, then start the machine in the state $[];[]
\triangleright M$.  It can be proved that the transitions in
Figure~\ref{fig:untyped-cek-machine} always preserve validity of states, so that
the machine can never enter a state such as $[] \triangleleft M$ or $s,
\texttt{(force \_)} \triangleleft \texttt{(lam}\ x\ A \ M\texttt{)}$ which isn't
covered by the rules.  If such a situation were to occur in an implementation
then it would indicate that the machine was incorrectly implemented or that it
was attempting to evaluate an ill-formed program (for example, one which attempts
to apply a variable to some other term).

\begin{figure}[H]
    \centering
    \[\begin{array}{lrclr}
    \textrm{State} & \Sigma & ::= & s;\rho \compute M \enspace | \enspace s \return V  \enspace |
       \enspace \cekerror{} \enspace | \enspace \cekhalt{V}\\
    \textrm{Stack} & s      & ::= & f^*\\
    \textrm{CEK value} & V &  ::= & \VCon{\tn}{c} \enspace | \enspace \VDelay{M}{\rho}
       \enspace| \enspace \VLamAbs{x}{M}{\rho} \enspace \\
       &&& | \enspace \VConstr{i}{\repetition{V}} \enspace | \enspace \VBuiltin{b}{\repetition{V}}{\eta}\\
    \textrm{Environment} & \rho & ::= & [] \enspace | \enspace \rho[x \mapsto V] \\
    \textrm{Expected builtin arguments} & \eta & ::= & [\iota] \enspace | \enspace \iota \cons \eta\\
    \end{array}\]
    \caption{Grammar of CEK machine states for Plutus Core}
    \label{fig:untyped-cek-states}
\end{figure}%
\nomenclature[Gb1]{$\Sigma$}{CEK machine state}%
\nomenclature[Gb2]{$s$}{CEK machine stack}%
\nomenclature[Gb2a]{$f$}{CEK stack frame: $\inForceFrame, \inAppLeftFrame{(M,\rho)}, \inAppRightFrame{V}, \inConstrFrame{i}{M_0 \ldots M_{i-1}}{V_{i+1} \ldots V_n}, \inCaseFrame{M_0 \ldots M_n}$}
\nomenclature[Gb3]{$V$}{CEK value: $\VCon{\tn}{c}, \VDelay{M}{\rho},\VLamAbs{x}{M}{\rho},\VBuiltin{b}{\repetition{V}}{\eta}$}%
\nomenclature[Gb4]{$\rho$}{CEK environment}%
\nomenclature[Gb5]{$\rho[x]$}{Value bound to variable $x$ in environment $\rho$}%
\nomenclature[Gb6]{$\eta$}{Arguments expected by partial builtin application}

\begin{figure}[H]
    \centering
    \[\begin{array}{lrclr}
        \textrm{Frame} & f  & ::=   & \inForceFrame                                                       & \textrm{force}\\
                       &    &       & \inAppLeftFrame{(M,\rho)}                                           & \textrm{left application to term}\\
                       &    &       & \inAppLeftFrame{V}                                                  & \textrm{left application to value}\\
                       &    &       & \inAppRightFrame{V}                                                 & \textrm{right application of value}\\
                       &    &       & \inConstrFrame{i}{\repetition{V}}{(\repetition{M}, \rho)}           & \textrm{constructor argument}\\
                       &    &       & \inCaseFrame{(\repetition{M}, \rho)}                                & \textrm{case scrutinee}

    \end{array}\]
    \caption{Grammar of CEK stack frames}
    \label{fig:untyped-cek-reduction-frames}
\end{figure}%

\kwxm{$\eta$ is the same as $\alpha$ except that we require it to be
  nonempty syntactically, whereas we put extra conditions on $\alpha$ in the
  definition of arities for builtins that make $\alpha(b)$ nonempty for all $b
  \in \Fun$.  This means that we can never have an empty $\eta$ in
  $\VBuiltin{b}{\repetition{V}}{\eta}$, which isn't entirely obvious.
  We'll have to revisit this if we ever have nullary builtins.}

\kwxm{Do we need to insist that CEK-values are well-formed, for example that
  there are enough variables in the environments to yield closed terms and that
  in $\VBuiltin{b}{\repetition{V}}{\eta}$ $\eta$ is a suffix of $\arity{b}$?
  Presumably the answer is no: you'd hope that a closed (and well-formed?) term
  $M$ will always yield a well-formed CEK value.}

\noindent Figures~\ref{fig:untyped-cek-states} and \ref{fig:untyped-cek-reduction-frames}
define some notation for \textit{states} of the CEK machine: these involve a
modified type of value adapted to the CEK machine, environments which bind names
to values, and a stack which stores partially evaluated terms whose evaluation
cannot proceed until some more computation has been performed (for example,
since Plutus Core is a strict language function arguments have to be reduced to
values before application takes place, and because of this a lambda term may
have to be stored on the stack while its argument is being reduced to a value).
Environments are lists of the form $\rho = [x_1 \mapsto V_1, \ldots, x_n \mapsto
  V_n]$ which grow by having new entries appended on the right; we say that
\textit{$x$ is \textit{bound} in the environment $\rho$} if $\rho$ contains an
entry of the form $x \mapsto V$, and in that case we denote by $\rho[x]$ the
value $V$ in the rightmost (ie, most recent) such entry.\footnote{The
  description of environments we use here is more general than necessary in that
  it permits a given variable to have multiple bindings; however, in what
  follows we never actually retrieve bindings other than the most recent one and
  we never remove bindings to expose earlier ones.  The list-based definition
  has the merit of simplicity and suffices for specification purposes but in an
  implementation it would be safe to use some data structure where existing
  bindings of a given variable are discarded when a new binding is added.}

To make the CEK machine fit into the built-in evaluation mechanism defined in
Section~\ref{sec:specify-builtins} we define $\Inputs = V$ and $\Con{\tn} =
\{\VCon{\tn}{c} : \tn \in \Uni, c \in \denote{\tn}\}$.

The rules in Figure~\ref{fig:untyped-cek-machine} show the transitions of the
machine; if any situation arises which is not included in these transitions (for
example, if a frame $\inAppRightFrame{\VCon{\tn}{c}}$ is encountered or if an
attempt is made to apply \texttt{force} to a partial builtin application which
is expecting a term argument), then the machine stops immediately in an error
state.


% Allow page break for (slightly) better placement
\begin{figure}[H]
  \begin{subfigure}[c]{\linewidth}
    \judgmentdef{$\Sigma \mapsto \Sigma'$}{Machine takes one step from state $\Sigma$ to state $\Sigma'$}

%\hspace{-1cm}
    \begin{minipage}{\linewidth}
\begin{alignat*}{2}
 s;\rho & \compute x                                 &~\mapsto~& s \return  \rho[x] \enskip \text{if $x$ is bound in $\rho$}\\
 s;\rho & \compute \con{\tn}{c}                       &~\mapsto~& s \return \VCon{\tn}{c}\\
 s;\rho & \compute \lamU{x}{M}                       &~\mapsto~& s \return \VLamAbs{x}{M}{\rho}\\
 s;\rho & \compute \delay{M}                         &~\mapsto~& s\return \VDelay{M}{\rho}\\
 s;\rho & \compute \force{M}                         &~\mapsto~& \inForceFrame{} \cons s;\rho \compute M \\
 s;\rho & \compute \appU{M}{N}                       &~\mapsto~& \inAppLeftFrame{(N,\rho)} \cons s ;\rho \compute M\\
 s;\rho & \compute \constr{i}{M \cons \repetition{M}} &~\mapsto~& \inConstrFrame{i}{}{(\repetition{M},\rho)} \cons s ;\rho \compute M\\
 s;\rho & \compute \constr{i}{[]} &~\mapsto~&  s \return \VConstr{i}{[]}\\
 s;\rho & \compute \kase{N}{\repetition{M}} &~\mapsto~&  \inCaseFrame{(\repetition{M},\rho)} \cons s ;\rho \compute N\\
% No nullary builtins (yet)
 s;\rho & \compute \builtin{b}                      &~\mapsto~& s \return \VBuiltin{b}{[]}{\arity{b}}\\
 s;\rho & \compute \errorU                           &~\mapsto~& \cekerror{}\\
\\[-10pt] %% Put some vertical space between compute and return rules, but not a whole line
[] & \return V                                    &~\mapsto~& \cekhalt{V}\\
\inAppLeftFrame{(M,\rho)}  \cons s            & \return V  &~\mapsto~& \inAppRightFrame{V} \cons s;\rho \compute M\\
\inAppRightFrame{\VLamAbs{x}{M}{\rho}} \cons s   & \return V  &~\mapsto~& s;\rho[x \mapsto V] \compute M\\
\inAppLeftFrame{V} \cons s   & \return \VLamAbs{x}{M}{\rho}  &~\mapsto~& s;\rho[x \mapsto V] \compute M\\
\inAppRightFrame{\VBuiltin{b}{\repetition{V}}{(\iota \cons \eta)}} \cons s & \return V &~\mapsto~&
                         s \return \VBuiltin{b}{(\repetition{V} \snoc V)}{\eta} \enskip \text{if $\iota \in \Unihash \cup \VarStar$}\\
\inAppLeftFrame{V} \cons s & \return \VBuiltin{b}{\repetition{V}}{(\iota \cons \eta)} &~\mapsto~&
                         s \return \VBuiltin{b}{(\repetition{V} \snoc V)}{\eta} \enskip \text{if $\iota \in \Unihash \cup \VarStar$}\\
\inAppRightFrame{\VBuiltin{b}{\repetition{V}}{[\iota]}} \cons s  & \return V &~\mapsto~&
                         \EvalCEK\,(s, b, \repetition{V}\snoc V) \enskip \text{if $\iota \in \Unihash \cup \VarStar$}\\
\inAppLeftFrame{V} \cons s & \return \VBuiltin{b}{\repetition{V}}{[\iota]} &~\mapsto~&
                         \EvalCEK\,(s, b, \repetition{V}\snoc V) \enskip \text{if $\iota \in \Unihash \cup \VarStar$}\\
\inForceFrame{} \cons s & \return \VDelay{M}{\rho}         &~\mapsto~& s;\rho \compute M\\
\inForceFrame{} \cons s & \return \VBuiltin{b}{\repetition{V}}{(\iota \cons \eta)} &~\mapsto~&
                         s \return \VBuiltin{b}{\repetition{V}}{\eta} \enskip \text{if $\iota \in \QVar$}\\
\inForceFrame{} \cons s & \return \VBuiltin{b}{\repetition{V}}{[\iota]}   &~\mapsto~&
                         \EvalCEK\,(s, b, \repetition{V}) \enskip \text{if $\iota \in \QVar$}\\
\inConstrFrame{i}{\repetition{V}}{(M \cons \repetition{M}, \rho)} \cons s & \return V   &~\mapsto~&
                         \inConstrFrame{i}{\repetition{V} \cons V}{(\repetition{M}, \rho)} \cons s;\rho \compute M \\
\inConstrFrame{i}{\repetition{V}}{([], \rho)} \cons s & \return V   &~\mapsto~&
                         s \return \VConstr{i}{\repetition{V} \cons V} \\
\inCaseFrame{(M_0 \ldots M_n, \rho)} \cons s & \return \VConstr{i}{V_1 \ldots V_m}   &~\mapsto~&
                         \inAppLeftFrame{V_m} \cons \cdots \cons \inAppLeftFrame{V_1} \cons s ;\rho \compute M_i \enskip \text{if $0 \leq i \leq n$}
\end{alignat*}
\end{minipage}
    \caption{CEK machine transitions for Plutus Core}
    \label{fig:untyped-cek-transitions}
\end{subfigure}

\bigskip
\begin{subfigure}[c]{\linewidth}
  \begin{align*}
 \EvalCEK(s, b, [V_1, \ldots, V_n]) =
  &   \begin{cases}
        \cekerror  & \text{if $r = \errorX$}\\
        \inAppLeftFrame{V^{\prime}_m} \cons \cdots \cons \inAppLeftFrame{V^{\prime}_1} \cons s \return V^{\prime}
          & \text{if $r = (V^{\prime}|V^{\prime}_1,\ldots,V^{\prime}_m) \in \R^{+}$}
      \end{cases}\\
  &  \;\;\text{where $r = \Eval\,(b,[V_1, \ldots, V_n])$}\\
   \end{align*}
  \caption{Evaluation of built-in functions}
  \label{fig:untyped-cek-builtins}
\end{subfigure}

  \caption{A CEK machine for Plutus Core}
\label{fig:untyped-cek-machine}
\end{figure}
  
\subsection{Converting CEK evaluation results into Plutus Core terms}
The purpose of the CEK machine is to evaluate Plutus Core terms, but in the
definition in Figure~\ref{fig:untyped-cek-machine} it does not return a Plutus
Core term; instead the machine can halt in two different ways:
\begin{itemize}
\item The machine can halt in the state $\cekhalt{V}$ for some CEK value $V$.
\item The machine can halt in the state $\cekerror{}$ .
\end{itemize}

\noindent To get a complete evaluation strategy for Plutus Core we must convert
these states into Plutus Core terms.  The term corresponding to $\cekerror{}$ is
$\errorU$, and to obtain a term from $\cekhalt{V}$ we perform a process which we
refer to as \textit{discharging} the CEK value $V$ (also known as
\textit{unloading}: see~\cite[pp. 129--130]{Plotkin-cbn-cbv},
\cite[pp. 71ff]{Felleisen-pllc}).  This process substitutes bindings in
environments for variables occurring in the value $V$ to obtain a term
$\unload{V}$: see Figure~\ref{fig:discharge-val}.  Since environments contain
bindings $x \mapsto W$ of variables to further CEK values, we have to
recursively discharge those bindings first before substituting: see
Figure~\ref{fig:discharge-env}, which defines an operation $\Unload{\rho}{}$ which
does this.  As before $[N/x]M$ denotes the usual (capture-avoiding) process of
substituting the term $N$ for all unbound occurrences of the variable $x$ in the
term $M$. Note that in Figure~\ref{fig:discharge-env} we substitute the
rightmost (ie, the most recent) bindings in the environment first.

\begin{figure}[H]
  \centering

  \begin{subfigure}[b]{\textwidth}
    \begin{align*}
      \unload{\VCon{\tn}{c}} &= \con{\tn}{c}\\
      \unload{\VDelay{M}{\rho}}
        &= \Unload{\rho}{\delay{M}}\\
      \unload{\VLamAbs{x}{M}{\rho}} &= \Unload{\rho}{\lamU{x}{M}}\\
      \unload{\VConstr{i}{\repetition{V}}} &= \constr{i}{\repetition{\unload{V}}}\\
      \unload{\VBuiltin{b}{V_1 V_2\ldots V_k}{\eta}} &=
      \appU{\ldots}  % [...[[(builtin b) V1!] V2!] ... Vk!]
           {\appU
             {\appU
               {\builtin{b}}
               {(\unload{V_1})}
             }
             {(\unload{V_2})}
             {\ldots (\unload{V_k})}
           }
    \end{align*}
    \caption{Discharging CEK values}
    \label{fig:discharge-val}
  \end{subfigure}

  \begin{subfigure}[c]{\textwidth}
    \begin{align*}
      \Unload{\rho}{M} &= [(\unload{V_1})/x_1]\cdots[(\unload{V_n})/x_n]M \quad
      \text{if $\rho = [x_1 \mapsto V_1, \ldots, x_n \mapsto V_n]$}
    \end{align*}
    \caption{Iterated substitution/discharging}
    \label{fig:discharge-env}
  \end{subfigure}

  \caption{Discharging CEK values to obtain Plutus Core terms}
  \label{fig:discharge-cek-val}
\end{figure}%
\nomenclature[Gc1]{$\unload{V}$}{Discharge a CEK value $V$ to obtain a Plutus Core term}%
\nomenclature[Gc2]{$\Unload{\rho}{M}$}{Discharge all variables bound by $\rho$ in the term $M$}

\noindent We can prove that if we evaluate a closed Plutus Core term in the CEK
machine and then convert the result back to a term using the above procedure
then we get the result that we should get according to the semantics in
Figure~\ref{fig:untyped-term-reduction}.

\section{Cost accounting for Untyped Plutus Core}
To follow.
\chapter{Typed Plutus Core}
To follow.
\chapter{Plutus Core on Cardano}
\section{Protocol Versions}
The Cardano blockchain controls the introduction of features through the use of \emph{protocol versions}.
Nodes will reject connections to peers with a higher major protocol version than they recognize, thus increasing the major protocol version corresponds to a hard-fork.

The current protocol version is part of the history of the chain.
When the protocol version changes, this only applies to new blocks, and old blocks are interpreted under the old protocol version.
Thus, conditioning on the protocol version is the main way in which we can introduce changes in behaviour while ensuring that old blocks are interpreted the old way in perpetuity.

Table~\ref{table:protocol-versions} lists the protocol versions that are relevant to the use of Plutus Core on Cardano.

\begin{table}[H]
  \centering
    \begin{tabular}{|l|l|l|}
        \hline
        \thead{Protocol version} & \thead{Codename} & \thead{Date} \\
        \hline
        5.0 & Alonzo & September 2021 \\
        7.0 & Vasil & June 2022 \\
        8.0 & Valentine & February 2023 \\
        9.0 & Conway & Upcoming \\
        \hline
    \end{tabular}
    \caption{Protocol versions}
    \label{table:protocol-versions}
\end{table}

\section{Ledger Languages}

The Cardano ledger uses Plutus Core as the programming language for \emph{scripts}.
The ledger in fact supports multiple different interpretations for scripts, and so each script is tagged with a \emph{ledger language} that tells the ledger how to interpret it.
Since the ledger must always be able to evaluate old scripts and get the same answer, the ledger language must pin down everything about how the script is evaluated, including:

\begin{enumerate}
  \item How to interpret the script itself (e.g. as a Plutus Core program, what versions of the Plutus Core language are allowable)
  \item Other configuration the script may need in order to run (e.g. the set of builtin types and functions and their interpretations, cost model parameters)
  \item How the script is called (e.g. after having certain arguments passed to it)
\end{enumerate}

There are currently three ``Plutus'' ledger languages (i.e. ledger languages whose underlying programming language is Plutus Core) in use on Cardano:\footnote{
  Note that ledger languages are completely distinct from the point of view of the ledger, the ``V1''/``V2'' naming is suggestive of the fact that these two ledger languages are related, but in the implementation they are completely independent.
}
\begin{enumerate}
  \item \LL{PlutusV1}
  \item \LL{PlutusV2}
  \item \LL{PlutusV2} (forthcoming)
\end{enumerate}

Table~\ref{table:ll-introduction} shows when each Plutus ledger language was introduced.
Ledger languages remain available permanently after they have been introduced.

\begin{table}[H]
  \centering
    \begin{tabular}{|l|l|}
        \hline
        \thead{Protocol version} & \thead{Ledger language introduced} \\
        \hline
        5.0 & \LL{PlutusV1} \\
        7.0 & \LL{PlutusV2} \\
        9.0 & \LL{PlutusV3} \\
        \hline
    \end{tabular}
    \caption{Introduction of Plutus ledger languages}
    \label{table:ll-introduction}
\end{table}

Ledger languages can evolve over time.
We can make backwards-compatible changes when the major protocol version changes, but backwards-incompatible changes can only be introduced by creating a whole new ledger language.\footnote{
  See~\cite{CIP-35} for more details on how the process of evolution works.
}
This means that to fully explain the behaviour of a ledger language we may need to also index by the protocol version.

The following tables show how Plutus ledger languages specify:
\begin{itemize}
  \item Which Plutus Core language versions are allowable (Table~\ref{table:lv-introduction})
  \item Which built-in functions and types are available (Table~\ref{table:b-introduction}, given in terms of tranches, see~\ref{sec:builtin-tranches})
  \item How to interpret the built-in functions and types (Table~\ref{table:bs-introduction}, given in terms of built-in semantics variants, see~\ref{sec:builtin-semantics-variants})
\end{itemize}

Currently, once we add a feature for any given protocol version/ledger language, we also make it available for all subsequent protocol versions/ledger languages.
For example, Tranche 2 of builtins was introduced in \LL{PlutusV2} at protool version 7.0, so it is also available in \LL{PlutusV2} at protocol versions after 7.0, and \LL{PlutusV3} at protocol versions after 9.0 (when \LL{PlutusV3} itself is first introduced).
Hence the tables are simplified to only show when something is \emph{introduced}.

\begin{table}[H]
  \centering
    \begin{tabular}{|l|l|l|}
        \hline
        \thead{Ledger language} & \thead{Protocol version} & \thead{Plutus Core language version introduced} \\
        \hline
        \LL{PlutusV1} & 5.0 & 1.0.0 \\
        \LL{PlutusV3} & 9.0 & 1.1.0 \\
        \hline
    \end{tabular}
    \caption{Introduction of Plutus Core language versions}
    \label{table:lv-introduction}
\end{table}

\begin{table}[H]
  \centering
    \begin{tabular}{|l|l|l|}
        \hline
        \thead{Ledger language} & \thead{Protocol version} & \thead{Built-in functions and types introduced} \\
        \hline
        \LL{PlutusV1} & 5.0 & Tranche 1 \\
        \LL{PlutusV2} & 7.0 & Tranche 2 \\
        \LL{PlutusV2} & 8.0 & Tranche 3 \\
        \LL{PlutusV3} & 9.0 & Tranche 4 \\
        \hline
    \end{tabular}
    \caption{Introduction of built-in functions and types}
    \label{table:b-introduction}
\end{table}

\begin{table}[H]
  \centering
    \begin{tabular}{|l|l|l|}
        \hline
        \thead{Ledger language} & \thead{Built-in semantics variant used} \\
        \hline
        \LL{PlutusV1} & Built-in semantics 1 \\
        \LL{PlutusV3} & Built-in semantics 2 \\
        \hline
    \end{tabular}
    \caption{Selection of built-in semantics variant}
    \label{table:bs-introduction}
\end{table}

\section{Built-in types and functions}
\label{sec:cardano-builtins}
\paragraph{Built-in batches.}
\label{sec:builtin-batches}

The built-in types and functions are defined in batches corresponding to how they were added to ledger languages.
These batches are given in the following sections.

\paragraph{Built-in semantics variants.}
\label{sec:builtin-semantics-variants}

In rare cases we can make a mistake or need to change the actual behaviour of a built-in function.
To handle this we define a series of built-in semantics variants, which indicate which behaviour should be used.
A fix will typically be deployed by defining a new semantics variant, and then using that variant for future ledger languages (but not existing ones, since this is usually a backwards-incompatible change).

Changes are listed alongside the original definition of the built-in function in its original batch, and are indexed in the following table.

\begin{table}[H]
  \centering
    \begin{tabular}{|l|l|l|}
        \hline
        \thead{Built-in semantics variant} & \thead{Changes from previous semantics} \\
        \hline
        Built-in semantics 1 & None \\
        Built-in semantics 2 & \TT{consByteString} (See~\ref{note:consbytestring}) \\
        \hline
    \end{tabular}
    \caption{Built-in semantics variants}
    \label{table:bs-variants}
\end{table}

\newcounter{notenumberA}
\newcommand{\note}[1]{
  \bigskip
  \refstepcounter{notenumberA}
  \noindent\textbf{Note \thenotenumberA. #1}
}

\newcommand{\utfeight}{\mathsf{utf8}}
\newcommand{\unutfeight}{\mathsf{utf8}^{-1}}
\newcommand{\vk}{\textit{vk}}  %% Verification key (ie public key) for signature verification functions.

\subsection{Batch 1}
\label{sec:default-builtins-1}

\subsubsection{Built-in types and type operators}
\label{sec:built-in-types-1}
The first batch of built-in types and type operators is defined in Tables~\ref{table:built-in-types-1}
and~\ref{table:built-in-type-operators-1}.  We also include concrete syntax for
these; the concrete syntax is not strictly part of the language, but may be
useful for tools working with Plutus Core.

\begin{table}[H]
  \centering
    \begin{tabular}{|l|p{6cm}|l|}
        \hline
        Type & Denotation & Concrete Syntax\\
        \hline
        \texttt{integer} &   $\mathbb{Z}$ & \texttt{-?[0-9]+}\\
        \texttt{bytestring}  & $ \B^*$, the set of sequences of bytes or 8-bit characters. & \texttt{\#([0-9A-Fa-f][0-9A-Fa-f])*}\\
        \texttt{string} & $\U^*$,  the set of sequences of Unicode characters. & See note below\\
        \texttt{bool} & \{\true, \false\} & \texttt{True | False}\\
        \texttt{unit} &  \{()\} & \texttt{()}\\
        \texttt{data} &  See below & See below\\
        \hline
    \end{tabular}
    \caption{Atomic types, Batch 1}
    \label{table:built-in-types-1}
\end{table}

\begin{table}[H]
  \centering
    \begin{tabular}{|l|p{14mm}|l|l|}
        \hline
        Operator $\op$ & $\left|\op\right|$  & Denotation & Concrete Syntax\\
        \hline
        \texttt{list} & 1 & $\denote{\listOf{t}} = \denote{t}^*$ & See below\\
        \texttt{pair} & 2 & $\denote{\pairOf{t_1}{t_2}} = \denote{t_1} \times \denote{t_2}$ & See below\\
        \hline
        \end{tabular}
   \caption{Type operators, Batch 1}
    \label{table:built-in-type-operators-1}
\end{table}

\paragraph{Concrete syntax for strings.} Strings are represented as sequences of Unicode characters
enclosed in double quotes, and may include standard escape sequences.  Surrogate
characters in the range \texttt{U+D800}--\texttt{U+DFFF} are replaced with the
Unicode replacement character \texttt{U+FFFD}.


\paragraph{Concrete syntax for lists and pairs.}
A list of type $\texttt{list}(t)$ is written as a syntactic list
\texttt{[$c_1, \ldots, c_n$]} where each $c_i$ lies in $\bitc_t$; a pair of type
$\texttt{pair}(t_1,t_2)$ is written as a syntactic pair $\texttt{(}c_1,c_2\texttt{)}$
with $c_1 \in \bitc_{t_1}$ and $c_2 \in \bitc_{t_2}$.  Some valid constant expressions
are thus

\begin{verbatim}
   (con (list integer) [11, 22, 33])
   (con (pair bool string) (True, "Plutus")).
   (con (list (pair bool (list bytestring)))
      [(True, []), (False, [#,#1F]), (True, [#123456, #AB, #ef2804])])
\end{verbatim}


\paragraph{The $\ty{data}$ type.}
We provide a built-in type $\ty{data}$ which permits the encoding of simple data
structures for use as arguments to Plutus Core scripts.  This type is defined in
Haskell as
\begin{alltt}
   data Data =
      Constr Integer [Data]
      | Map [(Data, Data)]
      | List [Data]
      | I Integer
      | B ByteString
\end{alltt}

\noindent In set-theoretic terms the denotation of $\ty{data}$ is
defined to be the least fixed point of the endofunctor $F$ on the category of
sets given by $F(X) = (\denote{\ty{integer}} \times X^*) \disj (X \times X)^* \disj
X^* \disj \denote{\ty{integer}} \disj \denote{\ty{bytestring}}$, so that
$$ \denote{\ty{data}} = (\denote{\ty{integer}} \times \denote{\ty{data}}^*)
               \disj (\denote{\ty{data}} \times \denote{\ty{data}})^*
               \disj \denote{\ty{data}}^*
               \disj \denote{\ty{integer}}
               \disj \denote{\ty{bytestring}}.
$$
We have injections
\begin{align*}
  \inj_C: \denote{\ty{integer}} \times \denote{\ty{data}}^* & \to \denote{\ty{data}} \\
  \inj_M: \denote{\ty{data}} \times \denote{\ty{data}}^*  & \to \denote{\ty{data}} \\
  \inj_L: \denote{\ty{data}}^* & \to \denote{\ty{data}} \\
  \inj_I: \denote{\ty{integer}} & \to \denote{\ty{data}} \\
  inj_B: \denote{\ty{bytestring}} & \to \denote{\ty{data}} \\
\end{align*}
\noindent and projections
\begin{align*}
  \proj_C: \denote{\ty{data}} & \to \withError{(\denote{\ty{integer}} \times \denote{\ty{data}}^*)}\\
  \proj_M: \denote{\ty{data}} & \to \withError{(\denote{\ty{data}} \times \denote{\ty{data}}^*)}\\
  \proj_L: \denote{\ty{data}} & \to \withError{\denote{\ty{data}}^* }\\
  \proj_I: \denote{\ty{data}} & \to \withError{\denote{\ty{integer}}}\\
  \proj_B: \denote{\ty{data}} & \to \withError{\denote{\ty{bytestring}} }\\
\end{align*}
\noindent which extract an object of the relevant type from a $\ty{data}$ object
$D$, returning $\errorX$ if $D$ does not lie in the expected component of the
disjoint union; also there are functions
$$
\is_C, \is_M, \is_L, \is_I, \is_B: \denote{\ty{data}} \to \denote{\ty{bool}}
$$
\noindent which determine whether a $\ty{data}$ value lies in the relevant component.

\paragraph{Note: \texttt{Constr} tag values.}
\label{note:constr-tag-values}
The \texttt{Constr} constructor of the \texttt{data} type is intended to
represent values from algebraic data types (also known as sum types and
discriminated unions, among other things; \texttt{data} itself is an example of
such a type), where $\mathtt{Constr}\, i\, [d_1,\ldots,d_n]$
represents a tuple of data items together with a tag $i$ indicating which of a
number of alternatives the data belongs to.  The definition above allows tags to
be any integer value, but because of restrictions in the serialisation format
for \texttt{data} (see Section~\ref{sec:encoding-data}) we recommend that in
practice \textbf{only tags $i$ with $0 \leq i \leq 2^{64}-1$ should be used}:
deserialisation will fail for \texttt{data} items (and programs which include
such items) involving tags outside this range.

Note also that \texttt{Constr} is unrelated to the $\keyword{constr}$ term in
Plutus Core itself. Both provide ways of representing structured data, but
the former is part of a built-in type whereas the latter is part of the language
itself.

\newcommand{\syn}[1]{c_{\mathtt{{#1}}}}

\paragraph{Concrete syntax for $\ty{data}$.}
The concrete syntax for $\ty{data}$ is given by

\begin{minipage}{0.6\linewidth}
    \centering
    \[\begin{array}{rcl}
    \syn{data} & ::= & \texttt{(Constr} \ \syn{integer} \ \syn{list(data)} \texttt{)}\\
               &  & \texttt{(Map} \ \syn{list(pair(data,data))} \texttt{)}\\
               &  & \texttt{(List} \ \syn{list(data)} \texttt{)}\\
               &  & \texttt{(I} \ \syn{integer} \texttt{)}\\
               &  & \texttt{(B} \ \syn{bytestring} \texttt{)}.
    \end{array}\]
    \label{fig:data-concrete-syntax}
\end{minipage}

\noindent We interpret these syntactic constants as elements of $\denote{\ty{data}}$ using
the various `$\inj$' functions defined earlier.  Some valid \texttt{data}
constants are

\begin{verbatim}
   (con data (Constr 1 [(I 2), (B #), (Map [])])
   (con data (Map [((I 0), (B #00)), ((I 1), (B #0F))]))
   (con data (List [(I 0), (I 1), (B #7FFF), (List []])))
   (con data (I -22))
   (con data (B #001A)).
\end{verbatim}
% TODO: be more relaxed about parenthesisation in general

\paragraph{Note.}  At the time of writing the syntax accepted by IOG's parser for textual Plutus Core
differs slightly from that above in that subobjects
of \texttt{Constr}, \texttt{Map} and \texttt{List} objects must \textit{not} be
parenthesised: for example one must write \verb|(con data (Constr 1 [I 2, B #,Map []])|.
This discrepancy will be resolved in the near future.


\subsubsection{Built-in functions}
\label{sec:built-in-functions-1}
The first batch of built-in functions is shown in
Table~\ref{table:built-in-functions-1}.  The table indicates which functions can
fail during execution, and conditions causing failure are specified either in
the denotation given in the table or in a relevant note.  Recall also that a
built-in function will fail if it is given an argument of the wrong type: this
is checked in conditions involving the $\sim$ relation and the $\Eval$ function
in Figures~\ref{fig:untyped-term-reduction} and~\ref{fig:untyped-cek-machine}.
Note also that some of the functions are
\#-polymorphic.  According to Section~\ref{sec:builtin-denotations} we
require a denotation for every possible monomorphisation of these; however all
of these functions are parametrically polymorphic so to simplify notation we
have given a single denotation for each of them with an implicit assumption that
it applies at each possible monomorphisation in an obvious way.

\setlength{\LTleft}{-18mm} % Shift the table left a bit to centre it on the page
\begin{longtable}[H]{|l|p{5cm}|p{5.5cm}|c|c|}
    \hline
    \text{Function} & \text{Signature} & \text{Denotation} & \text{Can} & \text{Note} \\
    & & & fail? & \\
    \hline
    \endfirsthead
    \hline
    \text{Function} & \text{Type} & \text{Denotation} & \text{Can} & \text{Note}\\
    & & & fail? & \\
    \hline
    \endhead
    \hline
    \caption{Built-in functions, Batch 1}
    % This caption goes on every page of the table except the last.  Ideally it
    % would appear only on the first page and all the rest would say
    % (continued). Unfortunately it doesn't seem to be easy to do that in a
    % longtable.
    \endfoot
    \caption[]{Built-in functions, Batch 1 (continued)}
    \label{table:built-in-functions-1}
    \endlastfoot
    \TT{addInteger}               & $[\ty{integer}, \ty{integer}] \to \ty{integer}$   & $+$ & No & \\[2mm]
    \TT{subtractInteger}          & $[\ty{integer}, \ty{integer}] \to \ty{integer}$   & $-$ & No & \\[2mm]
    \TT{multiplyInteger}          & $[\ty{integer}, \ty{integer}] \to \ty{integer}$   & $\times$ & No & \\[2mm]
    \TT{divideInteger}            & $[\ty{integer}, \ty{integer}] \to \ty{integer}$   & $\divfn$   & Yes & \ref{note:integer-division-functions}\\[2mm]
    \TT{modInteger}               & $[\ty{integer}, \ty{integer}] \to \ty{integer}$   & $\modfn$   & Yes & \ref{note:integer-division-functions}\\[2mm]
    \TT{quotientInteger}          & $[\ty{integer}, \ty{integer}] \to \ty{integer}$   & $\quotfn$  & Yes & \ref{note:integer-division-functions}\\[2mm]
    \TT{remainderInteger}         & $[\ty{integer}, \ty{integer}] \to \ty{integer}$   & $\remfn$   & Yes & \ref{note:integer-division-functions}\\[2mm]
    \TT{equalsInteger}            & $[\ty{integer}, \ty{integer}] \to \ty{bool}$      & $=$ & No & \\[2mm]
    \TT{lessThanInteger}          & $[\ty{integer}, \ty{integer}] \to \ty{bool}$      & $<$ & No & \\[2mm]
    \TT{lessThanEqualsInteger}    & $[\ty{integer}, \ty{integer}] \to \ty{bool}$      & $\leq$ & No & \\[2mm]
    %% Some of the signatures look like $ ... $ \text{\;\; $ ... $} to allow a break with some indentation afterwards
    \TT{appendByteString}         & $[\ty{bytestring}, \ty{bytestring}] $ \text{$\;\; \to \ty{bytestring}$}
                                           & $([c_1, \dots, c_m], [d_1, \ldots, d_n]) $ \text{$\;\; \mapsto [c_1,\ldots, c_m,d_1, \ldots, d_n]$} & No & \\[2mm]
    \TT{consByteString} (Variant 1) & $[\ty{integer}, \ty{bytestring}] $ \text{$\;\; \to \ty{bytestring}$}
                                          & $(c,[c_1,\ldots,c_n]) $ \text{$\;\;\mapsto [\text{mod}(c,256) ,c_1,\ldots,c_{n}]$} & No
                                          & \ref{note:consbytestring}\\[2mm]
    \TT{consByteString} (Variant 2) & $[\ty{integer}, \ty{bytestring}] $ \text{$\;\; \to \ty{bytestring}$}
                                          & $(c,[c_1,\ldots,c_n])$ \text{$\;\;\mapsto
                                                       \begin{cases}
                                                          [c,c_1,\ldots,c_{n}] & \text{if $0 \leq c \leq 255$} \\[2mm]
                                                         \errorX & \text{otherwise}
                                                       \end{cases}$} & Yes & \ref{note:consbytestring}\\[2mm]
    \TT{sliceByteString}        & $[\ty{integer}, \ty{integer}, \ty{bytestring]} $  \text {$\;\; \to  \ty{bytestring}$}
                                                   &   $(s,k,[c_0,\ldots,c_n])$ \text{$\;\;\mapsto [c_{\max(s,0)},\ldots,c_{\min(s+k-1,n-1)}]$}
                                                   & No & \ref{note:slicebytestring}\\[2mm]
    \TT{lengthOfByteString}       & $[\ty{bytestring}] \to \ty{integer}$ & $[] \mapsto 0, [c_1,\ldots, c_n] \mapsto n$ & No & \\[2mm]
    \TT{indexByteString}          & $[\ty{bytestring}, \ty{integer}] $ \text{$\;\; \to \ty{integer}$}
                                                   & $([c_0,\ldots,c_{n-1}],j)$ \text{$\;\;\mapsto
                                                       \begin{cases}
                                                         c_i & \text{if $0 \leq j \leq n-1$} \\[2mm]
                                                         \errorX & \text{otherwise}
                                                       \end{cases}$} & Yes & \\[2mm]
    \TT{equalsByteString}         & $[\ty{bytestring}, \ty{bytestring}] $ \text{$\;\; \to \ty{bool}$}   & = & No & \ref{note:bytestring-comparison}\\[2mm]
    \TT{lessThanByteString}       & $[\ty{bytestring}, \ty{bytestring}] $ \text{$\;\; \to \ty{bool}$}   & $<$ & No & \ref{note:bytestring-comparison}\\[2mm]
    \TT{lessThanEqualsByteString} & $[\ty{bytestring}, \ty{bytestring}] $ \text{$\;\; \to \ty{bool}$}   & $\leq$ & No & \ref{note:bytestring-comparison}\\[2mm]
    \TT{appendString}             & $[\ty{string}, \ty{string}] \to \ty{string}$
                                         & $([u_1, \dots, u_m], [v_1, \ldots, v_n]) $ \text{$\;\; \mapsto [u_1,\ldots, u_m,v_1, \ldots, v_n]$} & No & \\[2mm]
    \TT{equalsString}             & $[\ty{string}, \ty{string}] \to \ty{bool}$           & = & No & \\[2mm]
    \TT{encodeUtf8}               & $[\ty{string}] \to \ty{bytestring}$      & $\utfeight$ & & \ref{note:bytestring-encoding} \\[2mm]
    \TT{decodeUtf8}               & $[\ty{bytestring}] \to \ty{string}$      & $\unutfeight$ & Yes & \ref{note:bytestring-encoding} \\[2mm]
    \TT{sha2\_256}                & $[\ty{bytestring}] \to \ty{bytestring}$  & \text{Hash a $\ty{bytestring}$ using} \TT{SHA-}\TT{256}~\cite{FIPS-SHA2}. & No & \\[2mm]
    \TT{sha3\_256}                & $[\ty{bytestring}] \to \ty{bytestring}$  & \text{Hash a $\ty{bytestring}$ using} \TT{SHA3-}\TT{256}~\cite{FIPS-SHA3}. & No & \\[2mm]
    \TT{blake2b\_256}             & $[\ty{bytestring}] \to \ty{bytestring}$  & \text{Hash a $\ty{bytestring}$ using} \TT{Blake2b-}\TT{256}~\cite{IETF-Blake2}. & No & \\[2mm]
    \TT{verifyEd25519Signature}          & $[\ty{bytestring}, \ty{bytestring}, $ \text{$\;\; \ty{bytestring}] \to \ty{bool}$}
    & Verify an \TT{Ed25519} digital signature. &  Yes
    & \ref{note:digital-signature-verification-functions}, \ref{note:ed25519-signature-verification}\\[2mm]
    \TT{ifThenElse}               & $[\forall a_*, \ty{bool}, a_*, a_*] \to a_*$
                                                 & \text{$(\true,t_1,t_2) \mapsto t_1$}
                                                 \text{$(\false,t_1,t_2) \mapsto t_2$} & No & \\[2mm]
    \TT{chooseUnit}               & $[\forall a_*, \ty{unit}, a_*] \to a_*$        & $((), t) \mapsto t$ & No & \\[2mm]
    \TT{trace}                    & $[\forall a_*, \ty{string}, a_*] \to a_*$      & $ (s,t) \mapsto t$ & No & \ref{note:trace}\\[2mm]
    \TT{fstPair}                  & $[\forall a_\#, \forall b_\#, \pairOf{a_\#}{b_\#}] \to a_\#$       & $(x,y) \mapsto x$ & No & \\[2mm]
    \TT{sndPair}                  & $[\forall a_\#, \forall b_\#, \pairOf{a_\#}{b_\#}] \to b_\#$       & $(x,y) \mapsto y$ & No & \\[2mm]
    \TT{chooseList}               & $[\forall a_\#, \forall b_*, \listOf{a_\#}, b_*, b_*] \to b_*$
                                              & \text{$([], t_1, t_2) \mapsto t_1$,} \text{$([x_1,\ldots,x_n],t_1,t_2) \mapsto t_2\ (n \geq 1)$}. & No & \\[2mm]
    \TT{mkCons}                   & $[\forall a_\#, a_\#, \listOf{a_\#}] \to \listOf{a _\#}$  & $(x,[x_1,\ldots,x_n]) \mapsto [x,x_1,\ldots,x_n]$ & No & \\[2mm]
    \TT{headList}                 & $[\forall a_\#, \listOf{a_\#}] \to a_\#$               & $[]\mapsto \errorX, [x_1,x_2, \ldots, x_n] \mapsto x_1$ & Yes & \\[2mm]
    \TT{tailList}                 & $[\forall a_\#, \listOf{a_\#}] \to \listOf{a_\#}$
                                        &  \text{$[] \mapsto \errorX$,} \text{$ [x_1,x_2, \ldots, x_n] \mapsto [x_2, \ldots, x_n]$} & Yes & \\[2mm]
    \TT{nullList}                 & $[\forall a_\#, \listOf{a_\#}] \to \ty{bool}$            & $ [] \mapsto \true,
                                                                                                    [x_1,\ldots, x_n] \mapsto \false$ & No & \\[2mm]
    \TT{chooseData}               & $[\forall a_*, \ty{data}, a_*, a_*, a_*, a_*, a_*] \to a_*$
    & $ (d,t_C, t_M, t_L, t_I, t_B) $
    \smallskip
    \newline  % The big \{ was abutting the text above
    \text{$\;\;\mapsto
               \left\{ \begin{array}{ll}  %% This looks better than `cases`
                 t_C  & \text{if $\is_C(d)$} \\
                 t_M  & \text{if $\is_M(d)$} \\
                 t_L  & \text{if $\is_L(d)$} \\
                 t_I  & \text{if $\is_I(d)$} \\
                 t_B  & \text{if $\is_B(d)$} \\
               \end{array}\right.$}  & No & \\
    \TT{constrData}               & $[\ty{integer}, \listOf{\ty{data}}] \to \ty{data}$          & $\inj_C$ & No & \\[2mm]
    \TT{mapData}                  & $[\listOf{\pairOf{\ty{data}}{\ty{data}}}$ \text{$\;\; \to \ty{data}$}     & $\inj_M$ & No & \\[2mm]
    \TT{listData}                 & $[\listOf{\ty{data}}] \to \ty{data} $      & $\inj_L$ & No & \\[2mm]
    \TT{iData}                    & $[\ty{integer}] \to \ty{data} $            & $\inj_I$ & No & \\[2mm]
    \TT{bData}                    & $[\ty{bytestring}] \to \ty{data} $         & $\inj_B$& No & \\[2mm]
    \TT{unConstrData}             & $[\ty{data}]$ \text{$\;\; \to \pairOf{\ty{integer}}{\listOf{\ty{data}}}$} & $\proj_C$ & Yes & \\[2mm]
    \TT{unMapData}                & $[\ty{data}]$ \text{$\;\; \to \listOf{\pairOf{\ty{data}}{\ty{data}}}$}  & $\proj_M$ & Yes & \\[2mm]
    \TT{unListData}               & $[\ty{data}] \to \listOf{\ty{data}} $                          & $\proj_L$ & Yes & \\[2mm]
    \TT{unIData}                  & $[\ty{data}] \to \ty{integer} $                                & $\proj_I$ & Yes & \\[2mm]
    \TT{unBData}                  & $[\ty{data}] \to \ty{bytestring} $                             & $\proj_B$ & Yes & \\[2mm]
    \TT{equalsData}               & $[\ty{data}, \ty{data}] \to \ty{bool} $                        & $ = $ & & \\[2mm]
    \TT{mkPairData}               & $[\ty{data}, \ty{data}]$ \text{\;\; $\to \pairOf{\ty{data}}{\ty{data}}$}  & $(x,y) \mapsto (x,y) $ & No & \\[2mm]
    \TT{mkNilData}                & $[\ty{unit}] \to \listOf{\ty{data}} $                       & $() \mapsto []$ & No & \\[2mm]
    \TT{mkNilPairData}            & $[\ty{unit}] $ \text{$\;\; \to \listOf{\pairOf{\ty{data}}{\ty{data}}} $}   & $() \mapsto []$ & No & \\[2mm]
    \hline
\end{longtable}

\kwxm{Maybe try \texttt{tabulararray} to see what sort of output that gives for the big table.}

\note{Integer division functions.}
\label{note:integer-division-functions}
We provide four integer division functions: \texttt{divideInteger},
\texttt{modInteger}, \texttt{quotientInteger}, and \texttt{remainderInteger},
whose denotations are mathematical functions $\divfn, \modfn, \quotfn$, and
$\remfn$ which are modelled on the corresponding Haskell operations. Each of
these takes two arguments and will fail (returning $\errorX$) if the second one
is zero.  For all $a,b \in \Z$ with $b \ne 0$ we have
$$
\divfn(a,b) \times b + \modfn(a,b) = a
$$
$$
  |\modfn(a,b)| < |b|
$$\noindent and
$$
  \quotfn(a,b) \times b + \remfn(a,b) = a
$$
$$
  |\remfn(a,b)| < |b|.
$$
\nomenclature[Azz]{$\divfn$, $\modfn$}{Integer division operations}
\nomenclature[Azz]{$\quotfn$, $\remfn$}{Integer division operations}

\noindent The $\divfn$ and $\modfn$ functions form a pair, as do $\quotfn$ and $\remfn$;
$\divfn$ should not be used in combination with $\modfn$, not should $\quotfn$ be used
with $\modfn$.

For positive divisors $b$, $\divfn$ truncates downwards and $\modfn$ always
returns a non-negative result ($0 \leq \modfn(a,b) \leq b-1$).  The $\quotfn$
function truncates towards zero.  Table~\ref{table:integer-division-signs} shows
how the signs of the outputs of the division functions depend on the signs of
the inputs; $+$ means $\geq 0$ and $-$ means $\leq 0$, but recall that for $b=0$
all of these functions return the error value $\errorX$.
\begin{table}[H]
  \centering
    \begin{tabular}{|cc|cc|cc|}
        \hline
        a & b & $\divfn$ & $\modfn$ & $\quotfn$ & $\remfn$ \\
        \hline
        $+$ & $+$ & $+$ & $+$ & $+$ & $+$ \\
        $-$ & $+$ & $-$ & $+$ & $-$ & $-$ \\
        $+$ & $-$ & $-$ & $+$ & $+$ & $+$ \\
        $-$ & $-$ & $+$ & $-$ & $+$ & $-$ \\
        \hline
        \end{tabular}
   \caption{Behaviour of integer division functions}
   \label{table:integer-division-signs}
\end{table}
%% -------------------------------
%% |   n  d | div mod | quot rem |
%% |-----------------------------|
%% |  41  5 |  8   1  |   8   1  |
%% | -41  5 | -9   4  |  -8  -1  |
%% |  41 -5 | -9  -4  |  -8   1  |
%% | -41 -5 |  8  -1  |   8  -1  |
%% -------------------------------

\note{The \texttt{consByteString} function.}
\label{note:consbytestring}
In built-in semantics variant 1, the first argument of \texttt{consByteString}
is an arbitrary integer which will be reduced modulo 256 before being prepended
to the second argument.  In built-in semantics variant 2 we require that the first
argument lies between 0 and 255 (inclusive): in any other case an error will
occur.

\note{The \texttt{sliceByteString} function.}
\label{note:slicebytestring}
The application \texttt{[[(builtin sliceByteString) (con integer $s$)] (con
    integer $k$)] (con bytestring $b$)]} returns the substring of $b$ of length
$k$ starting at position $s$; indexing is zero-based, so a call with $s=0$
returns a substring starting with the first element of $b$, $s=1$ returns a
substring starting with the second, and so on.  This function always succeeds,
even if the arguments are out of range: if $b=[c_0, \ldots, c_{n-1}]$ then the
  application above returns the substring $[c_i, \ldots, c_j]$ where
  $i=\max(s,0)$ and $j=\min(s+k-1, n-1)$; if $j<i$ then the empty string is returned.

\note{Comparisons of bytestrings.}
\label{note:bytestring-comparison}
Bytestrings are ordered lexicographically in the usual way. If we have $a =
  [a_1, \ldots, a_m]$ and $b = [b_1, \ldots, b_n]$ then (recalling that if $m=0$
  then $a=[]$, and similarly for $b$),
\begin{itemize}
\item $a = b$ if and only if $m=n$ and $a_i = b_i$ for $1 \leq i \leq m$.

\item $a \leq b$ if and only if one of the following holds:
\begin{itemize}
  \item $a = []$
  \item $m,n > 0$ and $a_1 < b_1$
  \item $m,n > 0$ and $a_1 = b_1$ and $[a_2,\ldots,a_m] \leq [b_2,\ldots,b_n]$.
\end{itemize}
\item $a < b$ if and only if $a \leq b$ and $a \neq b$.
\end{itemize}
\noindent For example, $\mathtt{\#23456789} < \mathtt{\#24}$ and
$\mathtt{\#2345} < \mathtt{\#234500}$.  The empty bytestring is equal only to
itself and is strictly less than all other bytestrings.

\note{Encoding and decoding bytestrings.}
\label{note:bytestring-encoding}
The \texttt{encodeUtf8} and \texttt{decodeUtf8} functions convert between the
$\ty{string}$ type and the $\ty{bytestring}$ type.  We have defined
$\denote{\ty{string}}$ to consist of sequences of Unicode characters without
specifying any particular character representation, whereas
$\denote{\ty{bytestring}}$ consists of sequences of 8-bit bytes.  We define the
denotation of \texttt{encodeUtf8} to be the function
$$
\utfeight: \U^* \rightarrow \B^*
$$

\noindent which converts sequences of Unicode characters to sequences of bytes using the
well-known UTF-8 character encoding~\cite[Definition  D92]{Unicode-standard}.
The denotation of \texttt{decodeUtf8} is the partial inverse function

$$
\unutfeight: \B^* \rightarrow \U^*_{\errorX}.
$$

\noindent UTF-8 encodes Unicode characters encoded using between one and four
bytes: thus in general neither function will preserve the length of an object.
Moreover, not all sequences of bytes are valid representations of Unicode
characters, and \texttt{decodeUtf8} will fail if it receives an invalid input
(but \texttt{encodeUtf8} will always succeed).


\kwxm{In fact, strings are represented as sequences of UTF-16 characters, which
  use two or four bytes per character.  Do we need to mention that?  If we do,
  we'll need to be a little careful: there are sequences of 16-bit words that
  don't represent valid Unicode characters (for example, if the sequence uses
  surrogate codepoints improperly.  I don't think you can create a Haskell
  \texttt{Text} object (which is what our strings really are) that's invalid
  though.}


\note{Digital signature verification functions.}
\label{note:digital-signature-verification-functions}
We use a uniform interface for digital signature verification algorithms. A
digital signature verification function takes three bytestring arguments (in the
given order):
\begin{itemize}
  \item a public key $\vk$ (in this context $\vk$ is also known as a \textit{verification key})
  \item a message $m$
  \item a signature  $s$.
\end{itemize}
\noindent A signature verification function may require one
or more arguments to be well-formed in some sense (in particular an argument
may need to be of a specified length), and in this case the function will fail
(returning $\errorX$) if any argument is malformed. If all of the arguments are
well-formed then the verification function returns $\true$ if the private
key corresponding to $\vk$ was used to sign the message $m$ to produce $s$,
otherwise it returns $\false$.

\note{Ed25519 signature verification.}
\label{note:ed25519-signature-verification}
The \texttt{verifyEd25519Signature}
function performs cryptographic
signature verification using the Ed25519 scheme \cite{ches-2011-24091,
  rfc8032-EdDSA}, and conforms to the interface described in
Note~\ref{note:digital-signature-verification-functions}.  The arguments must
have the following sizes:
\begin{itemize}
\item $\vk$: 32 bytes
\item $m$: unrestricted
\item $s$: 64 bytes.
\end{itemize}

\note{The \texttt{trace} function.}
\label{note:trace}
An application \texttt{[(builtin trace) $s$ $v$]} ($s$ a \texttt{string}, $v$
any Plutus Core value) returns $v$.  We do not specify the semantics any
further.  An implementation may choose to discard $s$ or to perform some
side-effect such as writing it to a terminal or log file.

\newpage

% I tried resetting the note number from V1-builtins here, but that made
% some hyperlinks wrong.  To get note numbers starting at one in each section, I
% think we have to define a new counter every time.
\newcounter{notenumberB}
\renewcommand{\note}[1]{
  \bigskip
  \refstepcounter{notenumberB}
  \noindent\textbf{Note \thenotenumberB. #1}
}

\subsection{Batch 2}
\label{sec:default-builtins-2}

\subsubsection{Built-in functions}
\label{sec:built-in-functions-2}
The second batch of builtin operations is defined in Table~\ref{table:built-in-functions-2}.

\setlength{\LTleft}{20mm}  % Shift the table right a bit to centre it on the page
\begin{longtable}[H]{|l|l|l|c|c|}
    \hline
    \text{Function} & \text{Signature} & \text{Denotation} & \text{Can} & \text{Note} \\
    & & & fail? & \\
    \hline
    \endfirsthead
    \hline
    \text{Function} & \text{Type} & \text{Denotation} & \text{Can} & \text{Note}\\
    & & & fail? & \\
    \hline
    \endhead
    \hline
    \caption{Built-in Functions}
    \endfoot
    \caption[]{Built-in Functions}
    \label{table:built-in-functions-2}
    \endlastfoot
    \TT{serialiseData}                        & $[\ty{data}] \to \ty{bytestring}$   &  $\mathcal{E}_{\mathtt{data}}$ &
      & \ref{note:serialise-data}\\
\hline
\end{longtable}

\note{Serialising $\ty{data}$ objects.}
\label{note:serialise-data}
The \texttt{serialiseData} function takes a $\ty{data}$ object and converts it
into a bytestring using a CBOR encoding.  A full specification of the encoding
(including the definition of $\mathcal{E}_{\mathtt{data}}$) is provided in
Appendix~\ref{appendix:data-cbor-encoding}.

% I tried resetting the note number from V1-builtins here, but that made
% some hyperlinks wrong.  To get note numbers starting at one in each section, I
% think we have to define a new counter every time.
\newcounter{notenumberC}
\renewcommand{\note}[1]{
  \bigskip
  \refstepcounter{notenumberC}
  \noindent\textbf{Note \thenotenumberC. #1}
}

\subsection{Batch 3}
\label{sec:default-builtins-3}

\subsubsection{Built-in functions}
\label{sec:built-in-functions-3}
The third batch of built-in functions is defined in Table~\ref{table:built-in-functions-3}.
See~\cite{CIP-0049}.

\setlength{\LTleft}{-10mm}  % Shift the table left a bit to centre it on the page
\begin{longtable}[H]{|l|p{42mm}|p{35mm}|c|c|}
    \hline
    \text{Function} & \text{Signature} & \text{Denotation} & \text{Can} & \text{Note} \\
    & & & fail? & \\
    \hline
    \endfirsthead
    \hline
    \text{Function} & \text{Type} & \text{Denotation} & \text{Can} & \text{Note}\\
    & & & fail? & \\
    \hline
    \endhead
    \hline
    \caption{Built-in functions, batch 3}
    \endfoot
    \caption[]{Built-in functions, batch 3}
    \label{table:built-in-functions-3}
    \endlastfoot
    \TT{verifyEcdsaSecp256k1Signature}        & $[\ty{bytestring}, \ty{bytestring}, $ \text{$\;\; \ty{bytestring}] \to \ty{bool}$}
        & Verify an SECP-256k1 ECDSA signature & Yes & \ref{note:verify-ecdsa-secp256k1-signature}\\[2mm]
    \TT{verifySchnorrSecp256k1Signature}      & $[\ty{bytestring}, \ty{bytestring}, $ \text{$\;\; \ty{bytestring}] \to \ty{bool}$}
          & Verify an SECP-256k1 Schnorr signature & Yes & \ref{note:verify-schnorr-secp256k1-signature}\\[2mm]
\hline
\end{longtable}


\note{Secp256k1 ECDSA Signature verification.}
\label{note:verify-ecdsa-secp256k1-signature}
The \texttt{verifyEcdsaSecp256k1Signature} function performs elliptic curve
digital signature verification \cite{ANSI-X9.62, ANSI-x9.142,
  Johnson-Menezes-Vanstone-ECDSA} over the \texttt{secp256k1}
curve~\cite[\S2.4.1]{SECP256} and conforms to the interface described in
Note~\ref{note:digital-signature-verification-functions} of
Section~\ref{sec:built-in-functions-1}.  The arguments must have the
following sizes:
\begin{itemize}
\item $\vk$: 33 bytes
\item $m$: 32 bytes
\item $s$: 64 bytes.
\end{itemize}
The public key $\vk$ is expected to be in the 33-byte compressed form described
in~\cite{Bitcoin-ECDSA}.  Moreover, the ECDSA scheme admits two distinct valid
signatures for a given message and private key, and  we follow the restriction
imposed by Bitcoin (see~\cite{BIP-146},
\texttt{LOW\_S}) and \textbf{only accept the smaller signature};
\texttt{verifyEcdsa\-Secp\-256k1Signature} will return $\false$ if the larger
one is supplied.

% For more on the lower signature business, see
%    https://github.com/IntersectMBO/plutus/pull/4591#issuecomment-1120797931
% and
%    https://github.com/IntersectMBO/plutus/pull/4591#issuecomment-1121684776
% The C code that performs the check in the bitcoin implementation is at
%   https://github.com/bitcoin-core/secp256k1/blob/485f608fa9e28f132f127df97136617645effe81/src/secp256k1.c#L400
% and
%   https://github.com/bitcoin-core/secp256k1/blob/485f608fa9e28f132f127df97136617645effe81/src/scalar_low_impl.h#L85
%
% Schnorr signatures do something substantially different for ECDSA and I don't think
% the question of multiple signatures arises for verifySchnorrSecp256k1Signature.

\note{Secp256k1 Schnorr Signature verification.}
\label{note:verify-schnorr-secp256k1-signature}
The \texttt{verifySchnorrSecp256k1Signature} function performs verification of
Schnorr signatures~\cite{Schnorr89, BIP-340} over the \texttt{secp256k1} curve
and conforms to the interface described in
Note~\ref{note:digital-signature-verification-functions} of
Section~\ref{sec:built-in-functions-1}.  The arguments are expected to be
of the forms specified in BIP-340~\cite{BIP-340} and thus should have the
following sizes:
\begin{itemize}
\item $\vk$: 32 bytes
\item $m$: unrestricted
\item $s$: 64 bytes.
\end{itemize}

% I tried resetting the note number from V1-builtins here, but that made
% some hyperlinks wrong.  To get note numbers starting at one in each section, I
% think we have to define a new counter every time.
\newcounter{notenumberD}
\renewcommand{\note}[1]{
  \bigskip
  \refstepcounter{notenumberD}
  \noindent\textbf{Note \thenotenumberD. #1}
}

\newcommand{\itobsBE}{\mathsf{itobs_{BE}}}
\newcommand{\itobsLE}{\mathsf{itobs_{LE}}}
\newcommand{\bstoiBE}{\mathsf{bstoi_{BE}}}
\newcommand{\bstoiLE}{\mathsf{bstoi_{LE}}}

\subsection{Batch 4}
\label{sec:default-builtins-4}
The fourth batch of builtins adds support for
\begin{itemize}
\item The \texttt{Blake2b-224} and \texttt{Keccak-256} hash functions.
\item Conversion functions from integers to bytestrings and vice-versa.
\item BLS12-381 elliptic curve pairing operations
(see~\cite{CIP-0381}, \cite{BLS12-381}, \cite[4.2.1]{IETF-pairing-friendly-curves}, \cite{BLST-library}).
 For clarity these are described separately in Sections~\ref{sec:bls-types-4} and \ref{sec:bls-builtins-4}.
\end{itemize}

\subsubsection{Miscellaneous built-in functions}
\label{sec:misc-builtins-4}

\setlength{\LTleft}{-17mm} % Shift the table left a bit to centre it on the page
\begin{longtable}[H]{|l|p{45mm}|p{65mm}|c|c|}
    \hline \text{Function} & \text{Signature} & \text{Denotation} & \text{Can}
    & \text{Note} \\ & & & fail?
    & \\ \hline \endfirsthead \hline \text{Function} & \text{Type}
    % This caption goes on every page of the table except the last.  Ideally it
    % would appear only on the first page and all the rest would say
    % (continued). Unfortunately it doesn't seem to be easy to do that in a
    % longtable.
    \endfoot
%%    \caption[]{Built-in Functions}
    \caption[]{Miscellaneous built-in Functions}
    \label{table:misc-built-in-functions-4}
    \endlastfoot
%% G1
    \hline
    \TT{blake2b\_224} & $[\ty{bytestring}] \to \ty{bytestring}$  & \text{Hash a $\ty{bytestring}$ using
                                                                   \TT{Blake2b-224}~\cite{IETF-Blake2}.} & & \\
    \TT{keccak\_256}  & $[\ty{bytestring}] \to \ty{bytestring}$  & \text{Hash a $\ty{bytestring}$ using
                                                                   \TT{Keccak-256}~\cite{KeccakRef3}.} & & \\
    \hline\strut
    \TT{integerToByteString} & $[\ty{bool}, \ty{integer}, \ty{integer}]$  \text{\: $\to \ty{bytestring}$}
                                        & $(e, w, n) $ \text{$\mapsto \begin{cases}
                                        \itobsLE(w,n) & \text{if $e=\mathtt{false}$}\\
                                        \itobsBE(w,n) & \text{if $e=\mathtt{true}$}\\
                                        \end{cases}$}
                                        & Yes & \ref{note:itobs}\strut\\ \strut
    \TT{byteStringToInteger} & $[\ty{bool}, \ty{bytestring}] $ \text{\: $ \to \ty{bytestring}$}
                                        & $(e, [c_0, \ldots, c_{N-1}]) $ \text{\; $\mapsto \begin{cases}
                                        \sum_{i=0}^{N-1}c_{i}256^i & \text{if $e=\mathtt{false}$}\\
                                        \sum_{i=0}^{N-1}c_{i}256^{N-1-i} & \text{if $e=\mathtt{true}$}\\
                                        \end{cases}$}
                                        &  & \ref{note:bstoi}\\
    \hline
\end{longtable}

\note{Integer to bytestring conversion.}
\label{note:itobs}
The \texttt{integerToByteString} function converts non-negative integers to bytestrings.
It takes three arguments:
\begin{itemize}
\item A boolean endianness flag $e$.
\item An integer width argument $w$ with $0 \leq w < 8192$.
\item The integer $n$ to be converted: it is required that $0 \leq n < 256^{8192} = 2^{65536}$.
\end{itemize}

\noindent The conversion is little-endian if $e$ is \texttt{(con bool False)} and
big-endian if $e$ is \texttt{(con bool True)}. If the width $w$ is zero then the
output is a bytestring which is just large enough to hold the converted integer.
If $w>0$ then the output is exactly $w$ bytes long, and it is an error if $n$
does not fit into a bytestring of that size; if necessary, the output is padded
with \texttt{0x00} bytes (on the right in the little-endian case and the left in
the big-endian case) to make it the correct length.  For example, the five-byte
little-endian representation of the integer \texttt{0x123456} is the
bytestring \texttt{[0x56, 0x34, 0x12, 0x00, 0x00]} and the five-byte big-endian
representation is \texttt{[0x00, 0x00, 0x12, 0x34, 0x56]}.  In all cases an
error occurs error if $w$ or $n$ lies outside the expected range, and in
particular if $n$ is negative.

\newpage
\noindent The precise semantics of \texttt{integerToByteString} are given
by the functions $\itobsLE: \Z \times \Z \rightarrow \withError{\B^*}$ and $\itobsBE
: \Z \times \Z \rightarrow \withError{\B^*}$.  Firstly we deal with out-of-range cases and
the case $n=0$:

$$
\itobsLE (w,n) = \itobsBE (w,n) = 
\begin{cases}
  \errorX & \text{if $n<0$ or $n \geq 2^{65536}$}\\
  \errorX & \text{if $w<0$ or $w > 8192$}\\
  [] & \text{if $n=0$ and $0 \leq w \leq 8192$}\\
\end{cases}
$$

\noindent  Now assume that none of the conditions above hold, so $0 < n < 2^{65536}$ and
$0 \leq w \leq 8192$.  Since $n>0$ we can write
$n = \sum_{i=0}^{N-1}a_{i}256^i$ with $N \geq 1$ and $a_{N-1} \ne 0$.   We then  have

$$
\itobsLE (w,n) =
\begin{cases}
  [a_0, \ldots, a_{N-1}] & \text{if $w=0$} \\
  [b_0, \ldots, b_{w-1}] &  \text{if $w > 0$ and $N\leq w$, where }
      b_i = \begin{cases}
                a_i & \text{if $i \leq N-1$} \\
                0   & \text{if $i \geq N$}\\
            \end{cases}\\
  \errorX & \text{if $w > 0$ and $N > w$}
\end{cases}
$$

\noindent and

$$
\noindent
\itobsBE (w,n) =
\begin{cases}
  [a_{N-1}, \ldots, a_0] & \text{if $w=0$} \\
  [b_0, \ldots, b_{w-1}] &  \text{if $w > 0$ and $N\leq w$, where }
      b_i = \begin{cases}
                0   & \text{if $i \leq w-1-N$}\\
                a_{w-1-i} & \text{if $i \geq w-N$} \\
            \end{cases}\\
  \errorX & \text{if $w > 0$ and $N > w$.}
\end{cases}
$$

\note{Bytestring to integer conversion.}
\label{note:bstoi}
The \texttt{byteStringToInteger} function converts bytestrings to non-negative
integers.  It takes two arguments:
\begin{itemize}
\item A boolean endianness flag $e$.
\item The bytestring $s$ to be converted.
\end{itemize}
\noindent  
The conversion is little-endian if $e$ is \texttt{(con bool False)} and
big-endian if $e$ is \texttt{(con bool True)}. In both cases the empty bytestring is
converted to the integer 0. All bytestrings are legal inputs and there is no
limitation on the size of $s$.

\subsubsection{BLS12-381 built-in types}
\label{sec:bls-types-4}

\noindent Supporting the BLS12-381 operations involves adding three new types
and seventeen new built-in functions.  The description of the semantics of these
types and functions is quite complex and requires a considerable amount of
notation, most of which is used only in Sections~\ref{sec:bls-types-4} and~\ref{sec:bls-builtins-4}.

\bigskip
\noindent Table~\ref{table:built-in-types-4} describes three new built-in
types.

\newcommand{\TyMlResult}{\ty{bls12\_381\_mlresult}}
\newcommand{\MlResultDenotation}{H}
\newcommand{\Fq}{\mathbb{F}_q}
\newcommand{\Fqq}{\mathbb{F}_{q^2}}
\newcommand{\FF}{\mathbb{F}_{q^{12}}}

\begin{table}[H]
  \centering
    \begin{tabular}{|l|p{2cm}|l|}
        \hline
        Type & Denotation & Concrete Syntax\\
        \hline
        $\ty{bls12\_381\_G1\_element}$ &   $G_1$ & \texttt{0x[0-9A-Fa-f]\{96\}} \text{(see Note~\ref{note:bls-syntax})}\\
        $\ty{bls12\_381\_G2\_element}$ &   $G_2$ & \texttt{0x[0-9A-Fa-f]\{192\}} \text{(see Note~\ref{note:bls-syntax})}\\
        $\TyMlResult$    &   $\MlResultDenotation$  &  None (see Note~\ref{note:bls-syntax})\\
        \hline
    \end{tabular}
    \caption{Atomic Types}
    \label{table:built-in-types-4}
\end{table}

%% \paragraph{$G_1$ and $G_2$}.
\noindent Here $G_1$ and  $G_2$ are both additive cyclic groups of prime order $r$, where 
$$
r = \mathtt{0x73eda753299d7d483339d80809a1d80553bda402fffe5bfeffffffff00000001}.
$$
        
\paragraph{The fields $\Fq$ and $\Fqq$.}
\noindent To define the groups $G_1$ and $G_2$ we need the finite field $\Fq$ where
\begin{align*}
q = \mathtt{0x}&\mathtt{1a0111ea397fe69a4b1ba7b6434bacd764774b84f38512bf}\\
              &\mathtt{6730d2a0f6b0f6241eabfffeb153ffffb9feffffffffaaab}
\end{align*}

\noindent which is a 381-bit prime. The field $\Fq$ is isomorphic to $\Z_q$,
the ring of integers modulo $q$, and hence there is a natural epimorphism from
$\Z$ to $\Fq$ which we denote by $n \mapsto \bar{n}$.  Given $x \in \Fq$, we
denote by $\tilde{x}$ the smallest non-negative integer $n$ with $\bar{n} = x$.
We sometimes use literal integers to represent elements of $\Fq$ in the obvious
way.

We also make use of the field $\Fqq = \Fq[X]/(X^2+1)$; we may regard $\Fqq$ as
the set $\{a+bu: a,b \in \Fq\}$ where $u^2=-1$.

It is convenient to say that an element $a$ of $\Fq$ is \textit{larger} than
another element $b$ (and $b$ is \textit{smaller} than $a$) if $\tilde{a}
> \tilde{b}$ in $\Z$.  We extend this terminology to $\Fqq$ by saying that
$a+bu$ is larger than $c+du$ if either $b$ is larger than $d$ or $b=d$ and $a$
is larger than $c$.


\paragraph{The groups $G_1$ and $G_2$.}
\noindent There are elliptic curves $E_1$ defined over $\Fq$:
$$
E_1: Y^2 = X^3 + 4
$$

\noindent and $E_2$ defined over $\Fqq$:
$$
E_2: Y^2 = X^3 + 4(u+1).
$$

\noindent $E_1(\Fq)$ and  $E_2(\Fqq)$  are abelian groups under the
usual elliptic curve addition operations as described
in~\cite[III.2]{Silverman-Arithmetic-EC} or~\cite[2.1]{Costello-pairings}.
$G_1$ is a subgroup of $E_1(\Fq)$ and $G_2$ is a subgroup of $E_2(\Fqq)$;
explicit generators for $G_1$ and $G_2$ are given
in~\cite[4.2.1]{IETF-pairing-friendly-curves}.  We denote the identity element
(the point at infinity) in $G_1$ by $\mathcal{O}_{G_1}$ and that in $G_2$ by
$\mathcal{O}_{G_2}$.  Given an integer $n$ and a group element $a$ in $G_1$ or
$G_2$, the scalar multiple $na$ is defined as usual to be $a + \cdots + a$ ($n$
times) if $n>0$ and $-a + \cdots + -a$ ($-n$ times) if $n<0$; $0a$ is the
identity element of the group.

\paragraph{The \texttt{bls12\_381\_MlResult} type.}
\noindent Values of the \texttt{bls12\_381\_MlResult} type are completely
opaque and can only be obtained as a result of \texttt{bls12\_381\_millerLoop}
or by multiplying two existing elements of type \texttt{bls12\_381\_MlResult}.
We provide neither a serialisation format nor a concrete syntax for values of
this type: they exist only ephemerally during computation.  We do not specify
$\MlResultDenotation$, the denotation of $\TyMlResult$, precisely, but it
must be a multiplicative abelian group. See Note~\ref{note:pairing} for more on
this.

\subsubsection{BLS12-381 built-in functions}
\label{sec:bls-builtins-4}

\newcommand{\hash}{\mathsf{hash}}
\newcommand{\compress}{\mathsf{compress}}
\newcommand{\uncompress}{\mathsf{uncompress}}

\setlength{\LTleft}{-4mm} % Shift the table left a bit to centre it on the page
\begin{longtable}[H]{|l|p{5cm}|p{25mm}|c|c|}
    \hline \text{Function} & \text{Signature} & \text{Denotation} & \text{Can}
    & \text{Note} \\ & & & fail?
    & \\ \hline \endfirsthead \hline \text{Function} & \text{Type}
    & \text{Denotation} & \text{Can} & \text{Note}\\ & & & fail?
    & \\ \hline \endhead \hline \caption{BLS12-381 built-in Functions}
    % This caption goes on every page of the table except the last.  Ideally it
    % would appear only on the first page and all the rest would say
    % (continued). Unfortunately it doesn't seem to be easy to do that in a
    % longtable.
    \endfoot
%%    \caption[]{Built-in Functions}
    \caption[]{BLS12-381 built-in Functions (continued)}
    \label{table:built-in-functions-4}
    \endlastfoot
%% G1
    \TT{bls12\_381\_G1\_add}  &
    $[ \ty{bls12\_381\_G1\_element}$,
      \text{\; $\ty{bls12\_381\_G1\_element} ]$}
      \text{\: $ \to \ty{bls12\_381\_G1\_element}$} & $(a,b) \mapsto a+b$  &  No & \\
    \TT{bls12\_381\_G1\_neg}  &
      $ [ \ty{bls12\_381\_G1\_element} ]$  \text{\;\; $\to \ty{bls12\_381\_G1\_element}$} & $a \mapsto -a$  & No & \\
    \TT{bls12\_381\_G1\_scalarMul}  &
    $[ \ty{integer}$,
      \text{\; $\ty{bls12\_381\_G1\_element} ]$}
      \text{\: $ \to \ty{bls12\_381\_G1\_element}$} & $(n,a) \mapsto na$ &  No & \\
    \TT{bls12\_381\_G1\_equal}  &
    $[ \ty{bls12\_381\_G1\_element}$,
      \text{\; $\ty{bls12\_381\_G1\_element} ]$}
      \text{\: $ \to \ty{bool}$} & $=$ &  No & \\
    \TT{bls12\_381\_G1\_hashToGroup}  &
    $[ \ty{bytestring}, \ty{bytestring}]$
      \text{\: $ \to \ty{bls12\_381\_G1\_element}$} & $\hash_{G_1}$ &  Yes & \ref{note:hashing-into-group}\\
    \TT{bls12\_381\_G1\_compress}  &
    $[\ty{bls12\_381\_G1\_element}]$
      \text{\: $ \to \ty{bytestring}$} & $\compress_{G_1}$  &  No & \ref{note:group-compression}\\
    \TT{bls12\_381\_G1\_uncompress}  &
    $[ \ty{bytestring}]$
      \text{\: $ \to \ty{bls12\_381\_G1\_element}$} & $\uncompress_{G_1}$  &  Yes & \ref{note:group-uncompression}\\
    \hline 
%% G2
    \TT{bls12\_381\_G2\_add}  &
    $[ \ty{bls12\_381\_G2\_element}$,
      \text{\; $\ty{bls12\_381\_G2\_element} ]$}
      \text{\: $ \to \ty{bls12\_381\_G2\_element}$} & $(a,b) \mapsto a+b$ &  No & \\
    \TT{bls12\_381\_G2\_neg}  &
      $ [ \ty{bls12\_381\_G2\_element} ]$  \text{\;\; $\to \ty{bls12\_381\_G2\_element}$} & $a \mapsto -a$  & No & \\
    \TT{bls12\_381\_G2\_scalarMul}  &
    $[ \ty{integer}$,
      \text{\; $\ty{bls12\_381\_G2\_element} ]$}
      \text{\: $ \to \ty{bls12\_381\_G2\_element}$} & $(n,a) \mapsto na$ &  No & \\
    \TT{bls12\_381\_G2\_equal}  &
    $[ \ty{bls12\_381\_G2\_element}$,
      \text{\; $\ty{bls12\_381\_G2\_element} ]$}
      \text{\: $ \to \ty{bool}$} & $=$ &  No & \\
    \TT{bls12\_381\_G2\_hashToGroup}  &
    $[ \ty{bytestring}, \ty{bytestring}]$
      \text{\: $ \to \ty{bls12\_381\_G2\_element}$} & $\hash_{G_2}$  &  Yes & \ref{note:hashing-into-group}\\
    \TT{bls12\_381\_G2\_compress}  &
    $[\ty{bls12\_381\_G2\_element}]$
      \text{\: $ \to \ty{bytestring}$} & $\compress_{G_2}$  &  No & \ref{note:group-compression}\\
    \TT{bls12\_381\_G2\_uncompress}  &
    $[ \ty{bytestring}]$
      \text{\: $ \to \ty{bls12\_381\_G2\_element}$} & $\uncompress_{G_2}$  &  Yes & \ref{note:group-uncompression}\\
    \hline 
    \TT{bls12\_381\_millerLoop}  &
    $[ \ty{bls12\_381\_G1\_element}$,
      \text{\; $\ty{bls12\_381\_G2\_element} ]$}
    \text{\: $ \to \TyMlResult$} & $e$ &  No & \ref{note:pairing}\\
    \TT{bls12\_381\_mulMlResult}  &
    $[ \TyMlResult$,
    \text{\; $\TyMlResult]$}
    \text{\: $\to \TyMlResult$} & $(a,b) \mapsto ab$ & No & \ref{note:pairing}\\
    \TT{bls12\_381\_finalVerify}  &
    $[ \TyMlResult$,
    \text{\; $\TyMlResult] \to \ty{bool}$} & $\phi$ & No & \ref{note:pairing}\\
    \hline
\end{longtable}


\note{Hashing into $G_1$ and $G_2$.}
\label{note:hashing-into-group}
The denotations $\hash_{G_1}$ and $\hash_{G_2}$
of \texttt{bls12\_381\_G1\_hashToGroup} and
\texttt{bls12\_381\_G2\_hashToGroup} both take an arbitrary bytestring $b$ (the
\textit{message}) and a (possibly empty) bytestring of length at most 255 known as a \textit{domain
separation tag} (DST)~\cite[2.2.5]{IETF-hash-to-curve} and hash them to obtain a
point in $G_1$ or $G_2$ respectively.  The details of the hashing process are
described in~\cite{IETF-hash-to-curve} (see specifically Section 8.8), except
that
\textbf{we do not support DSTs of length greater than 255}: an attempt to use a
longer DST directly will cause an error.  If a longer DST is required then it
should be hashed to obtain a short DST as described
in~\cite[5.3.3]{IETF-hash-to-curve}, and then this should be supplied as the
second argument to the appropriate \texttt{hashToGroup} function.

%% Some hashing
%% implementations also allow a third argument (an ``augmentation string''), but we
%% do not support this since the same effect can be obtained by appending
%% (prepending?) the augmentation string to the message before hashing.

\newcommand{\ymin}{y_{\text{min}}}
\newcommand{\ymax}{y_{\text{max}}}

\note{Compression for elements of $G_1$ and $G_2$.} 
\label{note:group-compression}
Points in $G_1$ and $G_2$ are encoded as bytestrings in a ``compressed'' format
where only the $x$-coordinate of a point is encoded and some metadata is used to
indicate which of two possible $y$-coordinates the point has.  The encoding
format is based on the Zcash encoding for BLS12-381 points:
see~\cite{Zcash-serialisation} or~\cite[``Serialization'']{BLST-library}
or~\cite[Appendices C and D]{IETF-pairing-friendly-curves}.  In detail,

\begin{itemize}

\item Given an element $x$ of $\Fq$, $\tilde{x}$ can be written as a 381-bit
binary number: $\tilde{x} = \sum_{i=0}^{380}b_i2^i$ with $b_i \in \{0,1\}$.  We
define $\mathsf{bits}(x)$ to be the bitstring $b_{380}\cdots b_0$.

\item A non-identity element of $G_1$ can be written in the form $(x,y)$ with $x,y\in\Fq$.
Not every element $x$ of $\Fq$ is the $x$-coordinate of a point in $G_1$, but
those which are in fact occur as the $x$-coordinate of two distinct points in
$G_1$ whose $y$-coordinates are the negatives of each other.  A similar
statement is true for $\Fqq$ and $G_2$.  In both cases we denote the smaller of
the possible $y$-coordinates by $\ymin(x)$ and the larger by $\ymax(x)$.

\item For $(x,y) \in G_1\backslash \mathcal{O}_{G_1}$ we define
$$
\compress_{G_1} (x,y) = \begin{cases}
\mathsf{1}\mathsf{0}\mathsf{0}\cdot\mathsf{bits}(x) & \text{if $y=\ymin(x)$}\\
\mathsf{1}\mathsf{0}\mathsf{1}\cdot\mathsf{bits}(x) & \text{if $y=\ymax(x)$}\\
\end{cases}
$$
\item We encode the identity element of $G_1$ using
$$
\compress_{G_1}(\mathcal{O}_{G_1}) = \mathsf{1}\mathsf{1}\mathsf{0}\cdot\mathsf{0}^{381},
$$
\noindent where $\mathsf{0}^{381}$ denotes a string of 381 $\mathsf{0}$ bits.
\end{itemize}
\noindent Thus in all cases the encoding of an element of $G_1$ requires exactly 384 bits,
or 48 bytes.

\medskip 

\noindent 
\begin{itemize}
\item Similarly, every non-identity element of $G_2$ can be written
in the form $(x,y)$ with $x,y \in \Fqq$.  We define

$$
\compress_{G_2} (a+bu,y) = \begin{cases}
\mathsf{1}\mathsf{0}\mathsf{0}\cdot\mathsf{bits}(b)\cdot\mathsf{0}\mathsf{0}\mathsf{0}\cdot\mathsf{bits}(a)
& \text{if $y=\ymin(a+bu)$}\\
\mathsf{1}\mathsf{0}\mathsf{1}\cdot\mathsf{bits}(b)\cdot\mathsf{0}\mathsf{0}\mathsf{0}\cdot\mathsf{bits}(a) &
 \text{if $y=\ymax(a+bu)$}\\
\end{cases}
$$

\item The identity element of $G_2$ is encoded using
$$
\compress_{G_2}(\mathcal{O}_{G_2}) = \mathsf{1}\mathsf{1}\mathsf{0}\cdot\mathsf{0}^{765}.
$$

\end{itemize}
\noindent The encoding of an element of $G_2$ requires exactly 768 bits, or 96 bytes.

Note that in all cases the most significant bit of a compressed point is 1.  In
the Zcash serialisation scheme this indicates that the point is compressed;
Zcash also supports a serialisation format where both the $x$- and
$y$-coordinates of a point are encoded, and in that case the leading bit of the
encoded point is 0.  We do not support this format.

\note{Uncompression for elements of $G_1$ and $G_2$.} 
\label{note:group-uncompression}
There are two (partial) ``uncompression'' functions $\uncompress_{G_1}$ and
$\uncompress_{G_2}$ which convert bytestrings into group elements; these are
obtained by inverting the process described in
Note~\ref{note:group-compression}.

\paragraph{Uncompression for $G_1$ elements.}  Given a bytestring $b$, it is checked that
$b$ contains exactly 48 bytes.  If not, then $\uncompress_{G_1}(b) = \errorX$ (ie,
uncompression fails).  If the length is equal to 48 bytes, write $b$ as a
sequence of bits: $b = b_{383} \cdots b_0$.
\begin{itemize}
\item If $b_{383} \neq 1$, then $\uncompress_{G_1}(b) = \errorX$.
\item If $b_{383} = b_{382} = 1$ then
$\uncompress_{G_1}(b) =
\begin{cases}
\mathcal{O}_{G_1} & \text{if $b_{381} = b_{380} = \cdots = b_0 = 0$}\\
\errorX & \text{otherwise}.
\end{cases}$
\item If $b_{383}=1$ and $b_{382}=0$, let $c=\sum_{i=0}^{380}b_i2^i \in \N$.
\begin{itemize}
\item If $c \geq q$, $\uncompress_{G_1}(b) = \errorX$.
\item Otherwise, let $x = \bar{c} \in \Fq$ and let $z = x^3+4$. If $z$ is not a
square in $\Fq$, then $\uncompress_{G_1}(b) = \errorX$.
\item If $z$ is a square then let
$y=\begin{cases}
\ymin(x) & \text{if $b_{381}=0$}\\
\ymax(x) & \text{if $b_{381}=1$}.
\end{cases}$
\item Then $\uncompress_{G_1}(b) = \begin{cases}
(x,y) & \text{if $(x,y) \in G_1$}\\
\errorX & \text{otherwise}.
\end{cases}$
\end{itemize}
\end{itemize}


\paragraph{Uncompression for $G_2$ elements.}  Given a bytestring $b$, it is checked that
$b$ contains exactly 96 bytes.  If not, then $\uncompress_{G_2}(b) = \errorX$ (ie,
uncompression fails).  If the length is equal to 96 bytes, write $b$ as a
sequence of bits: $b = b_{767} \cdots b_0$.
\begin{itemize}
\item If $b_{767} \neq 1$, then $\uncompress_{G_2}(b) = \errorX$.
\item If $b_{767} = b_{766} = 1$ then $\uncompress_{G_2}(b) =
\begin{cases}
\mathcal{O}_{G_2} & \text{if $b_{765} = b_{764} = \cdots = b_0 = 0$}\\
\errorX & \text{otherwise}.
\end{cases}$
\item If $b_{767}=1$ and $b_{766} = 0$, let $c=\sum_{i=0}^{383}b_i2^i$ and $d=\sum_{i=384}^{764}b_i2^{i-384} \in \N$.
\begin{itemize}
\item If $c \geq q$ or $d \geq q$, $\uncompress_{G_2}(b) = \errorX$.
\item Otherwise, let $x = \bar{c}+\bar{d}u \in \Fqq$ and let $z = x^3+4(u+1)$.
If $z$ is not a square in $\Fqq$, then $\uncompress_{G_2}(b) = \errorX$.
\item If $z$ is a square then let
$y=\begin{cases}
\ymin(x) & \text{if $b_{765}=0$}\\
\ymax(x) & \text{if $b_{765}=1$}.
\end{cases}$
\item Then $\uncompress_{G_2}(b) = \begin{cases}
(x,y) & \text{if $(x,y) \in G_2$}\\
\errorX & \text{otherwise}.
\end{cases}$
\end{itemize}
\end{itemize}


\note{Concrete syntax for BLS12-381 types.}
\label{note:bls-syntax}
Concrete syntax for the $\ty{bls12\_381\_G1\_element}$ and
$\ty{bls12\_381\_G2\_element}$ types is provided via the compression and
decompression functions defined in Notes~\ref{note:group-compression}
and~\ref{note:group-uncompression}.  Specifically, a value of type
$\ty{bls12\_381\_G1\_element}$ is denoted by a term of the form \texttt{(con
bls12\_381\_G1\_element 0x...)} where \texttt{...}  consists of 96 hexadecimal
digits representing the 48-byte compressed form of the relevant point.
Similarly, a value of type $\ty{bls12\_381\_G2\_element}$ is denoted by a term
of the form \texttt{(con bls12\_381\_G2\_element 0x...)}  where \texttt{...}
consists of 192 hexadecimal digits representing the 96-byte compressed form of
the relevant point.  \textbf{This syntax is provided only for use in textual
Plutus Core programs}, for example for experimentation and testing.  We do not
support constants of any of the BLS12-381 types in serialised programs on the
Cardano blockchain: see Section~\ref{sec:flat-constants}.  However, for
$\ty{bls12\_381\_G1\_element}$ and $\ty{bls12\_381\_G2\_element}$ one can use
the appropriate uncompression function on a  bytestring constant at runtime:
for example, instead of
$$
\texttt{(con bls12\_381\_G1\_element 0xa1e9a0...)}
$$
write
$$
\texttt{[(builtin bls12\_381\_G1\_uncompress) (con bytestring \#a1e9a0...)]}.
$$

\noindent
No concrete syntax is provided for values of type
$\ty{bls12\_381\_mlresult}$. It is not possible to parse such values, and they
will appear as \texttt{(con bls12\_381\_mlresult <opaque>)} if output by a
program.


\note{Pairing operations.}
\label{note:pairing}
For efficiency reasons we split the pairing process into two parts:
the \texttt{bls12\_381\_millerLoop} and \texttt{bls12\_381\_finalVerify}
functions.  We assume that we have
\begin{itemize}
\item An intermediate multiplicative abelian group $H$.
\item A function (not necessarily itself a pairing) $e: G_1 \times
G_2 \rightarrow \MlResultDenotation$.
\item A cyclic group $\mu_r$ of order $r$.
\item An epimorphism $\psi: \MlResultDenotation \rightarrow \mu_r$ of groups such
that $\psi \circ e: G_1 \times G_2 \rightarrow \mu_r$ is a (nondegenerate,
bilinear) pairing.
\end{itemize}

\noindent Given these ingredients, we define
\begin{itemize}
\item $\denote{\TyMlResult} = \MlResultDenotation$.
\item $\denote{\mathtt{bls12\_381\_mulMlResult}} =$
the group multiplication operation in $\MlResultDenotation$.
\item $\denote{\mathtt{bls12\_381\_millerLoop}} = e$.
\item $\denote{\mathtt{bls12\_381\_finalVerify}} = \phi$,
where
$$
\phi(a,b) = \begin{cases}
               \mathtt{true} & \text{if $\psi(ab^{-1}) = 1_{\mu_r}$} \\
               \mathtt{false} & \text{otherwise.}
            \end{cases}
$$
\end{itemize}

\medskip
\noindent
We do not mandate specific choices for $\MlResultDenotation, \mu_r, e$, and $\phi$, but a
plausible choice would be
\begin{itemize}
\item $\MlResultDenotation = \units{\FF}$.
\item $e$ is the Miller loop associated with the optimal Ate pairing
for $E_1$ and $E_2$~\cite{Vercauteren}.
\item $\mu_r = \{x \in \units{\FF}: x^r=1\}$, the group of $r$th roots of unity in $\FF$.
(There are $r$ distinct $r$th roots of unity in $\FF$ because the embedding
degree of $E_1$ and $E_2$ with respect to $r$ is 12 (see~\cite[4.1]{Costello-pairings}).)
\item $\psi(x) = x^{\frac{q-1}{r}}$.
\end{itemize}

\noindent The functions \texttt{bls12\_381\_millerLoop} and (especially)
\texttt{bls12\_381\_finalVerify} are expected
to be expensive, so their use should be kept to a minimum.  Fortunately most
current use cases do not require many uses of these functions.


\begin{appendices}
\section{Expressiveness of \EUTXO{}}
\label{sec:expressiveness}

In this section, we introduce a class of state machines that can admit
a straightforward modelling of smart contracts running on an \EUTXO{}
ledger. The class we choose corresponds closely to Mealy
machines~\cite{mealy} (deterministic state transducers). The transition
function in a Mealy machine produces a value as well as a new
state. We use this value to model the emission of constraints which
apply to the current transaction in the ledger. We do not claim that
this class captures the full power of the ledger: instead we choose it
for its simplicity, which is sufficient to capture a wide variety of
use cases.

We demonstrate how one can represent a smart contracts using Mealy
machines and formalise a \textit{weak bisimulation} between the
machine model and the ledger model.  Furthermore, we have mechanised
our results in Agda\site{
https://github.com/\GitUser/formal-utxo/tree/\AgdaCommit/Bisimulation.agda
}, based on an executable
specification of the model described in
Section~\ref{sec:formal-model}.

\subsection{Constraint Emitting Machines}
We introduce Constraint Emitting Machines (\CEM{}) which are based on
Mealy machines. A \CEM{} consists of its type of states \s{S} and inputs
\s{I}, a predicate function $\s{final} : \s{S} \rightarrow \s{Bool}$
indicating which states are final and a valid set of transitions,
given as a function $\s{step} : \s{S} \rightarrow \s{I} \rightarrow
\s{Maybe}\ (\s{S} \times \s{TxConstraints})$\footnote{
The result may be \s{Nothing}, in case no valid transitions exist from a given state/input.
}
from source state and input symbol to target state and constraints and
denoted $\CStep{s}{i}{\txeq}$.

The class of state machines we are concerned with here diverge from
the typical textbook description of Mealy Machines in the following aspects:
\begin{itemize}
\item The set of states can be infinite.

\item There is no notion of \textbf{initial state}, since we would not
  be able to enforce it on the blockchain level. Therefore, each
  contract should first establish some initial trust to bootstrap the
  process.
  One possible avenue for overcoming this limitation is to built a notion of \textit{trace simulation}
  on top of the current relation between single states,
  thus guaranteeing that only valid sequences starting from initial states appear on the ledger.
  For instance, this could be used to establish inductive properties of a state machine and
  carry them over to the ledger; we plan to investigate such concerns in future work.

\item While \textbf{final states} traditionally indicate that the
  machine \textit{may} halt at a given point, allowing this
  possibility would cause potentially stale states to clutter the \UTXO{}
  set in the ledger. Thus, a \CEM{} final state indicates that the
  machine \textit{must} halt. It will have no continuing transitions
  from this point onward and the final state will not appear in the
  \UTXO{} set. This corresponds to the notion of a \emph{stopped}
  process~\cite{sangiorgi} which cannot make any transitions.

\item The set of output values is fixed to \emph{constraints} which
  impose a certain structure on the transaction that will implement
  the transition.  Our current formalisation considers a limited set
  of first-order constraints, but these can easily be extended without
  too many changes in the accompanying proofs.
\end{itemize}

\subsection{Transitions-as-transactions}

We want to compile a smart contract $\mathcal{C}$ defined as a \CEM{} into
a smart contract that runs on the chain. The idea is to derive a
validator script from the step function, using the data value to hold the
state of the machine, and the redeemer to provide the transition signal.
A valid transition in a \CEM{}
will correspond to a single valid transaction on the chain. The
validator is used to determine whether a transition is valid and the
state machine can advance to the next state. More specifically, this
validator should ensure that we are transitioning to a valid target
state, the corresponding transaction satisfies the emitted constraints
and that there are no outputs in case the target state is final:

\[
\mkValidator{\mathcal{C}}(s, i, \mi{txInfo}) = \left\{
  \begin{array}{lll}
  \true  & \mi{if} \ \CStep{s}{i}{\txeq} \\
         & \mi{and} \ \satisfies(\mi{txInfo}, \txeq) \\
         & \mi{and} \ \checkOutputs(s', \mi{txInfo}) \\
  \false & \mi{otherwise}
  \end{array}
\right.
\]

\noindent
Note that unlike the step function which returns the new state, the
validator only returns a boolean. On the chain the next state is
provided with the transaction output that ``continues'' the state
machine (if it continues), and the validator simply validates that
the correct next state was provided.\footnote{
A user can run the step function locally to determine the correct next state off-chain.
}

\subsection{Behavioural Equivalence}
We have explained how to compile state machines to smart contracts but
how do we convince ourselves that these smart contracts will behave as
intended? We would like to show
\begin{inparaenum}[(1)]
\item that any valid transition in a \CEM{} corresponds to a valid transaction
  on the chain, and
\item that any valid transaction on the chain corresponds to a valid transition.
\end{inparaenum}
We refer to these two properties as soundness and completeness below.

While state machines correspond to automata, the automata theoretic
notion of equivalence --- trace equivalence --- is too coarse when we
consider state machines as running processes. Instead we use
bisimulation which was developed in concurrency theory for exactly
this purpose, to capture when processes behave the
same~\cite{sangiorgi}. We consider both the state machine and the
ledger itself to be running processes.

If the state machine was the only user of the ledger then we could
consider so-called strong bisimulation where we expect transitions in
one process to correspond to transitions in the other and
vice-versa. But, as we expect there to be other unrelated transactions
occurring on the ledger we instead consider weak bisimulation where
the ledger is allowed to make additional so-called \emph{internal}
transitions that are unrelated to the behaviour we are interested in
observing.

The bisimulation proof relates steps of the \CEM{} to new transaction
submissions on the blockchain.  Note that we have a \textit{weak}
bisimulation, since it may be the case that a ledger step does not
correspond to a \CEM{} step.

\begin{definition}[Process relation]
A \CEM{} state $s$ corresponds to a ledger $l$ whenever $s$ appears
in the current \UTXO{} set,
locked by the validator derived from this \CEM{}:
\[
\Sim{l}{s}
\]
\end{definition}

\begin{definition}[Ledger step]
Concerning the blockchain transactions, we only consider valid
ledgers.\footnote{ In our formal development, we enforce validity
  statically at compile time.  }  Therefore, a valid step in the
ledger consists of submitting a new transaction $tx$,
valid w.r.t. to the current ledger $l$,
resulting in an extended ledger $l'$:
\[
\LStep{l}{tx}
\]
\end{definition}

\begin{proposition}[Soundness]
Given a valid \CEM{} transition $\CStep{s}{i}{\txeq}$ and a valid ledger
$l$ corresponding to source state $s$, we can construct a valid
transaction submission to get a new, extended ledger $l'$ that
corresponds to target state $s'$:
\[
\infer[\textsc{sound}]
  {\exists tx\ l'\ .\ \LStep{l}{tx}\ \wedge \Sim{l'}{s'}}
  {%
    \CStep{s}{i}{\txeq}
  & \Sim{l}{s}
  }
\]
\end{proposition}

\paragraph{Note.}
We also require that the omitted constraints are satisfiable in the
current ledger and the target state is not a final one, since there
would be no corresponding output in the ledger to witness
$\Sim{l'}{s'}$.  We could instead augment the definition of
correspondence to account for final states, but we refrained from
doing so for the sake of simplicity.

\begin{proposition}[Completeness]
Given a valid ledger transition $\LStep{l}{tx}$ and a \CEM{} state $s$
that corresponds to $l$, either $tx$ is irrelevant to the current \CEM{}
and we show that the extended ledger $l'$ still corresponds to source
state $s$, or $tx$ is relevant and we exhibit the corresponding \CEM{}
transition $\CStep{s}{i}{\txeq}$\footnote{ We cannot provide a
  correspondence proof in case the target state is final, as explained
  in the previous note.  }:
\[
\infer[\textsc{complete}]
  { \Sim{l'}{s}\ \vee\ \exists i\ s'\ \txeq\ .\ \CStep{s}{i}{\txeq} }
  { \LStep{l}{tx}
  & \Sim{l}{s}
  }
\]
\end{proposition}

Together, soundness and completeness finally give us weak bisimulation.
Note, however, that our notion of bisimulation differs from the textbook one (e.g. in Sangiorgi~\cite{sangiorgi}),
due to the additional hypotheses that concern our special treatment of constraints and final states.

\section{Serialising \texttt{data} Objects Using the CBOR Format}
\label{appendix:data-cbor-encoding}

\subsection{Introduction}
In this section we define a CBOR encoding for the \texttt{data} type introduced
in Section~\ref{sec:alonzo-built-in-types}.  For ease of reference we reproduce
the definition of the Haskell \texttt{Data} type, which we may regard as the
definition of the Plutus \texttt{data} type. Other representations are of course
possible, but this is useful for the present discussion.

\begin{alltt}
   data Data =
      Constr Integer [Data]
      | Map [(Data, Data)]
      | List [Data]
      | I Integer
      | B ByteString
\end{alltt}

\noindent The CBOR encoding defined here uses basic CBOR encodings as defined in
the CBOR standard~\cite{rfc8949-CBOR}, but with some refinements. Specifically

\begin{itemize}
\item We use a restricted encoding for bytestrings which requires that
  bytestrings are serialised as sequences of blocks, each block being at most 64
  bytes long.  Any encoding of a bytestring using our scheme is valid according
  to the CBOR specification, but the CBOR specification permits some encodings
  which we do not accept. The purpose of the size restriction is to prevent
  arbitrary data from bring stored on the blockchain.
\item Large integers (less than $-2^{64}$ or greater than $2^{64}-1$) are
  encoded via the restricted bytestring encoding; other integers are encoded as
  normal. Again, our restricted encodings are compatible with the CBOR
  specification.
\item The \texttt{Constr} case of the \texttt{data} type is encoded using a
  scheme which is an early version of a proposed extension of the CBOR
  specification to include encodings for discriminated unions.
  See~\cite{CBOR-alternatives} and \cite[Section 9.1]{CBOR-notable-tags}.
  \end{itemize}


\subsection{Notation}
We introduce some extra notation for use here and in
Appendix~\ref{appendix:flat-serialisation}.

\medskip 
\noindent The notation $f: X \rightharpoonup Y$ indicates that $f$ is a partial
map from $X$ to $Y$.  We denote the empty bytestring by $\epsilon$ and (as
in~\ref{sec:notation-lists}) use $\length(s)$ to denote the length of a
bytestring $s$ and $\cdot$ to denote the concatenation of two bytestrings, and
also the operation of prepending or appending a byte to a bytestring. We will
also make use of the $\divfn$ and $\modfn$ functions described in
Note~\ref{note:integer-division-functions} in
Appendix~\ref{appendix:default-builtins-alonzo}.%
\nomenclature[Bz]{$\epsilon$}{The empty bytestring}

\paragraph{Encoders and decoders.}
Recall that $\B = \Nab{0}{255}$, the set of integral values that can
be represented in a single byte, and that we identify bytestrings with elements
of $\B^*$. We will describe the CBOR encoding of the \texttt{data} type by
defining families of encoding functions (or \textit{encoders})
$$
\e_X : X \rightarrow \B^*
$$%
\nomenclature[HC]{$\e_X$}{CBOR encoder for \texttt{data}}
and decoding functions (or \textit{decoders})
$$
\d_X : \B^* \rightharpoonup \B^* \times X
$$%
\nomenclature[HC]{$\d_X$}{CBOR decoder for \texttt{data}}

\noindent for various sets $X$, such as the set $\Z$ of integers and the set of
all \texttt{data} items.  The encoding function $\e_X$ takes an element $x \in
X$ and converts it to a bytestring, and the decoding function $\d_X$ takes a
bytestring $s$, decodes some initial prefix of $s$ to a value $x \in X$, and
returns the remainder of $s$ together with $x$.  Decoders for complex types will
often be built up from decoders for simpler types.  Decoders are
\textit{partial} functions because they can fail, for instance, if there is
insufficient input, or if the input is not well formed, or if a decoded value is
outside some specified range. 

Many of the decoders which we define below involve a number of cases for
different forms of input, and we implicitly assume that the decoder fails if
none of the cases applies.  We also assume that if a decoder fails then so does
any other decoder which invokes it, so any failure when attempting to decode a
particular data item in a bytestring will cause the entire decoding process to
fail (immediately).

\subsection{The CBOR format}
A CBOR-encoded item consists of a bytestring beginning with a \textit{head}
which occupies 1,2,3,5, or 9 bytes.  Depending on the contents of the head, some
sequence of bytes following it may also contribute to the encoded item. The
first three bits of the head are interpreted as a natural number between 0 and 7
(the \textit{major type}) which gives basic information about the type of the
following data.  The remainder of the head is called the \textit{argument} of the
head and is used to encode further information, such as the value of an encoded
integer or the size of a list of encoded items.  Encodings of complex objects
may occupy the bytes following the head, and these will typically contain
further encoded items.

\subsection{Encoding and decoding the heads of CBOR items}
For $i \in \N$ we define a function $\byte_i: \N \rightarrow \B$ which returns
the $i$-th byte of an integer, with the 0-th byte being the least significant:
$$
  \byte_i(n) = \modfn(\divfn(n,256^i), 256).
$$

\noindent We use this to define for each $k \geq 1$ a partial function
$\intToBS_k: \N \rightharpoonup \B^*$ which converts a sufficiently small
integer to a bytestring of length $k$ (possibly with leading zeros):
$$
\intToBS_k(n) = [\byte_{k-1}(n), \ldots, \byte_0(n)]  \quad \text {if $n \leq 256^k-1$}.
$$
\noindent
This function fails if the input is too large to fit into a $k$-byte
bytestring.

We also define inverse functions $\bsToInt_k: \B^* \rightharpoonup \N$ which
decode a $k$-byte natural number from the start of a bytestring, failing if
there is insufficient input:
$$ \bsToInt_k(s) = (s', \sum_{i=0}^{k-1}256^ib_i) \qquad \text{if $s = [b_{k-1},
    \ldots, b_0] \cdot s'$}.
$$
 
\noindent We now define an encoder $\eHead: \Nab{0}{7} \times
\Nab{0}{2^{64}-1} \rightarrow \B^*$ which takes a major type and a
natural number and encodes them as a CBOR head using the standard encoding:

$$
  \eHead(m,n) =
  \begin{cases}
    [32m + n] & \text{if $n \leq 23$}\\
    (32m+24) \cdot \intToBS_1(n) & \text{if $24 \leq n \leq 255$}\\
    (32m+25) \cdot \intToBS_2(n) & \text{if $256 \leq n \leq 256^2-1$}\\
    (32m+26) \cdot \intToBS_4(n)& \text{if $256^2 \leq n \leq 256^4-1$}\\
    (32m+27) \cdot \intToBS_8(n) & \text{if $256^4 \leq n \leq 256^8-1$}.
  \end{cases}
$$

\noindent The corresponding decoder $\dHead: \B^* \rightharpoonup \B^* \times
\Nab{0}{7} \times \Nab{0}{2^{64}-1}$ is given by 

$$
  \dHead(n \cdot s) =
  \begin{cases}
    (s, \divfn(n,32), \modfn(n,32)) & \text{if $\modfn(n,32) \leq 23$}\\
    (s', \divfn(n,32), k) & \text{if $\modfn(n,32) = 24$ and $\bsToInt_1(s) = (s', k)$}\\
    (s', \divfn(n,32), k) & \text{if $\modfn(n,32) = 25$ and $\bsToInt_2(s) = (s', k)$}\\
    (s', \divfn(n,32), k) & \text{if $\modfn(n,32) = 26$ and $\bsToInt_4(s) = (s', k)$}\\
    (s', \divfn(n,32), k) & \text{if $\modfn(n,32) = 27$ and $\bsToInt_8(s) = (s', k)$}.
  \end{cases}
$$

\noindent This function is undefined if the input is the empty bytestring
$\epsilon$, if the input is too short, or if its initial byte is not of the
expected form.

\paragraph{Heads for indefinite-length items.}
The functions $\eHead$ and $\dHead$ defined above are used for a number of
purposes.  One use is to encode integers less than 64 bits, where the argument
of the head is the relevant integer.  Another use is for ``definite-length''
encodings of items such as bytestrings and lists, where the head contains the
length $n$ of the object and is followed by some encoding of the object itself
(for example a sequence of $n$ bytes for a bytestring or a sequence of $n$
encoded objects for the elements of a list).  It is also possible to have
``indefinite-length'' encodings of objects such as lists and arrays, which do
not specify the length of an object in advance: instead a special head with
argument 31 is emitted, followed by the encodings of the individual items; the
end of the sequence is marked by a ``break'' byte with value 255.  We define an
encoder $\eIndef: \Nab{2}{5} \rightarrow \B^*$ and a decoder
$\dIndef: \B^* \rightharpoonup \B^* \times \Nab{2}{5}$ which deal
with indefinite heads for a given major type:

\begin{align*}
  \eIndef(m) &= [32m+31]\\
  \dIndef(n \cdot s) & = (s, m) \qquad \text{if $n = 32m+31$}.
\end{align*}
  
\noindent Note that $\eIndef$ and $\dIndef$ are only defined for $m \in
\{2,3,4,5\}$ (and we shall only use them in these cases). The case $m=31$
corresponds to the break byte and for $m \in \{0,1,6\}$ the value is not well
formed: see~\cite[3.2.4]{rfc8949-CBOR}.

\subsection{Encoding and decoding bytestrings}
The standard CBOR encoding of bytestrings encodes a bytestring as either a
definite-length sequence of bytes (the length being given in the head) or as an
indefinite-length sequence of definite-length ``chunks'' (see~\cite[\S\S3.1 and
  3.4.2]{rfc8949-CBOR}).  We use a similar scheme, but only allow chunks of
length up to 64.  To this end, suppose that $a = [a_1, \ldots, a_{64k+r}] \in
\B^*\backslash\{\epsilon\}$ where $k \geq 0$ and $0 \leq r \leq 63$.  We define
the \textit{canonical 64-byte decomposition} $\bar{a}$ of $a$ to be
$$
\bar{a} = [[a_1, \ldots, a_{64}],
  [a_{65}, \ldots, a_{128}] ,\ldots, 
  [a_{64(k-1)+1}, \ldots, a_{64k}]] \in (\B^*)^*
$$
\noindent if $r=0$ and

$$
\bar{a} = [[a_1, \ldots, a_{64}],
  [a_{65}, \ldots, a_{128}], \ldots, 
  [a_{64(k-1)+1}, \ldots, a_{64k}], [a_{64k+1}, \ldots, a_{64k+r}]] \in (\B^*)^*
$$
\noindent if $r>0$.  The canonical decomposition of the empty list is $\bar{\epsilon} = []$.

\medskip
\noindent We define the encoder $\eBS: \B^* \rightarrow \B^*$ for bytestrings by
encoding bytestrings of size up to 64 using the standard CBOR encoding and
encoding larger bytestrings by breaking them up into 64-byte chunks (with the
final chunk possibly being less than 64 bytes long) and encoding them as an
indefinite-length list (major type 2 indicates a bytestring):
$$ \eBS(s) =
\begin{cases}
  \eHead(2,\length(s)) \cdot s & \text{if $\length(s) \leq 64$}\\
  \eIndef(2) \cdot \eHead(2,\length(c_1)) \cdot c_1 \cdot \eHead(2,\length(c_2)) \cdot \cdots \\
  \qquad  \cdots  \cdot c_{n-1} \cdot \eHead(2,\length(c_n)) \cdot c_n \cdot 255
  & \text{if $\length(s) > 64$ and $\bar{s} = [c_1, \ldots, c_n]$.}
\end{cases}
$$

\medskip

\noindent The decoder is slightly more complicated.  Firstly, for every $n \geq
0$ we define a decoder $\dBytes^{(n)}: \B^* \rightharpoonup \B^* \times \B^*$
which extracts an $n$-byte prefix from its input (failing in the case of
insufficient input):
$$
\dBytes^{(n)}(s) =
\begin{cases}
  (s, \epsilon) & \text{if $n=0$}\\
  (s'', b \cdot t) & \text{if $s = b \cdot s'$ and $\dBytes^{(n-1)}(s') = (s'', t)$}.
\end{cases}
$$

\noindent Secondly, we define a decoder $\dBlock: \B^* \rightharpoonup \B^*
\times \B^*$ which attempts to extract a bytestring of length at most 64
from its input; $\dBlock$ (and any other function which calls it) will
fail if it encounters a bytestring which is greater than 64 bytes.
$$
\dBlock(s) =
  \dBytes^{(n)}(s') \quad \text{if $\dHead(s) = (s', 2,n)$ and $n \leq 64$}.
$$

\noindent Thirdly, we define a decoder $\dBlocks: \B^* \rightharpoonup \B^*
\times \B^*$ which decodes a sequence of blocks and returns their concatenation.
$$
\dBlocks(s) =
\begin{cases}
  (s', \epsilon) & \text{if $s = 255 \cdot s'$}\\
  (s'', t \cdot t') &
  \text{if $\dBlock(s) = (s', t)$
    and $\dBlocks(s') = (s'', t')$}.
\end{cases}
$$

\noindent Finally we define the decoder $\dBS: \B^* \rightharpoonup \B^*
\times \B^*$ for bytestrings by
$$
\dBS(s) =
\begin{cases}
  (s', t) & \text{if $\dBlock(s) = (s', t)$}\\
  \dBlocks(s') & \text{if $\dIndef(s) = (s', 2)$}.
\end{cases}
$$

\noindent This looks for either a single block or an indefinite-length list of
blocks, in the latter case returning their concatenation.  It will accept the
output of $\eBS$ but will reject bytestring encodings containing any blocks
greater than 64 bytes long, even if they are valid bytestring encodings
according to the CBOR specification.

\subsection{Encoding and decoding integers}
As with bytestrings we use a specialised encoding scheme for integers which
prohibits encodings with overly-long sequences of arbitrary data.  We encode
integers in $\Nab{-2^{64}}{2^{64}-1}$ as normal (see ~\cite[\S
  3.1]{rfc8949-CBOR}: the major type is 0 for positive integers and 1 for
negative ones) and larger ones by emitting a CBOR tag (major type 6; argument 2
for positive numbers and 3 for negative numbers) to indicate the sign, then
converting the integer to a bytestring and emitting that using the encoder
defined above.  This encoding scheme is the same as the standard one except for
the size limitations.

\medskip
\noindent
We firstly define conversion functions $\itos : \N \rightarrow
\B^*$ and $\stoi: \B^* \rightarrow \N$ by
$$
\itos(n) =
\begin{cases}
  \epsilon & \text{if $n=0$}\\
  \itos(\divfn(n,256)) \cdot \modfn(n,256) & \text{if $n>0$.}\\
\end{cases}
$$
\noindent and
$$
\stoi(l) =
\begin{cases}
  0 & \text{if $l = \epsilon$}\\
  256\times\stoi(l') + n & \text{if $l=l' \cdot n$ with $n \in \B$.}\\
\end{cases}
$$

\noindent 
The encoder $\eZ: \Z \rightarrow \B^*$ for integers is now defined by
$$ \eZ(n) =
\begin{cases}
  \eHead(0,n)                             & \text{if $0\leq n \leq 2^{64}-1$}\\
  \eHead(6,2) \cdot \eBS(\itos(n))    & \text{if $n \geq 2^{64}$}\\
  \eHead(1,-n-1)                          & \text{if $-2^{64} \leq n \leq -1$}\\
  \eHead(6,3) \cdot \eBS(\itos(-n-1)) & \text{if $n \leq -2^{64}-1$}.
\end{cases}
$$
% 7 says it's a *tag*
% 2 -> positive bignum, 3 -> negative bignum (2.4.2)

\noindent The decoder $\dZ: \B^* \rightharpoonup \B^* \times \Z$ inverts this
process. The decoder is in fact slightly more permissive than the encoder
because it also accepts small integers encoded using the scheme for larger ones.
However, the CBOR standard permits integer encodings which contain bytestrings
longer than 64 bytes and it will not accept those.
$$ \dZ(s) =
\begin{cases}
  (s', n)               & \text{if $\dHead(s) = (s', 0,n)$}\\
  (s', -n-1)            & \text{if $\dHead(s) = (s', 1,n)$}\\
  (s'', \stoi(b))       & \text{if $\dHead(s) = (s', 6,2)$ and $\dBS(s') = (s'', b)$}\\
  (s'', -\stoi(b)-1)    & \text{if $\dHead(s) = (s', 6,3)$ and $\dBS(s') = (s'', b)$}.\\
\end{cases}
$$



\subsection{Encoding and decoding \texttt{data}}
\label{sec:encoding-data}
\newcommand\eData{\e_{\mathtt{data}}}
\newcommand\eDataStar{\e_{\mathtt{data^*}}}
\newcommand\eDataStarSq{\e_{\mathtt{(data^2)^*}}}

\newcommand\dData{\d_{\mathtt{data}}}
\newcommand\dDataStar{\d_{\mathtt{data^*}}}
\newcommand\dDataStarSq{\d_{\mathtt{(data^2)^*}}}

It is now quite straightforward to encode most \texttt{data} values.  The main
complication is in the encoding of constructor tags (the number $i$ in
$\mathtt{Constr}\: i\, l$).

\paragraph{The encoder.} The encoder is given by
\begin{alignat*}{2}
&  \eData(\mathtt{Map}\: l) && = \eHead(5,\length(l)) \cdot \eDataStarSq(l)\\ 
&  \eData(\mathtt{List}\: l) && = \eIndef(4) \cdot \eDataStar(l) \cdot 255\\ 
&  \eData(\mathtt{Constr}\: i\, l) && = \ecTag(i) \cdot \eIndef(4) \cdot  \eDataStar(l) \cdot 255\\
& \eData(\mathtt{I}\: n) && = \eZ(n)\\ 
&  \eData(\mathtt{B}\: s) && = \eBS(s).
\end{alignat*}

\noindent This definition uses encoders for lists of data items, lists of pairs
of data items, and constructor tags as follows:
$$
\eDataStar([d_1, \ldots, d_n]) = \eData(d_1) \cdot \cdots \cdot \eData(d_n)
$$
$$
\eDataStarSq([(k_1,d_1), \ldots, (k_n, d_n)]) = \eData(k_1) \cdot \eData(d_1) \cdot \cdots \cdot \eData(k_n) \cdot \eData(d_n)
$$
$$
\ecTag(i) =
\begin{cases}
  \eHead(6,121+i) & \text{if $0 \leq i \leq 6$}\\
  \eHead(6,1280+(i-7)) & \text{if $7 \leq i \leq 127$}\\
  \eHead(6,102) \cdot \eHead(4,2) \cdot \eZ(i) & \text{otherwise}.\\
  \end{cases}
$$

\noindent 
In the final case of $\ecTag$ we emit a head with major type 4 and argument
2. This indicates that an encoding of a list of length 2 will follow: the first
element of the list is the constructor number and the second is the argument
list of the constructor, which is actually encoded in $\eData$.  It might be
conceptually more accurate to have a single encoder which would encode both the
constructor tag and the argument list, but this would increase the complexity of
the notation even further.  Similar remarks apply to $\dcTag$ below.

%% kwxm: The CBOR specification says ``A map that has duplicate keys may be
%% well-formed, but it is not valid, and thus it causes indeterminate decoding.''
%% Presumably this is because potentially the map could be decoded into some data
%% structure that doesn't support duplicate keys, and exactly which one of multiple
%% entries with the same key ends up in the final structure could be
%% non-deterministic.  We don't say anything at all about the semantics of maps
%% (they really are just Scott-encoded lists of pairs in Plutus Core and it's up to
%% the user how they're treated), so I think we can just ignore this: our decoder
%% really does preserve entries with repeated keys.}


\paragraph{The decoder.} The decoder is given by
$$
\dData(s) =
\begin{cases}
  (s'', \mathtt{Map}\: l) & \text{if $\dHead(s) = (s', 5, n)$ and $\dDataStarSq^{(n)}(s') = (s'', l)$}\\
  (s', \mathtt{List}\: l) & \text{if $\dDataStar(s) = (s', l)$}\\
  (s'', \mathtt{Constr}\: i \, l) & \text{if $\dcTag(s) = (s', i)$ and $\dDataStar(s') = (s'', l)$}\\
  (s', \mathtt{I}\: n) & \text{if $\dZ(s) = (s', n)$}\\
  (s', \mathtt{B}\: b) & \text{if $\dBS(s) = (s', b)$}
\end{cases}
$$
where
$$
\dDataStar(s) =
\begin{cases}
  \dDataStar^{(n)}(s') & \text{if $\dHead(s) = (s', 4, n)$}\\
  \dDataStar^{\mathsf{indef}}(s') & \text{if $\dIndef(s) = (s', 4)$}
\end{cases}
$$

$$
\dDataStar^{(n)}(s) =
\begin{cases}
  (s, \epsilon) & \text{if $n = 0$}\\
  (s'', d \cdot l) & \text{if $\dData(s) = (s', d)$ and $\dDataStar^{(n-1)}(s') = (s'', l)$}\\
\end{cases}
$$

$$
\dDataStar^{\mathsf{indef}}(s) =
\begin{cases}
  (s', \epsilon) & \text{if $s = 255 \cdot s' $}\\
  (s'', d \cdot l) & \text{if $\dData(s) = (s', d)$ and $\dDataStar^{\mathsf{indef}}(s') = (s'', l)$}\\
\end{cases}
$$


\medskip
$$
\dDataStarSq^{(n)}(s) =
\begin{cases}
  (s, \epsilon) & \text{if $n=0$}\\
  (s''', (k,d) \cdot l) &
  \begin{cases}
    \text{if $n > 0$}\\
    \text{and $\dData(s) = (s', k)$}\\
    \text{and $\dData(s') = (s'', d)$}\\
    \text{and $\dDataStarSq^{(n-1)}(s'') = (s''', l)$}
  \end{cases}
\end{cases}
$$

\medskip
$$
\dcTag(s) =
\begin{cases}
  (s', i-121) & \text{if $\dHead(s) = (s', 6, i)$ and $121 \leq i \leq 127$}\\
  (s', (i-1280)+7) & \text{if $\dHead(s) = (s', 6, i)$ and $1280 \leq i \leq 1400$}\\
  (s''', i) &
  \begin{cases}
    \text{if $\dHead(s) = (s', 6, 102)$}\\
    \text{and $\dHead(s') = (s'', 4, 2)$}\\
    \text{and $\dZ(s'') = (s''', i)$}\\
    \text{and $0 \leq i \leq 2^{64}-1$}.
    \end{cases}\\
  \end{cases}
$$

\noindent
Note that the decoders for \texttt{List} and \texttt{Constr} accept both
definite-length and indefinite-length lists of encoded \texttt{data} values, but
the decoder for \texttt{Map} only accepts definite-length lists (and the length
is the number of \textit{pairs} in the map).  This is consistent with CBOR's
standard encoding of arrays and lists (major type 4) and maps (major type 5).

Note also that the encoder $\ecTag$ accepts arbitrary integer values for
\texttt{Constr} tags, but (for compatibility with~\cite{CBOR-alternatives}) the
decoder $\dcTag$ only accepts tags in $\Nab{0}{2^{64}-1}$.  This
means that some valid Plutus Core programs can be serialised but not
deserialised, and is the reason for the recommendation in
Section~\ref{sec:alonzo-built-in-types} that only constructor tags between 0 and
$2^{64}-1$ should be used.

\section{Serialising Plutus Core Terms and Programs Using the \texttt{flat} Format}
\label{appendix:flat-serialisation}
We use the \texttt{flat} format \cite{flat} to serialise Plutus Core
terms, and we regard this format as being the definitive concrete representation
of Plutus Core programs. For compactness we generally (and \textit{always} for
scripts on the blockchain) replace names with de Bruijn indices (see
Section~\ref{sec:grammar-notes}) in serialised programs.

We use bytestrings for serialisation, but it is convenient to define the
serialisation and deserialisation process in terms of strings of bits. Some
extra bits of padding are added at the end of the encoding of a program to
ensure that the number of bits in the output is a multiple of 8, and this allows
us to regard serialised programs as bytestrings in the obvious way.

See Section~\ref{sec:cardano-issues} for some restrictions on serialisation
specific to the Cardano blockchain.

\paragraph{Note: \texttt{flat} versus CBOR.}
Much of the Cardano codebase uses the CBOR format for serialisation; however, it
is important that serialised scripts not be too large. CBOR pays a price for
being a self-describing format. The size of the serialised terms is consistently
larger than a format that is not self-describing: benchmarks show that
\texttt{flat} encodings of Plutus Core scripts are smaller than CBOR encodings
by about 35\% (without using compression).



\subsection{Encoding and decoding}
Let $\Bits = \{\bits{0},\bits{1}\}^*$, the set of all finite sequences of
bits.  For brevity we write a sequence of bits in the form $b_{n-1} \cdots b_0$
instead of $[b_{n-1}, \ldots, b_0]$: thus $\bits{011001}$ instead of $[\bits{0},
  \bits{1},\bits{1},\bits{0},\bits{0},\bits{1}])$.  We denote the empty sequence
by $\epsilon$, and use $\length(s)$ to denote the length of a sequence of bits,
and $\cdot$ to denote concatenation (or prepending or appending a single bit to
a sequence of bits).%
\nomenclature[As]{$\Bits$}{The set of strings of bits}

\medskip
\noindent Similarly to the CBOR encoding for \texttt{data} described in
Appendix~\ref{appendix:data-cbor-encoding}, we will describe the flat encoding
by defining families of encoding functions (or \textit{encoders})
$$
\E_X : \Bits \times X \rightarrow \Bits
$$%
\nomenclature[HF]{$\E_X$}{Flat encoder}
and (partial) decoding functions (or \textit{decoders})
$$
\D_X : \Bits \rightharpoonup \Bits \times X
$$%
\nomenclature[HF]{$\D_X$}{Flat decoder}

\noindent for various sets $X$, such as the set $\Z$ of integers and the set of
all Plutus Core terms.  The encoding function $\E_X$ takes a sequence $s \in
\Bits$ and an element $x \in X$ and produces a new sequence of bits by appending
the encoding of $x$ to $s$, and the decoding function $\D_X$ takes a sequence of
bits, decodes some initial prefix of $s$ to a value $x \in X$, and returns the
remainder of $s$ together with $x$.


Encoding functions basically operate by decomposing an object into subobjects
and concatenating the encodings of the subobject; however it is sometimes
necessary to add some padding between subobjects in order to make sure that
parts of the output are aligned on byte boundaries, and for this reason (unlike
the CBOR encoding for \texttt{data}) all of our encoding functions have a first
argument containing all of the previous output, so that it can be examined to
determine how much alignment is required.

As in the case of CBOR, decoding functions are partial: they can fail if, for
instance, there is insufficient input, or if a decoded value is outside some
specified range.  To simplify notation we will mention any preconditions
separately, with the assumption that the decoder will fail if the preconditions
are not met; we also make a blanket assumption that all decoders fail if there
is not enough input for them to proceed.  Many of the definitions of decoders
construct objects by calling other decoders to obtain subobjects which are then
composed, and these are often introduced by a condition of the form ``if
$\D_X(s) = x$''.  Conditions like this should be read as implicitly saying that
if the decoder $\D_X$ fails then the whole decoding process fails.

\subsubsection{Padding}
The encoding functions mentioned above produce sequences of \textit{bits}, but
we sometimes need sequences of \textit{bytes}.  To this end we introduce a
functions $\pad: \Bits \rightarrow \Bits$ which adds a sequence of $\bits{0}$s
followed by a $\bits{1}$ to a sequence $s$ to get a sequence whose length is a
multiple of 8; if $s$ is a sequence such that $\length(s)$ is already a multiple of 8
then $\pad$ still adds an extra byte of padding; $\pad$ is used both for
internal alignment (for example, to make sure that the contents of a bytestring
are aligned on byte boundaries) and at the end of a complete encoding of a
Plutus Core program to to make the length a multiple of 8 bits.
Symbolically, 
$$
\pad(s)  = s \cdot \pp{k} \quad \text{if $\length(s) = 8n+k$ with $n,k \in \N$ and $0 \leq k \leq 7$}
$$
where
\begin{align*}
 \pp{0} &= \bits{00000001} \\
 \pp{1} &= \bits{0000001}  \\
 \pp{2} &= \bits{000001}   \\
 \pp{3} &= \bits{00001}    \\
 \pp{4} &= \bits{0001}     \\
 \pp{5} &= \bits{001}      \\
 \pp{6} &= \bits{01}       \\
 \pp{7} &= \bits{1}.
 \end{align*}

\noindent We also define a (partial) inverse function $\unpad: \Bits \rightharpoonup
\Bits$ which discards padding:
$$
  \unpad(q \cdot s) = s \quad \text{if $q = \pp{i}$ for some $i \in \{0,1,2,3,4,5,6,7\} $}.
$$

\noindent This can fail if the padding is not of the expected form or if the input is
the empty sequence $\epsilon$.

\subsection{Basic \texttt{flat} encodings}
\label{sec:basic-flat-encodings}
\subsubsection{Fixed-width natural numbers}
We often wish to encode and decode natural numbers which fit into some fixed
number of bits, and we do this simply by encoding them as their binary expansion
(most significant bit first), adding leading zeros if necessary.  More precisely
for $n \geq 1$ we define an encoder
$$
\E_n : \Bits \times \Nab{0}{2^{n-1}-1} \rightarrow \Bits
$$
by
$$
\E_n(s, \sum^{n-1}_{i=0}b_i2^i) = s \cdot b_{n-1} \cdots b_0 \quad \text{($b_i \in \{0,1\}$)}
$$
and a decoder
$$
\D_n : \Bits \rightharpoonup \Bits \times \Nab{0}{2^{n-1}-1}
$$
by
$$
\D_n(b_{n-1}\cdots{b_0} \cdot s)= (s,\sum^{n-1}_{i=0}b_i2^i).
$$ As in Appendix~\ref{appendix:data-cbor-encoding}, $\Nab{a}{b}$
denotes the closed interval of integers $\{n \in \Z : a \leq n \leq b\}$.  Note
that $n$ here is a variable (not a fixed label) so we are defining whole
families of encoders $\E_1, \E_2, \E_3, \ldots$ and and decoders $\D_1, \D_2,
\D_3\ldots$.


\subsubsection{Lists}
Suppose that we have a set $X$ for which we have defined an encoder $\E_X$ and a
decoder $\D_X$; we define an encoder $\Elist_X$ which encodes lists of elements
of $X$ by emitting the encodings of the elements of the list, each preceded by a
$\bits{1}$ bit, then emitting a $\bits{0}$ bit to mark the end of the list.
\begin{align*}
  \Elist_X(s,[]) &= s \cdot \bits{0} \\
  \Elist_X(s,[x_1, \ldots, x_n]) &= \Elist_X (s \cdot \bits{1} \cdot \E_X(x_1), [x_2, \ldots, x_n]).
\end{align*}

\noindent The corresponding decoder is given by
\begin{align*}
\Dlist_X(\bits{0} \cdot s) &= (s,[])\\
\Dlist_X(\bits{1} \cdot s) &= (s'', x \cdot l) \quad \text{if $D_X(s) = (s', x)$ and $\Dlist_X(s') = (s'', l).$}
\end{align*}

\subsubsection{Natural numbers}
We encode natural numbers by splitting their binary representations into
sequences of 7-bit blocks, then emitting these as a list with the \textbf{least
  significant block first}:

$$
\E_{\N} (s, \sum_{i=0}^{n-1}k_i2^{7i}) = \Elist_7(s, [k_0, \ldots, k_{n-1}])
$$
\noindent(where $k_i \in \Z$ and $0 \leq k_i \leq 127$).
\noindent The decoder is
$$
\D_{\N}(s) = (s', \sum_{i=0}^{n-1}k_i2^{7i}) \quad \text{if $\Dlist_7(s) = (s', [k_0, \ldots, k_{n-1}])$}.
$$

\subsubsection{Integers}
Signed integers are encoded by converting them to natural numbers using the
zigzag encoding ($0 \mapsto 0, -1 \mapsto 1, 1 \mapsto 2, -2 \mapsto 3, 2
\mapsto 4, \ldots$) and then encoding the result using $\E_{\N}$:
$$
\E_{\Z} (s, n) =
\begin{cases}
  \E_{\N}(s, 2n) & \text{if $n \geq 0$}\\
  \E_{\N}(s, -2n-1) & \text{if $n < 0$}.
\end{cases}
$$
The decoder is
$$
\D_{\Z}(s) =
\begin{cases}
  (s', \frac{n}{2}) & \text{if $n \equiv 0 \pmod 2$}\\
  (s', -\frac{n+1}{2}) & \text{if $n \equiv 1 \pmod 2$}
\end{cases} \quad\text{if $\D_{\N}(s) = (s', n)$}.
$$

\subsubsection{Bytestrings}  Bytestrings are encoded by dividing them into
nonempty blocks of up to 255 bytes and emitting each block in sequence.  Each
block is preceded by a single unsigned byte containing its length, and the end
of the encoding is marked by a zero-length block (so the empty bytestring is
encoded just as a zero-length block).  Before emitting a bytestring, the
preceding output is padded so that its length (in bits) is a multiple of 8; if
this is already the case a single padding byte is still added; this ensures that
contents of the bytestring are aligned to byte boundaries in the output.

Recall that $\B$ denotes the set of 8-bit bytes, $\{0,1, \ldots, 255\}$. For
specification purposes we may identify the set of bytestrings with the set
$\B^*$ of (possibly empty) lists of elements of $\B$.  We denote by $C$ the set
of \textit{bytestring chunks} of \textbf{nonempty} bytestrings of length at most
255: $C = \{[b_1, \ldots, b_n]: b_i \in \B, 1 \leq n \leq 255\}$, and define a
function $E_C: C \rightarrow \Bits$ by
$$
E_C ([b_1, \ldots, b_n]) = \E_8(n) \cdot \E_8(b_1) \cdot \cdots \cdot \E_8(b_n).
$$

\noindent
We define an encoder $\E_{C^*}$ for lists of chunks by 
$$
\E_{C^*} (s, [c_1, \ldots, c_n]) = s \cdot E_C(c_1) \cdot \cdots \cdot E_C(c_n) \cdot \bits{00000000}.
$$
\noindent Note that each $c_i$ is required to be nonempty but that we allow the
case $n = 0$, so that an empty list of chunks encodes as $\bits{00000000}$.

\medskip
\noindent To encode a bytestring we decompose it into a list $L$ of chunks and
then apply $\E_{C^*}$ to $L$.  However, there will usually be many ways to
decompose a given bytestring $a$ into chunks. For definiteness we recommend (but
do not demand) that $a$ is decomposed into a sequence of chunks of length 255
possibly followed by a smaller chunk.  Formally, suppose that $a = [a_1, \ldots,
  a_{255k+r}] \in \B^*\backslash\{\epsilon\}$ where $k \geq 0$ and $0 \leq r
\leq 254$.  We define the \textit{canonical 256-byte decomposition} $\tilde{a}$ of $a$ to
be
$$
\tilde{a} = [[a_1, \ldots, a_{255}],
  [a_{256}, \ldots, a_{510}],\ldots
  [a_{255(k-1)+1}, \ldots, a_{255k}]] \in C^*
$$
\noindent if $r=0$ and
$$
\tilde{a} = [[a_1, \ldots, a_{255}],
  [a_{256}, \ldots, a_{510}],\ldots
  [a_{255(k-1)+1}, \ldots, a_{255k}], [a_{255k+1}, \ldots, a_{255k+r}]] \in C^*
$$
\noindent if $r>0$.

\smallskip
\noindent For the empty bytestring we define
$$
\tilde{\epsilon} = [].
$$

\medskip
\noindent Given all of the above, we define the canonical encoding function
$\E_{\B^*}$ for bytestrings to be
$$
\E_{\B^*}(s, a) = E_{C^*}(\pad(s), \tilde{a}).
$$
\noindent Non-canonical encodings can be obtained by replacing $\tilde{a}$ with any
other decomposition of $a$ into nonempty chunks, and the decoder below will
accept these as well.

\bigskip

\noindent To define a decoder for bytestrings we first define a decoder
$\D_{C}$ for bytestring chunks:

$$
\D_{C}(s) = \D_C^{(n)}(s',[]) \quad \text{if $\D_8(s) = (s', n)$}
$$
where
$$
\D^{(n)}_C (s, l) =
\begin{cases}
  (s, l) & \text{if $n=0$}\\
  \D^{(n-1)}_C (s',l\cdot x)  & \text{if $n > 0$ and $\D_8(s) = (s',x)$.}
\end{cases}
$$
Now we define
$$
\D_{C^*}(s) =
\begin{cases}
  (s', []) & \text{if $D_C(s) = (s', [])$}\\
  (s'', x \cdot l) & \text{if $\D_C(s) = (s', x)$ with $x \ne []$ and $\D_{C^*}(s') = (s'', l)$}.
\end{cases}
$$
\noindent The notation is slightly misleading here: $\D_{C^*}$ does not
decode to a list of bytestring chunks, but to a single bytestring.  We
could alternatively decode to a list of bytestrings and then concatenate them
later, but this would have the same overall effect.

\medskip
\noindent Finally, we define the decoder for bytestrings by
$$
\D_{\B^*} (s) = \D_{C^*}(\unpad(s)).
$$

\subsubsection{Strings}
We have defined values of the \texttt{string} type to be sequences of Unicode
characters.  As mentioned earlier we do not specify any particular internal
representation of Unicode characters, but for serialisation we use the UTF-8
representation to convert between strings and bytestrings and then use the
bytestring encoder and decoder:

$$
\E_{\U^*}(s,u) = \E_{\B^*}(s,\utfeight(u))
$$

$$
\D_{\U^*}(s) = (s', \unutfeight(a)) \quad \text{if $\D_{\B^*}(s) = (s', a)$}
$$

\noindent
where $\utfeight$ and $\unutfeight$ are the UTF8 encoding and decoding functions
mentioned in Appendix~\ref{appendix:default-builtins-alonzo}. Recall that $\unutfeight$
is partial (not all bytestrings represent valid Unicode sequences), so
$\D_{\U^*}$ may fail if the input is invalid.


\subsection{Encoding and decoding Plutus Core}

\subsubsection{Programs}
A program is encoded by encoding the three components of the version number in
sequence then encoding the body, and possibly adding some padding to ensure that
the total number of bits in the output is a multiple of 8 (and hence the output
can be viewed as a bytestring).
$$
\Eprogram(\Prog{a}{b}{c}{t}) =
\pad(\Eterm(\E_{\N}(\E_{\N}(\E_{\N}(\epsilon, a), b), c), t)).
$$

\noindent The decoding process is the inverse of the encoding process: three
natural numbers are read to obtain the version number and then the body is
decoded.  After this we discard any padding in the remaining input and check
that all of the input has been consumed.
$$
\Dprogram(s) = \Prog{a}{b}{c}{t} \quad
\begin{cases}
  \text{ if }  &\D_{\N}(s) = (s', a)\\
  \text{ and } &\D_{\N}(s') = (s'', b)\\
  \text{ and } &\D_{\N}(s'') = (s''', c)\\
  \text{ and } &\Dterm(s''') = (r, t)\\
  \text{ and } &\unpad(r) = \epsilon.
\end{cases}
$$

\noindent 

\subsubsection{Terms}
Plutus Core terms are encoded by emitting a 4-bit tag identifying the type of
the term (see Table~\ref{table:term-tags}; recall that \texttt{[]} denotes
application) then emitting the encodings for any subterms.  We currently only
use eight of the sixteen available tags: the remainder are reserved for potential
future expansion.
\begin{table}[H]
\centering
\begin{tabular}{|l|c|c|}
  \hline
  \Strut
  Term type & Binary & Decimal\\
  \hline
  \Strut
  Variable         & $\bits{0000}$  & 0 \\
  \texttt{delay}   & $\bits{0001}$  & 1 \\
  \texttt{lam}     & $\bits{0010}$ & 2 \\
  \texttt{[]}      & $\bits{0011}$  & 3 \\
  \texttt{const}   & $\bits{0100}$  & 4 \\
  \texttt{force}   & $\bits{0101}$   & 5 \\
  \texttt{error}   & $\bits{0110}$  & 6 \\
  \texttt{builtin} & $\bits{0111}$  & 7 \\
  \hline
\end{tabular}
\caption{Term tags}
\label{table:term-tags}
\end{table}


\noindent The encoder for terms is given below: it refers to other encoders (for
names, types, and constants) which will be defined later.

\begin{alignat*}{2}
&  \Eterm(s,x)                 &&= \Ename(s \cdot \bits{0000},x) \\
&  \Eterm(s, \Delay{t})        &&=\Eterm(s \cdot \bits{0001}, t) \\
&  \Eterm(s, \Lam{x}{t})       &&= \Eterm(\Ebinder(s \cdot \bits{0010}, x), t) \\
&  \Eterm(s, \Apply{t_1}{t_2}) &&= \Eterm(\Eterm(s \cdot \bits{0011}, t_1), t_2)\\
&  \Eterm(s, \Const{tn}{c})    &&= \Econstant{tn}(\Etype(s \cdot \bits{0100}, \tn), c) \\
&  \Eterm(s, \Force{t})        &&= \Eterm(s \cdot \bits{0101}, t) \\
&  \Eterm(s, \Error)           &&= s \cdot \bits{0110} \\
&  \Eterm(s, \Builtin{b})      &&= \Ebuiltin(s \cdot \bits{0111}, b).
\end{alignat*}

\noindent The decoder for terms is given below.  To simplify the definition we
use some pattern-matching syntax for inputs to decoders: for example the
argument $\bits{0101} \cdot s$ indicates that when the input is a string
beginning with $\bits{0101}$ the definition after the $=$ sign should be used
(and the remainder of the input is available in $s$ there).  If the input is not
long enough to permit the indicated decomposition then the decoder fails.  The
decoder also fails if the input begins with a prefix which is not listed; that
does not happen here, but does in some later decoders.

\begin{alignat*}{5}
  \Dterm(\bits{0000} \cdot s)  &= (s', x) &&\quad \text{if } \Dname(s) = (s', x) \\
  \Dterm(\bits{0001} \cdot s)  &= (s', \Delay{t})  &&\quad \text{if}\ \Dterm(s) = (s', t) \\
  \Dterm(\bits{0010} \cdot s)  &= (s'', \Lam{x}{t})  &&\quad \text{if}\ \Dbinder(s) = (s', x)
                                                           &&\ \text{and}\ \Dterm(s') = (s'', t) \\
  \Dterm(\bits{0011} \cdot s)  &= (s'', \Apply{t_1}{t_2}) &&\quad \text{if}\ \Dterm(s) = (s', t_1)
                                                  &&\ \text{and}\ \Dterm(s') = (s'', t_2) \\
  \Dterm(\bits{0100} \cdot s)  &= (s'', \Const{tn}{c}) &&\quad \text{if}\ \Dtype(s) = (s', \tn)
                                                           &&\ \text{and}\ \dConstant{\tn}(s') =(s'', c) \\
  \Dterm(\bits{0101} \cdot s)  &= (s', \Force{t})  &&\quad \text{if}\ \Dterm(s) = (s', t) \\
  \Dterm(\bits{0110} \cdot s)  &= (s, \Error)  && \\
  \Dterm(\bits{0111} \cdot s)  &= (s', b) &&\quad \text{if } \Dbuiltin(s) = (s', b).
\end{alignat*}

\subsubsection{Built-in types}
Constants from built-in types are essentially encoded by emitting a sequence of
4-bit tags representing the constant's type and then emitting the encoding of
the constant itself.  However the encoding of types is somewhat complex because
it has to be able to deal with type operators such as $\ty{list}$ and
$\ty{pair}$.  The tags are given in Table~\ref{table:type-tags}: they include
tags for the basic types together with a tag for a type application operator.

\begin{table}[H]
\centering
\begin{tabular}{|l|c|c|}
  \hline
  \Strut
  Type & Binary  & Decimal \\
  \hline
  $\ty{integer}$     & $\bits{0000}$ & 0 \\
  $\ty{bytestring}$  & $\bits{0001}$ & 1 \\
  $\ty{string}$      & $\bits{0010}$ & 2 \\
  $\ty{unit}$        & $\bits{0011}$ & 3 \\
  $\ty{bool}$        & $\bits{0100}$ & 4 \\
  $\ty{list}$        & $\bits{0101}$ & 5 \\
  $\ty{pair}$        & $\bits{0110}$ & 6 \\
  (type application) & $\bits{0111}$ & 7 \\
  $\ty{data}$        & $\bits{1000}$ & 8 \\
  \hline
\end{tabular}
\caption{Type tags}
\label{table:type-tags}
\end{table}


\newcommand\etype{\mathsf{e}_{\mathsf{type}}}
\newcommand\dtype{\mathsf{d}_{\mathsf{type}}}

\noindent We define auxiliary functions $\etype: \Uni \rightarrow \N^*$ and
$\dtype: \N^* \rightharpoonup \N^* \times \Uni$ ($\dtype$ is partial and $\Uni$
denotes the universe of types used in Alonzo and Vasil) by

\begin{alignat*}{2}
  &\etype(\ty{integer})      &&= [0]  \\
  &\etype(\ty{bytestring})   &&= [1]  \\
  &\etype(\ty{string})       &&= [2]  \\
  &\etype(\ty{unit})         &&= [3]  \\
  &\etype(\ty{bool})         &&= [4]  \\
  &\etype(\listOf{t})        &&= [7,5] \cdot \etype(t) \\
  &\etype(\pairOf{t_1}{t_2}) &&= [7,7,6] \cdot \etype(t_1) \cdot \etype(t_2)\\
  &\etype(\ty{data})         &&= [8].
\end{alignat*}

\begin{alignat*}{3}
 &\dtype(0 \cdot l) &&= (l, \ty{integer})    \\
 &\dtype(1 \cdot l) &&= (l, \ty{bytestring}) \\
 &\dtype(2 \cdot l) &&= (l, \ty{string}))    \\
 &\dtype(3 \cdot l) &&= (l, \ty{unit})       \\
 &\dtype(4 \cdot l) &&= (l, \ty{bool})       \\
 &\dtype([7,5] \cdot l) &&= (l', \listOf{t}) &&\quad \text{if $\dtype(l) = (l', t)$}\\
 &\dtype([7,7,6] \cdot l) &&= (l'', \pairOf{t_1}{t_2}) 
  &&\ \begin{cases}
      \text{if} & \dtype(l) = (l', t_1)\\
      \text{and} & \dtype(l') = (l'', t_2)
    \end{cases}\\
  &\dtype(8 \cdot l) &&= (l, \ty{data}).
\end{alignat*}

\noindent The encoder and decoder for types is obtained by combining $\etype$
and $\dtype$ with $\Elist_4$ and $\Dlist_4$, the encoder and decoder for lists
of four-bit integers (see Section~\ref{sec:basic-flat-encodings}).

$$
\Etype(s,t) = \Elist_4 (s, \etype(t))
$$

$$
\Dtype(s) = (s', t) \quad \text{if $\Dlist_4(s) = (s', l)$ and $\dtype(l) = ([], t)$}.
$$

\subsubsection{Constants}
Values of built-in types can mostly be encoded quite simply by using encoders
already defined.  Note that the unit value \texttt{(con unit ())} does not have
an explicit encoding: the type has only one possible value, so there is no need
to use any space to serialise it.

The $\ty{data}$ type is encoded by converting to a bytestring using the CBOR
encoder $\eData$ described in Appendix~\ref{appendix:data-cbor-encoding} and
then using $\E_{\B^*}$.  The decoding process is the opposite of this: a
bytestring is obtained using $\D_{\B^*}$ and this is then decoded from CBOR
using $\dData$ to obtain a $\ty{data}$ object.


\begin{alignat*}{2}
  & \Econstant{\ty{integer}}(s,n)                  &&= \E_{\Z}(s, n) \\
  & \Econstant{\ty{bytestring}}(s,a)               &&= \E_{\B^*}(s, a) \\
  & \Econstant{\ty{string}}(s,t)                   &&= \E_{\U^*}(s, t) \\
  & \Econstant{\ty{unit}}(s,c)                     &&= s  \\
  & \Econstant{\ty{bool}}(s, \texttt{False})       &&= s \cdot \bits{0}\\
  & \Econstant{\ty{bool}}(s, \texttt{True})        &&= s \cdot \bits{1}\\
  & \Econstant{\listOf{\tn}}(s,l)                  &&= \Elist^{\tn}_{\mathsf{constant}}(s, l) \\
  & \Econstant{\pairOf{\tn_1}{\tn_2}}(s,(c_1,c_2))  &&= \Econstant{\tn_2}(\Econstant{\tn_1}(s, c_1), c_2)\\
  & \Econstant{\ty{data}}(s,d)                     &&= \E_{\B^*}(s, \eData(d)).
\end{alignat*}

\begin{alignat*}{3}
  &\dConstant{\ty{integer}}(s)              &&= \D_{\Z}(s) \\
  &\dConstant{\ty{bytestring}}(s)           &&= \D_{\B^*}(s) \\
  &\dConstant{\ty{string}}(s)               &&= \D_{\U^*}(s) \\
  &\dConstant{\ty{unit}}(s)                 &&= s  \\
  &\dConstant{\ty{bool}}(\bits{0} \cdot s)  &&= (s, \texttt{False}) \\
  &\dConstant{\ty{bool}}(\bits{1} \cdot s)  &&= (s, \texttt{True}) \\
  &\dConstant{\listOf{\tn}}(s)              &&= \Dlist^{\tn}_{\mathsf{constant}}(s, l) \\
  &\dConstant{\pairOf{\tn_1}{\tn_2}}(s)     &&= (s'', (c_1, c_2)) 
  && \begin{cases}
       \text{if}  & \dConstant{\tn_1}(s) = (s', c_1) \\
       \text{and} & \dConstant{\tn_2}(s') = (s'', c_2)
     \end{cases}\\
  &\dConstant{\ty{data}}(s)                  &&= (s', d) &&
                                           \text{if $\D_{\B*}(s) = (s', t)$
                                            and $\dData(t) = (t', d)$ for some $t'$}. 
\end{alignat*}

\subsubsection{Built-in functions}
Built-in functions are represented by seven-bit integer tags and encoded and
decoded using $\E_7$ and $\D_7$.  The tags are specified in
Tables~\ref{table:builtin-tags-alonzo} and \ref{table:builtin-tags-vasil}.  We
assume that there are (partial) functions $\Tag$ and $\unTag$ which convert back
and forth between builtin names and their tags.

\begin{alignat*}{2}
  & \Ebuiltin(s,b) &&= \E_7(s, \Tag(b))\\
  & \Dbuiltin(s)   &&= (s', \unTag(n)) \quad \text{if $\D_7(s) = (s', n)$}.\\
\end{alignat*}

\begin{table}[H]
\centering
\begin{tabular}{|l|c|c||l|c|c|}
  \hline
  \Strut
  Builtin & Binary & Decimal & Builtin & Binary & Decimal \\
  \hline
   \TT{addInteger}               &    $\bits{0000000}$  &   0    &        \TT{ifThenElse}               &    $\bits{0011010}$  &  26 \\
   \TT{subtractInteger}          &    $\bits{0000001}$  &   1    &        \TT{chooseUnit}               &    $\bits{0011011}$  &  27 \\
   \TT{multiplyInteger}          &    $\bits{0000010}$  &   2    &        \TT{trace}                    &    $\bits{0011100}$  &  28 \\
   \TT{divideInteger}            &    $\bits{0000011}$  &   3    &        \TT{fstPair}                  &    $\bits{0011101}$  &  29 \\
   \TT{quotientInteger}          &    $\bits{0000100}$  &   4    &        \TT{sndPair}                  &    $\bits{0011110}$  &  30 \\
   \TT{remainderInteger}         &    $\bits{0000101}$  &   5    &        \TT{chooseList}               &    $\bits{0011111}$  &  31 \\
   \TT{modInteger}               &    $\bits{0000110}$  &   6    &        \TT{mkCons}                   &    $\bits{0100000}$  &  32 \\
   \TT{equalsInteger}            &    $\bits{0000111}$  &   7    &        \TT{headList}                 &    $\bits{0100001}$  &  33 \\
   \TT{lessThanInteger}          &    $\bits{0001000}$  &   8    &        \TT{tailList}                 &    $\bits{0100010}$  &  34 \\
   \TT{lessThanEqualsInteger}    &    $\bits{0001001}$  &   9    &        \TT{nullList}                 &    $\bits{0100011}$  &  35 \\
   \TT{appendByteString}         &    $\bits{0001010}$  &  10    &        \TT{chooseData}               &    $\bits{0100100}$  &  36 \\
   \TT{consByteString}           &    $\bits{0001011}$  &  11    &        \TT{constrData}               &    $\bits{0100101}$  &  37 \\
   \TT{sliceByteString}          &    $\bits{0001100}$  &  12    &        \TT{mapData}                  &    $\bits{0100110}$  &  38 \\
   \TT{lengthOfByteString}       &    $\bits{0001101}$  &  13    &        \TT{listData}                 &    $\bits{0100111}$  &  39 \\
   \TT{indexByteString}          &    $\bits{0001110}$  &  14    &        \TT{iData}                    &    $\bits{0101000}$  &  40 \\
   \TT{equalsByteString}         &    $\bits{0001111}$  &  15    &        \TT{bData}                    &    $\bits{0101001}$  &  41 \\
   \TT{lessThanByteString}       &    $\bits{0010000}$  &  16    &        \TT{unConstrData}             &    $\bits{0101010}$  &  42 \\
   \TT{lessThanEqualsByteString} &    $\bits{0010001}$  &  17    &        \TT{unMapData}                &    $\bits{0101011}$  &  43 \\
   \TT{sha2\_256}                &    $\bits{0010010}$  &  18    &        \TT{unListData}               &    $\bits{0101100}$  &  44 \\
   \TT{sha3\_256}                &    $\bits{0010011}$  &  19    &        \TT{unIData}                  &    $\bits{0101101}$  &  45 \\       
   \TT{blake2b\_256}             &    $\bits{0010100}$  &  20    &        \TT{unBData}                  &    $\bits{0101110}$  &  46 \\       
   \TT{verifyEd25519Signature}   &    $\bits{0010101}$  &  21    &        \TT{equalsData}               &    $\bits{0101111}$  &  47 \\       
   \TT{appendString}             &    $\bits{0010110}$  &  22    &        \TT{mkPairData}               &    $\bits{0110000}$  &  48 \\
   \TT{equalsString}             &    $\bits{0010111}$  &  23    &        \TT{mkNilData}                &    $\bits{0110001}$  &  49 \\
   \TT{encodeUtf8}               &    $\bits{0011000}$  &  24    &        \TT{mkNilPairData}            &    $\bits{0110010}$  &  50 \\
   \TT{decodeUtf8}               &    $\bits{0011001}$  &  25    & & & \\
\hline
\end{tabular}
\caption{Tags for Alonzo builtins}
\label{table:builtin-tags-alonzo}
\end{table}

\begin{table}[H]
\centering
\begin{tabular}{|l|c|c|}
  \hline
  \Strut
  Builtin & Binary & Decimal\\
  \hline
 \TT{serialiseData}                   & $\bits{0110011}$  & 51  \\
 \TT{verifyEcdsaSecp256k1Signature}   & $\bits{0110100}$  & 52   \\
 \TT{verifySchnorrSecp256k1Signature} & $\bits{0110101}$  & 53   \\
\hline
\end{tabular}
\caption{Extra tags for Vasil builtins}
\label{table:builtin-tags-vasil}
\end{table}

\subsubsection{Variable names}
Variable names are encoded and decoded using the $\Ename$ and $\Dname$
functions, and variables bound in \texttt{lam} expressions are encoded and
decoded by the $\Ebinder$ and $\Dbinder$ functions.  

\paragraph{De Bruijn indices.}
We use serialised de Bruijn-indexed terms for script transmission because
this makes serialised scripts significantly smaller.  Recall from
Section~\ref{sec:grammar-notes} that when we want to use our syntax with de
Bruijn indices we replace names with natural numbers and the bound variable in a
\texttt{lam} expression with 0.  During serialisation the zero is ignored, and
during deserialisation no input is consumed and the index 0 is always returned:

$$
\Ebinder(s, n) = s
$$
$$
\Dbinder(s) = 0.
$$

\noindent For variables we always use indices which are greater than zero, and our
encoder and decoder for names are given by
$$
\Ename = \E_{\N}
$$
and
$$
\Dname (s) = (s', n) \quad \text{if $\D_{\N} = (s', n)$ and $n>0$}.
$$


\paragraph{Other types of name.}
One can serialise code involving other types of name by providing suitable
encoders and decoders for name.  For example, for textual names one could use
$\Ebinder = \Ename = \E_{\U^*}$ and $\Dbinder = \Dname = \D_{\U^*}$.  Depending
on the method used to represent variable names it may also be necessary to check
during deserialisation the more general requirement that variables are
well-scoped, but this problem will not arise if de Bruijn indices are used.

\subsection{Cardano-specific serialisation issues}
\label{sec:cardano-issues}
\subsubsection{Scope checking}
To execute a Plutus Core program on the blockchain it will be necessary to
deserialise it to some in-memory representation, and during or immediately after
deserialisation it should be checked that the body of the program is a closed
term (see the requirement in Section~\ref{sec:grammar-notes}); if this is not
the case then evaluation should fail immediately.

\subsubsection{CBOR wrapping}
Plutus Core programs are not stored on the Cardano chain directly as
\texttt{flat} bytestrings; for consistency with other objects used on the chain,
the \texttt{flat} bytestrings are in fact wrapped in a CBOR encoding.  This
should not concern most users, but we mention it here to avoid possible
confusion.


%% We could accept bytestrings split into blocks where the initial ones are all
%% exactly 64 bytes long: then the data would be interrupted by 0x40 ('@') every
%% 64 bytes.

\subsection{Example}
Consider the program
\begin{verbatim}
(program 5.0.2
 [
  [(builtin indexByteString)(con bytestring #1a5f783625ee8c)]
  (con integer 54321)
 ])
\end{verbatim}

\noindent Suppose this is stored in \texttt{index.uplc}.  We can convert it to
\texttt{flat} by running
\begin{verbatim}
$ cabal run exec uplc convert -- -i index.uplc --of flat -o index.flat
\end{verbatim}

\noindent The serialised program looks like this:

{\small
\begin{verbatim}
$ xxd -b index.flat
00000000: 00000101 00000000 00000010 00110011 01110001 11001001  ...3q.
00000006: 00010001 00000111 00011010 01011111 01111000 00110110  ..._x6
0000000c: 00100101 11101110 10001100 00000000 01001000 00111000  %...H8
00000012: 10110100 00000001 10000001
\end{verbatim}
}

\noindent Figure~\ref{fig:index-bytestring-example} shows how this encodes the
original program.  Sequences of bits are followed by explanatory comments and
lines beginning with \texttt{\#} provide further commentary on preceding bit
sequences.

\newcommand{\arrow}{$\rightarrow$}

% The \small below is just to make everything fit onto one page.  We could maybe
% remove it if the page breaks change.
\begin{figure}[H]
  \centering
  {\small  
        \begin{Verbatim}[commandchars=\\\{\}]
             00000101 : \textrm{Final integer chunk: \texttt{0000101} \arrow 5}
             00000000 : \textrm{Final integer chunk: \texttt{0000000} \arrow 0}
             00000010 : \textrm{Final integer chunk: \texttt{0000000} \arrow 2}
                      \# \textrm{Version: 5.0.2}
             0011     : \textrm{Term tag 3: apply}
             0011     : \textrm{Term tag 3: apply}
             0111     : \textrm{Term tag 7: builtin}
             0001110  : \textrm{Builtin tag 14}
                      \# builtin indexByteString
             0100     : \textrm{Term tag 4: constant}
             1        : \textrm{Start of type tag list}
             0001     : \textrm{Type tag 1}
             0        : \textrm{End of list}
                      \# \textrm{Type tags: [1] \arrow \texttt{bytestring}}
             001      : \textrm{Padding before bytestring}
             00000111 : \textrm{Bytestring chunk size: 7}
             00011010 : 0x1a
             01011111 : 0x5f
             01111000 : 0x78
             00110110 : 0x36
             00100101 : 0x25
             11101110 : 0xee
             10001100 : 0x8c
             00000000 : \textrm{Bytestring chunk size: 0 (end of list of chunks)}
                      \# con bytestring \#1a5f783625ee8c
             0100     : \textrm{Term tag 4: constant}
             1        : \textrm{Start of type tag list}
             0000     : \textrm{Type tag 0}
             0        : \textrm{End of list}
                      \# \textrm{Type tags: [0] \arrow \texttt{integer}}
             11100010 : \textrm{Integer chunk \texttt{1100010} (least significant)}
             11010000 : \textrm{Integer chunk \texttt{1010000}}
             00000110 : \textrm{Final integer chunk \texttt{0000110} (most significant)}
                      \# 0000110 \(\cons\) 1010000 \(\cons\) 1100010 \textrm{\arrow 108642 decimal}
                      \# \textrm{Zigzag encoding: 108642/2 \arrow +54321}
                      \# con integer 54321
             000001   : \textrm{Padding}
        \end{Verbatim}
        }
      \caption{\texttt{flat} encoding of \texttt{index.uplc}}
      \label{fig:index-bytestring-example}
  \end{figure}


\end{appendices}

\newpage
% \addcontentsline{toc}{section}{Bibliography}
\bibliographystyle{plain} %% ... or whatever
\bibliography{plutus-core-specification}

\newpage
\printnomenclature[2cm]  % Adjust horizontal space between notation and description

\end{document}
