\section{Introduction}
\label{sec:introduction}
Plutus Core (more correctly, Untyped Plutus Core) is an eagerly-evaluated
version of the untyped lambda calculus extended with some ``built-in'' types and
functions; it is intended for the implementation of validation scripts on the
Cardano blockchain.  This document presents the syntax and semantics of Plutus
Core, a specification of an efficient evaluator, a description of the built-in
types and functions available in various releases of Cardano, and
a specification of the binary serialisation format used by Plutus Core.

Since Plutus Core is intended for use in an environment where
computation is potentially expensive and excessively long computations can be
problematic we have also developed a costing infrastructure for Plutus Core
programs. A description of this will be added in a later version of this
document.

We also have a typed version of Plutus Core which provides extra robustness when
untyped Plutus Core is used as a compilation target, and we will eventually
provide a specification of the type system and semantics of Typed Plutus Core
here as well, together with its relationship to Untyped Plutus Core.

\subsection{Terminology and version numbers}
\label{sec:version-numbers}
Once a Plutus Core script has been deployed to the blockchain, it may be
re-evaluated at any point in the future when the history of the chain is
verified.  Plutus Core may evolve as time progresses, and because of this we
have to be quite careful with versioning.  A detailed discussion of this issue
can be found in~\cite{CIP-35}, but we reproduce two definitions here for
convenience.
\begin{itemize}
\item The language itself may evolve: for example some fundamental new construct
  may be added.  A \textit{Plutus Core language version} describes a version of
  the basic language with a particular set of features. A language version
  consists of three non-negative integers separated by decimal points, for
  example \texttt{1.4.2}.  Language versions are ordered lexicographically.
\item The built-in types and functions deployed on the chain along with a
  particular language version can change with time, and we refer to Plutus Core
  along with a particular set of types and functions as a \textit{Plutus Core
    ledger language version}.  These are referred to using version names like
  ``Plutus V1,'', ``Plutus V2'', and so on.
  \end{itemize}
 \noindent These two kinds of version are independent: all Plutus Core language
 versions must support all ledger language versions.

\subsection{Cardano support for different Plutus Core language and ledger language versions}

Cardano introduced support for Plutus Core in the Alonzo release, and
Table~\ref{table:versions} shows later releases which have introduced new
language versions and ledger language versions.  The features provided by
these versions are described later in this document.
 
\begin{table}[H]
  \centering
    \begin{tabular}{|l|l|l|l|}
        \hline
        Cardano release & Date & Language versions & Ledger language versions \\
        \hline
        Alonzo & September 2021 & 1.0.0 & V1 \\
        Vasil & June 2022 & 1.0.0 & V1, V2 \\
        (Forthcoming) & TBA & 1.0.0, 1.1.0 & V1, V2, V3 \\
        \hline
    \end{tabular}
    \caption{Plutus Core language versions and ledger language versions}
    \label{table:versions}
\end{table}
   


