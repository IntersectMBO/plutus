\section{Introduction}
\label{sec:introduction}
Plutus Core (more correctly, Untyped Plutus Core) is an eagerly-evaluated
version of the untyped lambda calculus extended with some ``built-in'' types and
functions; it is intended for the implementation of validation scripts on the
Cardano blockchain.  This document presents the syntax and semantics of Plutus
Core, a specification of an efficient evaluator, a description of the built-in
types and functions available in various releases of Cardano, and
a specification of the binary serialisation format used by Plutus Core.

Since Plutus Core is intended for use in an environment where
computation is potentially expensive and excessively long computations can be
problematic we have also developed a costing infrastructure for Plutus Core
programs. A description of this will be added in a later version of this
document.

We also have a typed version of Plutus Core which provides extra robustness when
untyped Plutus Core is used as a compilation target, and we will eventually
provide a specification of the type system and semantics of Typed Plutus Core
here as well, together with its relationship to Untyped Plutus Core.

\subsection{Version numbers and the Cardano blockchain}
\label{sec:version-numbers}
After a Plutus Core script has been deployed to the blockchain it may be
re-evaluated at any point in the future when the history of the chain is
verified.  Plutus Core may evolve as time progresses, and because of this we
have to be quite careful with versioning: a detailed discussion of this issue
can be found in~\cite{CIP-35}.  There are two types of version which are
relevant in this document:

\begin{itemize}
\item The language itself may evolve: for example some fundamental new construct
  may be added.  A \textit{Plutus Core language version} describes a version of
  the basic language with a particular set of features. A language version
  consists of three non-negative integers separated by decimal points, for
  example \texttt{1.4.2}.  Language versions are ordered lexicographically.
\item The built-in types and functions deployed on the chain along with a
  particular version of the Plutus Core language can change with time.  Plutus
  Core scripts on the chain are equipped with a tag called a \textit{ledger
    language version} which (among other things) determines which set of types
  and functions are to be used during evaluation of the script.  Ledger language
  versions have names names like ``Plutus V1'', ``Plutus V2'', and so on.  The
  semantics of a given built-in function may be different in different ledger
  language versions.  Each Plutus Core ledger language version is only valid
  with a specific Plutus Core language version, but a given language version may
  support multiple ledger language versions.
\end{itemize}

\subsection{Cardano support for different Plutus Core language and ledger language versions}

Cardano introduced support for Plutus Core in the Alonzo release;
Table~\ref{table:versions} shows all releases that have introduced new language
versions and/or new ledger language versions since then.  The features provided
by these versions are described later in this document.
\begin{table}[H]
  \centering
    \begin{tabular}{|l|l|l|l|}
        \hline
        Cardano release & Date & Language versions & Ledger language versions \\
        \hline
        Alonzo & September 2021 & 1.0.0 & V1 \\
        Vasil & June 2022 & 1.0.0 & V1, V2 \\
        (Forthcoming) & TBA & 1.0.0, 1.1.0 & V1, V2, V3 \\
        \hline
    \end{tabular}
    \caption{Plutus Core language versions and ledger language versions}
    \label{table:versions}
\end{table}
   

\noindent Table~\ref{table:version-dependencies} shows which ledger language
versions are compatible with each Plutus Core language version: if an on-chain
script specifies a combination which does not appear in this table then
attempting to evaluate it will cause an error.
 
%% See PlutusLedgerApi.Common.Versions.plcVersionsIntroducedIn
\begin{table}[H]
  \centering
    \begin{tabular}{|l|l|}
        \hline
        \thead{Plutus Core \\ language version} & \thead{Supported ledger \\ language versions}\\
        \hline
        1.0.0 & V1, V2\\
        1.0.1 & V3 \\
        \hline
    \end{tabular}
    \caption{Compatibility of Plutus Core language versions and ledger language versions}
    \label{table:version-dependencies}
\end{table}

