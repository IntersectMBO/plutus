\section{Some basic notation}
\label{sec:notation}
We begin with some notation which will be used throughout the document.

\subsection{Sets}
\label{sec:notation-sets}
\begin{itemize}
    
  \item $\N = \{0,1,2,3,\ldots\}$.%
    \nomenclature[Aaa1]{$\N$}{$\{0,1,2,3,\ldots\}$}

  \item $\Nplus = \{1,2,3,\ldots\}$.%
    \nomenclature[Aaa2]{$\Nplus$}{$\{1,2,3,\ldots\}$}

  \item The symbol $\disj$ denotes a disjoint union of sets;  for emphasis we often use this
    to denote the union of sets which we know to be disjoint.%
    \nomenclature[Aa1]{$\disj$}{Disjoint union of sets}%

  \item Given a set $X$, $X^*$ denotes the set of finite sequences of elements of $X$:
    $$
    X^* = \bigdisj{\{X^n: n \in \N\}}
    $$%
    and $X^+$ denotes the set of nonempty finite sequences of elements of $X$:
    $$
    X^+ = \bigdisj{\{X^n: n \in \Nplus\}}.
    $$
    We will sometimes write elements of $X^+$ in the form $(x|x_1,\ldots,x_n)$ with $n \geq 0$.
    \nomenclature[Aa2]{$X^*$}{The set of all finite sequences of elements of a set $X$}%
    \nomenclature[Aa3]{$X^+$}{The set of all nonempty finite sequences of elements of a set $X$}%
    \nomenclature[Aa4]{$(x \vert x_1,\ldots,x_n)$}{A member of $X^+$}% nomencl doesn't like |
 
 \item $\Nab{a}{b} = \{n \in \N: a \leq n \leq b\}$.%
    \nomenclature[Aaa3]{$\Nab{a}{b}$}{$\{n \in \N: a \leq n \leq b\}$}

  \item $\B = \Nab{0}{255}$, the set of 8-bit bytes.%
    \nomenclature[Aaa4]{$\B$}{$\{n \in \Z: 0 \leq n \leq 255\}$}%

  \item $\B^*$ is the set of all bytestrings.%
    \nomenclature[Ab2]{$\B^*$}{The set of all bytestrings}

  \item $\b = \{\mathtt{0}, \mathtt{1}\}$, the set of bits.%
    \nomenclature[Ac1]{$\b$}{$\{\mathtt{0}, \mathtt{1}\}$}%

  \item $\b^*$ is the set of all bitstrings.%
    \nomenclature[Ac2]{$\b^*$}{The set of all bitstrings}

  \item $\Z = \{\ldots, -2, -1, 0, 1, 2, \ldots\}$.%
    \nomenclature[Aw]{$\Z$}{$\{\ldots, -2, -1, 0, 1, 2, \ldots\}$}

  \item $\mathbb{F}_q$ denotes a finite field with $q$ elements ($q$ a prime power).%
    \nomenclature[Af]{$\mathbb{F}_q$}{The finite field with $q$ elements}

  \item $\units{\mathbb{F}_q}$ denotes the multiplicative group of nonzero elements of $\mathbb{F}_q$.%
    \nomenclature[Af]{$\units{\mathbb{F}_q}$}{The multiplicative group of $\mathbb{F}_q$}%

  \item $\U$ denotes the set of Unicode scalar values, as defined in~\cite[Definition D76]{Unicode-standard}.%
    \nomenclature[Au1]{$\U$}{The set of Unicode scalar values}%

  \item $\U^*$ is the set of all Unicode strings.%
    \nomenclature[Au2]{$\U^*$}{The set of Unicode strings}%

  \item We assume that there is a special symbol $\errorX$ which does not appear
    in any other set we mention.  The symbol $\errorX$ is used to indicate that
    some sort of error condition has occurred, and we will often need to consider
    situations in which a value is either $\errorX$ or a member of some set $S$.
    For brevity, if $S$ is a set then we define
    $$
    \withError{S} := S \disj \{\errorX\}.
    $$%
    \nomenclature[Ax]{$\withError{S}$}{$S \disj \{\errorX\}$ ($S$ a set)}%
\end{itemize}%

\subsection{Lists}
\label{sec:notation-lists}
\begin{itemize}
\item  The symbol $[]$ denotes an empty list.%
  \nomenclature[B1]{$[]$}{The empty list}%

\item The notation $[x_m, \ldots, x_n]$ denotes a list containing the elements
  $x_m, \ldots, x_n$.  If $m>n$ then the list is empty.%

\item The length of a list $L$ is denoted by $\length(L)$.%
\nomenclature[B6]{$\length(\cdot)$}{Length of a list or bytestring}

\item Given two lists $L = [x_1,\ldots, x_m]$ and $L' = [y_1,\ldots, y_n]$, $L\cdot L'$
  denotes their concatenation  $[x_1,\ldots, x_m,$ $y_1, \ldots, y_n]$.  % Broken in the middle to keep it out of the margin.%
  \nomenclature[B2]{$L \cdot L'$}{Concatenation of lists $L$ and $L'$}%

\item Given an object $x$ and a list $L = [x_1,\ldots, x_n]$,
  we denote the list $[x,x_1,\ldots, x_n]$ by $x \cons L$.%
  \nomenclature[B3]{$x \cdot L$}{$[x] \cdot L$}%

\item Given a list $L = [x_1, \ldots, x_n]$ and an object $x$,
  we denote the list $[x_1, \ldots, x_n, x]$ by $L \snoc x$.%
  \nomenclature[B4]{$L \cdot x$}{$L \cdot [x]$}%

%%\item We say that the list $L'$ is a \textit{proper prefix} of the list
%%  $L = [x_1, \ldots, x_n]$, and
%%  write $L' \prec L$,  if $L' = [x_1, \ldots, x_m]$ for some $m<n$.

\item Given a syntactic category $V$, the symbol $\repetition{V}$ denotes a
  possibly empty list $[V_1,\ldots, V_n]$ of elements $V_i \in V$.%
  \nomenclature[B5]{$\repetition{V}$}{A sequence $[V_1,\ldots, V_n]$}%
\end{itemize}

\subsection{Bytestrings and bitstrings}
\label{sec:notation-bytestrings}

We make frequent use of bytestrings and bitstrings and for the sake of
conciseness we occasionally use special notation.  We also define conversion
functions between bytestrings and bitstrings

\begin{itemize}
\item We typically index the bytes in bytestrings starting from the
  \textit{left} end but the bits in bitstrings from the \textit{right} end.

\item The bytestring $[c_0, \ldots, c_{n-1}]$ may be denoted by
  $c_0{\cdots}c_{n-1}$ ($n \geq 0$, $c_i \in \B$); the empty bytestring may be
  denoted by $\epsilon$.

\item The  bitstring $[b_{n-1}, \ldots, b_0]$ may be denoted by
  $b_{n-1}{\cdots}b_0$ ($n \geq 0$, $b_i \in \b$); the empty bitstring may be
  denoted by $\epsilon$: we also use this symbol for the empty bytestring, but
  this should not cause any confusion.

\item In the special case of bitstrings sometimes use notation such as
  \texttt{101110} to denote the list $[1,0,1,1,1,0]$; we use a teletype font to
  avoid confusion with decimal numbers.

\item A bytestring can naturally be viewed as a bitstring whose length is a
  multiple of 8 simply by concatenating the bits of the individual bytes, and
  vice-versa (by breaking the bitstring into groups of 8 bits).  To make this
  precise we define two conversion functions $\bitsof: \B^* \rightarrow \b^*$
  and $\bytesof: \{s \in \b^* : 8 \mid \length(s)\} \rightarrow \B^*$.  These
  depend on the fact that any $c \in \B$ can be written uniquely in the form
  $\Sigma_{i=0}^72^ib_i$ with $b_0, \ldots, b_7 \in \b$.
  \begin{itemize}
    \item $\bitsof([c_0, \ldots, c_{n-1}]) = [b_{8n-1}, \ldots, b_0]$ where $c_j=\Sigma_{i=0}^72^ib_{8(n-j-1)+i}$
    \item $\bytesof([b_{8n-1}, \ldots, b_0]) = [c_0, \ldots, c_{n-1}]$ where $c_j=\Sigma_{i=0}^72^ib_{8(n-j-1)+i}$.
  \end{itemize}
\end{itemize}

\subsection{Miscellaneous notation}
\begin{itemize}
\item Given integers $k \in \Z$ and $n \geq 1$ we write $k \bmod n = \min\{r \in \Z: r \geq 0 \text{ and } n | k - r \}$
\end{itemize}


