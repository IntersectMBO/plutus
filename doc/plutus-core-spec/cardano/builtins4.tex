% I tried resetting the note number from V1-builtins here, but that made
% some hyperlinks wrong.  To get note numbers starting at one in each section, I
% think we have to define a new counter every time.
\newcounter{notenumberD}
\renewcommand{\note}[1]{
  \bigskip
  \refstepcounter{notenumberD}
  \noindent\textbf{Note \thenotenumberD. #1}
}

\subsection{Batch 4}
\label{sec:default-builtins-4}
The fourth batch of builtins adds support for
\begin{itemize}
\item The \texttt{Blake2b-224} and \texttt{Keccak-256} hash functions (Section \ref{sec:misc-builtins-4})
\item BLS12-381 elliptic curve pairing operations
(see~\cite{CIP-0381}, \cite{BLS12-381}, \cite[4.2.1]{IETF-pairing-friendly-curves}, \cite{BLST-library}).
 For clarity these are described separately in Sections~\ref{sec:bls-types-4} and \ref{sec:bls-builtins-4}.
\end{itemize}

\subsubsection{Miscellaneous built-in functions}
\label{sec:misc-builtins-4}

\setlength{\LTleft}{-10mm} % Shift the table left a bit to centre it on the page
\begin{longtable}[H]{|l|p{5cm}|p{55mm}|c|c|}
    \hline \text{Function} & \text{Signature} & \text{Denotation} & \text{Can}
    & \text{Note} \\ & & & fail?
    & \\ \hline \endfirsthead \hline \text{Function} & \text{Type}
    % This caption goes on every page of the table except the last.  Ideally it
    % would appear only on the first page and all the rest would say
    % (continued). Unfortunately it doesn't seem to be easy to do that in a
    % longtable.
    \endfoot
%%    \caption[]{Built-in Functions}
    \caption[]{Miscellaneous built-in Functions}
    \label{table:misc-built-in-functions-4}
    \endlastfoot
%% G1
    \hline
    \TT{blake2b\_224} & $[\ty{bytestring}] \to \ty{bytestring}$  & \text{Hash a $\ty{bytestring}$ using}
                                                                   \TT{Blake2b-224}~\cite{IETF-Blake2}. &  & \\
    \TT{keccak\_256}  & $[\ty{bytestring}] \to \ty{bytestring}$  & \text{Hash a $\ty{bytestring}$ using}
                                                                   \TT{Keccak-256}~\cite{KeccakRef3}. &  & \\
    \hline
\end{longtable}

\subsubsection{BLS12-381 built-in types}
\label{sec:bls-types-4}

\noindent Supporting the BLS12-381 operations involves adding three new types
and seventeen new built-in functions.  The description of the semantics of these
types and functions is quite complex and requires a considerable amount of
notation, most of which is used only in Sections~\ref{sec:bls-types-4} and~\ref{sec:bls-builtins-4}.

\bigskip
\noindent Table~\ref{table:built-in-types-4} describes three new built-in
types.

\newcommand{\TyMlResult}{\ty{bls12\_381\_mlresult}}
\newcommand{\MlResultDenotation}{H}
\newcommand{\Fq}{\mathbb{F}_q}
\newcommand{\Fqq}{\mathbb{F}_{q^2}}
\newcommand{\FF}{\mathbb{F}_{q^{12}}}

\begin{table}[H]
  \centering
    \begin{tabular}{|l|p{2cm}|l|}
        \hline
        Type & Denotation & Concrete Syntax\\
        \hline
        $\ty{bls12\_381\_G1\_element}$ &   $G_1$ & \texttt{0x[0-9A-Fa-f]\{96\}} \text{(see Note~\ref{note:bls-syntax})}\\
        $\ty{bls12\_381\_G2\_element}$ &   $G_2$ & \texttt{0x[0-9A-Fa-f]\{192\}} \text{(see Note~\ref{note:bls-syntax})}\\
        $\TyMlResult$    &   $\MlResultDenotation$  &  None (see Note~\ref{note:bls-syntax})\\
        \hline
    \end{tabular}
    \caption{Atomic Types}
    \label{table:built-in-types-4}
\end{table}

%% \paragraph{$G_1$ and $G_2$}.
\noindent Here $G_1$ and  $G_2$ are both additive cyclic groups of prime order $r$, where 
$$
r = \mathtt{0x73eda753299d7d483339d80809a1d80553bda402fffe5bfeffffffff00000001}.
$$
        
\paragraph{The fields $\Fq$ and $\Fqq$.}
\noindent To define the groups $G_1$ and $G_2$ we need the finite field $\Fq$ where
\begin{align*}
q = \mathtt{0x}&\mathtt{1a0111ea397fe69a4b1ba7b6434bacd764774b84f38512bf}\\
              &\mathtt{6730d2a0f6b0f6241eabfffeb153ffffb9feffffffffaaab}
\end{align*}

\noindent which is a 381-bit prime. The field $\Fq$ is isomorphic to $\Z_q$,
the ring of integers modulo $q$, and hence there is a natural epimorphism from
$\Z$ to $\Fq$ which we denote by $n \mapsto \bar{n}$.  Given $x \in \Fq$, we
denote by $\tilde{x}$ the smallest non-negative integer $n$ with $\bar{n} = x$.
We sometimes use literal integers to represent elements of $\Fq$ in the obvious
way.

We also make use of the field $\Fqq = \Fq[X]/(X^2+1)$; we may regard $\Fqq$ as
the set $\{a+bu: a,b \in \Fq\}$ where $u^2=-1$.

It is convenient to say that an element $a$ of $\Fq$ is \textit{larger} than
another element $b$ (and $b$ is \textit{smaller} than $a$) if $\tilde{a}
> \tilde{b}$ in $\Z$.  We extend this terminology to $\Fqq$ by saying that
$a+bu$ is larger than $c+du$ if either $b$ is larger than $d$ or $b=d$ and $a$
is larger than $c$.


\paragraph{The groups $G_1$ and $G_2$.}
\noindent There are elliptic curves $E_1$ defined over $\Fq$:
$$
E_1: Y^2 = X^3 + 4
$$

\noindent and $E_2$ defined over $\Fqq$:
$$
E_2: Y^2 = X^3 + 4(u+1).
$$

\noindent $E_1(\Fq)$ and  $E_2(\Fqq)$  are abelian groups under the
usual elliptic curve addition operations as described
in~\cite[III.2]{Silverman-Arithmetic-EC} or~\cite[2.1]{Costello-pairings}.
$G_1$ is a subgroup of $E_1(\Fq)$ and $G_2$ is a subgroup of $E_2(\Fqq)$;
explicit generators for $G_1$ and $G_2$ are given
in~\cite[4.2.1]{IETF-pairing-friendly-curves}.  We denote the identity element
(the point at infinity) in $G_1$ by $\mathcal{O}_{G_1}$ and that in $G_2$ by
$\mathcal{O}_{G_2}$.  Given an integer $n$ and a group element $a$ in $G_1$ or
$G_2$, the scalar multiple $na$ is defined as usual to be $a + \cdots + a$ ($n$
times) if $n>0$ and $-a + \cdots + -a$ ($-n$ times) if $n<0$; $0a$ is the
identity element of the group.

\paragraph{The \texttt{bls12\_381\_MlResult} type.}
\noindent Values of the \texttt{bls12\_381\_MlResult} type are completely
opaque and can only be obtained as a result of \texttt{bls12\_381\_millerLoop}
or by multiplying two existing elements of type \texttt{bls12\_381\_MlResult}.
We provide neither a serialisation format nor a concrete syntax for values of
this type: they exist only ephemerally during computation.  We do not specify
$\MlResultDenotation$, the denotation of $\TyMlResult$, precisely, but it
must be a multiplicative abelian group. See Note~\ref{note:pairing} for more on
this.

\subsubsection{BLS12-381 built-in functions}
\label{sec:bls-builtins-4}

\newcommand{\hash}{\mathsf{hash}}
\newcommand{\compress}{\mathsf{compress}}
\newcommand{\uncompress}{\mathsf{uncompress}}

\setlength{\LTleft}{-4mm} % Shift the table left a bit to centre it on the page
\begin{longtable}[H]{|l|p{5cm}|p{25mm}|c|c|}
    \hline \text{Function} & \text{Signature} & \text{Denotation} & \text{Can}
    & \text{Note} \\ & & & fail?
    & \\ \hline \endfirsthead \hline \text{Function} & \text{Type}
    & \text{Denotation} & \text{Can} & \text{Note}\\ & & & fail?
    & \\ \hline \endhead \hline \caption{BLS12-381 built-in Functions}
    % This caption goes on every page of the table except the last.  Ideally it
    % would appear only on the first page and all the rest would say
    % (continued). Unfortunately it doesn't seem to be easy to do that in a
    % longtable.
    \endfoot
%%    \caption[]{Built-in Functions}
    \caption[]{BLS12-381 built-in Functions (continued)}
    \label{table:built-in-functions-4}
    \endlastfoot
%% G1
    \TT{bls12\_381\_G1\_add}  &
    $[ \ty{bls12\_381\_G1\_element}$,
      \text{\; $\ty{bls12\_381\_G1\_element} ]$}
      \text{\: $ \to \ty{bls12\_381\_G1\_element}$} & $(a,b) \mapsto a+b$  &  No & \\
    \TT{bls12\_381\_G1\_neg}  &
      $ [ \ty{bls12\_381\_G1\_element} ]$  \text{\;\; $\to \ty{bls12\_381\_G1\_element}$} & $a \mapsto -a$  & No & \\
    \TT{bls12\_381\_G1\_scalarMul}  &
    $[ \ty{integer}$,
      \text{\; $\ty{bls12\_381\_G1\_element} ]$}
      \text{\: $ \to \ty{bls12\_381\_G1\_element}$} & $(n,a) \mapsto na$ &  No & \\
    \TT{bls12\_381\_G1\_equal}  &
    $[ \ty{bls12\_381\_G1\_element}$,
      \text{\; $\ty{bls12\_381\_G1\_element} ]$}
      \text{\: $ \to \ty{bool}$} & $=$ &  No & \\
    \TT{bls12\_381\_G1\_hashToGroup}  &
    $[ \ty{bytestring}, \ty{bytestring}]$
      \text{\: $ \to \ty{bls12\_381\_G1\_element}$} & $\hash_{G_1}$ &  Yes & \ref{note:hashing-into-group}\\
    \TT{bls12\_381\_G1\_compress}  &
    $[\ty{bls12\_381\_G1\_element}]$
      \text{\: $ \to \ty{bytestring}$} & $\compress_{G_1}$  &  No & \ref{note:group-compression}\\
    \TT{bls12\_381\_G1\_uncompress}  &
    $[ \ty{bytestring}]$
      \text{\: $ \to \ty{bls12\_381\_G1\_element}$} & $\uncompress_{G_1}$  &  Yes & \ref{note:group-uncompression}\\
    \hline 
%% G2
    \TT{bls12\_381\_G2\_add}  &
    $[ \ty{bls12\_381\_G2\_element}$,
      \text{\; $\ty{bls12\_381\_G2\_element} ]$}
      \text{\: $ \to \ty{bls12\_381\_G2\_element}$} & $(a,b) \mapsto a+b$ &  No & \\
    \TT{bls12\_381\_G2\_neg}  &
      $ [ \ty{bls12\_381\_G2\_element} ]$  \text{\;\; $\to \ty{bls12\_381\_G2\_element}$} & $a \mapsto -a$  & No & \\
    \TT{bls12\_381\_G2\_scalarMul}  &
    $[ \ty{integer}$,
      \text{\; $\ty{bls12\_381\_G2\_element} ]$}
      \text{\: $ \to \ty{bls12\_381\_G2\_element}$} & $(n,a) \mapsto na$ &  No & \\
    \TT{bls12\_381\_G2\_equal}  &
    $[ \ty{bls12\_381\_G2\_element}$,
      \text{\; $\ty{bls12\_381\_G2\_element} ]$}
      \text{\: $ \to \ty{bool}$} & $=$ &  No & \\
    \TT{bls12\_381\_G2\_hashToGroup}  &
    $[ \ty{bytestring}, \ty{bytestring}]$
      \text{\: $ \to \ty{bls12\_381\_G2\_element}$} & $\hash_{G_2}$  &  Yes & \ref{note:hashing-into-group}\\
    \TT{bls12\_381\_G2\_compress}  &
    $[\ty{bls12\_381\_G2\_element}]$
      \text{\: $ \to \ty{bytestring}$} & $\compress_{G_2}$  &  No & \ref{note:group-compression}\\
    \TT{bls12\_381\_G2\_uncompress}  &
    $[ \ty{bytestring}]$
      \text{\: $ \to \ty{bls12\_381\_G2\_element}$} & $\uncompress_{G_2}$  &  Yes & \ref{note:group-uncompression}\\
    \hline 
    \TT{bls12\_381\_millerLoop}  &
    $[ \ty{bls12\_381\_G1\_element}$,
      \text{\; $\ty{bls12\_381\_G2\_element} ]$}
    \text{\: $ \to \TyMlResult$} & $e$ &  No & \ref{note:pairing}\\
    \TT{bls12\_381\_mulMlResult}  &
    $[ \TyMlResult$,
    \text{\; $\TyMlResult]$}
    \text{\: $\to \TyMlResult$} & $(a,b) \mapsto ab$ & No & \ref{note:pairing}\\
    \TT{bls12\_381\_finalVerify}  &
    $[ \TyMlResult$,
    \text{\; $\TyMlResult] \to \ty{bool}$} & $\phi$ & No & \ref{note:pairing}\\
    \hline
\end{longtable}


\note{Hashing into $G_1$ and $G_2$.}
\label{note:hashing-into-group}
The denotations $\hash_{G_1}$ and $\hash_{G_2}$
of \texttt{bls12\_381\_G1\_hashToGroup} and
\texttt{bls12\_381\_G2\_hashToGroup} both take an arbitrary bytestring $b$ (the
\textit{message}) and a (possibly empty) bytestring of length at most 255 known as a \textit{domain
separation tag} (DST)~\cite[2.2.5]{IETF-hash-to-curve} and hash them to obtain a
point in $G_1$ or $G_2$ respectively.  The details of the hashing process are
described in~\cite{IETF-hash-to-curve} (see specifically Section 8.8), except
that
\textbf{we do not support DSTs of length greater than 255}: an attempt to use a
longer DST directly will cause an error.  If a longer DST is required then it
should be hashed to obtain a short DST as described
in~\cite[5.3.3]{IETF-hash-to-curve}, and then this should be supplied as the
second argument to the appropriate \texttt{hashToGroup} function.

%% Some hashing
%% implementations also allow a third argument (an ``augmentation string''), but we
%% do not support this since the same effect can be obtained by appending
%% (prepending?) the augmentation string to the message before hashing.

\newcommand{\ymin}{y_{\text{min}}}
\newcommand{\ymax}{y_{\text{max}}}

\note{Compression for elements of $G_1$ and $G_2$.} 
\label{note:group-compression}
Points in $G_1$ and $G_2$ are encoded as bytestrings in a ``compressed'' format
where only the $x$-coordinate of a point is encoded and some metadata is used to
indicate which of two possible $y$-coordinates the point has.  The encoding
format is based on the Zcash encoding for BLS12-381 points:
see~\cite{Zcash-serialisation} or~\cite[``Serialization'']{BLST-library}
or~\cite[Appendices C and D]{IETF-pairing-friendly-curves}.  In detail,

\begin{itemize}

\item Given an element $x$ of $\Fq$, $\tilde{x}$ can be written as a 381-bit
binary number: $\tilde{x} = \sum_{i=0}^{380}b_i2^i$ with $b_i \in \{0,1\}$.  We
define $\mathsf{bits}(x)$ to be the bitstring $b_{380}\cdots b_0$.

\item A non-identity element of $G_1$ can be written in the form $(x,y)$ with $x,y\in\Fq$.
Not every element $x$ of $\Fq$ is the $x$-coordinate of a point in $G_1$, but
those which are in fact occur as the $x$-coordinate of two distinct points in
$G_1$ whose $y$-coordinates are the negatives of each other.  A similar
statement is true for $\Fqq$ and $G_2$.  In both cases we denote the smaller of
the possible $y$-coordinates by $\ymin(x)$ and the larger by $\ymax(x)$.

\item For $(x,y) \in G_1\backslash \mathcal{O}_{G_1}$ we define
$$
\compress_{G_1} (x,y) = \begin{cases}
\mathsf{1}\mathsf{0}\mathsf{0}\cdot\mathsf{bits}(x) & \text{if $y=\ymin(x)$}\\
\mathsf{1}\mathsf{0}\mathsf{1}\cdot\mathsf{bits}(x) & \text{if $y=\ymax(x)$}\\
\end{cases}
$$
\item We encode the identity element of $G_1$ using
$$
\compress_{G_1}(\mathcal{O}_{G_1}) = \mathsf{1}\mathsf{1}\mathsf{0}\cdot\mathsf{0}^{381},
$$
\noindent where $\mathsf{0}^{381}$ denotes a string of 381 $\mathsf{0}$ bits.
\end{itemize}
\noindent Thus in all cases the encoding of an element of $G_1$ requires exactly 384 bits,
or 48 bytes.

\medskip 

\noindent 
\begin{itemize}
\item Similarly, every non-identity element of $G_2$ can be written
in the form $(x,y)$ with $x,y \in \Fqq$.  We define

$$
\compress_{G_2} (a+bu,y) = \begin{cases}
\mathsf{1}\mathsf{0}\mathsf{0}\cdot\mathsf{bits}(b)\cdot\mathsf{0}\mathsf{0}\mathsf{0}\cdot\mathsf{bits}(a)
& \text{if $y=\ymin(a+bu)$}\\
\mathsf{1}\mathsf{0}\mathsf{1}\cdot\mathsf{bits}(b)\cdot\mathsf{0}\mathsf{0}\mathsf{0}\cdot\mathsf{bits}(a) &
 \text{if $y=\ymax(a+bu)$}\\
\end{cases}
$$

\item The identity element of $G_2$ is encoded using
$$
\compress_{G_2}(\mathcal{O}_{G_2}) = \mathsf{1}\mathsf{1}\mathsf{0}\cdot\mathsf{0}^{765}.
$$

\end{itemize}
\noindent The encoding of an element of $G_2$ requires exactly 768 bits, or 96 bytes.

Note that in all cases the most significant bit of a compressed point is 1.  In
the Zcash serialisation scheme this indicates that the point is compressed;
Zcash also supports a serialisation format where both the $x$- and
$y$-coordinates of a point are encoded, and in that case the leading bit of the
encoded point is 0.  We do not support this format.

\note{Uncompression for elements of $G_1$ and $G_2$.} 
\label{note:group-uncompression}
There are two (partial) ``uncompression'' functions $\uncompress_{G_1}$ and
$\uncompress_{G_2}$ which convert bytestrings into group elements; these are
obtained by inverting the process described in
Note~\ref{note:group-compression}.

\paragraph{Uncompression for $G_1$ elements.}  Given a bytestring $b$, it is checked that
$b$ contains exactly 48 bytes.  If not, then $\uncompress_{G_1}(b) = \errorX$ (ie,
uncompression fails).  If the length is equal to 48 bytes, write $b$ as a
sequence of bits: $b = b_{383} \cdots b_0$.
\begin{itemize}
\item If $b_{383} \neq 1$, then $\uncompress_{G_1}(b) = \errorX$.
\item If $b_{383} = b_{382} = 1$ then
$\uncompress_{G_1}(b) =
\begin{cases}
\mathcal{O}_{G_1} & \text{if $b_{381} = b_{380} = \cdots = b_0 = 0$}\\
\errorX & \text{otherwise}.
\end{cases}$
\item If $b_{383}=1$ and $b_{382}=0$, let $c=\sum_{i=0}^{380}b_i2^i \in \N$.
\begin{itemize}
\item If $c \geq q$, $\uncompress_{G_1}(b) = \errorX$.
\item Otherwise, let $x = \bar{c} \in \Fq$ and let $z = x^3+4$. If $z$ is not a
square in $\Fq$, then $\uncompress_{G_1}(b) = \errorX$.
\item If $z$ is a square then let
$y=\begin{cases}
\ymin(x) & \text{if $b_{381}=0$}\\
\ymax(x) & \text{if $b_{381}=1$}.
\end{cases}$
\item Then $\uncompress_{G_1}(b) = \begin{cases}
(x,y) & \text{if $(x,y) \in G_1$}\\
\errorX & \text{otherwise}.
\end{cases}$
\end{itemize}
\end{itemize}


\paragraph{Uncompression for $G_2$ elements.}  Given a bytestring $b$, it is checked that
$b$ contains exactly 96 bytes.  If not, then $\uncompress_{G_2}(b) = \errorX$ (ie,
uncompression fails).  If the length is equal to 96 bytes, write $b$ as a
sequence of bits: $b = b_{767} \cdots b_0$.
\begin{itemize}
\item If $b_{767} \neq 1$, then $\uncompress_{G_2}(b) = \errorX$.
\item If $b_{767} = b_{766} = 1$ then $\uncompress_{G_2}(b) =
\begin{cases}
\mathcal{O}_{G_2} & \text{if $b_{765} = b_{764} = \cdots = b_0 = 0$}\\
\errorX & \text{otherwise}.
\end{cases}$
\item If $b_{767}=1$ and $b_{766} = 0$, let $c=\sum_{i=0}^{383}b_i2^i$ and $d=\sum_{i=384}^{764}b_i2^{i-384} \in \N$.
\begin{itemize}
\item If $c \geq q$ or $d \geq q$, $\uncompress_{G_2}(b) = \errorX$.
\item Otherwise, let $x = \bar{c}+\bar{d}u \in \Fqq$ and let $z = x^3+4(u+1)$.
If $z$ is not a square in $\Fqq$, then $\uncompress_{G_2}(b) = \errorX$.
\item If $z$ is a square then let
$y=\begin{cases}
\ymin(x) & \text{if $b_{765}=0$}\\
\ymax(x) & \text{if $b_{765}=1$}.
\end{cases}$
\item Then $\uncompress_{G_2}(b) = \begin{cases}
(x,y) & \text{if $(x,y) \in G_2$}\\
\errorX & \text{otherwise}.
\end{cases}$
\end{itemize}
\end{itemize}


\note{Concrete syntax for BLS12-381 types.}
\label{note:bls-syntax}
Concrete syntax for the $\ty{bls12\_381\_G1\_element}$ and
$\ty{bls12\_381\_G2\_element}$ types is provided via the compression and
decompression functions defined in Notes~\ref{note:group-compression}
and~\ref{note:group-uncompression}.  Specifically, a value of type
$\ty{bls12\_381\_G1\_element}$ is denoted by a term of the form \texttt{(con
bls12\_381\_G1\_element 0x...)} where \texttt{...}  consists of 96 hexadecimal
digits representing the 48-byte compressed form of the relevant point.
Similarly, a value of type $\ty{bls12\_381\_G2\_element}$ is denoted by a term
of the form \texttt{(con bls12\_381\_G2\_element 0x...)}  where \texttt{...}
consists of 192 hexadecimal digits representing the 96-byte compressed form of
the relevant point.

No syntax is provided for values of type $\ty{bls12\_381\_mlresult}$. It is not
possible to parse such values, and they will appear as \texttt{(con
bls12\_381\_mlresult <opaque>)} if output by a program.


\note{Pairing operations.}
\label{note:pairing}
For efficiency reasons we split the pairing process into two parts:
the \texttt{bls12\_381\_millerLoop} and \texttt{bls12\_381\_finalVerify}
functions.  We assume that we have
\begin{itemize}
\item An intermediate multiplicative abelian group $H$.
\item A function (not necessarily itself a pairing) $e: G_1 \times
G_2 \rightarrow \MlResultDenotation$.
\item A cyclic group $\mu_r$ of order $r$.
\item An epimorphism $\psi: \MlResultDenotation \rightarrow \mu_r$ of groups such
that $\psi \circ e: G_1 \times G_2 \rightarrow \mu_r$ is a (nondegenerate,
bilinear) pairing.
\end{itemize}

\noindent Given these ingredients, we define
\begin{itemize}
\item $\denote{\TyMlResult} = \MlResultDenotation$.
\item $\denote{\mathtt{bls12\_381\_mulMlResult}} =$
the group multiplication operation in $\MlResultDenotation$.
\item $\denote{\mathtt{bls12\_381\_millerLoop}} = e$.
\item $\denote{\mathtt{bls12\_381\_finalVerify}} = \phi$,
where
$$
\phi(a,b) = \begin{cases}
               \mathtt{true} & \text{if $\psi(ab^{-1}) = 1_{\mu_r}$} \\
               \mathtt{false} & \text{otherwise.}
            \end{cases}
$$
\end{itemize}

\medskip
\noindent
We do not mandate specific choices for $\MlResultDenotation, \mu_r, e$, and $\phi$, but a
plausible choice would be
\begin{itemize}
\item $\MlResultDenotation = \FF^*$.
\item $e$ is the Miller loop associated with the optimal Ate pairing
for $E_1$ and $E_2$~\cite{Vercauteren}.
\item $\mu_r = \{x \in \FF^*: x^r=1\}$, the group of $r$th roots of unity in $\FF^*$.
(There are $r$ distinct $r$th roots of unity in $\FF^*$ because the embedding
degree of $E_1$ and $E_2$ with respect to $r$ is 12 (see~\cite[4.1]{Costello-pairings}).)
\item $\psi(x) = x^{\frac{q-1}{r}}$.
\end{itemize}

\noindent The functions \texttt{bls12\_381\_millerLoop} and (especially)
\texttt{bls12\_381\_finalVerify} are expected
to be expensive, so their use should be kept to a minimum.  Fortunately most
current use cases do not require many uses of these functions.
