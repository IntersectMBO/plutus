% I tried resetting the note number from V1-builtins here, but that made
% some hyperlinks wrong.  To get note numbers starting at one in each section, I
% think we have to define a new counter every time.
\newcounter{notenumberE}
\renewcommand{\note}[1]{
  \bigskip
  \refstepcounter{notenumberE}
  \noindent\textbf{Note \thenotenumberE. #1}
}

\newcommand\Xand{\mathsf{and}}
\newcommand\Xor{\mathsf{or}}
\newcommand\Xxor{\mathsf{xor}}
\newcommand{\bitzero}{\mathtt{0}}
\newcommand{\bitone}{\mathtt{1}}
\newcommand{\trunc}{\mathbf{\mathsf{trunc}}}
\newcommand{\ext}{\mathbf{\textsf{ext}}}

\subsection{Batch 5}
\label{sec:default-builtins-5}

\subsubsection{Built-in functions}
\label{sec:built-in-functions-5}
The fifth batch of built-in functions adds support for
\begin{itemize}
\item Logical and bitwise operations on bytestrings (see~\cite{CIP-0122} and~\cite{CIP-0123}).
\item The \texttt{RIPEMD-160} hash function.
\end{itemize}

\noindent In the table below most of the functions involve operating on individual bits.
We will often view byetstrings as bitstrings $b_{n-1}\cdots b_0$ with
$b_i \in \{\bitzero,\bitone\}$ (and $n$ necessarily a multiple of 8).  Strictly
we should use the functions $\bitsof$ and $\bytesof$ of
Section~\ref{sec:notation-bytestrings} to convert back and forth between
bytestrings and bitstrings, but we elide this for conciseness and reduce
ambiguity by using lower-case names like $b$,$c$ and $d$ for bits and upper-case
names like $B$ for bytes. We denote the complement of a bit
$b \in \{\bitzero,\bitone\}$ by $\bar{b}$, so $\bar{\bitzero} = \bitone$ and
$\bar{\bitone} = \bitzero$.


\setlength{\LTleft}{-6mm}  % Shift the table left a bit to centre it on the page
\begin{longtable}[H]{|l|p{50mm}|p{40mm}|c|c|}
    \hline
    \text{Function} & \text{Signature} & \text{Denotation} & \text{Can} & \text{Note} \\
    & & & fail? & \\
    \hline
    \endfirsthead
    \hline
    \text{Function} & \text{Type} & \text{Denotation} & \text{Can} & \text{Note}\\
    & & & fail? & \\
    \hline
    \endhead
    \hline
    \caption{Built-in functions, batch 5}
    \endfoot
    \caption[]{Built-in functions, batch 5}
    \label{table:built-in-functions-5}
    \endlastfoot
    \TT{andByteString} & $[\ty{bool}, \ty{bytestring}, \ty{bytestring}] $ \text{$\;\; \to \ty{bytestring}$} & $\Xand$ & No & \ref{note:bitwise-logical-ops} \\[2mm]
    \TT{orByteString} & $[\ty{bool}, \ty{bytestring}, \ty{bytestring}] $ \text{$\;\; \to \ty{bytestring}$} & $\Xor$ & No  & \ref{note:bitwise-logical-ops} \\[2mm]
    \TT{xorByteString} & $[\ty{bool}, \ty{bytestring}, \ty{bytestring}] $ \text{$\;\; \to \ty{bytestring}$} & $\Xxor$ & No & \ref{note:bitwise-logical-ops} \\[2mm]
    \TT{complementByteString} & $[\ty{bytestring}] \to \ty{bytestring}$
                              &  $ b_{n-1}{\cdots}b_0 \mapsto \bar{b}_{n-1}{\cdots}\bar{b}_0$  & No & \\[2mm]
    \TT{shiftByteString} & $[\ty{bytestring}, \ty{integer}] $ \text{$\;\; \to \ty{bytestring}$} &  $\mathsf{shift}$ & No & \ref{note:shift} \\[2mm]
    \TT{rotateByteString} & $[\ty{bytestring}, \ty{integer}] $ \text{$\;\; \to \ty{bytestring}$} &  $\mathsf{rotate}$ & No & \ref{note:rotate}\\[2mm]
    \TT{countSetBits} & $[\ty{bytestring}] \to \ty{integer}$
    & $b_{n-1}{\cdots}b_0 $ \text{$\mapsto \left|\{i: b_i =1\}\right|$} & No & \\[2mm]
    \TT{findFirstSetBit} & $[\ty{bytestring}] \to \ty{integer}$ & $\mathsf{ffs}$ & No & \ref{note:ffs}\\[2mm]
    \TT{readBit} & $[\ty{bytestring}, \ty{integer}] $ \text{$\;\; \to \ty{bytestring}$}
                                        & $(b_{n-1}{\cdots}b_0, i) $ \text{$\mapsto \begin{cases}
                                        b_i & \text{if $0 \leq i \leq n-1$}\\
                                        \errorX & \text{otherwise}
                                        \end{cases}$}
                                        & Yes & \\[14mm]  %% Odd, but it gives reasonable spacing
    \TT{writeBits} & $[\ty{bytestring}, \listOf{\ty{integer}}, $
        \text{$\;\; \ty{bool}] \to \ty{bytestring}$} & $\mathsf{writeBits}$ & Yes & \ref{note:writebits} \\[2mm]
    \TT{replicateByte} & $[\ty{integer}, \ty{integer}] $ \text{$\;\; \to \ty{bytestring}$}
                                        & $\mathsf{replicateByte}$
                                        & Yes & \ref{note:replicateByte}\\[2mm]
    \hline
    \TT{ripemd\_160} & $[\ty{bytestring}] \to \ty{bytestring}$ & Compute a RIPEMD-160 hash \cite{ripemd-2, ripemd-1} & No & \\
\hline
\end{longtable}

\note{Bitwise logical operations.}
\label{note:bitwise-logical-ops}
The bitwise logical operations $\Xand$, $\Xor$, and $\Xxor$ are defined by
extending the usual single-bit logical operations $\wedge$, $\vee$, and $\oplus$
(respectively) to bytestrings. However, there is a complication if the
bytestrings have different lengths.

\begin{itemize}
\item If the first argument of one of the bitwise logical operations is $\false$
then the longer bytestring is (conceptually) truncated on the right to have the
same length as the shorter one.
\item If the first argument is $\true$
then the shorter bytestring is (conceptually) extended on the right to have the
same length as the longer one.  In the case of $\Xand$ the shorter bytestring is
extended with $\bitzero$ bits and in the case of $\Xor$ and $\Xxor$ it is
extended with $\bitone$ bits.
\end{itemize}

\noindent Formally the truncation operation mentioned above is defined as
a function which removes the rightmost $k$ bits from a bitstring:

$$
\trunc\,(b_{n-1}{\cdots}b_0, k) = b_{n-1}{\cdots}b_{k} \quad\text{if $0 \leq k \leq n$}
$$

\noindent and the extension operation is defined as a function which appends $k$ bits to a bitstring:
$$
\ext\,(b_{n-1}{\cdots}b_0, k, x) = b_{n-1}{\cdots}b_0\cdot x_{k-1}{\cdots}x_0
  \quad\text{where $x\in\{\bitzero, \bitone\}$, $k \geq 0$, and $x_0 = x_1
  = \cdots = x_{k-1} = x$}.
$$

\noindent In both cases, the otuput is the same as the input when $k=0$.
\smallskip

\noindent The denotations of the bitwise logical functions are now defined on bitstring
representations of bytestrings as follows, where $b$ is a bitstring of length $m$ and $c$
is a bitstring of length $n$:

\begin{align*}
\Xand\,(\false, b, c) &= d_{l-1} \cdots d_0
\quad \text{where $l=\min\{m,n\}$ and $d_i = \trunc\,(b,m-l)_i \wedge \trunc\,(c,n-l)_i$}\\
\Xand\,(\true, b, c) &= d_{l-1} \cdots d_0
\quad \text{where $l=\max\{m,n\}$ and $d_i = \ext\,(b,l-m,\bitone)_i \wedge \ext\,(c,l-n,\bitone)_i$}
\end{align*}
%
\begin{align*}
\Xor\,(\false, b, c) &= d_{l-1} \cdots d_0
  \quad \text{where $l=\min\{m,n\}$ and $d_i = \trunc\,(b,m-l)_i \vee \trunc\,(c,n-l)_i$}\\
\Xor\,(\true, b, c) &= d_{l-1} \cdots d_0
\quad \text{where $l=\max\{m,n\}$ and $d_i = \ext\,(b,l-m,\bitzero)_i \vee \ext\,(c,l-n,\bitzero)_i$}
\end{align*}
%
\begin{align*}
\Xxor\,(\false, b, c) &= d_{l-1} \cdots d_0
  \quad \text{where $l=\min\{m,n\}$ and $d_i = \trunc\,(b,m-l)_i \oplus \trunc\,(c,n-l)_i$}\\
\Xxor\,(\true, b, c) &= d_{l-1} \cdots d_0
\quad \text{where $l=\max\{m,n\}$ and $d_i = \ext\,(b,l-m,\bitzero)_i \oplus \ext\,(c,l-n,\bitzero)_i$}.
\end{align*}

\noindent
Note that although $\trunc$ is applied to both arguments in the above
definitions, if the arguments have the same lengths then in fact neither will be
truncated and if the arguments have different lengths then only the longer one
will be truncated.  Similarly extension will only occur if the arguments are
of different lengths, in which case the shorter one will be extended.

\note{Shifting bits.}
\label{note:shift}
The \texttt{shiftByteString} builtin takes a bytestring $s$ and an integer $k$ and
shifts the bits of the bytestring $k$ places to the left if $k \geq 0$ and $k$
places to the right if $k < 0$, replacing vacated bits with $\bitzero$.  The
length of the output bytestring is the same as that of the input.  The
denotation (defined on the bitstring representation of $s$) is

$$
\mathsf{shift}\,(b_{n-1} \cdots b_0, k) =
  d_{n-1} \cdots d_0 \quad \text{where }
  d_i = \begin{cases}
     b_{i-k} & \text{if $0 \leq i-k \leq n-1$ }\\
     \bitzero & \text{otherwise.}\\
\end{cases}
$$

\note{Rotating bits.}
\label{note:rotate}
The \texttt{rotateByteString} builtin takes a bytestring $s$ and an integer $k$
and rotates the bits of $s$ $k$ places to the left if $k \geq 0$ and $k$ places
to the right if $k < 0$.  The length of the output bytestring is the same as
that of the input.  The denotation of
\texttt{rotateByteString} (defined on the bitstring representation of $s$) is

$$
\mathsf{rotate}\,(b_{n-1} \cdots b_0, k) = d_{n-1}\cdots d_0 \quad\text{where $d_i = b_{(i-k)\bmod n}$}.
$$

\note{Finding the first set bit in a bytestring.}
\label{note:ffs}
The \texttt{findFirstSetBit} builtin returns the index of the first nonzero bit
in a bytestring, counting from the \textit{right}. If the bytestring consists
entirely of zeros then it returns $-1$.  The denotation
$\mathsf{ffs}: \b^* \rightarrow \Z$ is

$$
\mathsf{ffs}\,(b_{n-1}\cdots b_0) =
\begin{cases}
  -1 & \text{if $b_i = \bitzero$ for $0 \leq i \leq n-1$}\\
  \min{\{i: b_i \ne \bitzero\}} & \text{otherwise.}
  \end{cases}
$$


\note{Writing bits.}
\label{note:writebits}
The denotation $\mathsf{writeBits}$ of the \texttt{writeBits} builtin takes a
bytestring $s$, a list $J$ of integer indices, and a boolean value $u$.  An
error occurs if any of the indices in $J$ is not a valid bit index for $s$.  If
all of the indices are within bounds then for each index $j$ in $J$ the $j$-th
bit of $s$ is updated according to the value of $u$ ($\bitzero$
for \textsf{false}, $\bitone$ for \textsf{true}).  The list $J$ is allowed to
contain repetitions.

\smallskip
\noindent Formally,
$$
\mathsf{writeBits}(b_{n-1}{\cdots}b_0, [j_0, \ldots, j_{l-1}], u) =
\begin{cases}
\errorX & \text{if $\exists k \in \{0, \ldots, l-1\}$ such that $j_k < 0$ or $j_k \geq n$} \\
d_{n-1}{\cdots}d_0 & \text{otherwise}
\end{cases}
$$

\noindent where
$$
  d_i =
  \begin{cases}
    b_i & \text{if $i \not{\in} \{j_0, \ldots, j_{l-1}\}$}\\
      \bitzero & \text{if $i \in \{j_0, \ldots, j_{l-1}\}$ and $u = \false$}\\
      \bitone & \text{if $i \in \{j_0, \ldots, j_{l-1}\}$ and $u = \true$}.
  \end{cases}
$$


\note{Replicating bytes.}
\label{note:replicateByte}
The \texttt{replicateByte} builtin takes a length $l$ between 0 and 8192 and an
integer $B$ between 0 and 255 and produces a bytestring of length $l$.  It fails
if either argument is outside the required bounds.  The denotation is

$$
\mathsf{replicateByte}\,(l,B) =
\begin{cases}
B_0{\cdots}B_{l-1} \quad\text{ ($B_i=B$ for all $i$}) & \text{if $0 \leq l \leq 8192$ and $B \in \B$}\\
\errorX      & \text{otherwise}.
\end{cases}
$$
\noindent Note that unlike the other denotations in this section we are
viewing the output as a bytestring here, not a bitstring.
