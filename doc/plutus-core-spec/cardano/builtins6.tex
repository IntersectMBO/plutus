% I tried resetting the note number from V1-builtins here, but that made
% some hyperlinks wrong.  To get note numbers starting at one in each section, I
% think we have to define a new counter every time.
\newcounter{notenumberF}
\renewcommand{\note}[1]{
  \bigskip
  \refstepcounter{notenumberF}
  \noindent\textbf{Note \thenotenumberF. #1}
}

\subsection{Batch 6}
\label{sec:default-builtins-6}

\subsubsection{Built-in functions}
\label{sec:built-in-functions-6}
The 6th batch of builtin operations is defined in Table~\ref{table:built-in-functions-6}.

\setlength{\LTleft}{2mm}  % Shift the table right a bit to centre it on the page
\begin{longtable}[H]{|l|p{67mm}|p{18mm}|c|c|}
    \hline
    \text{Function} & \text{Signature} & \text{Denotation} & \text{Can} & \text{Note} \\
    & & & fail? & \\
    \hline
    \endfirsthead
    \hline
    \text{Function} & \text{Type} & \text{Denotation} & \text{Can} & \text{Note}\\
    & & & fail? & \\
    \hline
    \endhead
    \hline
    \caption{Built-in functions, Batch 6}
    \endfoot
    \caption[]{Built-in functions, Batch 6}
    \label{table:built-in-functions-6}
    \endlastfoot
    \TT{expModInteger}        & $[\ty{integer}, \ty{integer}, $ \text{$\;\; \ty{integer}] \to \ty{integer}$}
        & $\mathsf{expMod} $  & Yes & \ref{note:exp-mod-integer}\\
\hline
\end{longtable}

\note{Modular Exponentiation.}
\label{note:exp-mod-integer}
The \texttt{expModInteger} function performs modular exponentiation.  Its denotation
$\mathsf{expMod}: \Z \times \Z \times \Z \rightarrow \Z_{\errorX}$ is given by

$$
\mathsf{expMod}(a,e,m) =
  \begin{cases}
     \modfn(a^e,m) & \text{if $m > 1$ and $e \geq 0$}\\
     \min\{r \in \N: ra^{-e} \equiv 1\pmod{m}\}\  & \text{if $m > 1, e < 0$ and $\mathrm{gcd}(a,m) = 1$}\\
     \errorX & \text{if $m>1,e<0$, and $\mathrm{gcd}(a,m) \neq 1$}\\
     0 & \text{if $m=1$}\\
     \errorX & \text{if $m \leq 0$.}
  \end{cases}
$$ 

\noindent The fact that this is well defined for the case $e<0$ and $\mathrm{gcd}(a,m) = 1$
follows, for example, from Proposition I.3.1 of~\cite{Koblitz-GTM}.  See
Note~\ref{note:integer-division-functions} of
Section~\ref{sec:built-in-functions-1} for the definition of $\modfn$.
