% I tried resetting the note number from V1-builtins here, but that made
% some hyperlinks wrong.  To get note numbers starting at one in each section, I
% think we have to define a new counter every time.
\newcounter{notenumberF}
\renewcommand{\note}[1]{
  \bigskip
  \refstepcounter{notenumberF}
  \noindent\textbf{Note \thenotenumberF. #1}
}
\newpage
\subsection{Batch 6}
\label{sec:default-builtins-6}
This section describes some new types and functions which have not been released
at the time of writing (July 2025), but are expected to be released at a later
date.

\subsubsection{Built-in type operators}
\label{sec:built-in-type-operators-6}
The sixth batch adds type operators defined in Table~\ref{table:built-in-type-operators-6}. 

\begin{table}[H]
  \centering
    \begin{tabular}{|l|p{14mm}|l|l|}
        \hline
        Operator $\op$ & $\left|\op\right|$  & Denotation & Concrete Syntax\\
        \hline
        \texttt{array} 
          & 1 
          & $\denote{\arrayOf{t}} = \denote{t}^*$ 
          & See below\\
        \hline
        \end{tabular}
   \caption{Type operators, Batch 6}
    \label{table:built-in-type-operators-6}
\end{table}

\paragraph{Concrete syntax for arrays.}
An array of type $\texttt{array}(t)$ is written as a syntactic list
\texttt{[$c_1, \ldots, c_n$]} where each $c_i$ lies in $\bitc_t$.
Some valid constant expressions are thus:
\begin{verbatim}
   (con (array bool) [True, False, True])
   (con (array integer) [])
\end{verbatim}

\subsubsection{Built-in functions}
\label{sec:built-in-functions-6}
The sixth batch of built-in types is defined in Table~\ref{table:built-in-type-operators-6}.
Operations are defined in Table~\ref{table:built-in-functions-6}.

\setlength{\LTleft}{-12mm}  % Shift the table left a bit to centre it on the page
\renewcommand*{\arraystretch}{1.25}  %% Stretch the space between the rows by a factor of 25%
\begin{longtable}[H]{|l|p{45mm}|p{70mm}|c|c|}
    \hline
    \text{Function} & \text{Signature} & \text{Denotation} & \text{Can} & \text{Note} \\
    & & & fail? & \\
    \hline
    \endfirsthead
    \hline
    \text{Function} & \text{Type} & \text{Denotation} & \text{Can} & \text{Note}\\
    & & & fail? & \\
    \hline
    \endhead
    \hline
    \caption{Built-in functions, Batch 6}
    \endfoot
    \caption[]{Built-in functions, Batch 6}
    \label{table:built-in-functions-6}
    \endlastfoot
    \TT{expModInteger}        & $[\ty{integer}, \ty{integer}, \ty{integer}]$ \text{$\;\;\; \to \ty{integer}$}
        & $\mathsf{expMod} $  & Yes & \ref{note:exp-mod-integer}\\
    \TT{dropList}        & $[\forall a_\#, \ty{integer}, \listOf{a_\#}]$ \text{$\;\;\; \to \listOf{a_\#}$}
        & $(k,[x_1,\ldots,x_n])$
        \smallskip
        \newline
        \text{$\;\;\mapsto \left\{ \begin{array}{ll}
            [x_1,\ldots, x_n]      &  \text{if $k \leq 0$} \\ \relax %
            % Without \relax the [ at the start of the next line is regarded as an argument to \\
            [x_{k+1}, \ldots, x_n]  & \text{if $1  \leq k \leq n-1$} \\ \relax
            []                     &\text{if $k \geq n$}\\
        \end{array}\right.$} &  & \\
    \TT{lengthOfArray} 
      & $[\forall a_\#, \arrayOf{a_\#}] \to \ty{integer}$ 
      & $[] \mapsto 0$
        \newline 
        $[x_1,\ldots,x_n] \mapsto n\ (n \geq 1)$ 
      &  
      & \\
    \TT{listToArray} 
      & $[\forall a_\#, \listOf{a_\#}] \to \arrayOf{a_\#}$ 
      & $[] \mapsto []$
        \newline 
        $[x_1,\ldots,x_n] \mapsto [x_1,\ldots,x_n]\ (n \geq 1)$
      & & \\
    \TT{indexArray} 
      & $[\forall a_\#, \arrayOf{a_\#}, \ty{integer}]$ \text{$\;\;\; \to \ty{a_\#}$}
      &  $([x_1,\ldots,x_n], k)$
        \smallskip
        \newline
        \text{$\;\;\mapsto \left\{ \begin{array}{ll}
            \errorX   & \text{if $k < 0$} \\ \relax %
            x_{k+1}   & \text{if $0 \leq k \leq n-1$} \\ \relax
            \errorX   & \text{if $k > n-1$}\\
        \end{array}\right.$}  
      & Yes & \\
\hline
\end{longtable}

\note{Modular Exponentiation.}
\label{note:exp-mod-integer}
The \texttt{expModInteger} function performs modular exponentiation.  Its denotation
$\mathsf{expMod}: \Z \times \Z \times \Z \rightarrow \Z_{\errorX}$ is given by

$$
\mathsf{expMod}(a,e,m) =
  \begin{cases}
     \modfn(a^e,m) & \text{if $m > 1$ and $e \geq 0$}\\
     \min\{r \in \N: ra^{-e} \equiv 1\pmod{m}\}\  & \text{if $m > 1, e < 0$ and $\mathrm{gcd}(a,m) = 1$}\\
     \errorX & \text{if $m>1,e<0$, and $\mathrm{gcd}(a,m) \neq 1$}\\
     0 & \text{if $m=1$}\\
     \errorX & \text{if $m \leq 0$.}
  \end{cases}
$$ 

\noindent The fact that this is well defined for the case $e<0$ and $\mathrm{gcd}(a,m) = 1$
follows, for example, from Proposition I.3.1 of~\cite{Koblitz-GTM}.  See
Note~\ref{note:integer-division-functions} of
Section~\ref{sec:built-in-functions-1} for the definition of $\modfn$.



