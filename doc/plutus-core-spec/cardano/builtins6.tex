% I tried resetting the note number from V1-builtins here, but that made
% some hyperlinks wrong.  To get note numbers starting at one in each section, I
% think we have to define a new counter every time.
\newcounter{notenumberF}
\renewcommand{\note}[1]{
  \bigskip
  \refstepcounter{notenumberF}
  \noindent\textbf{Note \thenotenumberF. #1}
}
\newpage
\subsection{Batch 6}
\label{sec:default-builtins-6}

\subsubsection{Built-in functions}
\label{sec:built-in-functions-6}
The sixth batch of builtin operations is defined in Table~\ref{table:built-in-functions-6}.

\setlength{\LTleft}{-12mm}  % Shift the table left a bit to centre it on the page
\begin{longtable}[H]{|l|p{45mm}|p{70mm}|c|c|}
    \hline
    \text{Function} & \text{Signature} & \text{Denotation} & \text{Can} & \text{Note} \\
    & & & fail? & \\
    \hline
    \endfirsthead
    \hline
    \text{Function} & \text{Type} & \text{Denotation} & \text{Can} & \text{Note}\\
    & & & fail? & \\
    \hline
    \endhead
    \hline
    \caption{Built-in functions, Batch 6}
    \endfoot
    \caption[]{Built-in functions, Batch 6}
    \label{table:built-in-functions-6}
    \endlastfoot
    \TT{expModInteger}        & $[\ty{integer}, \ty{integer}, \ty{integer}]$ \text{$\;\;\; \to \ty{integer}$}
        & $\mathsf{expMod} $  & Yes & \ref{note:exp-mod-integer}\\
    \TT{caseList}        & $[\forall a_\#, \forall\star, \star, \star, \listOf{a_\#}] \to \ap$
                                              & \text{$(t_1, t_2, []) \mapsto (t_1|)$,}
                                              \text{$(t_1, t_2,
                                              [x_1,\ldots,x_n]) \mapsto (t_2|x_1,[x_2,\ldots,x_n])\ (n \geq 1)$}
                                              & & \ref{note:case-list-case-data}\\[5mm]
    \TT{caseData}        & $[\forall\star, \star, \star, \star, \star, \star, \ty{data}] \to \ap$
    & $(t_C, t_M, t_L, t_I, t_B, d)$
    \smallskip
    \newline  % The big \{ was abutting the text above
    \text{$\;\;\mapsto
               \left\{ \begin{array}{ll}  %% This looks better than `cases`
                 (t_C|n,l)  & \text{if $d = \inj_C(n, l)$} \\
                 (t_M|l)  & \text{if $d = \inj_M(l)$} \\
                 (t_L|l)  & \text{if $d = \inj_L(l)$} \\
                 (t_I|n)  & \text{if $d =\inj_I(n)$} \\
                 (t_B|s)  & \text{if $d = \inj_B(s)$} \\
               \end{array}\right.$}  & & \ref{note:case-list-case-data}\\
\hline
\end{longtable}

\note{Modular Exponentiation.}
\label{note:exp-mod-integer}
The \texttt{expModInteger} function performs modular exponentiation.  Its denotation
$\mathsf{expMod}: \Z \times \Z \times \Z \rightarrow \Z_{\errorX}$ is given by

$$
\mathsf{expMod}(a,e,m) =
  \begin{cases}
     \modfn(a^e,m) & \text{if $m > 0$ and $e \geq 0$}\\
     \min\{r \in \N: ra^{-e} \equiv 1\pmod{m}\}\  & \text{if $m > 0, e < 0$ and $\mathrm{gcd}(a,m) = 1$}\\
     \errorX & \text{otherwise.}
  \end{cases}
$$ 

\noindent The fact that this is well defined for the case $e<0$ and $\mathrm{gcd}(a,m) = 1$
follows, for example, from Proposition I.3.1 of~\cite{Koblitz-GTM}.  See
Note~\ref{note:integer-division-functions} of
Section~\ref{sec:built-in-functions-1} for the definition of $\modfn$.

\note{\texttt{caseList} and \texttt{caseData} in UPLC.}
\label{note:case-list-case-data}
The \texttt{caseList} function takes two inputs $t_1$ and $t_2$ and a list $l$
and returns the delayed application $(t_1|)$ if $l$ is empty and
$(t_2|h,l^{\prime})$ if $l$ is nonempty with head $h$ and tail $l^{\prime}$.
The intended usage is that if $l$ is a list with elements of type $\alpha$ then
$t_1$ should be a term of some type $\beta$ and $t_2$ a function of type
$\alpha \rightarrow
\mathtt{list}(\alpha) \rightarrow \beta$; however, Untyped Plutus Core has no notion
of type for general terms and when \texttt{caseList} is applied there is no
check that the inputs $t_1$ and $t_2$ are of the expected type.  In UPLC, when
the list $l$ is nonempty the deferred application $(t_2|h,l^{\prime})$ will
immediately be turned into a genuine application by the evaluator, and at that
point there may be an error because the term $t_2$ cannot be applied to the
arguments $h$ and $l^{\prime}$ (although even then an error may not occur if,
for example $t_2 h l^{\prime}$ is not a saturated application (ie, $t_2$ expects
more than two arguments)) .  In the case of the empty list, the value $(t_1|)$
is effectively constant and no error will occur.  However, an error will occur
immediately if the final argument of \texttt{caseList} is not of
type \texttt{list}.

The \texttt{caseData} function behaves similarly: the argument $d$ is expected
to be an object of the built-in \texttt{data} type (see
Section~\ref{sec:built-in-types-1}) and $t_C, t_M, t_L, t_I$, and $t_B$ are
expected to be functions which will be applied to the contents of
the \texttt{Constr}, \texttt{Map}, \texttt{List}, \texttt{I} and \texttt{B}
constructors of the \texttt{data} argument respectively, but again there is no
typecheck of the functional arguments when \texttt{caseData} is applied and a
type mismatch can only be detected when the evaluator attempts to evaluate the
deferred application.  However, there will be an immediate error if the final
argument of \texttt{caseData} is not of type \texttt{data}.



