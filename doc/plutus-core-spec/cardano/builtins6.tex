% I tried resetting the note number from V1-builtins here, but that made
% some hyperlinks wrong.  To get note numbers starting at one in each section, I
% think we have to define a new counter every time.
\newcounter{notenumberE}
\renewcommand{\note}[1]{
  \bigskip
  \refstepcounter{notenumberE}
  \noindent\textbf{Note \thenotenumberE. #1}
}

\subsection{Batch X}
\label{sec:default-builtins-X}

\subsubsection{Built-in functions}
\label{sec:built-in-functions-X}
The Xth batch of builtin operations is defined in Table~\ref{table:built-in-functions-X}.

\setlength{\LTleft}{2mm}  % Shift the table right a bit to centre it on the page
\begin{longtable}[H]{|l|p{67mm}|p{18mm}|c|c|}
    \hline
    \text{Function} & \text{Signature} & \text{Denotation} & \text{Can} & \text{Note} \\
    & & & fail? & \\
    \hline
    \endfirsthead
    \hline
    \text{Function} & \text{Type} & \text{Denotation} & \text{Can} & \text{Note}\\
    & & & fail? & \\
    \hline
    \endhead
    \hline
    \caption{Built-in Functions}
    \endfoot
    \caption[]{Built-in Functions}
    \label{table:built-in-functions-X}
    \endlastfoot
    \TT{expModInteger}        & $[\ty{integer}, \ty{integer}, $ \text{$\;\; \ty{integer}] \to \ty{integer}$}
        & $ b^e (\modfn m)$  & Yes & \ref{note:exp-mod-integer}\\
\hline
\end{longtable}

\note{Modular Exponentiation.}
\label{note:exp-mod-integer}
The \texttt{expModInteger} function performs modular exponentiation,
an operation with many uses in cryptography. The implementation
is time and memory efficient when compared to the naive approach
of multiplying $e$ times  $(b \times b \times ... \times b)$ followed by $(\modfn m)$.

The operation fails with \texttt{error} in any of the following cases:
\begin{enumerate}
\item $m \le 0$
\item $e < 0$, and integer $b$ does not have a multiplicative inverse w.r.t. $(\modfn m)$
\end{enumerate}
