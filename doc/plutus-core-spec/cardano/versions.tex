\section{Protocol versions}
The Cardano blockchain controls the introduction of features through the use of \emph{protocol versions}, a field in the protocol parameters.
The major protocol version is used to indicate when forwards-incompatible changes (i.e. those that allow blocks that were not previously allowed) are made to the rules of the chain.
This is a hard fork of the chain.

The protocol version is part of the history of the chain, as are all protocol parameters.
That means that all blocks are associated with the protocol version from when they were created, so that they can be interpreted correctly.

In summary, conditioning on the protocol version is the main way in which we can introduce changes in behaviour.

Table~\ref{table:protocol-versions} lists the protocol versions that are relevant to the use of Plutus Core on Cardano.

\begin{table}[H]
  \centering
    \begin{tabular}{|c|l|l|}
        \hline
        \thead{Protocol version} & \thead{Codename} & \thead{Date} \\
        \hline
        5.0 & Alonzo & September 2021 \\
        7.0 & Vasil & June 2022 \\
        8.0 & Valentine & February 2023 \\
        9.0 & Conway & Upcoming \\
        \hline
    \end{tabular}
    \caption{Protocol versions}
    \label{table:protocol-versions}
\end{table}

\section{Ledger languages}

The Cardano ledger uses Plutus Core as the programming language for \emph{scripts}.
The ledger in fact supports multiple different interpretations for scripts, and so each script is tagged with a \emph{ledger language} that tells the ledger how to interpret it.
Since the ledger must always be able to evaluate old scripts and get the same answer, the ledger language must pin down everything about how the script is evaluated, including:

\begin{enumerate}
  \item How to interpret the script itself (e.g. as a Plutus Core program, what versions of the Plutus Core language are allowable)
  \item Other configuration the script may need in order to run (e.g. the set of builtin types and functions and their interpretations, cost model parameters)
  \item How the script is invoked (e.g. after having certain arguments passed to it)
\end{enumerate}

There are currently three ``Plutus'' ledger languages (i.e. ledger languages whose underlying programming language is Plutus Core) in use on Cardano:\footnote{
  Note that ledger languages are completely distinct from the point of view of the ledger, the ``V1''/``V2'' naming is suggestive of the fact that these two ledger languages are related, but in the implementation they are completely independent.
}
\begin{enumerate}
  \item \LL{PlutusV1}
  \item \LL{PlutusV2}
  \item \LL{PlutusV3} (forthcoming)
\end{enumerate}

\noindent Table~\ref{table:ll-introduction} shows when each Plutus ledger language was introduced.
Ledger languages remain available permanently after they have been introduced.

\begin{table}[H]
  \centering
    \begin{tabular}{|c|c|}
        \hline
        \thead{Protocol version} & \thead{Ledger language introduced} \\
        \hline
        5.0 & \LL{PlutusV1} \\
        7.0 & \LL{PlutusV2} \\
        9.0 & \LL{PlutusV3} \\
        \hline
    \end{tabular}
    \caption{Introduction of Plutus ledger languages}
    \label{table:ll-introduction}
\end{table}

\noindent Ledger languages can evolve over time.
We can make backwards-compatible changes when the major protocol version changes, but backwards-incompatible changes can only be introduced by creating a whole new ledger language.\footnote{
  See~\cite{CIP-35} for more details on how the process of evolution works.
}
This means that to fully explain the behaviour of a ledger language we may need to also index by the protocol version.

The following tables show how Plutus ledger languages determine:
\begin{itemize}
  \item Which Plutus Core language versions are allowable (Table~\ref{table:lv-introduction})
  \item Which built-in functions and types are available (Table~\ref{table:b-introduction}, given in terms of batches, see Section~\ref{sec:builtin-batches})
  \item How to interpret the built-in functions and types (Table~\ref{table:bs-introduction}, given in terms of built-in semantics variants, see Section~\ref{sec:builtin-semantics-variants})
\end{itemize}

Currently, once we add a feature for any given protocol version/ledger language, we also make it available for all subsequent protocol versions/ledger languages.
For example, Batch 2 of builtins was introduced in \LL{PlutusV2} at protool version 7.0, so it is also available in \LL{PlutusV2} at protocol versions after 7.0, and \LL{PlutusV3} at protocol versions after 9.0 (when \LL{PlutusV3} itself is first introduced).
Hence the tables are simplified to only show when something is \emph{introduced}.

\begin{table}[H]
  \centering
    \begin{tabular}{|c|c|c|}
        \hline
        \thead{Ledger language} & \thead{Protocol version} & \thead{Plutus Core language version introduced} \\
        \hline
        \LL{PlutusV1} & 5.0 & 1.0.0 \\
        \LL{PlutusV3} & 9.0 & 1.1.0 \\
        \hline
    \end{tabular}
    \caption{Introduction of Plutus Core language versions}
    \label{table:lv-introduction}
\end{table}

\begin{table}[H]
  \centering
    \begin{tabular}{|c|c|c|}
        \hline
        \thead{Ledger language} & \thead{Protocol version} & \thead{Built-in functions and types introduced} \\
        \hline
        \LL{PlutusV1} & 5.0 & Batch 1 \\
        \LL{PlutusV2} & 7.0 & Batch 2 \\
        \LL{PlutusV2} & 8.0 & Batch 3 \\
        \LL{PlutusV3} & 9.0 & Batch 4 \\
        \hline
    \end{tabular}
    \caption{Introduction of built-in functions and types}
    \label{table:b-introduction}
\end{table}

\begin{table}[H]
  \centering
    \begin{tabular}{|c|c|c|}
        \hline
        \thead{Ledger language} & \thead{Built-in semantics variant used} \\
        \hline
        \LL{PlutusV1} & Built-in semantics 1 \\
        \LL{PlutusV3} & Built-in semantics 2 \\
        \hline
    \end{tabular}
    \caption{Selection of built-in semantics variant}
    \label{table:bs-introduction}
\end{table}
