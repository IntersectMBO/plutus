% I tried resetting the note number from V1-builtins here, but that made
% some hyperlinks wrong.  To get note numbers starting at one in each section, I
% think we have to define a new counter every time.
\newcounter{notenumberB}
\renewcommand{\note}[1]{
  \bigskip
  \refstepcounter{notenumberB}
  \noindent\textbf{Note \thenotenumberB. #1}
}

\newpage

\section{PlutusV2 Built-in Types and Functions}
\label{appendix:default-builtins-V2}
PlutusV2 became available in the Vasil release of Cardano (June 2022). It
extends the set of built-in functions slightly.

\subsection{Built-in types and type operators}
\label{sec:V2-built-in-types} 
The built-in types and type operators remain unchanged from PlutusV1
(Appendix~\ref{sec:V1-built-in-types}).

\subsection{Built-in functions}
\label{sec:V2-built-in-functions}
PlutusV2 continues to support the PlutusV1 built-in functions (Table
\ref{table:V1-built-in-functions})
and adds three new ones: see Table~\ref{table:V2-built-in-functions}.

\setlength{\LTleft}{-10mm}  % Shift the table left a bit to centre it on the page
\begin{longtable}[H]{|l|p{42mm}|p{35mm}|c|c|}
    \hline
    \text{Function} & \text{Signature} & \text{Denotation} & \text{Can} & \text{Note} \\
    & & & fail? & \\
    \hline
    \endfirsthead
    \hline
    \text{Function} & \text{Type} & \text{Denotation} & \text{Can} & \text{Note}\\
    & & & fail? & \\
    \hline
    \endhead
    \hline
    \caption{Built-in Functions}
    \endfoot
    \caption[]{Built-in Functions}
    \label{table:V2-built-in-functions}
    \endlastfoot
    \TT{serialiseData}                        & $[\ty{data}] \to \ty{bytestring}$   &  $\mathcal{E}_{\mathtt{data}}$ &
      & \ref{note:serialise-data}\\
    \TT{verifyEcdsaSecp256k1Signature}        & $[\ty{bytestring}, \ty{bytestring}, $ \text{$\;\; \ty{bytestring}] \to \ty{bool}$}
        & Verify an SECP-256k1 ECDSA signature & Yes & \ref{note:verify-ecdsa-secp256k1-signature}\\
    \TT{verifySchnorrSecp256k1Signature}      & $[\ty{bytestring}, \ty{bytestring}, $ \text{$\;\; \ty{bytestring}] \to \ty{bool}$}
          & Verify an SECP-256k1 Schnorr signature & Yes & \ref{note:verify-schnorr-secp256k1-signature}\\
\hline 
\end{longtable}


\note{Serialising $\ty{data}$ objects.}
\label{note:serialise-data}
The \texttt{serialiseData} function takes a $\ty{data}$ object and converts it
into a bytestring using a CBOR encoding.  A full specification of the encoding
(including the definition of $\mathcal{E}_{\mathtt{data}}$) is provided in
Appendix~\ref{appendix:data-cbor-encoding}.


\note{Secp256k1 ECDSA Signature verification.}
\label{note:verify-ecdsa-secp256k1-signature}
The \texttt{verifyEcdsaSecp256k1Signature} function performs elliptic curve
digital signature verification \cite{ANSI-X9.62, ANSI-x9.142,
  Johnson-Menezes-Vanstone-ECDSA} over the \texttt{secp256k1}
curve~\cite[\S2.4.1]{SECP256} and conforms to the interface described in
Note~\ref{note:digital-signature-verification-functions} of
Section~\ref{sec:V1-built-in-functions}.  The arguments must have the
following sizes:
\begin{itemize}
\item $\vk$: 33 bytes
\item $m$: 32 bytes
\item $s$: 64 bytes.
\end{itemize}
The public key $\vk$ is expected to be in the 33-byte compressed form described
in~\cite{Bitcoin-ECDSA}.  Moreover, the ECDSA scheme admits two distinct valid
signatures for a given message and private key, and  we follow the restriction
imposed by Bitcoin (see~\cite{BIP-146},
\texttt{LOW\_S}) and \textbf{only accept the smaller signature};
\texttt{verifyEcdsa\-Secp\-256k1Signature} will return \texttt{false} if the larger
one is supplied.

% For more on the lower signature business, see
%    https://github.com/input-output-hk/plutus/pull/4591#issuecomment-1120797931
% and
%    https://github.com/input-output-hk/plutus/pull/4591#issuecomment-1121684776
% The C code that performs the check in the bitcoin implementation is at
%   https://github.com/bitcoin-core/secp256k1/blob/485f608fa9e28f132f127df97136617645effe81/src/secp256k1.c#L400
% and
%   https://github.com/bitcoin-core/secp256k1/blob/485f608fa9e28f132f127df97136617645effe81/src/scalar_low_impl.h#L85
%
% Schnorr signatures do something substantially different for ECDSA and I don't think
% the question of multiple signatures arises for verifySchnorrSecp256k1Signature.

\note{Secp256k1 Schnorr Signature verification.}
\label{note:verify-schnorr-secp256k1-signature}
The \texttt{verifySchnorrSecp256k1Signature} function performs verification of
Schnorr signatures~\cite{Schnorr89, BIP-340} over the \texttt{secp256k1} curve
and conforms to the interface described in
Note~\ref{note:digital-signature-verification-functions} of
Section~\ref{sec:V1-built-in-functions}.  The arguments are expected to be
of the forms specified in BIP-340~\cite{BIP-340} and thus should have the
following sizes:
\begin{itemize}
\item $\vk$: 32 bytes
\item $m$: unrestricted
\item $s$: 64 bytes.
\end{itemize}
