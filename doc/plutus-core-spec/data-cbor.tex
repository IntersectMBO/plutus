\chapter{Serialising \texttt{data} Objects Using the CBOR Format}
\label{appendix:data-cbor-encoding}

\section{Introduction}
In this section we define a CBOR encoding for the \texttt{data} type introduced
in Section~\ref{sec:built-in-types-1}.  For ease of reference we reproduce
the definition of the Haskell \texttt{Data} type, which we may regard as the
definition of the Plutus \texttt{data} type. Other representations are of course
possible, but this is useful for the present discussion.

\begin{alltt}
   data Data =
      Constr Integer [Data]
      | Map [(Data, Data)]
      | List [Data]
      | I Integer
      | B ByteString
\end{alltt}

\noindent The CBOR encoding defined here uses basic CBOR encodings as defined in
the CBOR standard~\cite{rfc8949-CBOR}, but with some refinements. Specifically

\begin{itemize}
\item We use a restricted encoding for bytestrings which requires that
  bytestrings are serialised as sequences of blocks, each block being at most 64
  bytes long.  Any encoding of a bytestring using our scheme is valid according
  to the CBOR specification, but the CBOR specification permits some encodings
  which we do not accept. The purpose of the size restriction is to prevent
  arbitrary data from being stored on the blockchain.
\item Large integers (less than $-2^{64}$ or greater than $2^{64}-1$) are
  encoded via the restricted bytestring encoding; other integers are encoded as
  normal. Again, our restricted encodings are compatible with the CBOR
  specification.
\item The \texttt{Constr} case of the \texttt{data} type is encoded using a
  scheme which is an early version of a proposed extension of the CBOR
  specification to include encodings for discriminated unions.
  See~\cite{CBOR-alternatives} and \cite[Section 9.1]{CBOR-notable-tags}.
  \end{itemize}


\section{Notation}
We introduce some extra notation for use here and in
Appendix~\ref{appendix:flat-serialisation}.

\medskip
\noindent The notation $f: X \rightharpoonup Y$ indicates that $f$ is a partial
map from $X$ to $Y$.  We denote the empty bytestring by $\epsilon$ and (as
in~\ref{sec:notation-lists}) use $\length(s)$ to denote the length of a
bytestring $s$ and $\cdot$ to denote the concatenation of two bytestrings, and
also the operation of prepending or appending a byte to a bytestring. We will
also make use of the $\divfn$ and $\modfn$ functions described in
Note~\ref{note:integer-division-functions} in
Section~\ref{sec:default-builtins-1}.%
\nomenclature[Bz]{$\epsilon$}{The empty bytestring}

\paragraph{Encoders and decoders.}
Recall that $\B = \Nab{0}{255}$, the set of integral values that can
be represented in a single byte, and that we identify bytestrings with elements
of $\B^*$. We will describe the CBOR encoding of the \texttt{data} type by
defining families of encoding functions (or \textit{encoders})
$$
\e_X : X \rightarrow \B^*
$$%
\nomenclature[IC]{$\e_X$}{CBOR encoder for \texttt{data}}
and decoding functions (or \textit{decoders})
$$
\d_X : \B^* \rightharpoonup \B^* \times X
$$%
\nomenclature[IC]{$\d_X$}{CBOR decoder for \texttt{data}}

\noindent for various sets $X$, such as the set $\Z$ of integers and the set of
all \texttt{data} items.  The encoding function $\e_X$ takes an element $x \in
X$ and converts it to a bytestring, and the decoding function $\d_X$ takes a
bytestring $s$, decodes some initial prefix of $s$ to a value $x \in X$, and
returns the remainder of $s$ together with $x$.  Decoders for complex types will
often be built up from decoders for simpler types.  Decoders are
\textit{partial} functions because they can fail, for instance, if there is
insufficient input, or if the input is not well formed, or if a decoded value is
outside some specified range.

Many of the decoders which we define below involve a number of cases for
different forms of input, and we implicitly assume that the decoder fails if
none of the cases applies.  We also assume that if a decoder fails then so does
any other decoder which invokes it, so any failure when attempting to decode a
particular data item in a bytestring will cause the entire decoding process to
fail (immediately).

\section{The CBOR format}
A CBOR-encoded item consists of a bytestring beginning with a \textit{head}
which occupies 1,2,3,5, or 9 bytes.  Depending on the contents of the head, some
sequence of bytes following it may also contribute to the encoded item. The
first three bits of the head are interpreted as a natural number between 0 and 7
(the \textit{major type}) which gives basic information about the type of the
following data.  The remainder of the head is called the \textit{argument} of the
head and is used to encode further information, such as the value of an encoded
integer or the size of a list of encoded items.  Encodings of complex objects
may occupy the bytes following the head, and these will typically contain
further encoded items.

\section{Encoding and decoding the heads of CBOR items}
For $i \in \N$ we define a function $\byte_i: \N \rightarrow \B$ which returns
the $i$-th byte of an integer, with the 0-th byte being the least significant:
$$
  \byte_i(n) = \modfn(\divfn(n,256^i), 256).
$$

\noindent We use this to define for each $k \geq 1$ a partial function
$\intToBS_k: \N \rightharpoonup \B^*$ which converts a sufficiently small
integer to a bytestring of length $k$ (possibly with leading zeros):
$$
\intToBS_k(n) = [\byte_{k-1}(n), \ldots, \byte_0(n)]  \quad \text {if $n \leq 256^k-1$}.
$$
\noindent
This function fails if the input is too large to fit into a $k$-byte
bytestring.

We also define inverse functions $\bsToInt_k: \B^* \rightharpoonup \N$ which
decode a $k$-byte natural number from the start of a bytestring, failing if
there is insufficient input:
$$ \bsToInt_k(s) = (s', \sum_{i=0}^{k-1}256^ib_i) \qquad \text{if $s = [b_{k-1},
    \ldots, b_0] \cdot s'$}.
$$

\noindent We now define an encoder $\eHead: \Nab{0}{7} \times
\Nab{0}{2^{64}-1} \rightarrow \B^*$ which takes a major type and a
natural number and encodes them as a CBOR head using the standard encoding:

$$
  \eHead(m,n) =
  \begin{cases}
    [32m + n] & \text{if $n \leq 23$}\\
    (32m+24) \cdot \intToBS_1(n) & \text{if $24 \leq n \leq 255$}\\
    (32m+25) \cdot \intToBS_2(n) & \text{if $256 \leq n \leq 256^2-1$}\\
    (32m+26) \cdot \intToBS_4(n)& \text{if $256^2 \leq n \leq 256^4-1$}\\
    (32m+27) \cdot \intToBS_8(n) & \text{if $256^4 \leq n \leq 256^8-1$}.
  \end{cases}
$$

\noindent The corresponding decoder $\dHead: \B^* \rightharpoonup \B^* \times
\Nab{0}{7} \times \Nab{0}{2^{64}-1}$ is given by

$$
  \dHead(n \cdot s) =
  \begin{cases}
    (s, \divfn(n,32), \modfn(n,32)) & \text{if $\modfn(n,32) \leq 23$}\\
    (s', \divfn(n,32), k) & \text{if $\modfn(n,32) = 24$ and $\bsToInt_1(s) = (s', k)$}\\
    (s', \divfn(n,32), k) & \text{if $\modfn(n,32) = 25$ and $\bsToInt_2(s) = (s', k)$}\\
    (s', \divfn(n,32), k) & \text{if $\modfn(n,32) = 26$ and $\bsToInt_4(s) = (s', k)$}\\
    (s', \divfn(n,32), k) & \text{if $\modfn(n,32) = 27$ and $\bsToInt_8(s) = (s', k)$}.
  \end{cases}
$$

\noindent This function is undefined if the input is the empty bytestring
$\epsilon$, if the input is too short, or if its initial byte is not of the
expected form.

\paragraph{Heads for indefinite-length items.}
The functions $\eHead$ and $\dHead$ defined above are used for a number of
purposes.  One use is to encode integers less than 64 bits, where the argument
of the head is the relevant integer.  Another use is for ``definite-length''
encodings of items such as bytestrings and lists, where the head contains the
length $n$ of the object and is followed by some encoding of the object itself
(for example a sequence of $n$ bytes for a bytestring or a sequence of $n$
encoded objects for the elements of a list).  It is also possible to have
``indefinite-length'' encodings of objects such as lists and arrays, which do
not specify the length of an object in advance: instead a special head with
argument 31 is emitted, followed by the encodings of the individual items; the
end of the sequence is marked by a ``break'' byte with value 255.  We define an
encoder $\eIndef: \Nab{2}{5} \rightarrow \B^*$ and a decoder
$\dIndef: \B^* \rightharpoonup \B^* \times \Nab{2}{5}$ which deal
with indefinite heads for a given major type:

\begin{align*}
  \eIndef(m) &= [32m+31]\\
  \dIndef(n \cdot s) & = (s, m) \qquad \text{if $n = 32m+31$}.
\end{align*}

\noindent Note that $\eIndef$ and $\dIndef$ are only defined for $m \in
\{2,3,4,5\}$ (and we shall only use them in these cases). The case $m=31$
corresponds to the break byte and for $m \in \{0,1,6\}$ the value is not well
formed: see~\cite[3.2.4]{rfc8949-CBOR}.

\section{Encoding and decoding bytestrings}
The standard CBOR encoding of bytestrings encodes a bytestring as either a
definite-length sequence of bytes (the length being given in the head) or as an
indefinite-length sequence of definite-length ``chunks'' (see~\cite[\S\S3.1 and
  3.4.2]{rfc8949-CBOR}).  We use a similar scheme, but only allow chunks of
length up to 64.  To this end, suppose that $a = [a_1, \ldots, a_{64k+r}] \in
\B^*\backslash\{\epsilon\}$ where $k \geq 0$ and $0 \leq r \leq 63$.  We define
the \textit{canonical 64-byte decomposition} $\bar{a}$ of $a$ to be
$$
\bar{a} = [[a_1, \ldots, a_{64}],
  [a_{65}, \ldots, a_{128}] ,\ldots,
  [a_{64(k-1)+1}, \ldots, a_{64k}]] \in (\B^*)^*
$$
\noindent if $r=0$ and

$$
\bar{a} = [[a_1, \ldots, a_{64}],
  [a_{65}, \ldots, a_{128}], \ldots,
  [a_{64(k-1)+1}, \ldots, a_{64k}], [a_{64k+1}, \ldots, a_{64k+r}]] \in (\B^*)^*
$$
\noindent if $r>0$.  The canonical decomposition of the empty list is $\bar{\epsilon} = []$.

\medskip
\noindent We define the encoder $\eBS: \B^* \rightarrow \B^*$ for bytestrings by
encoding bytestrings of size up to 64 using the standard CBOR encoding and
encoding larger bytestrings by breaking them up into 64-byte chunks (with the
final chunk possibly being less than 64 bytes long) and encoding them as an
indefinite-length list (major type 2 indicates a bytestring):
$$ \eBS(s) =
\begin{cases}
  \eHead(2,\length(s)) \cdot s & \text{if $\length(s) \leq 64$}\\
  \eIndef(2) \cdot \eHead(2,\length(c_1)) \cdot c_1 \cdot \eHead(2,\length(c_2)) \cdot \cdots \\
  \qquad  \cdots  \cdot c_{n-1} \cdot \eHead(2,\length(c_n)) \cdot c_n \cdot 255
  & \text{if $\length(s) > 64$ and $\bar{s} = [c_1, \ldots, c_n]$.}
\end{cases}
$$

\medskip

\noindent The decoder is slightly more complicated.  Firstly, for every $n \geq
0$ we define a decoder $\dBytes^{(n)}: \B^* \rightharpoonup \B^* \times \B^*$
which extracts an $n$-byte prefix from its input (failing in the case of
insufficient input):
$$
\dBytes^{(n)}(s) =
\begin{cases}
  (s, \epsilon) & \text{if $n=0$}\\
  (s'', b \cdot t) & \text{if $s = b \cdot s'$ and $\dBytes^{(n-1)}(s') = (s'', t)$}.
\end{cases}
$$

\noindent Secondly, we define a decoder $\dBlock: \B^* \rightharpoonup \B^*
\times \B^*$ which attempts to extract a bytestring of length at most 64
from its input; $\dBlock$ (and any other function which calls it) will
fail if it encounters a bytestring which is greater than 64 bytes.
$$
\dBlock(s) =
  \dBytes^{(n)}(s') \quad \text{if $\dHead(s) = (s', 2,n)$ and $n \leq 64$}.
$$

\noindent Thirdly, we define a decoder $\dBlocks: \B^* \rightharpoonup \B^*
\times \B^*$ which decodes a sequence of blocks and returns their concatenation.
$$
\dBlocks(s) =
\begin{cases}
  (s', \epsilon) & \text{if $s = 255 \cdot s'$}\\
  (s'', t \cdot t') &
  \text{if $\dBlock(s) = (s', t)$
    and $\dBlocks(s') = (s'', t')$}.
\end{cases}
$$

\noindent Finally we define the decoder $\dBS: \B^* \rightharpoonup \B^*
\times \B^*$ for bytestrings by
$$
\dBS(s) =
\begin{cases}
  (s', t) & \text{if $\dBlock(s) = (s', t)$}\\
  \dBlocks(s') & \text{if $\dIndef(s) = (s', 2)$}.
\end{cases}
$$

\noindent This looks for either a single block or an indefinite-length list of
blocks, in the latter case returning their concatenation.  It will accept the
output of $\eBS$ but will reject bytestring encodings containing any blocks
greater than 64 bytes long, even if they are valid bytestring encodings
according to the CBOR specification.

\section{Encoding and decoding integers}
As with bytestrings we use a specialised encoding scheme for integers which
prohibits encodings with overly-long sequences of arbitrary data.  We encode
integers in $\Nab{-2^{64}}{2^{64}-1}$ as normal (see ~\cite[\S
  3.1]{rfc8949-CBOR}: the major type is 0 for positive integers and 1 for
negative ones) and larger ones by emitting a CBOR tag (major type 6; argument 2
for positive numbers and 3 for negative numbers) to indicate the sign, then
converting the integer to a bytestring and emitting that using the encoder
defined above.  This encoding scheme is the same as the standard one except for
the size limitations.

\medskip
\noindent
We firstly define conversion functions $\itos : \N \rightarrow
\B^*$ and $\stoi: \B^* \rightarrow \N$ by
$$
\itos(n) =
\begin{cases}
  \epsilon & \text{if $n=0$}\\
  \itos(\divfn(n,256)) \cdot \modfn(n,256) & \text{if $n>0$.}\\
\end{cases}
$$
\noindent and
$$
\stoi(l) =
\begin{cases}
  0 & \text{if $l = \epsilon$}\\
  256\times\stoi(l') + n & \text{if $l=l' \cdot n$ with $n \in \B$.}\\
\end{cases}
$$

\noindent
The encoder $\eZ: \Z \rightarrow \B^*$ for integers is now defined by
$$ \eZ(n) =
\begin{cases}
  \eHead(0,n)                             & \text{if $0\leq n \leq 2^{64}-1$}\\
  \eHead(6,2) \cdot \eBS(\itos(n))    & \text{if $n \geq 2^{64}$}\\
  \eHead(1,-n-1)                          & \text{if $-2^{64} \leq n \leq -1$}\\
  \eHead(6,3) \cdot \eBS(\itos(-n-1)) & \text{if $n \leq -2^{64}-1$}.
\end{cases}
$$
% 7 says it's a *tag*
% 2 -> positive bignum, 3 -> negative bignum (2.4.2)

\noindent The decoder $\dZ: \B^* \rightharpoonup \B^* \times \Z$ inverts this
process. The decoder is in fact slightly more permissive than the encoder
because it also accepts small integers encoded using the scheme for larger ones.
However, the CBOR standard permits integer encodings which contain bytestrings
longer than 64 bytes and it will not accept those.
$$ \dZ(s) =
\begin{cases}
  (s', n)               & \text{if $\dHead(s) = (s', 0,n)$}\\
  (s', -n-1)            & \text{if $\dHead(s) = (s', 1,n)$}\\
  (s'', \stoi(b))       & \text{if $\dHead(s) = (s', 6,2)$ and $\dBS(s') = (s'', b)$}\\
  (s'', -\stoi(b)-1)    & \text{if $\dHead(s) = (s', 6,3)$ and $\dBS(s') = (s'', b)$}.\\
\end{cases}
$$



\section{Encoding and decoding \texttt{data}}
\label{sec:encoding-data}
\newcommand\eData{\e_{\mathtt{data}}}
\newcommand\eDataStar{\e_{\mathtt{data^*}}}
\newcommand\eDataStarSq{\e_{\mathtt{(data^2)^*}}}

\newcommand\dData{\d_{\mathtt{data}}}
\newcommand\dDataStar{\d_{\mathtt{data^*}}}
\newcommand\dDataStarSq{\d_{\mathtt{(data^2)^*}}}

It is now quite straightforward to encode most \texttt{data} values.  The main
complication is in the encoding of constructor tags (the number $i$ in
$\mathtt{Constr}\: i\, l$).

\paragraph{The encoder.} The encoder is given by
\begin{alignat*}{2}
&  \eData(\mathtt{Map}\: l) && = \eHead(5,\length(l)) \cdot \eDataStarSq(l)\\
&  \eData(\mathtt{List}\: l) && = \eIndef(4) \cdot \eDataStar(l) \cdot 255\\
&  \eData(\mathtt{Constr}\: i\, l) && = \ecTag(i) \cdot \eIndef(4) \cdot  \eDataStar(l) \cdot 255\\
& \eData(\mathtt{I}\: n) && = \eZ(n)\\
&  \eData(\mathtt{B}\: s) && = \eBS(s).
\end{alignat*}

\noindent This definition uses encoders for lists of data items, lists of pairs
of data items, and constructor tags as follows:
$$
\eDataStar([d_1, \ldots, d_n]) = \eData(d_1) \cdot \cdots \cdot \eData(d_n)
$$
$$
\eDataStarSq([(k_1,d_1), \ldots, (k_n, d_n)]) = \eData(k_1) \cdot \eData(d_1) \cdot \cdots \cdot \eData(k_n) \cdot \eData(d_n)
$$
$$
\ecTag(i) =
\begin{cases}
  \eHead(6,121+i) & \text{if $0 \leq i \leq 6$}\\
  \eHead(6,1280+(i-7)) & \text{if $7 \leq i \leq 127$}\\
  \eHead(6,102) \cdot \eHead(4,2) \cdot \eZ(i) & \text{otherwise}.\\
  \end{cases}
$$

\noindent
In the final case of $\ecTag$ we emit a head with major type 4 and argument
2. This indicates that an encoding of a list of length 2 will follow: the first
element of the list is the constructor number and the second is the argument
list of the constructor, which is actually encoded in $\eData$.  It might be
conceptually more accurate to have a single encoder which would encode both the
constructor tag and the argument list, but this would increase the complexity of
the notation even further.  Similar remarks apply to $\dcTag$ below.

%% kwxm: The CBOR specification says ``A map that has duplicate keys may be
%% well-formed, but it is not valid, and thus it causes indeterminate decoding.''
%% Presumably this is because potentially the map could be decoded into some data
%% structure that doesn't support duplicate keys, and exactly which one of multiple
%% entries with the same key ends up in the final structure could be
%% non-deterministic.  We don't say anything at all about the semantics of maps
%% (they really are just Scott-encoded lists of pairs in Plutus Core and it's up to
%% the user how they're treated), so I think we can just ignore this: our decoder
%% really does preserve entries with repeated keys.}


\paragraph{The decoder.} The decoder is given by
$$
\dData(s) =
\begin{cases}
  (s'', \mathtt{Map}\: l) & \text{if $\dHead(s) = (s', 5, n)$ and $\dDataStarSq^{(n)}(s') = (s'', l)$}\\
  (s', \mathtt{List}\: l) & \text{if $\dDataStar(s) = (s', l)$}\\
  (s'', \mathtt{Constr}\: i \, l) & \text{if $\dcTag(s) = (s', i)$ and $\dDataStar(s') = (s'', l)$}\\
  (s', \mathtt{I}\: n) & \text{if $\dZ(s) = (s', n)$}\\
  (s', \mathtt{B}\: b) & \text{if $\dBS(s) = (s', b)$}
\end{cases}
$$
where
$$
\dDataStar(s) =
\begin{cases}
  \dDataStar^{(n)}(s') & \text{if $\dHead(s) = (s', 4, n)$}\\
  \dDataStar^{\mathsf{indef}}(s') & \text{if $\dIndef(s) = (s', 4)$}
\end{cases}
$$

$$
\dDataStar^{(n)}(s) =
\begin{cases}
  (s, \epsilon) & \text{if $n = 0$}\\
  (s'', d \cdot l) & \text{if $\dData(s) = (s', d)$ and $\dDataStar^{(n-1)}(s') = (s'', l)$}\\
\end{cases}
$$

$$
\dDataStar^{\mathsf{indef}}(s) =
\begin{cases}
  (s', \epsilon) & \text{if $s = 255 \cdot s' $}\\
  (s'', d \cdot l) & \text{if $\dData(s) = (s', d)$ and $\dDataStar^{\mathsf{indef}}(s') = (s'', l)$}\\
\end{cases}
$$


\medskip
$$
\dDataStarSq^{(n)}(s) =
\begin{cases}
  (s, \epsilon) & \text{if $n=0$}\\
  (s''', (k,d) \cdot l) &
  \begin{cases}
    \text{if $n > 0$}\\
    \text{and $\dData(s) = (s', k)$}\\
    \text{and $\dData(s') = (s'', d)$}\\
    \text{and $\dDataStarSq^{(n-1)}(s'') = (s''', l)$}
  \end{cases}
\end{cases}
$$

\medskip
$$
\dcTag(s) =
\begin{cases}
  (s', i-121) & \text{if $\dHead(s) = (s', 6, i)$ and $121 \leq i \leq 127$}\\
  (s', (i-1280)+7) & \text{if $\dHead(s) = (s', 6, i)$ and $1280 \leq i \leq 1400$}\\
  (s''', i) &
  \begin{cases}
    \text{if $\dHead(s) = (s', 6, 102)$}\\
    \text{and $\dHead(s') = (s'', 4, 2)$}\\
    \text{and $\dZ(s'') = (s''', i)$}\\
    \text{and $0 \leq i \leq 2^{64}-1$}.
    \end{cases}\\
  \end{cases}
$$

\noindent
Note that the decoders for \texttt{List} and \texttt{Constr} accept both
definite-length and indefinite-length lists of encoded \texttt{data} values, but
the decoder for \texttt{Map} only accepts definite-length lists (and the length
is the number of \textit{pairs} in the map).  This is consistent with CBOR's
standard encoding of arrays and lists (major type 4) and maps (major type 5).

Note also that the encoder $\ecTag$ accepts arbitrary integer values for
\texttt{Constr} tags, but (for compatibility with~\cite{CBOR-alternatives}) the
decoder $\dcTag$ only accepts tags in $\Nab{0}{2^{64}-1}$.  This
means that some valid Plutus Core programs can be serialised but not
deserialised, and is the reason for the recommendation in
Section~\ref{sec:built-in-types-1} that only constructor tags between 0 and
$2^{64}-1$ should be used.
