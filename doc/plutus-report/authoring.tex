\section{Authoring: the Plutus Haskell SDK}
\label{sec:sdk}

Just like web applications, \glspl{app} have two separate components which are deployed in separate execution environments:
\begin{itemize}
\item \gls{on-chain} which is stored on the blockchain and executed during the validation of new transactions (similar to the server component of a web application), and
\item \gls{off-chain} code which is deployed to and executed on the client machine of a blockchain user with access to the user’s wallet (similar to the client portion of a web application running in a user’s web browser\footnote{
In fact, in other systems off-chain code often executes in a web browser as well!
})
\end{itemize}

Why is this decomposition necessary? The on-chain code contains the \gls{app}'s enforceable components.
It needs to enforce that only transactions that meet the enforceable obligations are successfully validated and added to the chain.
In other words, the integrity of an \gls{app} depends on the integrity of the on-chain code: thus we need to store it on the cryptographically immutable blockchain to prevent tampering.
Moreover, \glspl{slot-leader} need to execute on-chain code to check that it is in fact satisfied.

Conversely, the \gls{off-chain}, which submits new transactions to the chain for validation and inclusion, necessarily needs to run in close association with a contract user’s wallet.
After all, each transaction needs to be paid for with transaction fees, and a wallet is the only place where the necessary credentials are held (anything else would compromise the security of the funds).

Existing blockchains and their smart contract and dapp frameworks use separate languages for the \gls{on-chain} and \gls{off-chain} (in Ethereum, Solidity and JavaScript), and they tend to invent new languages for the on-chain component (e.g., Solidity).
This comes with the same disadvantages as using different languages for the client and server component of web apps.
However, when new languages are invented the situation is even worse because of the enormous overhead involved in creating a new language, compilers and other tools, libraries, teaching material, and generally growing a new language community. We would like to avoid this (\cref{req:source-lang-conservatism}).

We overcome these problems by using Haskell for both \gls{on-chain} and \gls{off-chain} code.
This enables us to build on the existing Haskell ecosystem and to share datatypes and code between the two.
However, a downside of this approach is that GHC forms part of our compilation toolchain, which we do not control and which may change unexpectedly (this can make it hard to satisfy \cref{req:compilation-stability,req:compilation-reproducibility}, for example).

The \gls{plutus-sdk} is our library and tooling support for writing \glspl{app} --- both their \gls{on-chain} and their \gls{off-chain} together.

\subsection{Requirements}
\begin{requirement}[Conservatism]
\label{req:source-lang-conservatism}
Designing a new programming language is hard.
Designing a new \emph{source} programming language (one written by users directly) is even worse.
One must worry about syntax, tooling, build systems, libraries and their ecosystem, etc.

Ideally, we would like to avoid all this by reusing existing languages as much as possible.
\end{requirement}

\begin{requirement}[Lifting values at runtime]
\label{req:runtime-args}
The programs which we put on the chain cannot be entirely static (i.e. determined at \gls{off-chain} compile time).
It must be possible to parameterize them or partially generate them at runtime.

One reason for this is that we may want to configure our code.
For example, a crowdfunding \gls{app} might want to parameterize its \gls{on-chain} by the crowdfunding target, or the beneficiary of the funds.\footnote{
In some instances one can get away with putting this information in the \gls{datum}.
The crowdfunding example is interesting because many people pay to the \gls{address} spontaneously, and the owner cannot control what those people put in the \glspl{datum}.
But they can control the \gls{address} people send to, and hence the \gls{script}.
So it is necessary to bake the parameters into the \gls{script} in this instance.
}
How can we parameterize a \gls{plutus-core} program?
One way is to write the program as a function, compile that function statically, then construct the argument at runtime and apply the program to the argument.
In pseudo-Haskell:
\begin{quote}
\begin{verbatim}
compile (\parameter -> ...) `apply` lift argument
\end{verbatim}
\end{quote}
\noindent This requires a way to ``lift'' a runtime value into an appropriate \gls{plutus-core} term so we can actually apply our compiled program to it.

This is just generally a very handy thing to be able to do and we want to be able to do it.
\end{requirement}

\begin{requirement}[Stability]
\label{req:compilation-stability}
When the \gls{plutus-core} program that makes up the \gls{on-chain} of a \gls{app} changes, so does its hash, and hence its \gls{address}.
This can be a big problem for applications: you cannot spend a \gls{script-output} without presenting a \gls{script} with \emph{exactly} that hash.
If your tooling won't produce such a \gls{script} any more, then you can't get the money!

We therefore want our tooling to be as stable and deterministic as possible, so we don't change the output unnecessarily.

At the very least, we must not be sensitive to:
\begin{itemize}
\item The platform we are working on (linux/macos etc.)
\item Any random or mutable conditions
\end{itemize}
\end{requirement}

\begin{requirement}[Compilation reproducibility]
\label{req:compilation-reproducibility}
As discussed in \cref{req:compilation-stability}, if the output of compilation changes unexpectedly, then that can be a big problem.
But if the user changes their source or their tooling (e.g. their version of GHC), then that may just genuinely change the input to our compiler.

In this instance there is not a great deal we can do at a technical level, except help people to build their applications reproducibly, so that
they can at least revert to previous states reliably.
\end{requirement}

\subsection{Haskell}

We need a source language for users to write \glspl{app} in.
We don't want to write our own one (\cref{req:source-lang-conservatism}), so we would really like to reuse an existing one.
We decided to use Haskell for a number of reasons:
\begin{itemize}
\item
  It is a powerful functional programming language.
\item
  It has an industrial-grade compiler, adequate tooling, and a good community and ecosystem.
\item
  It has good metaprogramming facilities.
\item
  We are familiar with it as a team and a company.
\end{itemize}

However, this means we need a way to compile (as subsection of) a Haskell program into \gls{plutus-core}.
Our solution to this is \gls{plutus-tx}.

\subsection{Plutus Tx}
\label{sec:plutus-tx}

The \gls{plutus-tx} compiler compiles \gls{ghc-core} into \gls{plutus-core}.
That is, it takes Haskell after it has been desugared from its source representation into GHC's internal representation (\gls{ghc-core}), and compiles that further.
This approach allows us to support all of source Haskell while only having to deal with the much smaller \gls{ghc-core} language.

\subsubsection{Plugins for custom compilation}

GHC core-to-core plugins enable us to inject our own code including the \gls{plutus-tx} compiler into the GHC pipeline.
The \gls{plutus-tx} plugin,
\begin{enumerate}
\item locates \gls{ghc-core} fragments representing to \gls{on-chain},
\item compiles them to \gls{plutus-core}, and
\item replaces each \gls{ghc-core} AST subtree representing \gls{on-chain} code with an AST representing a serialised version of the generated \gls{plutus-core}.
\end{enumerate}

Overall, we end up with compiled \gls{off-chain} that embeds blobs of \gls{on-chain} in its serialised \gls{plutus-core} representation, ready to be submitted to the blockchain attached to transactions generated by the \gls{off-chain}.

There just seem to be two problems:
\begin{inparaenum}
\item how does the plugin identify on-chain code and
\item how do we ensure that the type of the serialised on-chain code lines up with the source code?\footnote{
\Gls{ghc-core} is a typed intermediate language; hence, any code transformation needs to be type-preserving.
}
\end{inparaenum}
We achieve this using a trick that to the best of our knowledge was first used in the \code{inline-java} package embedding Java into Haskell.
This packages uses GHC plugins to extract type information at a Template Haskell splice point \autocite{inline-java-blog-post}.
The idea is to wrap the target Haskell code inside a splice of a Template Haskell function that inserts a marker around that AST fragment.

This function does not actually compile the AST of the target program fragment.
Instead, it inserts a marker function that is picked up by the \gls{plutus-tx} compiler injected with the plugin.
When the plugin runs, it finds the marker, compiles the code, and inserts the serialized form back into the program.
By taking a little care over the types of our marker function, we can ensure that the expression in question remains well-typed at each stage of this process.

\subsubsection{Compiling GHC Core to Plutus Core}

Both \gls{ghc-core} and \gls{plutus-core} are extensions of \gls{system-f}.
\Gls{ghc-core} is a much more generous extension.
It adds mutually-recursive binding groups, algebraic data types, case expressions, coercions, and more.
In contrast, \gls{plutus-core}, as discussed in \cref{sec:plutus-core}, is much more minimal.

How do we deal with the extra features of \gls{ghc-core}?
First, we split the problem in half, by defining an intermediate language, \gls{plutus-ir} (see \cref{sec:plutus-ir}), which is much closer to \gls{ghc-core}.
Most of the theoretical complexity is therefore moved to the \gls{plutus-ir} compiler.

The remaining work of the \gls{plutus-tx} compiler is then to \emph{lower} the \gls{ghc-core} terms and types into their corresponding \gls{plutus-ir} variants, emitting errors as appropriate if we encounter features we do not support.

\subsubsection{Supporting Haskell's features}

As alluded to in the previous section, we do not support the entirety of Haskell.
Thanks to the design of GHC, we get a great deal for free, as we compile programs after they have been converted to \gls{ghc-core}, which means that most of the complex source-level features of Haskell have already been desugared into a smaller set of simpler features.

While we support most ``standard'' Haskell, there are quite a few things we do not support. A non-exhaustive list of features that we do not support is:
\begin{itemize}
\item Not implemented yet
  \begin{itemize}
  \item Mutually recursive datatypes (should be done by release)
  \end{itemize}
\item Incompatible with the design of \gls{plutus-core}
  \begin{itemize}
    \item \code{PolyKinds}, \code{DataKinds}, anything that moves towards ``Dependent Haskell''
  \end{itemize}
\item Technically difficult
  \begin{itemize}
  \item Literal patterns
  \end{itemize}
\item Requires access to function definitions (might be fixed with some GHC work)
  \begin{itemize}
  \item Function usage without \code{INLINEABLE} or \code{-fexpose-all-unfoldings}
  \item Typeclass dictionaries
  \end{itemize}
\item Use of coercions required
  \begin{itemize}
  \item GADTs
  \item \code{Data.Coerce}
  \item \code{DerivingVia}, \code{GeneralizedNewtypeDeriving}, etc.
  \end{itemize}
\item Assumes ``normal'' codegen
  \begin{itemize}
  \item FFI
  \item Numeric types other than integers
  \item Unlifted/\code{MagicHash} types
  \item Machine words, C strings, etc.
  \end{itemize}
\end{itemize}

\subsubsection{Strictness}
\label{sec:plutus-tx-strictness}

Haskell is a lazy language and \gls{plutus-core} is a strict language.
How can we compile a lazy language into a strict language efficiently?

The answer is that we handle this partially.
We generally compile Haskell as though it were strict, but the key exception is for non-value let-bindings.
That is, if we see a let-binding whose right-hand-side is not a value (i.e. may evaluate further), then we compile it as a non-strict let-binding (see \cref{sec:pir-non-strict}).

Unfortunately we have no proof that this approach is sound, which is an area for future work.

\subsubsection{Lifting values at runtime}
\label{sec:plutus-tx-lifting}

We are going to great lengths to compile \gls{on-chain} \glspl{validator} at \gls{off-chain} compile time.
However, we may also need to create some \gls{plutus-core} programs from \emph{runtime} values (\cref{req:runtime-args}).

Unfortunately we cannot use the main \gls{plutus-tx} compiler for this: the \gls{plutus-tx} compiler turns Haskell \emph{programs} represented as \gls{ghc-core} into \gls{plutus-core}.
It cannot do anything with the \emph{runtime} representation of a Haskell value!

We therefore need to replicate what we \emph{would} do with the \gls{plutus-tx} compiler, but at runtime.
Fortunately, we can reuse the \gls{plutus-ir} compiler, which helps a lot.
We define a pair of typeclasses inspired by the Haskell typeclasses for ``lifting'' runtime values into metaprograms: \code{Lift} and \code{Typeable}.
Our typeclasses look something like this (\code{Term} and \code{Type} are the types for \gls{plutus-ir} terms and types; class constraints on the methods are omitted for simplicity):

\begin{quote}
\begin{verbatim}
class Lift a where
    lift :: (...) => a -> m (Term TyName Name ())

class Typeable a where
    typeRep :: (...) => Proxy a -> m (Type TyName ())
\end{verbatim}
\end{quote}

With some effort, we are able to generate instances for these with Template Haskell, so the burden on users is minimal.

Why do we output \gls{plutus-ir} here, rather than running the \gls{plutus-ir} compiler each time and just producing \gls{plutus-core}?
The reason is that the \gls{plutus-ir} compiler has some support for \emph{sharing} definitions, and it is important that programs generated from multiple calls to \code{lift} (e.g. from one implementation calling another, as is common) share the same definition of their shared types.

\subsection{Plutus IR}
\label{sec:plutus-ir}

\Gls{plutus-ir} is an intermediate language that sits between \gls{ghc-core} and \gls{plutus-core}.
Many of the core ideas are published in \textcite{peytonjones2019unraveling}, including the complex parts of compilation and the typesystem.

We give a high-level overview of the language here, further details can be found in the above reference.

\Gls{plutus-ir} is essentially \gls{plutus-core}, but with the addition of:
\begin{itemize}
\item Datatypes, including recursive and mutually recursive datatypes
\item Let terms, including recursive and mutually recursive bindings
\end{itemize}

Compiling recursive datatypes and recursive values are the two trickiest compilation problems, and are covered in \textcite{peytonjones2019unraveling}.

\subsubsection{Non-strict let bindings}
\label{sec:pir-non-strict}

\Gls{plutus-ir} has an additional feature which isn't discussed in \textcite{peytonjones2019unraveling}: non-strict let-bindings.
By default (and in the paper), let-bindings are \emph{strict}, meaning that the right-hand-side of the binding is evaluated before the body of the term.

However, it is useful to support \emph{non-strict} let-bindings, particularly because these correspond more closely to the semantics of Haskell (see \cref{sec:plutus-tx-strictness}).
We can desugar these into strict let-bindings simply by inserting a \code{delay} on the binding right-hand-side and a \code{force} at every use site.

\subsubsection{Optimization}

We do a small amount of optimization in the \gls{plutus-ir} compilation pipeline.
We don't want to do too much in case we make the generated code too unstable (\cref{req:compilation-stability}).

\paragraph{Dead code elimination}
\label{para:dead-code}

Dead code elimination is a straightforward optimization and close to a clear win:
\begin{inparaenum}
\item it reduces code size,
\item it makes the code easier to read, and
\item it has no effect on the semantics.
\end{inparaenum}

It is particularly helpful as the \gls{plutus-tx} compiler introduces definitions for all the builtins, some of which will be unused.

\subsubsection{Compilation}

The \gls{plutus-ir} compiler works via a series of small passes that eliminate individual features of \gls{plutus-ir} in turn, until the remaining program is pure \gls{plutus-core} and can simply be lowered into that AST type.

The passes are:
\begin{itemize}
\item Non-strict let-bindings into strict let-bindings by inserting thunks
\item Type bindings and datatypes into simple type and lambda abstractions
\item Recursive term bindings into non-recursive term bindings
  \begin{itemize}
  \item We do another dead code elimination pass (\cref{para:dead-code}) as this can introduce dead bindings.
  \end{itemize}
\item Non-recursive term bindings into lambda abstractions
\end{itemize}

\subsection{Cross-compilation}
\label{sec:cross-compilation}

To support \cref{req:app-dist}, we want to be able to compile \glspl{app} into easily redistributable \glspl{app-exe}.

The current approach is to target Javascript or WebAssembly as our format for distribution, and leverage cross-compilation of Haskell to actually produce the executables.

\subsubsection{Cross-compilers}

At the moment IOHK is working on two cross-compilation efforts:
\begin{description}
  \item[GHCJS] GHCJS is a Haskell cross-compiler which targets Javascript \autocite{ghcjs-repo}.
  \item[Asterius] Asterius is a Haskell cross-compiler which targets WebAssembly \autocite{asterius-repo}.
\end{description}

\noindent We may use either or both of these in the end.

\subsubsection{haskell.nix}

Cross-compilation of Haskell projects is not easy.
Neither of the major Haskell build tools (\code{cabal} and \code{stack}) support cross-compilation well.

To address this issue, IOHK has developed the \code{haskell.nix} framework for building Haskell projects using Nix \autocite{haskell-nix-repo}.
In addition to supporting cross-compilation well, Nix is well-suited to ensuring that builds are reproducible, which helps with \cref{req:compilation-reproducibility}.

\subsection{Developer tooling}
\label{sec:tooling}

Since the \gls{plutus-sdk} uses Haskell for development, there is much less need to create specialized development tooling, since generic Haskell tooling will work perfectly well.\footnote{
It is true that Haskell development tooling is generally considered to not be very good.
However, it is improving rapidly, and while it might be sensible for IOHK to contribute to the community's efforts on this front, that will be significantly less work than developing completely new tools.
}

It is possible that we may want to develop some tools, particularly for testing and visualization, but this has not been decided yet.
