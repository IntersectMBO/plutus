\section{Related work}
\label{sec:related}

Bitcoin Covenants~\cite{moser2016bitcoin} allow Bitcoin transactions
to restrict how the transferred value can be used in the future,
including propagating themselves to ongoing outputs. This provides
contract continuity and allows the implementation of simple state
machines. Our work is inspired by Covenants, although our addition of
a datum is novel and simplifies the state passing.

The Bitcoin Modelling Language (BitML)~\cite{bitml} is an idealistic process calculus
that specifically targets smart contracts running on Bitcoin.
The semantics of BitML contracts essentially comprise a (labelled) \textit{transition system}, aka a state machine.
Nonetheless, due to the constrained nature of the plain \UTXO{} model without any extensions,
the construction is far from straightforward and requires quite a bit of off-chain communication to set everything up.
Most importantly, the BitML compilation scheme only concerns a restricted form of state machines,
while ours deals with a more generic form that admits any user-defined type of states and inputs.
BitML builds upon an abstract model of Bitcoin transactions by the same
authors~\cite{formal-model-of-bitcoin-transactions};
one of our main contributions is an extended version of such an abstract model,
which also accounts for the added functionality apparent in \Cardano{}.

Ethereum and its smart contract language, Solidity~\cite{Solidity}, are powerful
enough to implement state machines, due to their native support for
global contract instances and state. However, this approach has some major downsides,
notably that contract state is global, and must be kept indefinitely by all core nodes.
In the \EUTXO{} model, contract state is localised to where it is used, and
it is the responsibility of clients to manage it.

Scilla~\cite{scilla} is a intermediate-level language for writing smart
contracts as state machines. It compiles to Solidity and is amendable to formal verification.
Since Scilla supports the asynchronous messaging capabilities of Ethereum,
Scilla contracts correspond to a richer class of automata, called
\textit{Communicating State Transition Systems}~\cite{csta}.
In the future, we plan to formally compare this class of state machines with our own class of \CEM{}s,
which would also pave the way to a systematic comparison of Ethereum's account-based model against \Cardano{}'s \UTXO{}-based one.

Finally, there has been an attempt to model Bitcoin contracts using \textit{timed automata}~\cite{timed-btc},
which enables semi-automatic verification using the UPPAAL model checker~\cite{uppaal}.
While this provides a pragmatic way to verify temporal properties of concrete smart contracts,
there is no formal claim that this class of automata actually corresponds to the semantics of Bitcoin smart contracts.
In contrast, our bisimulation proof achieves the bridging of this semantic gap.
