\section{Related work}
\label{sec:related-work}

We discuss other efforts to use state machines to model smart
contracts in our previous paper~\cite{eutxo-1-paper} and we compare our
 approach to a multi-asset ledger with other systems (including Waves~\cite{waves}, Stellar~\cite{stellar}, and Zilliqa~\cite{scilla-arxiv}) in the companion
paper~\cite{plain-multicurrency}.
Here we focus on approaches that reason formally about the
properties of such contracts as state machines, which is the essence
of this paper's main contribution.
We do not know of any other approaches that use tokens as state threads.

\paragraph{Scilla.}
%
Scilla~\cite{scilla} is a intermediate-level language for writing smart
contracts as state machines.
The Scilla authors have used Coq to reason about contracts written in
Scilla, proving a variety of temporal properties such as safety,
liveness, and others; hence their goals are similar to ours.
Since our meta-theory enjoys property preservation over any trace predicate,
we can also formally prove these temporal properties.

However, we are targeting a very different ledger model.
This means that we need to do additional work: the major contribution
of this paper is using tokens to provide state machine instances with
an ``identity'', which comes for free on Ethereum.
Another Ethereum feature that widens the gap between our approaches is
support for asynchronous message passing,
which renders Scilla unsuitable as a source language for a
\UTXO{}-based ledger,
and explains the different choice of \textit{communicating automata}~\cite{scilla-arxiv}
as the backbone of its model.
Nonetheless, it would be interesting to develop a Scilla-like language
that was suitable for our ledger model.

\paragraph{BitML.}
%
The \textit{Bitcoin Modelling Language} (BitML)~\cite{bitml} allows the definition
of smart contracts running on Bitcoin by means of a restricted class
of state machines.
The BitML compilation process has been proven to be computationally
sound (although this proof has not yet been mechanised), which allows
trace-based properties about the BitML contract to be transferred to
the implementation, as in our system.
This proof is used, for example, in~\cite{atzei2019developing} to prove
and transfer LTL properties of BitML contracts to their implementations.
Most importantly, LTL formulas can be automatically verified using a dedicated \textit{model checker}.

Again, our work is closely related in spirit, although our ledger
model is different and we use a more expressive class of state
machines. In the future, we plan to add support for LTL formulas in our framework.

\paragraph{VeriSolid.}
%
VeriSolid~\cite{mavridou2019verisolid} synthesises Solidity smart
contracts from a state machine specification, and verifies temporal
properties of the state machine using CTL.
They use this to prove safety, liveness, deadlock-freedom, and others.
Again, we expect to support CTL formulas in the near future.

In contrast, the present work focuses on establishing a formal
connection between the state machine model and the real implementation
on the ledger --- in particular, our proofs are mechanised.
We also target a \UTXO\ ledger model, whereas VeriSolid targets the Ethereum ledger.
Finally, our approach is agnostic about the logic or checker used to
prove the properties that we assert on state machines and, by way of
the results in this paper, transfer to an \EUTXOma\ implementation of
the same state machine.

\subsubsection*{Acknowledgments.} We thank Gabriele Keller for comments on an earlier version of this paper.
