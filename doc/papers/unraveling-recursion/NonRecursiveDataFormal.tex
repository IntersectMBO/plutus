\subsection{Non-recursive datatype bindings}
\label{sec:non-recursive-data}

Non-recursive datatypes are fairly easy to compile. We will generalize the
Scott-encoding approach described in \cref{sec:data-encoding}.

We define the compilation scheme for non-recursive datatype bindings in
\cref{fig:compile-datatypes}, along with a number of auxiliary functions in
addition to those in \cref{fig:fir_aux}.

\begin{figure}[!t]
  \centering
  \begin{displaymath}
  \begin{array}{l@{\ }l@{\ }l}
  \multicolumn{3}{l}{\textsf{Throughout this figure when $d$ or $c$ is an argument}}\\
  \multicolumn{3}{l}{d = \datatype{X}{(\seq{Y :: K})}{x}{(\seq{c})}} \\
  \multicolumn{3}{l}{c = x(\seq{T})}\\
  \\
  \multicolumn{3}{l}{\textsc{Auxiliary functions}}\\
  \unveil{d}{t}
  &=& \substIn{X}{\scottTy{d}}{t} \\
  \constr{d}{c}{k}
  &=& \unveil{d}{\Lambda (\seq{Y::K}) . 
  \lambda (\seq{a : T}). 
  \Lambda R . 
  \lambda (\seq{b : \branchTy{c}{\fixed{R}}}) 
  ~b_k ~ \seq{a}}\\
  \constrs{d} &=& \seq{\constr{\fixed{d}}{c_j}{j}}^j\\
  \match{d}
  &=& \Lambda (\seq{Y::K}). \lambda (x : (\scottTy{d}\ \seq{Y}) . x\\
  \\
  \multicolumn{3}{l}{\textsc{Compilation function}}\\
  \multicolumn{3}{l}{\compiledata(\tlet\, d\, \tin\, t)}\\
  &=& (\Lambda (\dataBind{d}) . \lambda (\constrBinds{d}) . \lambda (\matchBind{d}) . t)\\
  &&\{\scottTy{d}\}\ \\
  &&\constrs{d}\\
  &&\match{d}
  \end{array}
  \end{displaymath}

  \captionof{figure}{Compilation of non-recursive datatype bindings}
  \label{fig:compile-datatypes}
\end{figure}

\noindent Let's go through the auxiliary functions in turn (both those in \cref{fig:fir_aux} and
\cref{fig:compile-datatypes}), using the Maybe datatype as an example.
\begin{displaymath}
d \defeq \datatype{\Maybe}{A}{\Match}{(\Nothing (), \Just (A))}
\end{displaymath}
\begin{itemize}
\item $\branchTy{c}{R}$ computes the type of a function which consumes all the
  arguments of the given constructor, producing a value of type $R$.
  \begin{align*}
  \branchTy{\Nothing ()}{R} &= R\\
  \branchTy{\Just A}{R} &= A \rightarrow R
  \end{align*}
\item $\dataKind{d}$ computes the kind of the datatype type. This is a kind
  arrow from the kinds of all the type arguments to $\Type$.
  $$\dataKind{\Maybe} = \Type \kindArrow \Type$$
\item $\scottTy{d}$ computes the Scott-encoded datatype. This binds the type
  variables with a lambda and then constructs the pattern matching function type
  using the branch types.
  $$\scottTy{d} = \lambda A . \forall R . R \rightarrow (A \rightarrow R) \rightarrow R$$
\item $\constrTy{c}{T}$ computes the type of a constructor of the datatype $d$.
  \begin{align*}
  \constrTy{\Nothing ()}{\Maybe} &= \forall A . \Maybe A\\
  \constrTy{\Just A}{\Maybe} &= \forall A . A \rightarrow \Maybe A
  \end{align*}
\item $\unveil{d}{t}$ ``unveils'' the datatype inside a type or term, replacing
  the abstract definition with the concrete, Scott-encoded one. We apply this to
  the \emph{definition} of the constructors, for a reason we will see shortly.
  This makes no difference for non-recursive datatypes, but will matter for
  recursive ones.
  $$\unveil{d}{t} = \substIn{\Maybe}{\lambda A . \forall R . R \rightarrow (A \rightarrow R) \rightarrow R}{t}$$
\item $\constr{d}{c}{k}$ computes the definition of the $k$th constructor of a datatype. To match
  the signature of the constructor, this is type abstracted over the type
  variables and takes arguments corresponding to each of the constructor
  arguments. Then it constructs a pattern matching function which takes branches
  for each alternative and uses the $k$th branch on the constructor arguments.
  \begin{align*}
  \constr{d}{\Nothing ()}{1} &= \Lambda A . \Lambda R . \lambda (b_1 : R) (b_2 : A -> R) . b_1\\
  \constr{d}{\Just (A)}{2} &= \Lambda A . \lambda (v : A) . \Lambda R . \lambda (b_1 : R) (b_2 : A -> R) . b_2\ v
  \end{align*}
\item $\matchTy{d}$ computes the type of the datatype matcher, which converts
  from the abstract datatype to a pattern-matching function \textemdash{} that
  is, precisely the Scott-encoded type.
  $$\matchTy{d} = \forall A . \Maybe A \rightarrow (\forall R . R \rightarrow (A \rightarrow R) \rightarrow R)$$
\item $\match{d}$ computes the definition of the matcher of the datatype, which is 
  the identity.
  $$\match{d} = \Lambda A . \lambda (v : \Maybe A) . v$$
\end{itemize}

\noindent The basic idea of the compilation scheme itself is straightforward:
use type abstraction and lambda abstraction to bind names for the
type itself, its constructors, and its match function.

There is one quirk: usually when encoding let-bindings we create an
\emph{immediately} applied type- or lambda-abstraction, but here they are
\emph{interleaved}. The reason for this is that the datatype must be
abstract inside the \emph{signature} of the constructors and the match
function, since otherwise any uses of those functions inside the body will not
typecheck. But inside the \emph{definitions} the datatype must be concrete, since
the definitions make use of the concrete structure of the type.
This explains why we needed to use $\unveil{d}{t}$ on the definitions of the
constructors, since they appear outside the scope in which we define the
abstract type. Note that this means we really must perform a
substitution rather than creating a let-binding, since that would simply create
another abstract type.\footnote{It is
well-known that abstract datatypes can be encoded with existential types (\cite{mitchell1988abstract}). The
presentation we give here is equivalent to using a value of existential type which is immediately
unpacked, and where existential types are given the typical encoding using
universal types.} 

Consider the following example:
\begin{align*}
  &\compiledata(\tlet\, \datatype{\Maybe}{A}{\Match}{(\Nothing (), \Just (A))}\\
  &\quad\quad \tin\, \Match\ \{\Int\}\ (\Just \{\Int\}1)\ 0\ (\lambda x: \Int . x+1))\\
  =\ &(\Lambda (\Maybe :: \Type \kindArrow \Type) . \tag{signature of $\Maybe$}\\
  &\lambda (\Nothing : \forall A . \Maybe A) . \tag{signature of $\Nothing$}\\
  &\lambda (\Just : \forall A . A \rightarrow \Maybe A) . \tag{siganture of $\Just$}\\
  &\lambda (\Match : \forall A . \Maybe A \rightarrow \forall R . R \rightarrow (A \rightarrow R) \rightarrow R). \tag{signature of $\Match$}\\
  &\Match\ \{\Int\}\ (\Just \{\Int\}1)\ 0\ (\lambda x: \Int . x+1)) \tag{body of the let}\\
  &(\lambda A . \forall R . R \rightarrow (A \rightarrow R) \rightarrow R) \tag{definition of $\Maybe$}\\
  &(\Lambda A . \Lambda R . \lambda (b_1 : R)\ (b_2 : A \rightarrow R) . b_1) \tag{definition of $\Nothing$}\\
  &(\Lambda A . \lambda (v_1 : A) . \Lambda R . \lambda (b_1 : R)\ (b_2 : A \rightarrow R) . b_2\ v_1) \tag{definition of $\Just$}\\
  &(\Lambda A . \lambda (v : \forall R . R \rightarrow (A \rightarrow R) \rightarrow R) . v) \tag{definition of $\Match$}
\end{align*}

\noindent Here we can see that:
\begin{itemize}
  \item $\Just$ needs to produce the abstract type inside the body of the let, otherwise the
    application of $\Match$ will be ill-typed.
  \item The definition of $\Just$ produces the Scott-encoded type.
  \item $\Match$ maps from the abstract type to the Scott-encoded type inside
    the body of the let.
  \item The definition of $\Match$ is the identity on the Scott-encoded type.
\end{itemize}
