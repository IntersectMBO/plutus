\subsection{Recursive types}

\FOM{} is very powerful, but does not allow us to define (non-positive) recursive
types. Adding a type-level fixed point operator enables us to do
this (see e.g. \cite[Chapter 20]{pierce2002types}).
However, we must make a number of choices about the precise nature of our type-level fixed points. 

\subsubsection{Isorecursive and equirecursive types.}

The first choice we have is between two approaches to exposing the fixpoint
property of our recursive types.
Systems with \emph{equirecursive} types identify $(\fixo f)$ and $f (\fixo f)$; whereas
systems with \emph{isorecursive} types provide an isomorphism between
the two, using a term $\unwrap$ to convert the first into the second, and a term
$\wrap$ for the other direction.

The tradeoff is that equirecursive types add no additional terms to the language, but
have a more complicated metatheory. Indeed, typechecking \FOMF{} with equirecursive types is not
known to be decidable in general (\cite{dreyer:recursive-modules,cai-giarrusso-ostermann}). Isorecursive types, on the other hand,
have a simpler metatheory, but require additional terms. It is not
too important for an IR to be easy to program by hand, so we opt for
isorecursive types, with our witness terms being $\wrap$ and $\unwrap$.

\subsubsection{Choosing an appropriate fixpoint operator.}

We also have a number of options for \emph{which} fixpoint operator to add. The most obvious choice is a fixpoint
operator $\fixo$ which takes fixpoints of type-level endofunctions 
at any kind $K$ (i.e. it has signature $\fixo : (K \kindArrow K) \kindArrow
K$). In contrast, our language \FOMF{} has a fixpoint operator $\ifix$
(``indexed fix'') which allows us to
take fixpoints only at kinds $K \kindArrow \Type$. 

\noindent The key advantage of $\ifix$ over $\fixo$ is that it is much easier to give
fully-synthesizing type rules for $\ifix$. To see this, suppose we had a $\fixo$
operator in our language, with corresponding $\wrap$
and $\unwrap$ terms. We now want to write typing rules for $\wrap$.
However, $\fixo$ allows us to take fixpoints at \emph{arbitrary} kinds, whereas
$\wrap$ and $\unwrap$ are terms, which always have types of kind $\Type$.
Thus, the best we can hope for is to use $\wrap$ and $\unwrap$ with \emph{fully applied} fixed points, i.e.:
\begin{displaymath}
\begin{array}{l@{\ }l@{\ }l@{\ }l@{\ }l}
\wrap_0 f_0            & t &: \fixo f_0 & \quad\textrm{where} & t : f_0\ (\fixo f_0)\\
\wrap_1 f_1  ~ a1      & t &: \fixo f_1\ a1 & \quad\textrm{where} & t : f_1\ (\fixo f_1)\ a1\\
\wrap_2 f_2  ~ a1 ~ a2 & t &: \fixo f_2\ a1\ a2 & \quad\textrm{where} & t : f_2\ (\fixo f_2)\ a1\ a2\\
\dots & & & & 
\end{array}
\end{displaymath}
This must be accounted for in our typing rules for fixed points. 

It is possible to give typing rules for $\wrap$ that will do the right thing
regardless of how the fixpoint type is applied. One approach is to use
\emph{elimination contexts}, which represent the context in which a type will be
eliminated (i.e. applied). This is the approach taken in
\cite{dreyer2005understanding}. However, this incurs a cost, since we cannot
\emph{guess} the elimination context (since type synthesis is bottom-up), so we
must attach elimination contexts to our terms somehow.

An alternative approach is to pick a more restricted fixpoint operator. Using $\ifix$
avoids the problems of $\fixo$: it always produces fixpoints at kind $K \kindArrow
\Type$, which must be applied to precisely one argument of kind $K$ before
producing a type of kind $\Type$. This means we can give relatively straightforward typing rules as shown
in \cref{fig:fir_typing}.

\subsubsection{Adequacy of $\ifix$.}

Perhaps surprisingly, $\ifix$ is powerful enough to give us fixpoints at any
kind $K$. We give a semantic argument here, but the idea is simply 
stated: we can ``CPS-transform'' a kind $K$ into
$(K \kindArrow \Type) \kindArrow \Type$, which then has the correct shape for
$\ifix$.
 
\begin{definition}
  Let $J$ and $K$ be kinds. Then $J$ is a \emph{retract} of $K$ if
  there exist functions $\phi: J \kindArrow K$ and $\psi : K \kindArrow J$
  such that $\psi \circ \phi = id$.
\end{definition}

\begin{proposition}
   \label{prop:retracts-fixed-points}
  Suppose $J$ is a retract of $K$ and there is a fixpoint operator \textnormal{$\fixo_K$} on $K$.
  Then there is fixpoint operator \textnormal{$\fixo_J$} on $J$.
\end{proposition}
\begin{proof}
  Take $\fixo_J(f) = \psi (\fixo_K (\phi \circ f \circ \psi))$.
\end{proof}

\begin{proposition}
  \label{prop:kind-structure}
  Let $K$ be a kind in \FOMF{}. Then there is a unique (possibly empty) sequence
  of kinds $(K_0, \dots, K_n)$ such that $K = \seqKindArr{K}{\Type}$.
\end{proposition}
\begin{proof}
  Simple structural induction.
\end{proof}

\begin{proposition}
  For any kind $K$ in \FOMF{}, $K$ is a retract of $(K \kindArrow \Type)
  \kindArrow \Type$.
\end{proposition}
\begin{proof}
  Let $K = \seqKindArr{K}{\Type}$ (by \cref{prop:kind-structure}), and take
  \begin{align*}
    \phi &: K \kindArrow (K \kindArrow \Type) \kindArrow \Type \\
    \phi &= \lambda (x :: K) . \lambda (f :: K \kindArrow \Type) . f\ x\\
    \psi &: ((K \kindArrow \Type) \kindArrow \Type) \kindArrow K\\
    \psi &= \lambda (w :: (K \kindArrow \Type) \kindArrow \Type) . \lambda (\seq{a :: K}) . w (\lambda (o :: K) . o\ \seq{a})\\
  \end{align*}
\end{proof}

\begin{corollary}
  If there is a fixpoint operator at kind $(K \kindArrow \Type) \kindArrow \Type$ then there is a fixpoint operator at any kind $K$.
\end{corollary}

We can instantiate $\ifix$ with $K \kindArrow \Type$ to get fixpoints at $(K
\kindArrow \Type) \kindArrow \Type$, so $\ifix$ is sufficient to get fixpoints
at any kind.

Note that since our proof relies on \cref{prop:kind-structure}, it will not go
through for arbitrary kinds when there are additional kind forms beyond $\Type$
and $\kindArrow$. However, it will still be true for all kinds of the structure
shown in \cref{prop:kind-structure}.

The fact that retractions preserve the fixed point property is
well-known in the context of algebraic topology: see
~\cite[Exercise 4.7]{fulton1995algebraic} or
~\cite[Proposition 23.9]{bredon1993topology} for example. While retractions
between datatypes are a common tool in theoretical computer science
(see e.g. \cite{stirling}), we have been unable to find a version of
\cref{prop:retracts-fixed-points} in the computer science
literature. Nonetheless, we suspect this to be widely known.
