\documentclass[runningheads]{llncs}

% correct bad hyphenation here
\hyphenation{}

%\usepackage{natbib}
\usepackage{url}

% *** MATHS PACKAGES ***
%
\usepackage[cmex10]{amsmath}
\usepackage{amssymb}
\usepackage{stmaryrd}
%\usepackage{amsthm}

% *** ALIGNMENT PACKAGES ***
%
\usepackage{array}
\usepackage{float}  %% Try to improve placement of figures.  Doesn't work well with subcaption package.
\usepackage{subcaption}
\usepackage{caption}

\usepackage{subfiles}
\usepackage{geometry}
\usepackage{listings}
\usepackage[dvipsnames]{xcolor}
\usepackage{verbatim}
\usepackage{listings}% http://ctan.org/pkg/listings
\lstset{
  basicstyle=\ttfamily,
  mathescape
}
\usepackage{alltt}
\usepackage{paralist}

\usepackage{todonotes}
%\usepackage[disable]{todonotes}

% This has to go at the end of the packages.
\usepackage[colorlinks=true,linkcolor=MidnightBlue,citecolor=ForestGreen,urlcolor=Plum]{hyperref}

% Stuff for splitting figures over page breaks
%\DeclareCaptionLabelFormat{continued}{#1~#2 (Continued)}
%\captionsetup[ContinuedFloat]{labelformat=continued}

% *** MACROS ***

\newcommand\agdaRepo{https://github.com/omelkonian/formal-utxo/tree/ed72}

\newcommand{\todochak}[1]{\todo[inline,color=purple!40,author=chak]{#1}}
\newcommand{\todompj}[1]{\todo[inline,color=yellow!40,author=Michael]{#1}}
\newcommand{\todokwxm}[1]{\todo[inline,color=blue!20,author=kwxm]{#1}}
\newcommand{\todojm}[1]{\todo[inline,color=purple!40,author=Jann]{#1}}

\newcommand{\red}[1]{\textcolor{red}{#1}}
\newcommand{\redfootnote}[1]{\red{\footnote{\red{#1}}}}
\newcommand{\blue}[1]{\textcolor{blue}{#1}}
\newcommand{\bluefootnote}[1]{\blue{\footnote{\blue{#1}}}}

%% A version of ^{\prime} for use in text mode
\makeatletter
\DeclareTextCommand{\textprime}{\encodingdefault}{%
  \mbox{$\m@th'\kern-\scriptspace$}%
}
\makeatother

\newcommand{\code}{\texttt}
\renewcommand{\i}{\textit}  % Just to speed up typing: replace these in the final version
\renewcommand{\t}{\texttt}  % Just to speed up typing: replace these in the final version
\newcommand{\s}{\textsf}  % Just to speed up typing: replace these in the final version
\newcommand{\msf}[1]{\ensuremath{\mathsf{#1}}}
\newcommand{\mi}[1]{\ensuremath{\mathit{#1}}}

%% A figure with rules above and below.
\newcommand\rfskip{3pt}
%\newenvironment{ruledfigure}[1]{\begin{figure}[#1]\hrule\vspace{\rfskip}}{\vspace{\rfskip}\hrule\end{figure}}
\newenvironment{ruledfigure}[1]{\begin{figure}[#1]}{\end{figure}}

%% Various text macros
\newcommand{\true}{\textsf{true}}
\newcommand{\false}{\textsf{false}}

\newcommand{\hash}[1]{\ensuremath{#1^{\#}}}

\newcommand\mapsTo{\ensuremath{\mapsto}}
\newcommand\cL{\ensuremath{\{}}
\newcommand\cR{\ensuremath{\}}}

\newcommand{\List}[1]{\ensuremath{\s{List}[#1]}}
\newcommand{\Set}[1]{\ensuremath{\s{Set}[#1]}}
\newcommand{\FinSet}[1]{\ensuremath{\s{FinSet}[#1]}}
\newcommand{\Interval}[1]{\ensuremath{\s{Interval}[#1]}}
\newcommand{\FinSup}[2]{\ensuremath{\s{FinSup}[#1,\linebreak[0]#2]}}
% ^ \linebeak is to avoid a bad line break when we talk about finite
% maps.  We may be able to remove it in the final version.
\newcommand{\supp}{\msf{supp}}

\newcommand{\FPScript}{\ensuremath{\s{Script}}}
\newcommand{\Script}{\FPScript}
\newcommand{\scriptAddr}{\msf{scriptAddr}}
\newcommand{\ctx}{\ensuremath{\s{Context}}}
\newcommand{\toData}{\ensuremath{\s{toData}}}
\newcommand{\fromData}{\msf{fromData}}

\newcommand{\verify}{\msf{verify}}

\newcommand{\mkContext}{\ensuremath{\s{mkContext}}}

\newcommand{\applyScript}[1]{\ensuremath{\llbracket#1\rrbracket}}

% Macros for eutxo things.
\newcommand{\tx}{\mi{tx}}
\newcommand{\TxId}{\ensuremath{\s{TxId}}}
\newcommand{\txId}{\msf{txId}}
\newcommand{\txrefid}{\mi{id}}
\newcommand{\Address}{\ensuremath{\s{Address}}}
\newcommand{\DataHash}{\ensuremath{\s{DataHash}}}
\newcommand{\hashData}{\msf{dataHash}}
\newcommand{\idx}{\mi{index}}
\newcommand{\inputs}{\mi{inputs}}
\newcommand{\outputs}{\mi{outputs}}
\newcommand{\validityInterval}{\mi{validityInterval}}
\newcommand{\scripts}{\mi{scripts}}
\newcommand{\forge}{\mi{forge}}
\newcommand{\sigs}{\mi{sigs}}
\newcommand{\fee}{\mi{fee}}
\newcommand{\addr}{\mi{addr}}
\newcommand{\pubkey}{\mi{pubkey}}
\newcommand{\val}{\mi{value}}  %% \value is already defined

\newcommand{\validator}{\mi{validator}}
\newcommand{\redeemer}{\mi{redeemer}}
\newcommand{\datum}{\mi{datum}}
\newcommand{\datumHash}{\mi{datumHash}}
\newcommand{\datumWits}{\mi{datumWitnesses}}
\newcommand{\Data}{\ensuremath{\s{Data}}}
\newcommand{\Input}{\ensuremath{\s{Input}}}
\newcommand{\Output}{\ensuremath{\s{Output}}}
\newcommand{\OutputRef}{\ensuremath{\s{OutputRef}}}
\newcommand{\Signature}{\ensuremath{\s{Signature}}}
\newcommand{\Ledger}{\ensuremath{\s{Ledger}}}

\newcommand{\outputref}{\mi{outputRef}}
\newcommand{\txin}{\mi{in}}
\newcommand{\id}{\mi{id}}
\newcommand{\lookupTx}{\msf{lookupTx}}
\newcommand{\getSpent}{\msf{getSpentOutput}}

\newcommand{\consumes}[1]{\msf{consumes(#1)}}
\newcommand{\consumesOne}[1]{\msf{consumesOne(#1)}}
\newcommand{\cid}{\mi{cid}}
\newcommand{\inputValue}{\mi{inputValue}}
\newcommand{\rMin}{r_{\mi{min}}}
\newcommand{\rMax}{r_{\mi{max}}}

\newcommand{\Tick}{\ensuremath{\s{Tick}}}
\newcommand{\currentTick}{\msf{currentTick}}
\newcommand{\spent}{\msf{spentOutputs}}
\newcommand{\unspent}{\msf{unspentOutputs}}
\newcommand{\txunspent}{\msf{unspentTxOutputs}}
\newcommand{\utxotx}{\msf{Tx}}

\newcommand{\Quantity}{\ensuremath{\s{Quantity}}}
\newcommand{\Asset}{\ensuremath{\s{Asset}}}
\newcommand{\Policy}{\ensuremath{\s{PolicyID}}}
\newcommand{\Quantities}{\ensuremath{\s{Quantities}}}
\newcommand{\nativeCur}{\ensuremath{\mathrm{nativeC}}}
\newcommand{\nativeTok}{\ensuremath{\mathrm{nativeT}}}

\newcommand{\PublicKey}{\ensuremath{\s{PubKey}}}
\newcommand{\PrivateKey}{\ensuremath{\s{PrivateKey}}}

\newcommand{\pkey}{\ensuremath{\pi_{\mathsf{p}}}}
\newcommand{\skey}{\ensuremath{\pi_{\mathsf{s}}}}

\newcommand\B{\ensuremath{\mathbb{B}}}
\newcommand\N{\ensuremath{\mathbb{N}}}
\newcommand\Z{\ensuremath{\mathbb{Z}}}
\renewcommand\H{\ensuremath{\mathbb{H}}}
%% \H is usually the Hungarian double acute accent
\newcommand{\emptyBs}{\ensuremath{\emptyset}}

\newcommand{\emptymap}{\ensuremath{\{\}}}

\usepackage{etoolbox}

% For anonymisation
\newtoggle{anonymous}
\toggletrue{anonymous}
\iftoggle{anonymous}{
  \newcommand{\Cardano}{CHAIN}
  \newcommand{\Plutus}{LANG}
}{
  \newcommand{\Cardano}{Cardano}
  \newcommand{\Plutus}{Plutus Core}
}

% Names, for consistency
\newcommand{\UTXO}{UTXO}
\newcommand{\UTXOma}{UTXO$_{\textsf{ma}}$}
\newcommand{\EUTXO}{E\UTXO{}}
\newcommand{\ExUTXO}{Extended \UTXO{}}
\newcommand{\CEM}{CEM}

% relaxed float placement
\renewcommand{\topfraction}{.95}
\renewcommand{\bottomfraction}{.7}
\renewcommand{\textfraction}{.15}
\renewcommand{\floatpagefraction}{.66}
\renewcommand{\dbltopfraction}{.66}
\renewcommand{\dblfloatpagefraction}{.66}
\setcounter{topnumber}{9}
\setcounter{bottomnumber}{9}
\setcounter{totalnumber}{20}
\setcounter{dbltopnumber}{9}

%% ------------- Start of document ------------- %%

\begin{document}

\lstset{
  basicstyle=\ttfamily,
  columns=fullflexible,
  keepspaces=true,
}

\title{\UTXOma: \UTXO\ with Multi-Asset Support}
%\title{\UTXOma: \UTXO\ with Multi-Asset Support\\ --- DRAFT --- DRAFT ---}

% First names are abbreviated in the running head.
% If there are more than two authors, 'et al.' is used.

\author{
  Manuel M.T. Chakravarty\inst{1}
  \and
  James Chapman\inst{1}
  \and
  Kenneth MacKenzie\inst{1}
  \and
  Orestis Melkonian\inst{1,2}
  \and
  Jann M\"uller\inst{1}
  \and
  Michael Peyton Jones\inst{1}
  \and
  Polina Vinogradova\inst{1}
  \and
  Philip Wadler\inst{1,2}
  \and
  Joachim Zahnentferner\inst{3}
}

\authorrunning{Chakravarty et al.}
%\authorrunning{--- DRAFT --- DRAFT --- DRAFT ---}

\institute{
  IOHK,
  \email{firstname.lastname@iohk.io}
  \and
  University of Edinburgh,
  \email{orestis.melkonian@ed.ac.uk, wadler@inf.ed.ac.uk}
  \and
  \email{chimeric.ledgers@protonmail.com}
}

\maketitle

\begin{abstract}
A prominent use case of Ethereum smart contracts is the creation of a wide range of \emph{user-defined tokens} or \emph{assets} by way of smart contracts. User-defined assets are \emph{non-native} on Ethereum; i.e., they are not directly supported by the ledger, but require repetitive custom code. This makes them unnecessarily inefficient, expensive, and complex. It also makes them insecure as numerous incidents on Ethereum have demonstrated. Even without stateful smart contracts, the lack of perfect fungibility of Bitcoin assets allows for implementing user-defined tokens as layer-two solutions, which also adds
an additional layer of complexity.

In this paper, we explore an alternative design based on Bitcoin-style \UTXO\ ledgers. Instead of introducing general scripting capabilities together with the associated security risks, we propose an extension of the \UTXO\ model, where we replace the accounting structure of a single cryptocurrency with a new structure that manages an unbounded number of user-defined, native tokens, which we call \emph{token bundles.} Token creation is controlled by \emph{forging policy scripts} that, just like Bitcoin validator scripts, use a small domain-specific language with bounded computational expressiveness, thus favouring Bitcoin's security and computational austerity. The resulting approach is lightweight, i.e., custom asset creation and transfer is cheap, and it avoids use of any global state in the form of an asset registry or similar.

The proposed \UTXOma\ model and the semantics of the scripting language have been formalised in the Agda proof assistant.
\end{abstract}

\keywords{blockchain \and \UTXO{} \and native tokens \and functional programming.}

\section{The Plutus Platform}

The \gls{plutus-platform} is a platform for writing \emph{applications} that interact with a \emph{distributed ledger} featuring \emph{scripting capabilities}, in particular the \gls{cardano} blockchain.
A high-level architecture of the \gls{plutus-platform} on \gls{cardano} is shown in \cref{fig:platform-architecture}, with an emphasis on applications.

\paragraph{Ledgers.}
The \gls{plutus-platform} is designed to work with distributed ledgers (henceforth simply ``ledgers'').\footnote{
More specifically, it is designed to work with the \gls{cardano} blockchain, although it could function on other systems that implement the requisite ledger functionality.
}
While the design of the Platform supports writing very simple applications that do nothing but occasionally submit asset-transfer transactions to the ledger, the main focus is on the applications which are enabled by a ledger with scripting functionality.
Specifically, we look at \gls{utxo} ledgers which implement what we call the \gls{eutxo-model}.

The ledger model is described in \cref{sec:ledger}.

\paragraph{Scripting.}
Ledgers without scripting functionality can only support very simple applications.
Much of the interest in applications that interact with blockchains has come from the ability to put some part of the application code \emph{on} the chain itself (such code is generally referred to as a ``\gls{script}''), such that it is guaranteed to be executed correctly as part of the consensus protocol of the blockchain.
This enables applications to have small kernels of ``trusted'' code which ensures that critical safety conditions are met.

The \gls{plutus-platform} requires a particular scripting model from the ledger, which is described with the general ledger model in \cref{sec:ledger}.
This scripting model is agnostic about the scripting \emph{language} which is used for \glspl{script}, but the \gls{plutus-sdk} is designed to work with a particular scripting language, namely \gls{plutus-core}. \Gls{plutus-core} is described in \cref{sec:plutus-core}.

\paragraph{Applications.}
In the \gls{plutus-platform} a \gls{app} is simply a program that interacts with a distributed ledger.
We use the rather generic word ``application'' over the more usual ``smart contract'' or ``distributed application'' to emphasize the generic nature of the programs we are considering --- indeed, they need not scripting functionality or have distributed participants!

The point of the Platform is to enable such applications by providing support in the ledger, as well as support for authoring, distributing, and running them.
While the modifications to the ledger may seem like the most technically substantive part of the work, most of the application behaviour is in the part that runs off the chain, and this requires a commensurate amount of support.

The \gls{app} model is described in \cref{sec:paf}, and our support for authoring \glspl{app} in \cref{sec:sdk}.

\begin{figure}[t]
  \centering
  \includegraphics[width=\textwidth]{platform-architecture.png}
  \caption{Architecture of the \gls{plutus-platform}}
  \label{fig:platform-architecture}
\end{figure}

\section{Multi-Asset Support}
\label{sec:multicurrency}

In Bitcoin's ledger model~\cite{Nakamoto,formal-model-of-bitcoin-transactions,Zahnentferner18-UTxO}, transactions spend as yet \emph{unspent transaction outputs \textup{(}\!\UTXO{}s\textup{)}}, while supplying new unspent outputs to be consumed by subsequent transactions.
Each individual \UTXO\ locks a specific \emph{quantity} of cryptocurrency by imposing specific conditions that need to be met to spend that quantity, such as for example signing the spending transaction with a specific secret cryptographic key, or passing some more sophisticated conditions enforced by a \emph{validator script}.
Quantities of cryptocurrency in a transaction output are represented as an integral number of the smallest unit of that particular cryptocurrency --- in Bitcoin, these are Satoshis.
To natively support multiple currencies in transaction outputs, we generalise those integral quantities to natively support the dynamic creation of new user-defined \emph{assets} or \emph{tokens}. Moreover, we require a means to forge tokens in a manner controlled by an asset's \emph{forging policy}.

We achieve all this by the following three extensions to the basic \UTXO\ ledger model that are further detailed in
the remainder of this section.
%
\begin{enumerate}
\item Transaction outputs lock a \emph{heterogeneous token bundle} instead of only an integral value of one cryptocurrency.
\item We extend transactions with a \emph{forge} field. This is a token bundle of tokens that are created (minted) or destroyed (burned) by that transaction.
\item We introduce \emph{forging policy scripts \textup{(}FPS\textup{)}} that govern the creation and destruction of assets in forge fields. These scripts are not unlike the validators locking outputs in \UTXO.
\end{enumerate}

\subsection{Token bundles}

We can regard transaction outputs in an \UTXO\ ledger as pairs \((\val, \nu)\) consisting of a locked value $\val$ and a validator script $\nu$ that encodes the spending condition. The latter may be proof of ownership by way of signing the spending transaction with a specific secret cryptography key or a temporal condition that allows an output to be spent only when the blockchain has reached a certain height (i.e. a certain number of blocks have been produced).

To conveniently use multiple currencies in transaction outputs, we want each output to be able to lock varying quantities of multiple different currencies at once in its $\val$ field.
This suggests using finite maps from some kind of \emph{asset identifier} to an integral quantity as a concrete representation, e.g. $\texttt{Coin} \mapsto 21$.
Looking at the standard \UTXO\ ledger rules~\cite{Zahnentferner18-UTxO}, it becomes apparent that cryptocurrency quantities need to be monoids.
It is a little tricky to make finite maps into a monoid, but the solution is to think of them as \emph{finitely supported functions} (see Section~\ref{sec:fsfs} for details).

If want to use \emph{finitely supported functions} to achieve a uniform
representation that can handle groups of related, but \emph{non-fungible} tokens, we need to go a step further.
In order to not lose the grouping of related non-fungible tokens (all house tokens issued by a specific entity, for example) though, we need to move to a two-level structure --- i.e., finitely-supported functions of finitely-supported functions. Let's consider an example. Trading of rare in-game items is popular in modern, multi-player computer games. How about representing ownership of such items and trading of that ownership on our multi-asset \UTXO\ ledger? We might need tokens for ``hats'' and ``swords'', which form two non-fungible assets with possibly multiple tokens of each asset --- a hat is interchangeable with any other hat, but not with a sword, and also not with the currency used to purchase these
items. Here our two-level structure pays off in its full generality, and we can represent currency to purchase items together with sets of items, where some can be multiples, e.g.,
%
\begin{align*}
  & \{\mathsf{Coin} \mapsto \{\mathsf{Coin} \mapsto 2\}, \mathsf{Game} \mapsto \{\mathsf{Hat} \mapsto 1, \mathsf{Sword} \mapsto 4\}\} \\
  + \ & \{\mathsf{Coin} \mapsto \{\mathsf{Coin} \mapsto 1\}, \mathsf{Game} \mapsto \{\mathsf{Sword} \mapsto 1, \mathsf{Owl} \mapsto 1\}\} \\
  = \ & \{\mathsf{Coin} \mapsto \{\mathsf{Coin} \mapsto 3\}, \mathsf{Game} \mapsto \{\mathsf{Hat} \mapsto 1, \mathsf{Sword} \mapsto 5, \mathsf{Owl} \mapsto 1\}\} \ .
\end{align*}

\subsection{Forge fields}

If new tokens are frequently generated (such as issuing new hats whenever an in-game achievement has been reached) and destroyed (a player may lose a hat forever if the wind picks up), these operations need to be lightweight and cheap. We achieve this by adding a forge field to every transaction. It is a token bundle (just like the $\val$ in an output), but admits positive quantities (for minting new tokens) and negative quantities (for burning existing tokens). Of course, minting and burning needs to be strictly controlled.

\subsection{Forging policy scripts}

The script validation mechanism for locking \UTXO\ outputs is as follows :
in order to for a transaction to spend an output \((\val, \nu)\), the validator
script $\nu$ needs to be executed and approve of the spending transaction.
Similarly, the forging
policy scripts associated with the tokens being minted or burned by a transaction
are run in order to validate those actions.
In the spirit of the Bitcoin Miniscript approach, we chose to include a simple
scripting language supporting forging policies for several common usecases, such as
single issuer, non-fungible, or one-time issue tokens, etc.
(see Section~\ref{sec:fps-language} for all the usecases).

In order to establish a permanent association between the forging policy and the
assets controlled by it, we propose a hashing approach, as opposed to a global registry
lookup. Such a registry requires a specialized access control scheme, as well
as a scheme for cleaning up unused entries.
In the representation of custom assets we propose, each token is associated with the
hash of the forging policy script required to validate at the time of forging
the token, eg.
in order to forge the value
\(\{\mathsf{HASHVALUE} \mapsto \{\mathsf{Owl} \mapsto 1\}\}\), a script whose
hash is $\mathsf{HASHVALUE}$ will be run.

Relying on permanent hash associations to identify asset forging policies and their assets also has its disadvantages.
For example, policy hashes are long strings that, in our model, will have multiple copies stored on the ledger.
Such strings are not human-readable, take up valuable ledger real estate, and increase transaction-size-based fees.

\section{Formal ledger rules}
\label{sec:model}

\begin{ruledfigure}{t}
  \begin{displaymath}
    \begin{array}{rll}
      \B{} && \mbox{the type of Booleans}\\
      \N{} && \mbox{the type of natural numbers}\\
      \Z{} && \mbox{the type of integers}\\
      \H{} && \mbox{the type of bytestrings: } \bigcup_{n=0}^{\infty}\{0,1\}^{8n}\\
      (\phi_1 : T_1, \ldots, \phi_n : T_n) && \mbox{a record type with fields $\phi_1, \ldots, \phi_n$ of types $T_1, \ldots, T_n$}\\
      t.\phi && \mbox{the value of $\phi$ for $t$, where $t$ has type $T$ and $\phi$ is a field of $T$}\\
      \Set{T} && \mbox{the type of (finite) sets over $T$}\\
      \List{T} && \mbox{the type of lists over $T$, with $\_[\_]$ as indexing and $|\_|$ as length}\\
      h::t && \mbox{the list with head $h$ and tail $t$}\\
      x \mapsto f(x) && \mbox{an anonymous function}\\
      \hash{c} && \mbox{a cryptographic collision-resistant hash of $c$}\\
      \Interval{A} && \mbox{the type of intervals over a totally-ordered set $A$}\\
      \FinSup{K}{M} && \mbox{the type of finitely supported functions from a type $K$ to a monoid $M$}
    \end{array}
  \end{displaymath}
  \caption{Basic types and notation}
  \label{fig:basic-types}
\end{ruledfigure}
%
Our formal ledger model follows the style of the UTXO-with-scripts model from~\cite{Zahnentferner18-UTxO} adopting the notation from~\cite{eutxo-1-paper} with basic types defined as in Figure~\ref{fig:basic-types}.

\paragraph{Finitely-supported functions.}
\label{sec:fsfs}
%
We model token bundles as finitely-supported functions.
If $K$ is any type and $M$ is a monoid with identity element $0$, then a function $f: K \rightarrow M$ is \textit{finitely supported} if $f(k) \ne 0$ for only finitely many $k \in K$.
More precisely, for $f: K \rightarrow M$ we define the \textit{support} of $f$ to be
%%
$\supp(f) = \{k \in K : f(k) \ne 0\}$
%%
and
%%
$\FinSup{K}{M} = \{f : K \rightarrow M : \left|\supp(f)\right| < \infty \}$.
%%

If $(M,+,0)$ is a monoid then $\FinSup{K}{M}$ also becomes a monoid if we define addition pointwise (i.e., $(f+g)(k) = f(k) + g(k)$), with the identity element being the zero map.
Furthermore, if $M$ is an abelian group then $\FinSup{K}{M}$ is also an abelian group under this construction, with $(-f)(k) = -f(k)$.
Similarly, if $M$ is partially ordered, then so is $\FinSup{K}{M}$ with comparison defined pointwise: $f \leq g$ if and only if $f(k) \leq g(k)$ for all $k \in K$.

It follows that if $M$ is a (partially ordered) monoid or abelian group then so is $\FinSup{K}{\FinSup{L}{M}}$ for any two sets of keys $K$ and $L$.
We will make use of this fact in the validation rules presented later in the paper (see Figure~\ref{fig:validity}).
Finitely-supported functions are easily implemented as finite maps, with a failed map lookup corresponding to returning 0.

\subsection{Ledger types}

%
\begin{ruledfigure}{htb}
  \begin{displaymath}
    \begin{array}{rll}
      \multicolumn{3}{l}{\textsc{Ledger primitives}}\\
      \Quantity && \mbox{an amount of currency, forming an abelian group (typically \Z{})}\\
      \Asset && \mbox{a type consisting of identifiers for individual asset classes}\\
      \Tick && \mbox{a tick}\\
      \Address && \mbox{an ``address'' in the blockchain}\\
      \TxId && \mbox{the identifier of a transaction}\\
      \txId : \utxotx \rightarrow \TxId && \mbox{a function computing the identifier of a transaction}\\
      \lookupTx : \Ledger \times \TxId \rightarrow \utxotx && \mbox{retrieve the unique transaction with a given identifier}\\
      \verify : \PublicKey\times\H\times\H \rightarrow \B && \mbox{signature verification}\\
      \FPScript && \mbox{forging policy scripts}\\
      \scriptAddr : \Script \rightarrow \Address && \mbox{the address of a script}\\
      \applyScript{\_} : \Script \to (\Address \times \utxotx \times \Set{\Output}) \to
      \B && \mbox{apply script inside brackets to its arguments}\\
    \\
    \multicolumn{3}{l}{\textsc{Ledger types}} \\
    \Policy &=& \Address \qquad \mbox{(an identifier for a custom currency)}\\
    \Signature &=& \H\\
    \\
    \Quantities   &=& \FinSup{\Policy}{\FinSup{\Asset}{\Quantity}}\\
    \\
    \Output &=& (\addr: \Address, \val: \Quantities)\\
    \\
    \OutputRef &=& (\txrefid: \TxId, \idx: \s{Int})\\
    \\
    \Input &=& ( \outputref : \OutputRef\\
             & &\ \validator: \Script)\\
    \\
    \utxotx &=&(\inputs: \Set{\Input},\\
               & &\ \outputs: \List{\Output},\\
               & &\ \validityInterval: \Interval{\Tick},\\
               & &\ \forge: \Quantities\\
               & &\ \scripts: \Set{\FPScript},\\
               & &\ \sigs: \Set{\Signature})\\
    \\
    \Ledger &=&\!\List{\utxotx}\\
    \end{array}
  \end{displaymath}
  \caption{Ledger primitives and basic types}
  \label{fig:ledger-types}
\end{ruledfigure}
%
Figure~\ref{fig:ledger-types} defines the ledger primitives and types that we need to define the \UTXOma\ model.
All outputs use a pay-to-script-hash scheme, where an output is locked with the hash of a script. We use a single scripting language for forging policies and to define output locking scripts. Just as in Bitcoin, this is a restricted domain-specific language (and not a general-purpose language); the details follow in Section~\ref{sec:fps-language}.
We assume that each transaction has a unique identifier derived from its value by a hash function. This is the basis of the $\lookupTx$ function to look up a transaction, given its unique identifier.

\paragraph{Token bundles.}

We generalise per-output transferred quantities from a plain \Quantity\ to a bundle of \Quantities.
A \Quantities{} represents a token bundle: it is a mapping from a policy and an \emph{asset}, which defines the asset class, to a \Quantity{} of that asset.\footnote{
  We have chosen to represent \Quantities{} as a finitely-supported function whose values are themselves finitely-supported functions (in an implementation, this would be a nested map).
  We did this to make the definition of the rules simpler (in particular Rule~\ref{rule:forging}).
  However, it could equally well be defined as a finitely-supported function from tuples of \Policy{}s and \Asset{}s to \Quantity{}s.
}
Since a \Quantities\ is indexed in this way, it can represent any combination of tokens from any assets (hence why we call it a token \emph{bundle}).

\paragraph{Asset groups and forging policy scripts.}

A key concept is the \emph{asset group}.
An asset group is identified by the hash of special script that controls the creation and destruction of asset tokens of that asset group.
We call this script the \emph{forging policy script}.

\paragraph{Forging.}

Each transaction gets a $\forge$ field, which simply modifies the required balance of the transaction by the $\Quantities$ inside it: thus a positive $\forge$ field indicates the creation of new tokens.
In contrast to outputs, $\Quantities$ in forge fields can
also be negative, which effectively burns existing tokens.\footnote{
The restriction on outputs is enforced by Rule~\ref{rule:all-outputs-are-non-negative}.  We simply do not impose such a restriction on the $\forge$ field: this lets us define rules in a simpler way, with cleaner notation. }

Additionally, transactions get a $\scripts$ field holding a set of forging policy scripts: \(\Set{\FPScript}\).
This provides the forging policy scripts that are required as part of validation when tokens are minted or destroyed (see Rule~\ref{rule:forging} in Figure~\ref{fig:validity}). The forging scripts of the assets being forged are
executed and the transaction is only considered valid if the execution of the script returns $\true$.
A forging policy script is executed in a context that provides access to the main components of the forging transaction, the UTXOs it spends, and the policy ID.
The passing of the context provides a crucial piece of the puzzle regarding self-identification: it includes the script's own $\Policy$, which avoids the problem of trying to include the hash of a script inside itself.

\paragraph{Validity intervals.}
\label{para:validity-intervals}

A transaction's \emph{validity interval} field contains an interval of ticks (monotonically increasing units of ``time'', from~\cite{eutxo-1-paper}).
The validity interval states that the transaction must only be validated if the current tick is within the interval. The validity interval, rather than the actual current chain tick value, must be used
for script validation. In an otherwise valid transaction, passing the current
tick to the evaluator
could result in different script validation outcomes at different ticks, which
would be problematic.

\paragraph{Language clauses.}

In our choice of the set of predicates $\texttt{p1, ..., pn}$ to include in the
scripting language definition, we adhere to the following heuristic: we only
admit predicates with quantification over finite structures
passed to the evaluator in the transaction-specific data, i.e. sets,
maps, and lists. The computations we allow in the predicates themselves are well-known
computable functions, such as hashing, signature checking, arithmetic operations, comparisons,
etc.

The gamut of policies expressible in the model we propose here is
fully determined by the collection of predicates,
assembled into a single script by logical connectives
\texttt{\&\&}, \texttt{||}, and \texttt{Not}.
Despite being made up of only hard-coded predicates and connectives,
the resulting policies can be quite
expressive, as we will demonstrate in the upcoming applications section.
When specifying forging predicates, we use $\texttt{tx.\_}$ notation to access
the fields of a transaction.

\subsection{Transaction validity}
\label{sec:validity}

\begin{ruledfigure}{t}
  \begin{displaymath}
  \begin{array}{lll}
  \multicolumn{3}{l}{\txunspent : \utxotx \rightarrow \Set{\s{OutputRef}}}\\
  \txunspent(t) &=& \{(\txId(t),1), \ldots, (\txId(id),\left|t.outputs\right|)\}\\
  \\
  \multicolumn{3}{l}{\unspent : \s{Ledger} \rightarrow \Set{\s{OutputRef}}}\\
  \unspent([]) &=& \emptymap \\
  \unspent(t::l) &=& (\unspent(l) \setminus t.\inputs) \cup \txunspent(t)\\
  \\
  \multicolumn{3}{l}{\getSpent : \s{Input} \times \s{Ledger} \rightarrow \s{Output}}\\
  \getSpent(i,l) &=& \lookupTx(l, i.\outputref.\id).\outputs[i.\outputref.\idx]
  \end{array}
  \end{displaymath}
  \caption{Auxiliary validation functions}
  \label{fig:validation-functions}
\end{ruledfigure}
%
\begin{ruledfigure}{t}
\begin{enumerate}

\item
  \label{rule:slot-in-range}
  \textbf{The current tick is within the validity interval}
  \begin{displaymath}
    \msf{currentTick} \in t.\i{validityInterval}
  \end{displaymath}

\item
  \label{rule:all-outputs-are-non-negative}
  \textbf{All outputs have non-negative values}
  \begin{displaymath}
    \textrm{For all } o \in t.\outputs,\ o.\val \geq 0
  \end{displaymath}

\item
  \label{rule:all-inputs-refer-to-unspent-outputs}
  \textbf{All inputs refer to unspent outputs}
  \begin{displaymath}
    \{i.\outputref: i \in t.\inputs \} \subseteq \unspent(l).
  \end{displaymath}

\item
  \label{rule:value-is-preserved}
  \textbf{Value is preserved}
  \begin{displaymath}
    t.\forge + \sum_{i \in t.\inputs} \getSpent(i, l) = \sum_{o \in t.\outputs} o.\val
  \end{displaymath}

\item
  \label{rule:no-double-spending}
  \textbf{No output is double spent}
  \begin{displaymath}
    \textrm{If } i_1, i \in t.\inputs \textrm{ and }  i_1.\outputref = i.\outputref
    \textrm{ then } i_1 = i.
  \end{displaymath}

\item
  \label{rule:all-inputs-validate}
  \textbf{All inputs validate}
  \begin{displaymath}
    \textrm{For all } i \in t.\inputs,\
    \applyScript{i.\validator}(\scriptAddr(i.\validator), t,
    \{\getSpent(i, l) ~\vert~ i~\in~ t.\inputs\}) = \true
  \end{displaymath}

\item
  \label{rule:validator-scripts-hash}
  \textbf{Validator scripts match output addresses}
  \begin{displaymath}
    \textrm{For all } i \in t.\inputs,\ \scriptAddr(i.\validator) = \getSpent(i, l).\addr
  \end{displaymath}

\item
  \label{rule:forging}
  \textbf{Forging}\\
  A transaction with a non-zero \forge{} field is only
  valid if either:
  \begin{enumerate}
  \item the ledger $l$ is empty (that is, if it is the initial transaction).
  \item \label{rule:custom-forge}
    for every key $h \in \supp(t.\forge)$, there
    exists $s \in t.\scripts$ with
    $h = \scriptAddr(s)$.
  \end{enumerate}
  \medskip
  % ^ There's no space between these items without this, but all the other items have space due to \displaymath

\item
  \label{rule:all-mps-validate}
  \textbf{All scripts validate}
  \begin{displaymath}
    \textrm{For all } s \in t.\scripts,\ \applyScript{s}(\scriptAddr(s), t,
    \{\getSpent(i, l) ~\vert~ i~\in~ t.\inputs\}) = \true
  \end{displaymath}

\end{enumerate}
\caption{Validity of a transaction $t$ in a ledger $l$}
\label{fig:validity}
\end{ruledfigure}
%
Figure~\ref{fig:validity} defines what it means for a transaction $t$ to be valid for a valid ledger $l$ during the tick \currentTick, using some auxiliary functions from Figure~\ref{fig:validation-functions}. A ledger $l$ is \textit{valid} if either $l$ is empty or $l$ is of the form $t::l^{\prime}$ with $l^{\prime}$ valid and $t$ valid for $l^{\prime}$.

The rules follow the usual structure for an UTXO ledger, with a number of modifications and additions.
The new \textbf{Forging} rule (Rule~\ref{rule:forging}) implements the support for forging policies by requiring that the currency's forging policy is included in the transaction --- along with Rule~\ref{rule:all-mps-validate} which ensures that they are actually run!
The arguments that a script is applied to are the ones discussed earlier.

When forging policy scripts are run, they are provided with the appropriate transaction data, which allows them to enforce conditions on it.
In particular, they can inspect the $\forge$ field on the transaction, and so a forging policy script can identify how much of its own currency was forged, which is typically a key consideration in whether to allow the transaction.

We also need to be careful to ensure that transactions in our new system preserve value correctly.
There are two aspects to consider:
\begin{itemize}
\item
  We generalise the type of value to \Quantities.
  However, since \Quantities\ is a monoid (see Section~\ref{sec:model}), Rule~\ref{rule:value-is-preserved} is (almost) identical to the one in the original UTXO model, simply with a different monoid.
  Concretely, this amounts to preserving the quantities of each of the individual token classes in the transaction.
\item
  We allow forging of new tokens by including the forge field into the balance in Rule~\ref{rule:value-is-preserved}.
\end{itemize}

\section{A stateless forging policy language}
\label{sec:fps-language}

The domain-specific language for forging policies strikes a balance between expressiveness and simplicity. It particular, it is stateless and of bounded computational complexity. Nevertheless, it is sufficient to support the applications described in Section~\ref{sec:applications}.

\paragraph{Semantically meaningful token names.}

The policy ID is associated with a policy script (it is the hash of it),
so it has a semantic meaning that is identified with that of the script.
In the clauses of our language, we give semantic meaning
to the names of the tokens as well. This allows us to make some
judgements about them in a programmatic way, beyond confirming that the
preservation of value holds, or
which ones are fungible with each other.
For example, the $\texttt{FreshTokens}$ constructor gives us a way to programmatically generate
token names which, by construction, mean that these tokens are unique, without
ever checking the global ledger state.

\paragraph{Forging policy scripts as output-locking scripts.}

As with
currency in the non-digital world, it is a harder problem to control the transfer of assets once
they have come into circulation (see also Section~\ref{sec:discussion}). We can, however,
specify directly in the forging policy that the assets being forged must
be locked by an output script of our choosing.
Moreover, since both output addresses and policies are hashes of scripts,
we can use the asset policy ID and the address interchangeably.
The $\texttt{AssetToAddress}$ clause is used for this purpose.

\begin{figure}[t]
    \begin{lstlisting}
    $\llbracket$JustMSig(msig)$\rrbracket$(h, tx, utxo) = checkMultiSig(msig, tx)

    $\llbracket$SpendsOutput(o)$\rrbracket$(h, tx, utxo) = o $\in $ { i.outputRef : i $\in $ tx.inputs }

    $\llbracket$TickAfter(tick1)$\rrbracket$(h, tx, utxo) = tick1 $\leq$ min(tx.validityInterval)

    $\llbracket$Forges(tkns)$\rrbracket$(h, tx, utxo) = (h $\mapsto$ tkns $\in$ tx.forge) && (h $\mapsto$ tkns $\geq$ 0)

    $\llbracket$Burns(tkns)$\rrbracket$(h, tx, utxo) = (h $\mapsto$ tkns $\in$ tx.forge) && (h $\mapsto$ tkns $\leq$ 0)

    $\llbracket$FreshTokens$\rrbracket$(h, tx, utxo) =
       $\forall$ pid $\mapsto$ tkns $\in$ tx.forge, pid == h $\Rightarrow$
         $\forall$ t $\mapsto$ q $\in$ tkns,
           t == hash(indexof(t, tkns), tx.inputs) && q == 1

    $\llbracket$AssetToAddress(addr)$\rrbracket$(h, tx, utxo) =
       $\forall$ pid $\mapsto$ tkns $\in$ utxo.balance, pid == h $\Rightarrow$
          addr == _ $\Rightarrow$ (h, pid $\mapsto$ tkns) $\in$ utxo
        $\wedge$ addr $\neq$ _ $\Rightarrow$ (addr, pid $\mapsto$ tkns) $\in$ utxo

    $\llbracket$DoForge$\rrbracket$(h, tx, utxo) = h $\in$ supp(tx.forge)

    $\llbracket$SignedByPIDToken$\rrbracket$(h, tx, utxo) =
       $\forall$ pid $\mapsto$ tkns $\in$ utxo.balance, pid == h $\Rightarrow$
         $\forall$ s $\in$ tx.sigs, $\exists$ t $\in$ supp(tkns),
           isSignedBy(tx, s, t)

    $\llbracket$SpendsCur(pid)$\rrbracket$(h, tx, utxo) =
         pid == _ $\Rightarrow$ h $\in$ supp(utxo.balance)
       $\wedge$ pid $\neq$ _ $\Rightarrow$ pid $\in$ supp(utxo.balance)
    \end{lstlisting}
    \caption{Forging Policy Language}
    \label{figure:fps-language}
\end{figure}
%
\paragraph{Language clauses.}
%
The various clauses of the validator and forging policy language are as described below, with their formal semantics as in Figure~\ref{figure:fps-language}. In this figure, we use the notation
$\texttt{x}\mapsto \texttt{y}$ to represent a single key-value pair of a finite map.
Recall form Rule~\ref{rule:all-mps-validate} that the arguments passed to the validation
function $\llbracket \texttt{s} \rrbracket$ $\texttt{h}$ are: the hash of the forging (or output
locking) script being validated, the transaction $\texttt{tx}$ being validated,
and the ledger outputs which the transaction $\texttt{tx}$ is spending (we denote
these $\texttt{utxo}$ here).
%
\begin{itemize}
  \item \texttt{JustMSig(msig)}
  verifies that the $m$-out-of-$n$ signatures required by $s$ are in the
  set of signatures provided by the transaction. We do not give the
  multi-signature script evaluator details as this is a common concept, and
  assume a procedure \texttt{checkMultiSig} exists.

  \item \texttt{SpendsOutput(o)} checks that the transaction spends
  the output referenced by $\texttt{o}$ in the \UTXO.

  \item \texttt{TickAfter(tick1)} checks that the validity interval
  of the current transaction starts after time $\texttt{tick1}$.

  \item \texttt{Forges(tkns)} checks that the transaction forges exactly
  $\texttt{tkns}$ of the asset with the policy ID that is being validated.

  \item \texttt{Burns(tkns)} checks that the transaction burns exactly
  $\texttt{tkns}$ of the asset with the policy ID that is being validated.

  \item \texttt{FreshTokens}
  checks that all tokens of the asset being forged are non-fungible.

  This script must check that the names of the tokens in this token bundle
  are generated by hashing some unique data. This data must be unique to both
  the transaction itself and the token within the asset being forged.
  In particular, we can hash a pair of
  %
  \begin{enumerate}
    \item some \emph{output} in the \UTXO\ that the
    transaction consumes, and
    \item the \emph{index} of the token name in (the
    list representation of) the map of tokens being
    forged (under the specific policy, by this transaction). We denote the
    function that gets the index of a key in a key-value map by $\texttt{indexof}$.
  \end{enumerate}

  \item \texttt{AssetToAddress(addr)}
  checks that all the tokens associated with the policy ID that
  is equal to the hash of the script being run are output
  to an \UTXO\ with the address $\texttt{addr}$. In the case that no $\texttt{addr}$ value is
  provided (represented by $\_$), we use the $\texttt{addr}$ value passed to the evaluator as
  the hash of the policy of the asset being forged.

  \item \texttt{DoForge}
  checks that this transaction forges tokens in the bundle controlled by the policy ID that is passed
  to the FPS script evaluator (here, again, we make use of the separate passing of
  the FPS script and the policy ID).

  \item \texttt{SignedByPIDToken(pid)} verifies the hash of every key that has signed the transaction.

  \item \texttt{SpendsCur(pid)}
  verifies that the transaction is spending assets in the token bundle with policy
  ID $\texttt{pid}$ (which is specified as part of the \emph{constructor}, and may be different
  than the policy ID passed to the evaluator).

\end{itemize}

\section{Applications}
\label{sec:applications}

\UTXOma\ is able to support a large number of standard use cases for multi-asset ledgers, as well as some novel ones.
In this section we give a selection of examples.
There are some common themes: (1) Tokens as resources can be used to reify many non-obvious things, which makes them first-class tradeable items; (2) cheap tokens allow us to solve many small problems with \emph{more tokens}; and (3) the power of the scripting language affects what examples can be implemented.

\subsection{Simple single token issuance}
%
To create a simple currency $\mathsf{SimpleCoin}$ with a fixed supply of $\texttt{s = 1000 SimpleCoins}$ tokens, we might try to use the simple policy script $\texttt{Forges(s)}$ with a single forging transaction. Unfortunately, this is not sufficient as somebody else could submit another transaction forging another $\texttt{1000 SimpleCoins}$.

In other words, we need to ensure that there can only ever be a single transaction on the ledger that successfully forges $\mathsf{SimpleCoin}$. We can achieve that by requiring that the forging transaction consumes a specific \UTXO. As \UTXO{}s are guaranteed to be (1) unique and (2) only be spent once, we are being guaranteed that the forging policy can only be used once to forge tokens. We can use the script:

\begin{alltt}
  simple_policy(o, v) = SpendsOutput(o) && Forges(v)
\end{alltt}

\noindent where $\texttt{o}$ is an output that we create specifically for this purpose in a preceding setup transaction, and $\texttt{v = s}$.

\subsection{Reflections of off-ledger assets}
\label{sec:asset-tokens}

Many tokens are used to represent (be backed by) off-ledger assets on the ledger. An
important example of this is \emph{backed stablecoins}. Other noteworthy
examples of such assets include video game tokens, as well as service
tokens (which represent service provider obligations).

A typical design for such a system is that a trusted party (the ``issuer'') is responsible for creation and destruction of the asset tokens on the ledger.
The issuer is trusted to hold one of the backing off-ledger assets for every token that exists on the ledger, so the only role that the on-chain policy can play is to verify that the forging of the
token is signed by the trusted issuer.
This can be implemented with a forging policy that enforces
an $m$-out-of-$n$ multi-signature scheme, and no additional clauses:
\begin{alltt}
  trusted_issuer(msig) = JustMSig(msig)
\end{alltt}

\subsection{Vesting}

A common desire is to release a supply of some asset on some schedule.
Examples include vesting schemes for shares, and staged releases of newly minted tokens.
This seems tricky in our simple model: how is the forging policy supposed to know which tranches have already been released without some kind of global state which tracks them?
However, this is a problem that we can solve with more tokens. We start building
this policy by following the single issuer scheme, but we need to express more.

Given a specific output \code{o}, and two tranches of tokens \code{tr1} and \code{tr2} which should be released after \code{tick1} and \code{tick2}, we can write a forging policy such as:
\begin{alltt}
  vesting = SpendsOutput(o) && Forges(\cL"tr1" \mapsTo 1, "tr2" \mapsTo 1\cR)
         || TickAfter(tick1) && Forges(tr1bundle) && Burns(\cL"tr1" \mapsTo 1\cR)
         || TickAfter(tick2) && Forges(tr2bundle) && Burns(\cL"tr2" \mapsTo 1\cR)
\end{alltt}
%
This disjunction has three clauses:
\begin{itemize}
\item
  Once only, you may forge two unique tokens \code{tranche1} and \code{tranche2}.
\item
  If you spend and burn \code{tr1} and it is after \code{tick1}, then you may forge all the tokens in \code{tr1bundle}.
\item
  If you spend and burn \code{tr2} and it is after \code{tick2}, then you may forge all the tokens in \code{tr2bundle}.
\end{itemize}
%
By reifying the tranches as tokens, we ensure that they are unique and can be used precisely once.
As a bonus, the tranche tokens are themselves tradeable.

\subsection{Inventory tracker: tokens as state}

We can use tokens to carry some data for us, or to represent state.
A simple example is inventory tracking, where the inventory listing can only be modified by a set of trusted parties.
To track inventory on-chain, we want to have a single output containing all of the tokens of an ``inventory tracking'' asset.
If the trusted keys are represented by the multi-signature $\texttt{msig}$, the inventory tracker tokens should always be kept in a \UTXO\ entry with the following output:
\begin{alltt}
  (hash(msig) , \cL{}hash(msig) \mapsTo \cL{}hats \mapsTo 3, swords \mapsTo 1, owls \mapsTo 2\cR\cR)
\end{alltt}

The inventory tracker is
an example of an asset that should indefinitely be controlled by a specific script
(which ensures only authorized users can update the inventory), and
we enforce this condition in the forging script itself:

\begin{alltt}
  inventory_tracker(msig) = JustMSig(msig) && AssetToAddress(_)
\end{alltt}

In this case, $\texttt{inventory\_tracker(msig)}$ is both the forging
script and the output-locking script. The blank value supplied as the
argument means that the policy ID (and also the address) are both
assumed to be the hash of the $\texttt{inventory\_tracker(msig)}$
script.
Defined this way, our script is run at initial forge time, and any time
the inventory is updated. Each time
it only validates if all the inventory tracker tokens in the transaction's
outputs are always locked by this exact output script.

\subsection{Non-fungible tokens}

A common case is to want an asset group where \emph{all} the tokens are non-fungible.
A simple way to do this is to simply have a different asset policy for each token, each of which can only be run once by requiring a specific \UTXO\ to be spent. However, this is clumsy, and typically we want to have a set of non-fungible tokens all controlled by the same policy. We can do this with the \texttt{FreshTokens} clause.
If the policy always asserts that the token names are hashes of data unique to the transaction and token, then the tokens will always be distinct.

\subsection{Revocable permission}

An example where we employ this dual-purpose nature of scripts is revocable permission.
We will express permissions as a \emph{credential token}.

The list of users (as a list of hashes of their public keys) in a credential token is
composed by some central accreditation authority. Users usually trust that this authority
has verified some
real-life data, e.g. that a KYC accreditation authority has checked off-chain
that those it accredits meet some standard.\footnote{
KYC stands for ``know your customer'', which
is the process of verifying a customer's identity before allowing the customer
to use a company's service.
}
Note here that we significantly
simplify the function of KYC credentials for brevity of our example.

For example, suppose that
exchanges are only willing to transfer funds to those that have proved that
they are KYC-accredited.

In this case, the accreditation authority could issue an asset that looks like
\begin{alltt}
  \cL{}KYC_accr_authority \mapsTo \cL{}accr_key_1 \mapsTo 1, accr_key_2 \mapsTo 1, accr_key_3 \mapsTo 1\cR\cR
\end{alltt}

\noindent where the token names are the public keys of the accredited users.
We would like to make sure that

\begin{itemize}
  \item only the
authority has the power to ever forge or burn tokens controlled by this policy,
and it can do so at any time,
  \item all the users with listed keys are able to spend this asset
  as on-chain proof that they are KYC-accredited, and
  \item once a user is able to prove they have the credentials, they should be allowed
 to receive funds from an exchange.
\end{itemize}

We achieve this with a script of the following form:
\begin{alltt}
  credential_token(msig) = JustMSig(msig) && DoForge
                        || AssetToAddress(_) && Not DoForge && SignedByPIDToken(_)
\end{alltt}

Here, forges (i.e. updates to credential tokens) can only be done by the
$\texttt{msig}$ authority,
but every user whose key hash is included in the token names can spend from
this script, provided they return the asset to the same script.
To make a script that only allows spending from it if the user doing so
is on the list of key hashes in the credential token made by $\texttt{msig}$, we write

\begin{alltt}
  must_be_on_list(msig) = SpendsCur(credential_token(msig))
\end{alltt}

In our definition of the credential token, we have used all the strategies
we discussed above to extend the expressivity of an FPS language.
We are not yet using the \UTXO\ model to its full potential, as we are just
using the \UTXO\ to store some information that cannot be traded. However, we
could consider updating
our credential token use policy to associate spending it with another action,
such as adding a pay-per-use clause. Such a change really relies on the
\UTXO\ model.

\section{Related work}

\paragraph{Ethereum.}

Ethereum's ERC token standards~\cite{erc20,erc721} are one of the better known multi-asset
implementations. They are a non-native implementation and so come with a set of drawbacks,
such as having to implement ledger functionality (such as asset transfers) using smart contracts, rather
than using the underlying ledger.

Augmenting the Ethereum ledger with functionality similar to that of the model
we present here would likely be possible. Additionally, access to global
contract state would make it easier to define forging policies that care about
global information, such as the total supply of an asset.

\paragraph{Waves.}

Waves~\cite{waves} is an account-based multi-asset ledger, supporting its own smart contract
language. In Waves, both accounts and assets themselves can
be associated with contracts. In both cases, the association is made by
adding the associated script
to the account state (or the state of the account containing the asset).
A script associated with an asset imposes conditions on the use of this asset,
including minting and burning restrictions, as well as transfer restrictions.

\paragraph{Stellar.}

Stellar~\cite{stellar} is an account-based native multi-asset system geared towards tracking
real-world item ownership via associated blockchain tokens.
Stellar is optimised to allow a token issuer to maintain a level of control over
the use of their token even once it changes hands. The Stellar ledger also features
a distributed exchange listing, which is used to
facilitate matching (by price) exchanges between different tokens.

\paragraph{Zilliqa.}

Zilliqa
is an account-based platform with an approach to smart contract implementation
similar to that of Ethereum~\cite{scilla-arxiv}.
The Zilliqa fungible and non-fungible tokens are
designed in a way that mimics the ERC-20 and ERC-721 tokens, respectively.
While this system is designed to be statically analysable, it does not offer new solutions to
the problem of dependency on the global state.

\paragraph{DAML.}

DAML~\cite{daml} is a smart contract language designed to be used on the DAML ledger model.
The DAML ledger model does not support keeping records of asset ownership, but instead,
only stores current contract states in the following way:
a valid transaction interacting with a contract results in the creation of new
contracts that are the next
steps of the original contract, and removal of the original contract from the ledger.

Only the contracts with which a transaction
interacts are relevant to validating it, which is similar to our approach
of validation without global context.
Although this system does not have built-in multi-asset support (or any
ledger-level accounting),
the transfer of any type of asset can be represented on the ledger via contracts.
Due to the design of the system to operate entirely by listing contracts on the ledger,
each
action, including accepting funds transferred by a contract, requires consent. This
is another significant way
in which this model is different from ours.

%\todompj{This doesn't actually say anything about multi-asset stuff?}

\paragraph{Bitcoin.}

Bitcoin popularised \UTXO\ ledgers, but has neither native nor non-native
multi-asset support on the main chain.
The Bitcoin ledger model does not appear to have the accounting infrastructure
or sufficiently expressive
smart contracts for implementing multi-asset support in a generic way.
There have been several layer-two approaches to implementing custom Bitcoin assets.
Because each mined Bitcoin is unique, a particular
Bitcoin can represent a specific custom asset, as is done in~\cite{coloredcoins}.
A more sophisticated accounting strategy is implemented in another layer-two
custom asset approach~\cite{nick2020liquid}.
There have also been attempts to implement custom tokens using Lightning network
channels~\cite{lntokens}.

\paragraph{Tezos.}

Tezos~\cite{tezos} is an account-based platform with its own smart contract language.
It has been used to implement an ERC-20-like fungible token standard (FA1.2),
with a unified token standard in the works (see~\cite{tezosMA}).
The custom tokens for both multi-asset standards are non-native, and thus have shortcomings
similar to those of Ethereum token standards.

%\todompj{This is very vague.}

\paragraph{Nervos CKB.}

Nervos CKB~\cite{nervos} is a \UTXO{}-inspired platform that operates on a broader notion of a Cell,
rather than the usual output balance amount and address, as the value stored in
an entry. A Cell entry can contain any type of data,
including a native currency balance, or any type of code.
This platform comes with a Turing-complete scripting language that can be used to define custom
native tokens. There is, however, no dedicated accounting infrastructure to handle
trading custom assets in a similar way as the base currency type.

\section{Discussion}
\label{sec:discussion}

\subsection{General observations}
\subsubsection{Asset registries and distributed exchanges.}

The most obvious way to manage custom assets might be to add some kind of global \emph{asset registry}, which associates a new asset group with its policy.
Once we have an asset registry, this becomes a natural place to put other kinds of infrastructure that rely on global state associated with assets, such as decentralised exchanges.

However, our system provides us with a way to associate forging policies and the assets controlled by them \emph{without} any global state. This simplifies the ledger implementation in the concurrent and distributed environment of a blockchain.
Introducing global state into our model would result in disrupted synchronisation (on which \cite{chakravarty2020hydra} relies to great effect to implement fast, optimistic settlement), as well as slow and costly state updates at the time of asset registration.
Hence, on balance we think it is better to have a stateless system, even if it relegates  features like decentralised exchanges to be Layer 2 solutions.

% MPJ: I don't think this adds anything

%Similarly, when we think about facilitating the trading custom tokens, we might want to include or facilitate a
%\emph{decentralised exchange} as part of our system.
%However, again such a listing requires global state to track the offers.

%Note that our model allows for a fair and reliable way for users
%to make offers of MA tokens
%for sale or exchange on the ledger. To make such an offer, a user would place
%their tokens in an output locked by a script. This script requires the transaction spending
%this output to \emph{pay} for the token.

%It is possible to implement a decentralised exchange listing or a global
%asset registry in a
%in a way that does not compromise security guarantees of our current model, but
%at the expense of high overhead, complicated validation rules,
%and frequent transaction validation failure.
%All transactions submitted, but not processed, before a registry update would
%necessarily become invalid, which would be a frequent occurence.

%Even with a design of a registry that solves these problems, it does not make
%sense to have either registry be the main
%source of truth in our model, since their functionality must be specified
%elsewhere, i.e. the relevant forging and output scripts.

\subsubsection{Spending policies.}

Some platforms we discussed provide ways to express restrictions on the \emph{transfer} of
tokens, not just on their forging and burning, which we refer to as \textit{spending policies}.
Unlike forging policies, spending polices are not a native part of our system.
We have considered a number of approaches to adding spending policies, but we
have not found a solution that does not put an undue burden on the \emph{users}
of such tokens, both humans and programmatic users such as layer-two protocols
(e.g. Lightning). For example, it would be necessary to ensure that spending
policies are not in conflict with forging or output-locking scripts (any time the asset is spent).

Forging tokens requires a specific action by the user (providing and satisfying a forging policy script), but this action is always taken knowingly by a user who is specifically trying to forge those tokens.
\emph{Spending} tokens is, however, a completely generic operation that works over arbitrary bundles of tokens.
Indeed, a virtue of our system is that custom tokens all look and behave uniformly.
In contrast, spending policies make custom tokens extremely difficult to handle in a generic way, in particular for automated systems.

These arguments do not invalidate the usefulness of spending policies, but instead highlight that
they are not obviously compatible with trading of native assets in a generic way, and an approach that addresses these issues in an ergonomic way is much needed.
This problem is not ours alone: spending policies in other systems we have looked at here (such as Waves),
do not provide universal solutions to the issues we face with spending policies either.

\subsubsection{Viral scripts.}

One way to emulate spending policies in our system is to lock all the tokens with a particular script that ensures
that they \emph{remain} locked by the same script when transferred.
We call such a script a ``viral'' script (since it spreads to any new outputs that are ``infected'' with the token).

This allows the conditions of the script to be enforced on every transaction that uses the tokens, but at significant costs.
In particular such tokens can never be locked by a \emph{different} script, which prevents such tokens from being used in smart contracts, as well as preventing an output from containing tokens from two such viral asset groups (since \emph{both} would require that their validator be applied to the output!).
In some cases, however, this approach is exactly what we want.
For example, in the case of credential tokens, we want
the script locking the credential to permanently allow the issuer access to the credential (in
order to maintain their ability to revoke it).

\subsubsection{Global State.}

There are limitations of our model due to the fact that global information about the ledger is not available to the forging policy script.
Many global state constraints can be accommodated with workarounds (such as in the case of provably unique fresh tokens), but some cannot:
for example, a forging policy that allows a variable amount to be forged in every block depending on the current total supply of assets of another policy.
This is an odd policy to have, but nevertheless, not one that can be defined in our model.

\subsection{Conclusions}

We present a multi-asset ledger model which extends a UTXO-based ledger so that it can support custom fungible and non-fungible tokens.
We do this in a way that does not require smart contract functionality.
We add a small language with the ability to express exactly the logic we need for a particular set of usecases.
Our \UTXOma{} ledger together with this language allows us to use custom assets to support a wide variety of usecases, even those that are not normally based on ``assets'', while still remaining deterministic, high-assurance, and analysable.

Our design has a number of limitations, some of which have acceptable workarounds, and some that do not.
In particular, access to global state in a general way cannot be supported, and spending policies are not easy to implement.
It is also not possible to explicitly restrict payment to an address.
We consider these worthy directions for future improvements to our model.


% MPJ: This doesn't seem relevant? Nothing to do with multi-asset really
%\paragraph{Transfer restrictions imposed by the recepient.}
%
%In our system, we do not provide a default way to restrict payment to
%an address. We have discussed two platforms where this is possible, Nervos and
%DAML, both of which have fundamentally different designs that compromise other
%features we require in ours. These types of restriction policies
%are closely related to and suffer from many of the same problems as
%spending policies, such as high overhead, clashing constraints with sequetial or
%parallel contracts, and the possibility of circumvention.

% Doesn't tell anything new -=chak
%\section{Conclusion}
%
%We have presented \UTXOma{}, a \UTXO{} ledger model with native, lightweight, stateless mulitasset support.
%\UTXOma{} supports a wide range of key use cases, including fungible and non-fungible tokens.
%
%Moreover, we show that a simple scripting model is adequate to implement a variety of both useful forging policies and output-locking scripts.
%Using these together we are able to realise a wide variety of applications.

% MPJ: I think we can get away with a simpler conclusion

%With our two-level token classification structure, we are able to both
%\begin{itemize}
%  \item infer the fungibility relation directly by looking at the names and IDs of
%  token classes, without executing code
%  \item retain the ability to impose fungibility constraints, as tokens under a single
%  policy are minted and burned, by rejecting attempts to forge tokens that do not satisfy them
%\end{itemize}

%As promised, we gave examples of clauses that a language would need in order to express
%a number of interesting and realistic use cases.
%Defining language clauses in this way, however, is one step away from hard-coding scripts into
%rules about processing transactions, and frequently requires updating them
%every time we encounter a new use case.

%There are many advantages to making this system more expressive, with a fully
%formed scripting language. The approach we present here, however, gives use the opportunity to highlight
%just how versatile and intuitive our MA structure us due to being native and
%represented by the two-level map we described. Even with a minimally expressive
%smart contract language (only checking multi-signatures), we can already allow users to
%define asset policies governing the forging of custom tokens.


\bibliographystyle{splncs04}
\bibliography{utxoma}

\end{document}
