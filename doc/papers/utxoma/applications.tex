\section{Applications}
\label{sec:applications}

\UTXOma\ is able to support a large number of standard use cases for multi-asset ledgers, as well as some novel ones.
In this section we give a selection of examples.
There are some common themes: (1) Tokens as resources can be used to reify many non-obvious things, which makes them first-class tradeable items; (2) cheap tokens allow us to solve many small problems with \emph{more tokens}; and (3) the power of the scripting language affects what examples can be implemented.

\subsection{Simple single token issuance}
%
To create a simple currency $\mathsf{SimpleCoin}$ with a fixed supply of $\texttt{s = 1000 SimpleCoins}$ tokens, we might try to use the simple policy script $\texttt{Forges(s)}$ with a single forging transaction. Unfortunately, this is not sufficient as somebody else could submit another transaction forging another $\texttt{1000 SimpleCoins}$.

In other words, we need to ensure that there can only ever be a single transaction on the ledger that successfully forges $\mathsf{SimpleCoin}$. We can achieve that by requiring that the forging transaction consumes a specific \UTXO. As \UTXO{}s are guaranteed to be (1) unique and (2) only be spent once, we are being guaranteed that the forging policy can only be used once to forge tokens. We can use the script:

\begin{alltt}
  simple_policy(o, v) = SpendsOutput(o) && Forges(v)
\end{alltt}

\noindent where $\texttt{o}$ is an output that we create specifically for this purpose in a preceding setup transaction, and $\texttt{v = s}$.

\subsection{Reflections of off-ledger assets}
\label{sec:asset-tokens}

Many tokens are used to represent (be backed by) off-ledger assets on the ledger. An
important example of this is \emph{backed stablecoins}. Other noteworthy
examples of such assets include video game tokens, as well as service
tokens (which represent service provider obligations).

A typical design for such a system is that a trusted party (the ``issuer'') is responsible for creation and destruction of the asset tokens on the ledger.
The issuer is trusted to hold one of the backing off-ledger assets for every token that exists on the ledger, so the only role that the on-chain policy can play is to verify that the forging of the
token is signed by the trusted issuer.
This can be implemented with a forging policy that enforces
an $m$-out-of-$n$ multi-signature scheme, and no additional clauses:
\begin{alltt}
  trusted_issuer(msig) = JustMSig(msig)
\end{alltt}

\subsection{Vesting}

A common desire is to release a supply of some asset on some schedule.
Examples include vesting schemes for shares, and staged releases of newly minted tokens.
This seems tricky in our simple model: how is the forging policy supposed to know which tranches have already been released without some kind of global state which tracks them?
However, this is a problem that we can solve with more tokens. We start building
this policy by following the single issuer scheme, but we need to express more.

Given a specific output \code{o}, and two tranches of tokens \code{tr1} and \code{tr2} which should be released after \code{tick1} and \code{tick2}, we can write a forging policy such as:
\begin{alltt}
  vesting = SpendsOutput(o) && Forges(\cL"tr1" \mapsTo 1, "tr2" \mapsTo 1\cR)
         || TickAfter(tick1) && Forges(tr1bundle) && Burns(\cL"tr1" \mapsTo 1\cR)
         || TickAfter(tick2) && Forges(tr2bundle) && Burns(\cL"tr2" \mapsTo 1\cR)
\end{alltt}
%
This disjunction has three clauses:
\begin{itemize}
\item
  Once only, you may forge two unique tokens \code{tranche1} and \code{tranche2}.
\item
  If you spend and burn \code{tr1} and it is after \code{tick1}, then you may forge all the tokens in \code{tr1bundle}.
\item
  If you spend and burn \code{tr2} and it is after \code{tick2}, then you may forge all the tokens in \code{tr2bundle}.
\end{itemize}
%
By reifying the tranches as tokens, we ensure that they are unique and can be used precisely once.
As a bonus, the tranche tokens are themselves tradeable.

\subsection{Inventory tracker: tokens as state}

We can use tokens to carry some data for us, or to represent state.
A simple example is inventory tracking, where the inventory listing can only be modified by a set of trusted parties.
To track inventory on-chain, we want to have a single output containing all of the tokens of an ``inventory tracking'' asset.
If the trusted keys are represented by the multi-signature $\texttt{msig}$, the inventory tracker tokens should always be kept in a \UTXO\ entry with the following output:
\begin{alltt}
  (hash(msig) , \cL{}hash(msig) \mapsTo \cL{}hats \mapsTo 3, swords \mapsTo 1, owls \mapsTo 2\cR\cR)
\end{alltt}

The inventory tracker is
an example of an asset that should indefinitely be controlled by a specific script
(which ensures only authorized users can update the inventory), and
we enforce this condition in the forging script itself:

\begin{alltt}
  inventory_tracker(msig) = JustMSig(msig) && AssetToAddress(_)
\end{alltt}

In this case, $\texttt{inventory\_tracker(msig)}$ is both the forging
script and the output-locking script. The blank value supplied as the
argument means that the policy ID (and also the address) are both
assumed to be the hash of the $\texttt{inventory\_tracker(msig)}$
script.
Defined this way, our script is run at initial forge time, and any time
the inventory is updated. Each time
it only validates if all the inventory tracker tokens in the transaction's
outputs are always locked by this exact output script.

\subsection{Non-fungible tokens}

A common case is to want an asset group where \emph{all} the tokens are non-fungible.
A simple way to do this is to simply have a different asset policy for each token, each of which can only be run once by requiring a specific \UTXO\ to be spent. However, this is clumsy, and typically we want to have a set of non-fungible tokens all controlled by the same policy. We can do this with the \texttt{FreshTokens} clause.
If the policy always asserts that the token names are hashes of data unique to the transaction and token, then the tokens will always be distinct.

\subsection{Revocable permission}

An example where we employ this dual-purpose nature of scripts is revocable permission.
We will express permissions as a \emph{credential token}.

The list of users (as a list of hashes of their public keys) in a credential token is
composed by some central accreditation authority. Users usually trust that this authority
has verified some
real-life data, e.g. that a KYC accreditation authority has checked off-chain
that those it accredits meet some standard.\footnote{
KYC stands for ``know your customer'', which
is the process of verifying a customer's identity before allowing the customer
to use a company's service.
}
Note here that we significantly
simplify the function of KYC credentials for brevity of our example.

For example, suppose that
exchanges are only willing to transfer funds to those that have proved that
they are KYC-accredited.

In this case, the accreditation authority could issue an asset that looks like
\begin{alltt}
  \cL{}KYC_accr_authority \mapsTo \cL{}accr_key_1 \mapsTo 1, accr_key_2 \mapsTo 1, accr_key_3 \mapsTo 1\cR\cR
\end{alltt}

\noindent where the token names are the public keys of the accredited users.
We would like to make sure that

\begin{itemize}
  \item only the
authority has the power to ever forge or burn tokens controlled by this policy,
and it can do so at any time,
  \item all the users with listed keys are able to spend this asset
  as on-chain proof that they are KYC-accredited, and
  \item once a user is able to prove they have the credentials, they should be allowed
 to receive funds from an exchange.
\end{itemize}

We achieve this with a script of the following form:
\begin{alltt}
  credential_token(msig) = JustMSig(msig) && DoForge
                        || AssetToAddress(_) && Not DoForge && SignedByPIDToken(_)
\end{alltt}

Here, forges (i.e. updates to credential tokens) can only be done by the
$\texttt{msig}$ authority,
but every user whose key hash is included in the token names can spend from
this script, provided they return the asset to the same script.
To make a script that only allows spending from it if the user doing so
is on the list of key hashes in the credential token made by $\texttt{msig}$, we write

\begin{alltt}
  must_be_on_list(msig) = SpendsCur(credential_token(msig))
\end{alltt}

In our definition of the credential token, we have used all the strategies
we discussed above to extend the expressivity of an FPS language.
We are not yet using the \UTXO\ model to its full potential, as we are just
using the \UTXO\ to store some information that cannot be traded. However, we
could consider updating
our credential token use policy to associate spending it with another action,
such as adding a pay-per-use clause. Such a change really relies on the
\UTXO\ model.
