% Plutus Core Specification
\title{Formal Specification of\\the Plutus Core Language (version 2.1)}
\date{6th April 2021}

\renewcommand{\thefootnote}{\fnsymbol{footnote}}

\documentclass[a4paper]{article}

\usepackage{blindtext, graphicx}
\usepackage{url}
\usepackage{natbib}

% *** MATH PACKAGES ***
%
\usepackage[cmex10]{amsmath}
\usepackage{stmaryrd}

% *** ALIGNMENT PACKAGES ***
%
\usepackage{array}
\usepackage{float}  %% Try to improve placement of figures.  Doesn't work well with subcaption package.
\usepackage{subcaption}
\usepackage{caption}

% Stuff for splitting figures over page breaks
\DeclareCaptionLabelFormat{continued}{#1~#2 (Continued)}
\captionsetup[ContinuedFloat]{labelformat=continued}


% correct bad hyphenation here
\hyphenation{}

\usepackage{subfiles}
\usepackage{geometry}
\usepackage{pdflscape}
\usepackage[title]{appendix}


% *** IMPORTS FOR PLUTUS LANGUAGE ***

\usepackage[T1]{fontenc}
\usepackage{bussproofs,amsmath,amssymb}
\usepackage[dvipsnames]{xcolor}
\usepackage{alltt}

\usepackage{listings}
\lstset{
  basicstyle=\ttfamily,
  columns=fullflexible,
  mathescape=true,
  escapeinside={|}{|}   %% Inside listings you can say things like |\textit{blah blah}|
}

\usepackage[colorlinks=true,linkcolor=MidnightBlue,citecolor=ForestGreen,urlcolor=Plum]{hyperref}
% You're supposed to make this the final package


% *** DEFINITIONS FOR PLUTUS LANGUAGE ***

%%% General Misc. Definitions

\newcommand{\red}[1]{\textcolor{red}{#1}}
\newcommand{\redfootnote}[1]{\red{\footnote{\red{#1}}}}
\newcommand{\blue}[1]{\textcolor{blue}{#1}}
\newcommand{\bluefootnote}[1]{\blue{\footnote{\blue{#1}}}}


\newcommand{\diffbox}[1]{\text{\colorbox{lightgray}{\(#1\)}}}
\newcommand{\judgmentdef}[2]{\fbox{#1}

\vspace{0.5em}

#2}
\newcommand{\hyphen}{\operatorname{-}}
\newcommand{\repetition}[1]{\overline{#1}}
\newcommand{\Fomega}{F$^{\omega}$}
\newcommand{\keyword}[1]{\texttt{#1}}
\newcommand{\construct}[1]{\texttt{(} #1 \texttt{)}}

\newcommand\discharge[1]{\widehat{#1}}

%%% Term Grammar

\newcommand{\sig}[3]{[#1](#2)#3}
\newcommand{\constsig}[1]{#1}
\newcommand{\con}[2]{\construct{\keyword{con} ~ #1 ~ #2}}
\newcommand{\abs}[3]{\construct{\keyword{abs} ~ #1 ~ #2 ~ #3}}
\newcommand{\inst}[2]{\texttt{\{}#1 ~ #2\texttt{\}}}
\newcommand{\lam}[3]{\construct{\keyword{lam} ~ #1 ~ #2 ~ #3}}
\newcommand{\app}[2]{\texttt{[} #1 ~ #2 \texttt{]}}
\newcommand{\iwrap}[3]{\construct{\keyword{iwrap} ~ #1 ~ #2 ~ #3}}
\newcommand{\wrap}{\iwrap}
%% ^ Temporary fix to avoid substituting all occurrences of new keyword
\newcommand{\unwrap}[1]{\construct{\keyword{unwrap} ~ #1}}
\newcommand{\builtin}[3]{\construct{\keyword{builtin} ~ \texttt{\{} \mathit{#1} ~ #2 \texttt{\}} ~ #3}}
\newcommand{\error}[1]{\construct{\keyword{error} ~ #1}}

%% Extra untyped terms
\newcommand{\lamU}[2]{\construct{\keyword{lam} ~ #1 ~ #2        }}
\newcommand{\appU}[2]{\texttt{[} #1 ~ #2 \texttt{]}}
\newcommand{\errorU}{\construct{\keyword{error}}}
\newcommand{\builtinU}[2]{\construct{\keyword{builtin} ~ \mathit{#1} ~ #2}}
\newcommand{\delay}[1]{\construct{\keyword{delay} ~ #1}}
\newcommand{\force}[1]{\construct{\keyword{force} ~ #1}}

\newcommand{\erase}[1]{\llbracket#1\rrbracket}

%%%  Type Grammar

\newcommand{\funT}[2]{\construct{\keyword{fun} ~ #1 ~ #2}}
\newcommand{\ifixT}[2]{\construct{\keyword{ifix} ~ #1 ~ #2}}
\newcommand{\fixT}{\ifixT}
%% ^ Temporary fix to avoid substituting all occurrences of new keyword
\newcommand{\allT}[3]{\construct{\keyword{all} ~ #1 ~ #2 ~ #3}}
\newcommand{\conUnitType}{\keyword{unit}}
\newcommand{\conBooleanType}{\keyword{bool}}
\newcommand{\conIntegerType}{\keyword{integer}}
\newcommand{\conBytestringType}{\keyword{bytestring}}
\newcommand{\conStringType}{\keyword{string}}
\newcommand{\conT}[1]{\construct{\keyword{con} ~ #1}}
\newcommand{\lamT}[3]{\construct{\keyword{lam} ~ #1 ~ #2 ~ #3}}
\newcommand{\appT}[2]{\texttt{[} #1 ~ #2 \texttt{]}}

\newcommand{\typeK}{\construct{\keyword{type}}}
\newcommand{\funK}[2]{\construct{\keyword{fun} ~ #1 ~ #2}}



%%% Program Grammar

\newcommand{\version}[2]{\construct{\keyword{program} ~ #1 ~ #2}}



%%% Judgments

\newcommand{\hypJ}[2]{#1 \vdash #2}
\newcommand{\ctxni}[2]{#1 \ni #2}
\newcommand{\validJ}[1]{#1 \ \operatorname{valid}}
\newcommand{\termJ}[2]{#1 : #2}
\newcommand{\typeJ}[2]{#1 :: #2}
\newcommand{\istermJ}[2]{#1 : #2}
\newcommand{\istypeJ}[2]{#1 :: #2}



%%% Contextual Normalization

\newcommand{\ctxsubst}[2]{#1\{#2\}}
\newcommand{\typeStep}[2]{#1 ~ \rightarrow_{ty} ~ #2}
\newcommand{\typeMultistep}[2]{#1 ~ \rightarrow_{ty}^{*} ~ #2}
\newcommand{\typeBoundedMultistep}[3]{#2 ~ \rightarrow_{ty}^{#1} ~ #3}
\newcommand{\step}[2]{#1 ~ \rightarrow ~ #2}
\newcommand{\normalform}[1]{\lfloor #1 \rfloor}
\newcommand{\subst}[3]{[#1/#2]#3}
\newcommand{\kindEqual}[2]{#1 =_{\mathit{k}} #2}
\newcommand{\typeEqual}[2]{#1 =_{\mathit{ty}} #2}
\newcommand{\typeEquiv}[2]{#1 \equiv_{\mathit{ty}} #2}


\newcommand{\inConTFrame}[1]{\conT{#1}}
\newcommand{\inAppTLeftFrame}[1]{\appT{\_}{#1}}
\newcommand{\inAppTRightFrame}[1]{\appT{#1}{\_}}
\newcommand{\inFunTLeftFrame}[1]{\funT{\_}{#1}}
\newcommand{\inFunTRightFrame}[1]{\funT{#1}{\_}}
\newcommand{\inAllTFrame}[2]{\allT{#1}{#2}{\_}}
\newcommand{\inFixTLeftFrame}[1]{\fixT{\_}{#1}}
\newcommand{\inFixTRightFrame}[1]{\fixT{#1}{\_}}
\newcommand{\inLamTFrame}[2]{\lamT{#1}{#2}{\_}}

\newcommand{\inBuiltin}[5]{\builtin{#1}{#2}{#3 #4 #5}}



%%% CK Machine Normalization

\newcommand{\ckerror}{\blacklozenge}
\newcommand{\compute}{\triangleright}
\newcommand{\return}{\triangleleft}

\newcommand{\bcompute}{\color{blue}\compute}  
\newcommand{\breturn}{\color{blue}\return}  
\newcommand{\bmapsto}{\color{blue}\mapsto}
% This is to get blue symbols after an '&' inside an alignat
% environment.  It seems to mess up the spacing if you do anything
% else.



\newcommand{\inInstLeftFrame}[1]{\inst{\_}{#1}}
\newcommand{\inWrapRightFrame}[2]{\iwrap{#1}{#2}{\_}}
\newcommand{\inUnwrapFrame}{\unwrap{\_}}
\newcommand{\inAppLeftFrame}[1]{\app{\_}{#1}}
\newcommand{\inAppRightFrame}[1]{\app{#1}{\_}}

% Extra frames for untyped term normalisation
\newcommand{\inForceFrame}{\force{\_}}
\newcommand{\inBuiltinU}[4]{\builtinU{#1}{#2 #3 #4}}

% These are for use inside listings and $...$.  If you just use
% \textit in listings it uses the italic tt font and the spacing
% inside the words is a bit strange.  Spacing is also bad if you
% just put something like "integer" in math text.
\newcommand\unit{\ensuremath{\mathit{unit}}}
\newcommand\one{\ensuremath{\mathit{one}}}
\newcommand\boolean{\ensuremath{\mathit{boolean}}}
\newcommand\integer{\ensuremath{\mathit{integer}}}
\newcommand\bytestring{\ensuremath{\mathit{bytestring}}}
\newcommand\str{\ensuremath{\mathit{str}}}
\newcommand\case{\ensuremath{\mathit{case}}}
\newcommand\signed{\ensuremath{\mathit{signed}}}
\newcommand\txhash{\ensuremath{\mathit{txhash}}}
\newcommand\pubkey{\ensuremath{\mathit{pubkey}}}
\newcommand\blocknum{\ensuremath{\mathit{blocknum}}}
%% \newcommand\uniqmem[1]{#1^{\blacktriangledown}}  %% Unique member of size type

\newcommand\true{\ensuremath{\mathtt{True}}}
\newcommand\false{\ensuremath{\mathtt{False}}}


%%% Spacing in tables

\newcommand\sep{4pt}
% The table of abbreviations previously had \\\\ at the end of each line, which
% made it quite long. Lines are now separated by a vertical space of size \sep.
% This makes it a bit more readable than no spacing at all, but not too long

\newcommand{\Strut}{\rule[-2mm]{0mm}{6mm}}

\begin{document}
%
% paper title
% can use linebreaks \\ within to get better formatting as desired

%% 6.0: "User-friendly" version with lots of explication for the reader.
%% 16th April 2019

\maketitle
\begin{center}
\noindent\framebox[0.9\textwidth]{%
\parbox{0.8\textwidth}{%
\begin{center}
  \LARGE{\red{\textsf{SUPERSEDED}}}
\end{center}
\textsf{This version of the Plutus Core specification has largely
 been superseded by a new version which can be found
 at \url{https://github.com/IntersectMBO/plutus/tree/master/doc/plutus-core-spec}.
 The new version currently only deals with Untyped Plutus Core; this version has
 been temporarily retained for its account of Typed Plutus Core.  The main
 semantic difference is that in this version built-in functions must be fully
 applied but in the new version they can be partially applied.}
\\
\rule{0cm}{3mm}
}
}
\end{center}

\thispagestyle{plain}
\pagestyle{plain}


\section{Plutus Core}
Plutus Core is a typed, eagerly evaluated $\lambda$-calculus designed
for use as a transaction validation language in blockchain
systems. More precisely, Plutus Core is the System $F_\omega$ of
Girard and Reynolds (see \citep{Girard-thesis},
\citep{Reynolds-type-structure}, \citep[\S30]{Pierce:TAPL}) with a
number of extensions (isorecursive types, higher kinds, and a library
of built-in basic types and functions).  Plutus Core is meant to be a
compilation target, and this is reflected in the design of the
language: while writing large Plutus Core programs by hand is
difficult, the language is relatively straightforward to formalise
with a proof assistant.  Indeed, the specification in the present
document has been formalised in the Agda proof assistant~\citep{Agda}.

We have tried to keep the Plutus Core language small. There is no explicit
support for algebraic datatypes, but they are representable using, for example,
the Scott encoding (\citep{Scott-encoding}: see also \citep{Koopman:2014} and
~\citep{Geuvers-2014} for example; note that the Scott encoding coincides with
the Church encoding for non-recursive types). Similarly, there is no explicit
constructs for recursion, but the type system is sufficiently expressive to
allow us to use standard combinators (for example the $Z$ combinator) for such
purposes.

\paragraph{Implementation.} A compiler and evaluator for Plutus Core can be found
in~\citep{Plutus-exe-repo}.

\paragraph{Formalisation.} The Agda formalisation of Plutus Core mentioned earlier
can be found in~\citep{Plutus-metatheory-repo},
and the techniques used are described in ~\citep{System-F-in-Agda}.

\subsection{Blockchain issues}
\label{sec:blockchain-issues}
Plutus Core code (and code intended for execution on a blockchain in
general) can be executed in two different environments:
\textit{off-chain} and \textit{on-chain}.  As the name suggests,
off-chain execution doesn't happen on the blockchain itself, but in
some other environment, such as in an electronic wallet on a
smartphone or PC.  In contrast, on-chain execution takes place on
\textit{core nodes}, machines which are actually maintaining the
blockchain.

It is important that core nodes process transactions quickly and
efficiently; moreover, core nodes benefit from executing transactions
and compete with each other to do so.  Denial of service thus becomes
an issue for on-chain code, and it is important that a malicious user
cannot submit code which consumes excessive amounts of processor time
or memory. To deal with this problem, a charge is levied when a core
node runs on-chain code.  Units of so-called \textit{gas} are consumed
as a program runs, and users must pay a fee in advance to cover gas
costs.  If at any point during execution the amount of gas consumed
exceeds the amount than has been paid for then the program is
immediately terminated and the fee is forfeited.

Plutus Core is a typed language which is intended as a compiler
target.  The fact that it is typed helps to increase the robustness of
compilers from higher-level languages, since compiler output can be
typechecked at the end of compilation, and any type errors will
indicate errors in the compilation process.

However, resource limitations preclude complicated on-chain code
analysis and verification, and since programs are transmitted across a
network for execution on the chain it is also important that programs
are as small as possible.  Types take up a lot of space, so we also
define an \textit{untyped} variant of Plutus core for deployment to
and execution on the chain (see Section~\ref{sec:untyped-plc}).
Untyped Plutus Core is obtained from typed Plutus Core by a simple
erasure process, and this makes it easy to analyse and typecheck
a typed program off chain and check that after type erasure it
is identical to an untyped on-chain program.

Plutus Core is intended to be used for validation of on-chain
transactions on the Cardano blockchain~\citep{Cardano}.  The details
of the validation process are outside the scope of this document, but
are explained in~\citep{EUTXO}. For details of Cardano's smart
contract infrastructure (including examples of the kind of validation
code which is compiled into Plutus Core) see the
book~\citep{Plutus-book}.



\section{Syntax}

The grammar of Plutus Core is given in
Figures~\ref{fig:lexical-grammar} and
\ref{fig:grammar} (but see the note on revised syntax in Section~\ref{sec:syntax-update}).
This grammar describes the abstract syntax trees of Plutus Core in a convenient
notation which can also be used as concrete syntax for purposes such as
experimentation and debugging.

Lexemes are described in standard regular expression notation.  The
following lexemes are reserved as keywords for use in concrete syntax:
\texttt{abs}, \texttt{all}, \texttt{builtin}, 
\texttt{con}, \texttt{error}, \texttt{fun}, \texttt{ifix},
\texttt{iwrap}, \texttt{lam}, \texttt{program},
\texttt{type}, \texttt{unwrap}.  The only other lexemes
are round brackets $\texttt{(}\ldots\texttt{)}$, square brackets
$\texttt{[}\ldots\texttt{]}$, and braces
$\texttt{\{}\ldots\texttt{\}}$.  Spaces and tabs are allowed anywhere,
and have no effect save to separate lexemes.

Application in both terms and types is indicated by square
brackets, and instantiation in terms is indicated by braces. We
permit the use of multi-argument application and instantiation as
syntactic sugar for iterated application.
For instance,
\[
  [M_1 ~ M_2 ~ M_3 ~ M_4]
  \]
\noindent is short for
\[
  [[[M_1 ~ M_2] ~ M_3] ~ M_4]
\]
All subsequent definitions assume iterated application and instantiation
has been expanded out, and use only the binary form. Implementations
may use the multi-argument forms.


\subfile{figures/LexicalGrammar}

\subfile{figures/TypedGrammar}

\newcommand\fixtype[1]{\mu\,\alpha.#1}  % Just for talking about the fix operator in the notes.

\subsection{Notes on the grammar}
\paragraph{Terms.}
\begin{itemize}
\item $\con{tn}{cn}$ represents a constant belonging to some built-in
  type, with \textit{cn} being of the correct lexical form for the type \textit{tn}:
  for example \texttt{(con integer 28)} is an integer with the value 28.
  See Section~\ref{sec:builtins} for more on this.
\item $\abs{\alpha}{K}{M}$ represents a polymorphic value abstracted
  over a type; this would often be denoted by $\Lambda\alpha{::}K.M$.
\item $\inst{M}{A}$ represents a polymorphic term instantiated at a particular type.
\item $\iwrap{A}{B}{M}$ and $\unwrap{M}$: see the note on recursive types below.
\item $\lam{x}{A}{M}$ is standard lambda-abstraction, $\lambda{}x{:}{A}.{M}$.
\item $\app{M}{N}$ is standard function application.
\item $\builtin{bn}{A_1 \ldots A_k}{M_1 \ldots M_n}$ denotes application of a built-in function
to the terms $M_1, \ldots, M_n$.
Built-in functions are allowed to be polymorphic, and
$\texttt{\{}\mathit{bn} ~ A_1 \ldots A_k\texttt{\}}$ indicates that the
function denoted by $\mathit{bn}$ is instantiated with the given
types.  For brevity we allow the syntax
$\construct{\keyword{builtin} ~ \mathit{bn} ~ M_1 \ldots M_n}$ for
application of non-polymorhpic built-in functions.
Section~\ref{sec:builtins} for more information on the currently
available built-in functions.
\item $\error{A}$ causes an error, terminating computation.
\end{itemize}

\paragraph{Values.} Figure~\ref{fig:values} shows a class of terms of a particular
syntactic form which we call \textit{values}.  In
Section~\ref{sec:reduction} below we define reduction rules for Plutus
Core terms, and it can be shown (formally, in
Agda (\cite{Plutus-metatheory-repo})) that values are precisely those
terms which cannot undergo any further reduction.

\paragraph{Types.} Plutus Core contains a copy of the simply typed lambda calculus
  at the type level, together with a few extensions.
\begin{itemize}
\item $\funT{A}{B}$ is the type of functions from $A$ to $B$, $A \rightarrow B$.
\item $\allT{\alpha}{K}{A}$ represents the type of a polymorphic term (eg a type abstraction), $\forall \alpha{::}K.A$.
\item $\fixT{A}{B}$ represents a recursive type: see the note in Section~\ref{sec:ifix-note} below for more information.
\item $\lamT{\alpha}{K}{A}$ is abstraction of types over types, $\lambda \alpha{::}K.A$.
\item $\appT{A}{B}$ is function application at the type level.
\item $\conT{tn}$ represents a built-in type: for example, $\conT{\textrm{integer}}$
is the type of integers.  See Section~\ref{sec:builtins} for more information.
\end{itemize}


\paragraph{Type Values.} Section~\ref{sec:reduction} also gives
reduction rules for types.  All types can be reduced to a type value
(or \textit{normalised type}) which cannot undergo any further
reduction, and it can be shown that the type values are precisely
those whose structure is given in Figure~\ref{fig:type-values} (again,
these statements have been formalised and proved in
Agda~\citep{Agda}).  The type normalisation process always terminates:
the proof is similar to the proof of the fact that reduction of terms
in the simply typed lambda calculus always
terminates \citep[\S12]{Pierce:TAPL}; this is important because type
normalisation has to be performed during type checking to check that
two types are equal. The type-level structure of Plutus Core is
slightly more complicated than the simply typed lambda calculus, but
the extra forms (\texttt{ifix}, \texttt{all}, built-in base types)
don't provide any new opportunities for type reduction and thus have
little effect on the proof.

\paragraph{Kinds.} In Plutus Core we have a copy of the
simply typed lambda calculus at the type level. The types of the
simply typed lambda calculus are lifted to the level of
\textit{kinds} in Plutus Core, allowing the type system to talk about
operations which themselves occur at the level of types.  The basic
kind (often written as $\star$ in the literature) is denoted
by \texttt{(type)}.

\paragraph{Signatures.}  A signature
$$[\alpha_1::K_1, \ldots, \alpha_k::K_k](t_1, \ldots, t_n)u$$ represents
the type
$$\forall \alpha_1::K_1\,.\, \ldots \,.\, \forall \alpha_m::K_k\,.\, t_1
\rightarrow \ldots \rightarrow t_n \rightarrow u.$$
Both $m$ and $n$
may be zero, in which case the corresponding list is empty.
Signatures are not used in the concrete syntax of Plutus Core, but we
do use them later in this document, in typing rules and in the
description of built-in values and functions.

\paragraph{Recursive types.}
\label{sec:ifix-note}
\noindent
The operator \texttt{ifix} allows one to define recursive types in
Plutus Core: \texttt{ifix A B} is a type such that \texttt{ifix A B
  $\cong$ A (ifix A) B} where \texttt{A} is a \textit{pattern
  functor}~\citep[2.4]{backhouseetal98} and \texttt{B} is an
index. Pattern functors that \texttt{ifix} receives bind two
variables: one for building recursive occurrences and the other to act
as an index. Indices can be used in order to get parameterised data
types, but also in order to control the shape of the data type:
depending on an index, recursive occurrences can be instantiated
differently.  This allows us to encode a wide variety of data types,
including non-regular and mutually recursive data types:
see~\citep[3.1]{unravelling-recursion} for more information.

Note that we have an \textit{isomorphism} of types here rather than
equality: fixpoint types are
\textit{isorecursive}~\citep[20.2]{Pierce:TAPL} with explicit maps

$$
\texttt{iwrap} : \texttt{A (ifix A) B} \rightarrow \texttt{ifix A B}
$$

\noindent and

$$
\texttt{unwrap} : \texttt{ifix A B} \rightarrow  \texttt{A (ifix A) B}
$$

\noindent such that

$$
\texttt{iwrap} \circ \texttt{unwrap} = id_{ \texttt{ifix A B}}
\mbox{\qquad and \qquad}
\texttt{unwrap} \circ \texttt{iwrap} = id_{ \texttt{A (ifix A) B}}
$$
%% I really hate not having a full stop after the second equation, but it looks weird.

\noindent The use of isorecursive types makes it somewhat more difficult to
write \textit{terms}, but makes it much easier to reason about
\textit{types}, and in particular simplifies the typechecking process
considerably.

\paragraph{Programs and Versions.} A complete Plutus Core program
consists of a standard version number (1.0.2, for example) indicating
the Plutus Core version and a \textit{closed} term (i.e., a term with no
free variables) forming the body of the program.  The version number
is used by the Plutus Core evaluator and other tools to check that
they are dealing with code conforming to the correct version of the
Plutus Core language.

\subsection{\red{Note: revised syntax}}
\label{sec:syntax-update}
\noindent In the current (April 2021) implementation of Plutus Core the syntax of built-in
functions has changed somewhat.  In the syntax above and the semantics and
typing rules below, built-in functions are always required to be \textit{fully
applied}: when a built-in function is applied (or instantiated, in the
polymorphic case), all of its arguments have to be present.  \textbf{This
restriction has now been relaxed.}  We now have a term
\\
$$
   \texttt{(builtin \textit{bn})}
$$
\noindent
which can be applied to arguments one by
one. Thus \texttt{(builtin addInteger)}, \texttt{[(builtin addInteger) (con
integer 5)]}, and \texttt{[(builtin addInteger) (con integer 5) (con int\-eger 6)]}
are all valid terms.  The first two of these are values which cannot immediately
be reduced (but could, for example, be passed as arguments to other functions);
the third will evaluate to \texttt{(con integer 11)}.  In the polymorphic
case type instantiations and applications to term arguments can be interleaved.

The new syntax is used in the examples in Section~\ref{sec:examples}, but
elsewhere in the document we use the earlier form.  We are in the process of
producing a revised version of this document which will give full typing rules
and semantics for the updated syntax, but in the interim the reader should keep
in mind that things have changed slightly.

\section{Type correctness}

We define for Plutus Core a number of typing judgments which explain
ways that a program can be well-formed. First, in Figure
\ref{fig:contexts}, we define the grammar of \textit{contexts},
which are sequences of variables accompanied by kind and type
information. In System F there are two different sorts of variables:
term variables and type variables. For succinctness of presentation we
keep information about both sorts of variables in the same
context. Type variables carry their name and a kind and term variables
carry their name and a type. The rules for context validity explain
how valid contexts can be constructed: the empty context is valid, and
given a valid context $\Gamma$, it can be extended with a fresh
variable and kind or a fresh variable and a valid type in
$\Gamma$. Note that we do not need a separate judgment for valid
kinds. They are so simple that any kind that can be constructed
according to the grammar is a valid kind.

%We also consider variable judgements in Figure
%\ref{fig:contexts}. Technically this judgement is defined
%mutually with the typing and kinding judgement defined in Figures
%\ref{fig:type_synthesis} and
%\ref{fig:kind_synthesis}. We give axiomatic rules for term
%and type variables at the right-hand end of the context. For term and
%type variables defined further to the left in the context the
%judgement is dependent on a judgement about a shorter context. These
%two styles of rules for all combinations of term and type variables
%allow us to give judgements for variables in all positions in the
%context. This style of presentation is analogous to how one might
%define instrinsically typed/kinded de Bruijn indices.

%% ---------------- Contexts ---------------- %%

\begin{figure}[H]
{
    \[\begin{array}{lrclr}
        \textrm{Ctx} & \Gamma  & ::= & \epsilon                    & \textrm{empty context} \\
                     &         &     & \Gamma, \typeJ{\alpha}{K}   & \textrm{type variable} \\
                     &         &     & \Gamma, \termJ{x}{A}        & \textrm{term variable} \\
    \end{array}\]

    \judgmentdef{\(\validJ{\Gamma}\)}{Context $\Gamma$ is valid}

    \begin{prooftree}
        \AxiomC{}
        \UnaryInfC{\(\validJ{\epsilon}\)}
    \end{prooftree}

    \begin{prooftree}
        \AxiomC{\(\validJ{\Gamma}\)}
        \AxiomC{$\alpha$ is free in $\Gamma$}
        \BinaryInfC{\(\validJ{\Gamma, \typeJ{\alpha}{K}}\)}
    \end{prooftree}

    \begin{prooftree}
        \AxiomC{\(\validJ{\Gamma}\)}
        \AxiomC{$x$ is free in $\Gamma$}
        \AxiomC{\(\hypJ{\Gamma}{\istypeJ{A}{\typeK{}}}\)}
        \TrinaryInfC{\(\validJ{\Gamma, \termJ{x}{A}}\)}
    \end{prooftree}

%    \judgmentdef{\(\ctxni{\Gamma}{J}\)}{In valid context $\Gamma$ we can make judgements about variables where $J$ is either a typing or a kinding judgement.}
%
%    \begin{prooftree}
%        \AxiomC{}
%        \UnaryInfC{\(\ctxni{\Gamma, \typeJ{\alpha}{K}}{\typeJ{\alpha}{K}}\)}
%    \end{prooftree}
%
%    \begin{prooftree}
%        \AxiomC{}
%        \UnaryInfC{\(\ctxni{\Gamma, \termJ{x}{A}}{\termJ{x}{A}}\)}
%    \end{prooftree}
%
%    \begin{prooftree}
%        \AxiomC{\(\ctxni{\Gamma}{\typeJ{\alpha}{K}}\)}
%        \AxiomC{\(\alpha \not= \beta\)}
%        \BinaryInfC{\(\ctxni{\Gamma, \typeJ{\beta}{J}}{\typeJ{\alpha}{K}}\)}
%    \end{prooftree}
%
%     \begin{prooftree}
%        \AxiomC{\(\ctxni{\Gamma}{\typeJ{\alpha}{K}}\)}
%        \UnaryInfC{\(\ctxni{\Gamma, \termJ{y}{T}}{\typeJ{\alpha}{K}}\)}
%    \end{prooftree}
%
%    \begin{prooftree}
%        \AxiomC{\(\ctxni{\Gamma}{\termJ{x}{A}}\)}
%        \UnaryInfC{\(\ctxni{\Gamma, \typeJ{\beta}{J}}{\termJ{x}{A}}\)}
%    \end{prooftree}
%
%     \begin{prooftree}
%        \AxiomC{\(\ctxni{\Gamma}{\termJ{x}{A}}\)}
%        \AxiomC{\(x \not= y\)}
%        \BinaryInfC{\(\ctxni{\Gamma, \termJ{y}{B}}{\termJ{x}{A}}\)}
%    \end{prooftree}


}
    \captionof{figure}{Contexts}
    \label{fig:contexts}
\end{figure}

\newpage
\noindent Figure \ref{fig:kind_synthesis} defines what
it means for a type to synthesise a kind. Plutus Core is a
higher-kinded version of System F, so we have a number of standard
System F rules (\textsf{tyvar},\textsf{tyall},\textsf{tyfun}) together
with some extensions with extensions to higher kinds
(\textsf{tylam},\textsf{tyapp}) and to indexed recursive types
(\textsf{tyfix}). We also introduce built-in types (\textsf{tycon}) to support
integers and bytestrings.

%% ---------------- Kind synthesis ---------------- %%

\begin{figure}[H]
{
    \judgmentdef{\(\hypJ{\Gamma}{\istypeJ{A}{K}}\)}{In valid context $\Gamma$, type $A$ has kind $K$}

    \begin{prooftree}
        \AxiomC{\(({\typeJ{\alpha}{K}}) \in \Gamma\)}
        \RightLabel{\textsf{tyvar}}
        \UnaryInfC{\(\hypJ{\Gamma}{\istypeJ{\alpha}{K}}\)}
    \end{prooftree}

    \begin{prooftree}
        \AxiomC{\(\hypJ{\Gamma, \typeJ{\alpha}{K}}{\istypeJ{A}{\typeK{}}}\)}
        \RightLabel{\textsf{tyall}}
        \UnaryInfC{\(\hypJ{\Gamma}{\istypeJ{\allT{\alpha}{K}{A}}{\typeK{}}}\)}
    \end{prooftree}

    \begin{prooftree}
		\AxiomC{\(\hypJ{\Gamma}{\istypeJ{B}{K}}\)}
		\AxiomC{\(\hypJ{\Gamma}{\istypeJ{A}{\funK{\funK{K}{\typeK{}}}{\funK{K}{\typeK{}}}}}\)}
        \RightLabel{\textsf{tyfix}}
        \BinaryInfC{\(\hypJ{\Gamma}{\istypeJ{\fixT{A}{B}}{\typeK{}}}\)}
    \end{prooftree}

    \begin{prooftree}
        \AxiomC{\(\hypJ{\Gamma}{\istypeJ{A}{\typeK{}}}\)}
        \AxiomC{\(\hypJ{\Gamma}{\istypeJ{B}{\typeK{}}}\)}
        \RightLabel{\textsf{tyfun}}
        \BinaryInfC{\(\hypJ{\Gamma}{\istypeJ{\funT{A}{B }}{\typeK{}}}\)}
    \end{prooftree}

    \begin{prooftree}
        \AxiomC{\(\hypJ{\Gamma, \typeJ{\alpha}{J}}{\istypeJ{A}{K}}\)}
        \RightLabel{\textsf{tylam}}
        \UnaryInfC{\(\hypJ{\Gamma}{\istypeJ{\lamT{\alpha}{J}{A}}{\funK{J}{K}}}\)}
    \end{prooftree}

    \begin{prooftree}
        \AxiomC{\(\hypJ{\Gamma}{\istypeJ{A}{\funK{J}{K}}}\)}
        \AxiomC{\(\hypJ{\Gamma}{\istypeJ{B}{J}}\)}
        \RightLabel{\textsf{tyapp}}
        \BinaryInfC{\(\hypJ{\Gamma}{\istypeJ{\appT{A}{B}}{K}}\)}
    \end{prooftree}

    \begin{prooftree}
        \AxiomC{$tn$ is a type constant in in Figure \ref{fig:type-constants}}
        \RightLabel{\textsf{tycon}}
        \UnaryInfC{\(\hypJ{\Gamma}{\istypeJ{\conT{tn}}{\typeK{}}}\)}
    \end{prooftree}

    \captionof{figure}{Kind Synthesis}
    \label{fig:kind_synthesis}
}
\end{figure}


\noindent In Figure \ref{fig:type-synthesis}, we define the type
synthesis judgment, which explains how a term synthesises a type. We
have rules analogous to STLC (\textsf{var},\textsf{lam},\textsf{app}),
extensions to System F (\textsf{abs},\textsf{inst}). Iso-recursive
types introduce terms for wrapping and unwrapping recursive values
(\textsf{wrap},\textsf{unwrap}), and to support constants and builtins
we have \textsf{con}, \textsf{builtin} and \textsf{error}.
Higher-kinding introduces computation in types so we need the rule
\textsf{conv}. 

%% ---------------- Type synthesis ---------------- %%

\begin{figure}[H]
    \judgmentdef{\(\hypJ{\Gamma}{\istermJ{M}{A}}\)}{In valid context
    $\Gamma$, term $M$ has type $A$}

    \begin{prooftree}
        \AxiomC{\(({\termJ{x}{A}}) \in \Gamma\)}
        \RightLabel{\textsf{var}}
        \UnaryInfC{\(\hypJ{\Gamma}{\istermJ{x}{A}}\)}
    \end{prooftree}

    \begin{prooftree}
        \AxiomC{$cn$ is a valid constant of type $\constsig{tn}$}
        \RightLabel{\textsf{con}}
        \UnaryInfC{\(\hypJ{\Gamma}{\istermJ{\con{tn}{cn}}{\conT{tn}}}\)}
    \end{prooftree}

    \begin{prooftree}
        \AxiomC{\(\hypJ{\Gamma, \typeJ{\alpha}{K}}{\istermJ{M}{B}}\)}
        \RightLabel{\textsf{abs}}
        \UnaryInfC{\(\hypJ{\Gamma}{\istermJ{\abs{\alpha}{K}{M}}{\allT{\alpha}{K}{B}}}\)}
    \end{prooftree}

    \begin{prooftree}
        \AxiomC{\(\hypJ{\Gamma}{\istermJ{L}{C}}\)}
        \AxiomC{\(\typeEquiv{C}{\allT{\alpha}{K}{B}}\)}
        \AxiomC{\(\hypJ{\Gamma}{\istypeJ{A}{K}}\)}
        \RightLabel{\textsf{inst}}
        \TrinaryInfC{\(\hypJ{\Gamma}{\istermJ{\inst{L}{A}}{\subst{A}{\alpha}{B}}}\)}
    \end{prooftree}

    \begin{prooftree}
    	\AxiomC{\(\hypJ{\Gamma}{\istypeJ{B}{K}}\)}
		\alwaysNoLine
		\UnaryInfC{\(\hypJ{\Gamma}{\istypeJ{A}{\funK{\funK{K}{\typeK{}}}{\funK{K}{\typeK{}}}}}\)}
		\UnaryInfC{\(\hypJ{\Gamma}{\istermJ{M}{C}}\)}
		\UnaryInfC{\(\typeEquiv{C}{\appT{\appT{A}{\lamT{\beta}{K}{\fixT{A}{\beta}}}}{B}}\)}
		\alwaysSingleLine
    	\RightLabel{\textsf{wrap}}
        \UnaryInfC{\(\hypJ{\Gamma}{\istermJ{\wrap{A}{B}{M}}{\fixT{A}{B}}}\)}
    \end{prooftree}

    \begin{prooftree}
    	\AxiomC{\(\hypJ{\Gamma}{\istermJ{M}{C}}\)}
		\AxiomC{\(\typeEquiv{C}{\fixT{A}{B}}\)}
		\AxiomC{\(\hypJ{\Gamma}{\istypeJ{B}{K}}\)}
		\RightLabel{\textsf{unwrap}}
        \TrinaryInfC{\(\hypJ{\Gamma}{\istermJ{\unwrap{M}}{\appT{\appT{A}{\lamT{\beta}{K}{\fixT{A}{\beta}}}}{B}}}\)}
    \end{prooftree}

    \begin{prooftree}
        \AxiomC{\(\hypJ{\Gamma}{\istypeJ{A}{\typeK{}}}\)}
        \AxiomC{\(\hypJ{\Gamma, \termJ{y}{A}}{\istermJ{M}{B}}\)}
        \RightLabel{\textsf{lam}}
        \BinaryInfC{\(\hypJ{\Gamma}{\istermJ{\lam{y}{A}{M}}{\funT{A}{B}}}\)}
    \end{prooftree}

    \begin{prooftree}
        \AxiomC{\(\hypJ{\Gamma}{\istermJ{L}{C}}\)}
        \AxiomC{\(\typeEquiv{C}{\funT{A}{B}}\)}
        \AxiomC{\(\hypJ{\Gamma}{\istermJ{M}{A'}}\)}
        \AxiomC{\(\typeEquiv{A}{A'}\)}
        \RightLabel{\textsf{app}}
        \QuaternaryInfC{\(\hypJ{\Gamma}{\istermJ{\app{L}{M}}{B}}\)}
    \end{prooftree}

    \begin{prooftree}
        \alwaysNoLine
        \AxiomC{$\mathit{bn}$ has signature $\sig{\alpha_1 :: K_1, ..., \alpha_k :: K_k}{B_1, ..., B_n}{C}$ in Figure \ref{fig:builtins}}
        \UnaryInfC{\(\hypJ{\Gamma}{\istypeJ{A_i}{K_i}}\)}
        \UnaryInfC{\(\hypJ{\Gamma}{\istermJ{M_i}{D_i}}\)}
        \UnaryInfC{\(\typeEquiv{D_i}{\subst{A_1, ..., A_k}{\alpha_1, ..., \alpha_k}{B_i}}\)}
        \alwaysSingleLine
        \RightLabel{\textsf{builtin}}
        \UnaryInfC{\(\hypJ{\Gamma}{\istermJ{\builtin{bn}{A_1 ... A_k}{M_1 ... M_n}}{\subst{A_1, ..., A_k}{\alpha_1, ..., \alpha_k}{C}}}\)}
    \end{prooftree}

    \begin{prooftree}
        \AxiomC{\(\hypJ{\Gamma}{\istypeJ{A}{\typeK{}}}\)}
        \RightLabel{\textsf{error}}
        \UnaryInfC{\(\hypJ{\Gamma}{\istermJ{\error{A}}{A}}\)}
    \end{prooftree}

    \begin{prooftree}
    	\AxiomC{\(\hypJ{\Gamma}{\istermJ{M}{A}}\)}
		\AxiomC{\(\typeEquiv{A}{A'}\)}
		\RightLabel{\textsf{conv}}
		\BinaryInfC{\(\hypJ{\Gamma}{\istermJ{M}{A'}}\)}
    \end{prooftree}

    \captionof{figure}{Type Synthesis}
    \label{fig:type-synthesis}

\end{figure}

\noindent Finally, type synthesis for builtins are elaborated in tabular form
rather than in inference rule form, in Figure
\ref{fig:builtins}, which also gives the reduction
semantics. This table also specifies what conditions trigger an error.


\section{Reduction and execution}
\label{sec:reduction}

In this section we define a standard eager,
small-step contextual semantics~\citep[5.3]{Harper:PFPL} (or
\textit{reduction semantics}~\citep[\S2]{Felleisen-Hieb}) for Plutus
Core in terms of the reduction relation for types
(\(\typeStep{A}{A'}\)) (Figure~\ref{fig:type-reduction})
and terms (\(\step{M}{M'}\)) (Figure~\ref{fig:term-reduction}), which
incorporates both $\beta$-reduction and contextual congruence. We make
use of the transitive closure of these stepping relations via the
usual Kleene star notation.
% It seems that Harper talks about "contextual semactics" in the
% first edition of his book, but changed it to "contextual dymanics"
% in the second.  It doesn't seem to be a standard term elsewhere though.


\subsection{Type reduction}
Because the Plutus Core type system contains a copy of the simply
typed lambda calculus, complex computations can take place at the
level of types.  Reductions in the type system always transform a
given type into an equivalent type, and so have no effect on the term
level.  In conjunction with the fact that type reduction always
terminates, this allows us to perform normalisation (i.e., reduction
to a form which cannot undergo any further reduction) statically,
independently of term reduction.  Figure~\ref{fig:type-reduction}
contains the rules for type reduction, and
Figure~\ref{fig:type-equivalence} contains rules for type equivalence.
Here and elsewhere we use syntax of the form $[X/x]Y$ to denote
(capture-avoiding) substitution of some entity (a type, term, or
value) $X$ for all occurrences of a name $x$ within some other entity
$Y$.

\subfile{figures/TypeReduction}


\newpage
\subsection{Term reduction}
Execution of Plutus Core programs is performed by (possibly
non-terminating) reduction of well-typed terms.  The reduction rules
are contained in Figure~\ref{fig:term-reduction}, and give us a fairly
standard operational semantics.

\subfile{figures/TermReduction}

\noindent Note that there is no term $M^{\prime}$ such that
$\step{\error{B}}{M^{\prime}}$, so at most one of the final two rules
applies.

\subsection{An abstract machine for evaluating Plutus Core programs}
\label{sec:typed-ck-machine}
This section contains a description of an abstract machine for
executing Plutus Core.  This is based on the CK machine of Felleisen
and Friedman~\citep{Felleisen-CK-CEK}.

This machine is intended as a reference implementation which is
amenable to formalisation.  For example, it can be proved to implement
the operational semantics described in Section~\ref{sec:reduction}.
The CK machine is inefficient because it implements application as
$\beta$-reduction, so that when evaluating $(\lambda x.M)V$, $V$ must
be substituted bodily for $x$ wherever it occurs in $M$.  This process
can require considerable amounts of time and space.  More efficient
machines are available, and we describe one of these, the CEK machine,
in Appendix~\ref{appendix:typed-cek-machine}; however, the CK machine still
provides a useful bridge for formalisation purposes.

\subfile{figures/TypedCkMachine}

\noindent The machine alternates between two main phases: the
\textit{compute} phase ($\triangleright$), where it recurses down
the AST looking for values, saving surrounding contexts as frames (or
\textit{reduction contexts}) on a stack as it goes; and the
\textit{return} phase ($\triangleleft$), where it has obtained a value and
pops a frame off the stack to tell it how to proceed next.  In
addition there is an error state $\blacklozenge$ which halts execution
with an error, and a stop state $\square$ which halts execution and
returns a value to the outside world.

To evaluate a program $\texttt{(program}\ v\ M \texttt{)}$, we first
check that the version number $v$ is valid, then start the machine in
the state $\cdot \triangleright M$.  It can be proved that the
transitions in Figure~\ref{fig:typed-ck-machine} always preserve
validity of states, so that the machine can never enter a state such as
  $\cdot \triangleleft M$
or
$s, \texttt{(unwrap \_)} \triangleleft \texttt{(lam}\ x\ A \ M\texttt{)}$
which isn't covered by the rules.  If such a
situation were to occur in an implementation then it would indicate
that the machine was incorrectly implemented or that it was attempting
to evaluate an invalid program which should have been detected during
parsing or typechecking.

For efficiency reasons the behaviour of the machine defined in
Figure~\ref{fig:typed-ck-machine} differs in two respects from the semantics
defined in Figure~\ref{fig:term-reduction}:
\begin{itemize}
\item The CK machine does not perform type instantiation since this
  is computationally irrelevant.
\item When it encounters an \texttt{error} term the machine terminates
  immediately.  In Figure~\ref{fig:term-reduction} the \texttt{error}
  term propagates to the top of the stack of reduction frames,
  changing its type to match the surrounding context as it does so:
  this is to ensure that the program has a well-defined type.
\end{itemize}
\noindent It can be proved that when the CK machine as defined in
Figure~\ref{fig:typed-ck-machine} evaluates a program $P$, the machine
terminates in the $\blacklozenge$ state if and only if $P$ terminates
with an \texttt{error} value according to the rules of
Figure~\ref{fig:term-reduction}, and that the machine terminates with
a non-error value $V$ if and only if $P$ terminates according to the
reduction rules with a value $V^{\prime}$ which is equal to $V$
modulo erasure of type annotations.


\section{Built in types, functions, and values}
\label{sec:builtins}
Plutus Core comes with a predefined set of built-in types and
functions which will be useful for blockchain applications.  The set
of built-ins is fixed, although in future we may provide a framework
to allow customisation for specialised blockchains.

There are four basic types: \conUnitType, \conBooleanType, \conIntegerType,
and \conBytestringType\footnote{The implementation also supports a type
\conStringType{} of Unicode strings, but this is only used for debugging and we won't
discuss it any further here.}.  These types are given in
Figure~\ref{fig:type-constants}: for example \texttt{(con \conIntegerType)} is
the type of signed integers.  We provide standard arithmetic and comparison
operations for integers and a number of list-like functions for bytestrings. The
details are given in Figure~\ref{fig:builtins}, using a number of abbreviations
given in Figure~\ref{fig:type-abbreviations}.


Note the following:
\begin{itemize}
\item Some of the entries in Figure~\ref{fig:builtins}
  contain \textit{success conditions}.  If a success condition is
  violated then the function immediately returns $\errorU$.

\item Every built-in function returns either a value or the term \texttt{(error)}.

\item The \texttt{ifThenElse} operation is polymorphic and must be instantiated
    with the type of its branches before being applied: see the example below.

\item We provide two versions of the division and remainder operations
  for integers.  These differ in their treatment of negative
  arguments.

  \begin{itemize}

  \item \texttt{divideInteger} and \texttt{modInteger} implement the
    standard mathematical integer division operations (at least when
    the divisor is positive): \texttt{divideInteger} rounds
    downwards and the sign of \texttt{modInteger} $n$ $d$ is the same
    as the sign of $d$.  These correspond to Haskell's \texttt{div}
    and \texttt{mod} operators.

  \item \texttt{quotientInteger} and \texttt{remainderInteger}
    represent the operations found in many computer languages and
    CPUs: \texttt{quotientInteger} rounds towards zero and the sign
    of \texttt{remainderInteger $n$ $d$} is the same as the sign of
    $n$.  These correspond to Haskell's \texttt{rem} and \texttt{quot}
    operators.  \end{itemize}

% NOTE. The standard mathematical definition of integer
% division/remainder (the so-called _division algorithm_, or
% _Euclidean division_) is as follows: if n, d \in Z and d != 0
% then there exist unique integers q, r \in Z such that n=qb+r and
% 0 <= r < |d|; q is the _quotient_ and r is the _remainder_.
% Neither of our sets of operations implements this definition!
% For d>0, divideInteger and modInteger give the correct results,
% but they can give negative remainders for d<0.  In any sensible
% computation we'd probably have d>0, but it's worth being careful.

% For comparison, here are the results for some sample inputs.
% The last two columns give the results for Euclidean division.
%
%     n  d | div mod | quot rem |  q   r
%   --------------------------------------
%    41  5 |  8   1  |   8   1  |  8   1
%   -41  5 | -9   4  |  -8  -1  | -9   4
%    41 -5 | -9  -4  |  -8   1  | -8   1
%   -41 -5 |  8  -1  |   8  -1  |  9   4
%

\end{itemize}
\subfile{figures/Builtins}

\section{Examples}
\label{sec:examples}
The section contains some simple examples of Plutus Core code. The
first example is an entire program, but for simplicity subsequent ones
are presented as functions. The latter can easily be embedded into
programs (and applied to arguments) as in the first example.

\paragraph{Integer subtraction.}
The following program subtracts 11 from 55, returning 44.

\begin{verbatim}
  (program 1.0.0
    [(builtin subtractInteger) (con integer 55) (con integer 11)]
  )
\end{verbatim}


\paragraph{Absolute value.} The function below returns the absolute value of an
integer.
\begin{verbatim}
  (lam n (con integer)
     [
       [
        { (builtin ifThenElse) (con integer) } 
           [(builtin lessThanInteger) n (con integer 0)]
       ]
       [(builtin subtractInteger) (con integer 0) n]
       n
     ]
  )
\end{verbatim}

\noindent Note that both branches above will always be evaluated because Plutus
Core has strict evaluation.  Thus even if $n$ is positive the code above will
still calculate $0-n$ and then discard it.  To avoid this we can delay
the evaluation of the branches by making them into functions taking
arguments of type \textit{unit} and only applying the function once we
know the outcome of the test:

\begin{verbatim}
  (lam n (con integer)
    [
     [
      { (builtin ifThenElse) (fun (con unit) (con integer)) } 
        [(builtin lessThanInteger) n (con integer 0)]
         (lam u (con unit) [(builtin subtractInteger) (con integer 0) n])
         (lam u (con unit) n)
      ]
      (con unit ())
    ]
  )
\end{verbatim}

\paragraph{Other examples.} The Plutus Core evaluator in~\citep{Plutus-exe-repo}
has the ability to print out sample programs, including recursive
programs to calculate Fibonacci numbers and factorials.  The available
examples can be listed using \verb|plc example -a| and specific
examples can be printed out using, for example,
\verb|plc example -s factorial| (this particular example is quite lengthy, mostly because
it uses the techniques from~\citep{unravelling-recursion} to implement
recursion using an \verb|ifix|-based analogue of the $Z$ combinator).

% I was going to add the factorial function, but \verb|plc example -s
% factorial| gives you something 399 lines long, thanks
% to \verb|ifix|.

\section{Untyped Plutus Core}
\label{sec:untyped-plc}
For code transmission and on-chain execution it is important to have
small programs.  In this section we define an untyped version of
Plutus Core together with a type erasure function converting typed
Plutus Core terms to untyped ones.

As mentioned in Section~\ref{sec:blockchain-issues}, it is important to
limit the consumption of computational resources in on-chain code.  A
future version of this document will include a cost model which will
provide fine-grained costs for individual operations and built-in
functions, enabling accurate (dynamic) monitoring of execution costs.

\subfile{figures/UntypedGrammar}

\bigskip
\noindent The grammar is essentially that of typed Plutus Core with all types
and type-related operations discarded.  However, we require two new
items: \texttt{delay} and \texttt{force}.  These are needed because
type-level abstraction (\texttt{abs}) and instantiation (\texttt{\{\}})
have a computational effect.  The body of a type abstraction will not
be executed until the type is instantiated; thus we cannot just
replace $\abs{\alpha}{K}{M}$ with $M$, since the latter is evaluated
immediately.  As we will see below, \texttt{delay} and \texttt{force}
are used to preserve the effects of type abstraction and
instantiation (a more standard strategy of abstracting over the unit value was
rejected because it is less space-efficient).%
%% \footnote{A more standard, and semantically equivalent,
%% way to implement \texttt{delay} and \texttt{force} would be to use
%% $\lamU{u}{e}$ instead of $\delay{e}$ and $\appU{e}{\lamU{x}{x}}$
%% instead of $\force{e}$.  This has the
%% advantage of not requiring any new additions to the language, but is
%% less space-efficient.  Compared with erasing \texttt{abs}
%% and \texttt{\{\}} completely (which is semantically unsound), using
%% \texttt{delay} and \texttt{force} increasese script size by about 10\%, but
%% abstacting over the unit value increases sizes by about 20\%.  Small
%% script sizes are important, so we opted for the former.}

As with typed Plutus Core, we can show that values (terms which can
not be reduced) have a particular syntactic form: see
Figure~\ref{fig:untyped-values}.

\subsection{Type erasure}
Figure~\ref{fig:type-erasure} defines a type erasure operation which
converts a typed Plutus Core term $M$ into an untyped one $\erase{M}$
(objects from the typed grammar on the left of $=$, untyped on the
right).  We extend this to sequences of terms in the obvious way:
given $M^* = M_1 \ldots M_n$ we define $\erase{M^*}
= \erase{M_1} \ldots \erase{M_n}$.  Note that typed values erase to
untyped values.  Note also that it follows
from \cite{Wells-96-typability} that, given an untyped term $M$, it is
undecidable whether there is a typed term $\widetilde{M}$ with $\erase{\widetilde{M}} = M$.

For a typed program $P$ we use $\erase{P}$ to denote the untyped
program with the same version whose body is the erasure of the body of
$P$.

\subfile{figures/TypeErasure}


%% The next section is a rather vague attempt to try to explain what's
%% going on with untyped builtins in the absence of a proper
%% specification.  When we do have a specification this can be
%% removed, and we should talk about untyped Plutus Core rather than
%% type-erased Plutus Core (captions of CK machine figure etc.)

\subsubsection{Note. Built-in functions in untyped Plutus Core}
\label{sec:untyped-ck-builtins}
We assume that untyped Plutus Core contains a set of built-in functions
corresponding to those in Figure~\ref{fig:builtins}, and that these
behave consistently with respect to type erasure in the sense that
$$
\erase{\builtin{bn}{A_1 \ldots A_k\}{V_1 \ldots V_n}}} =
\builtinU{bn}{\erase{V_1} \ldots \erase{V_n}}
$$
for all typed values $V_1, \ldots, V_n$ (typed evaluation on the left,
untyped on the right).\footnote{We also assume that evaluation of
built-in functions always terminates, in both the typed and untyped
languages.}  In Section~\ref{sec:untyped-term-reduction} below we
define semantics for type-erased programs, and if the aforementioned
consistency condition is satisfied then it can be shown that the
erasure operation preserves semantics in an appropriate sense: if $P$
is a well-typed Plutus Core program then
\begin{itemize}
\item Evaluation of $P$ terminates if and only if that of $\erase{P}$ does.
\item The evaluation of $P$ returns \texttt{(error $A$)} for some type $A$
if and only if the evaluation of $\erase{P}$ returns \texttt{(error)}.
\item $P$ evaluates to the value $V$ if and only if $\erase{P}$ evaluates to the value
$\erase{V}$.
\end{itemize}
This allows
us to give semantics to those untyped programs which are obtained by
erasing types from well-typed typed programs.  A future version of
this document will contain a detailed specification of built-in types
and functions which will allow us to give precise semantics
for \textit{all} untyped terms and programs.

\subsection{Term Reduction}
\label{sec:untyped-term-reduction}
Figure~\ref{fig:untyped-term-reduction} defines a reduction semantics
for type-erased Plutus Core, analogously to
Figure~\ref{fig:term-reduction}.

\subfile{figures/UntypedTermReduction}

\subsection{An abstract machine for evaluating erased Plutus Core programs}
\label{sec:untyped-ck-machine}
Figure~\ref{fig:untyped-ck-machine} describes a modified version of
the CK machine (Figure~\ref{fig:typed-ck-machine}) which evaluates
type-erased Plutus Core programs, implementing the semantics of
Figure~\ref{fig:untyped-term-reduction}. As in the typed case, the CK
machine is not very efficient; a more efficient CEK machine is
described in Appendix~\ref{appendix:untyped-cek-machine}.

\subfile{figures/UntypedCkMachine}


%% ------------------------------- Appendices --------------------------------- %%

\begin{appendices}
\section{Algorithmic type system}

For implementation purposes it is useful to have a variant of the type
system that is more algorithmic in its presentation. We give this
here, showing the figures that have been changed (relative to the
declarative version above), and highlighting the specific parts of
each rule that is different, where possible
(Figures~\ref{fig:contexts-algorithmic-unrestricted}
and~\ref{fig:type-synthesis-algorithmic-unrestricted}) .

\subfile{figures/TypeSynthesis-Algorithmic.tex}


\section{A CEK machine for typed Plutus Core}
\label{appendix:typed-cek-machine}
In Section~\ref{sec:typed-ck-machine} we described the CK machine, an
abstract machine for executing Plutus Core programs.  As we pointed
out, the CK machine is inefficient because it performs application by
substitution: if one is evaluating $(\lambda x:t.e)e^{\prime}$ then
the whole of $e^{\prime}$ has to be substituted for every occurrence
of $x$ in $e$, which can be expensive in terms of both time and space.
This section presents a version of the \textit{CEK machine} of
~\citep{Felleisen-CK-CEK}: this uses environments to store bindings of
values to variables, where they can be looked up when required. This
makes evaluation considerably more efficient, to the extent that the
CEK machine can be used in practical situations.

\subsection{Structure of the CEK machine}
The CEK machine avoids the inefficiency of repeated substitution by
extending the state of the machine with
\textit{environments} mapping variable names to \textit{closures} consisting of
pairs $(V,\rho)$ where $V$ is a value and $\rho$ is another
environment containing bindings for free variables occurring in $V$.
The result of looking up a variable $x$ in an environment $\rho$ is
denoted by $\rho[x]$: if $x$ is has multiple bindings in $\rho$ then
the most recent one is returned.\footnote{In the implementation, names
are supplied with unique integer IDs which make them unambiguous; the
environments in fact map these IDs to closures.} If $x$ does not occur
in $\rho$ then an \texttt{error} is returned; however, it can be shown
that if the machine is started with a closed term $M$ as input then
variable lookups will never fail.  When the machine is evaluating
$[(\lambda x.M)\, N]$ in an environment $\rho$ it first evaluates $N$
in $\rho$ to produce a value $V$, then extends $\rho$ with the mapping
$x \mapsto (V,\rho)$ to obtain a new environment $\rho^{\prime}$, then
proceeds to evaluate $M$ in the environment $\rho^\prime$: when any
occurrences of $x$ are encountered, the machine retrieves the value
$V$ from the environment and proceeds with evaluation as normal.  It
is necessary to use closures because values may now contain variable
names, and we have to know what the variables were bound to at the
time when the value was created.

\subsubsection{Discharging closures}
Given a closure $(V,\rho)$ we denote by $\discharge{(V,\rho)}$ the
result of \textit{discharging} the closure, ie, recursively replacing
all of the variables occurring in $V$ by their bindings in
$\rho$. Note that variables are bound to closures which may themselves
need to be discharged: hence the use of the word `recursively'. It
can be shown that when any closure produced by the CEK machine is
discharged it yields a value.

\subsection{Definition of the CEK machine}
The CEK machine is described in Figure~\ref{fig:cek-machine}.  It is
  is very similar to the CK machine of Figure~\ref{fig:typed-ck-machine},
  except that
\begin{itemize}
\item Every rule now involves environments
\item A new rule has been added to deal with the evaluation of variables
\item The rule involving lambda application has changed to replace substitution with
the use of an extended environment
\item When the machine terminates without error it now
returns a closure $(V,\rho)$ rather than a value.  The invoker
of the machine may wish to discharge this closure to obtain a value.
\end{itemize}
\noindent The major changes are highlighted in \blue{blue}. To evaluate a program
\texttt{(program $v$ $M$)} we check the version number $v$ and then evaluate the body $M$
with an empty stack and environment: $\cdot;[] \compute M$.

\subfile{figures/TypedCekMachine.tex}

\subsubsection{Note. Built-in functions in the CEK machine}
\label{sec:typed-cek-builtins}
In the CK machine the arguments of built-in functions are always
closed values, but in the CEK machine arguments are closures.  We
assume that the mechanism for evaluating built-in functions in the CEK
machine can accept closures as arguments, and that
\begin{itemize}
\item Evaluation of built-in functions always terminates, returning either a value or \texttt{(error)}
(this is true for all of the current built-in functions).


\item For closures $C_1, \ldots, C_n$, evaluation of
  $\builtin{bn}{A_1 \ldots A_k}{C_1 \ldots C_n}$ fails in the CEK
machine (returning $\error{A}$ for some type $A$) if and only
if evaluation of $\builtin{bn}{A_1 \\ \ldots  %% Extends way into the margin without this
A_k}{\discharge{C_1} \ldots \discharge{C_n}}$ fails (returning $\errorU$) according to
Figure~\ref{fig:builtins}.

\item For closures $C_1, \ldots, C_n$,
  $\builtin{bn}{A_1 \ldots A_k}{C_1 \ldots C_n}$
evaluates to a value $V$ in the CEK machine if and only if 
  $\builtin{bn}{A_1 \ldots A_k}{\discharge{C_1} \ldots \discharge{C_n}}$
evaluates to the same value $V$ according to Figure~\ref{fig:builtins}.% <- No space before footnote marker.
\footnote{One way to ensure that these conditions hold would be to discharge all of the
closures and then use the same mechanism as in the CK machine, but this would
reintroduce the very inefficiencies which the CEK machine is designed to avoid.}

\end{itemize}
A future version of this document will contain a
specification of built-in function evaluation which will cover this
issue in detail.

% [NOTE] We'll need to assume (or deduce from the specification of
% builtins) that builtins always return terms that are well-typed and
% which don't contain any out-of-scope variables (We'll need these
% conditions for the CK machine as well).  This doesn't matter at the
% moment because our fixed set of builtins always return values of
% built-in types or `error` (which annoyingly isn't a value), but when
% we have extensible builtins (including polymorphic ones) things will
% be trickier.


\subsubsection{Comparison of the CK and CEK machines}
If the CK and CEK machines are run with the same program as input then
we can prove the following facts (assuming that the condition in
Section~\ref{sec:typed-cek-builtins} holds):
\begin{itemize}
\item The CEK machine terminates if and only if the CK machine terminates.
\item The CEK machine terminates in the error state $\blacklozenge$ if
and only if the CK machine terminates in the state $\blacklozenge$.
\item The CEK machine returns a closure $C$ if and only if the CK machine returns a
value $V$ with $\discharge{C} = V$.
\end{itemize}

\section{A CEK machine for type-erased Plutus Core}
\label{appendix:untyped-cek-machine}
As in the typed situation, the CK machine for type-erased Plutus Core
(Figure~\ref{fig:untyped-ck-machine}) can be replaced with a more
efficient CEK machine.  This is obtained by applying modifications
similar to those which we used to obtain the CEK machine for typed
Plutus Core from the CK machine for typed Plutus Core.  We will refer
to this machine as the \textit{untyped CEK machine}, but we emphasise
that the version presented here is only to be used to evaluate terms
which are obtained by erasing types from well-typed Plutus Core terms.
A later version of this document will contain a specification for
evaluation of built-in functions which will allow us to evaluate
arbitrary untyped Plutus Core terms.

\subsection{Structure of the untyped CEK machine}
The untyped CEK machine uses closures and environments similar to
those used in the typed CEK machine
(Appendix~\ref{appendix:typed-cek-machine}), except that environments now
bind variables to terms in untyped Plutus Core.  Again,
$\discharge{(V,\rho)}$ denotes the value obtained by discharging a
closure $(V,\rho)$.

\subfile{figures/UntypedCekMachine.tex}

\subsubsection{Note. Built-in functions in the untyped CEK machine}
\label{sec:untyped-cek-builtins}
As in Sections~\ref{sec:untyped-ck-builtins}
and~\ref{sec:typed-cek-builtins} we have to say something about how
evaluation of built-in functions is performed in the untyped CEK
machine. We assume that the mechanism for evaluating built-in
functions in the untyped CEK machine can accept closures as arguments,
and that built-in evaluation commutes with type erasure and discharge
of environments in an appropriate sense; more precisely, we require
that the following conditions hold.

\medskip
\noindent Suppose that
\begin{itemize}
\item $A_1, \ldots, A_k$ are Plutus Core types.
\item $V_1, \ldots, V_n$ are well-typed typed Plutus Core values.
\item $C_1, \ldots, C_n$ are closures in the untyped
CEK machine such that $\discharge{C_i} = \erase{V_i} $ for
$1 \le i \le n$.
\end{itemize}

\noindent Then we require that
\begin{itemize}
\item Evaluation of built-in functions always terminates, returning either a value or \texttt{(error)}.

\item In the untyped CEK machine, evaluation of 
  $\builtinU{bn}{C_1 \ldots C_n}$
fails (returning $\errorU$) if and only if evaluation of 
  $\builtin{bn}{A_1 \ldots A_k}{V_1 \ldots V_n}$
fails according to Figure~\ref{fig:builtins}.

\item In the untyped CEK machine,
  $\builtinU{bn}{C_1 \ldots C_n}$
evaluates to the untyped value $V$ if and only if
  $\builtin{bn}{A_1 \ldots A_k}{V_1 \ldots V_n}$
evaluates to a typed term $W$ (according to Figure~\ref{fig:builtins}) with $\erase{W} = V$.
\end{itemize}

\subsubsection{Comparison of the typed CK machine and the untyped CEK machine}
Suppose that the conditions of Section~\ref{sec:untyped-cek-builtins}
hold, and that $P$ is a typed Plutus Core program.  Then if $P$ is
evaluated using the typed CK machine and $\erase{P}$ is evaluated
using the untyped CEK machine we can prove the following:
\begin{itemize}
\item The typed CK machine terminates if and only if the untyped CEK machine terminates.
\item The typed CK machine halts in the error state $\blacklozenge$ if and only if
the untyped CEK machine halts in the $\blacklozenge$ state.
\item The typed CK machine returns a value $V$ if and only if the untyped CEK machine returns a
closure $C$ with $\discharge{C} = \erase{V}$.
\end{itemize}

\section{Serialisation}

We use the \texttt{flat} \citep{flat} format to serialise Plutus Core terms. The
\texttt{flat} format encodes sum types as tagged unions and products by
concatenating their contents. We proceed by defining the structure and the data
types of untyped Plutus Core and how they get serialised.

\subsection{Variable length data}

\texttt{Non-empty lists} are encoded by prefixing the element stored with `0'
if this is the \texttt{last} element or `1' if there is \texttt{more} data following.

\noindent We encode \texttt{Integers} as a non-empty list of chunks, 7 bits each,
with the least significant chunk first and the most significant bit first in the chunk.

\medskip
\noindent Let's calculate the encoding of the \texttt{32768} index (unsigned, arbitrary
length integer):
\begin{enumerate}
  \item Converting \texttt{32768} to binary: \\
    \verb|32768| $\rightarrow$ \verb|0b1000000000000000|
  \item Split into 7 bit chunks: \\
    \verb|0b1000000000000000| $\rightarrow$ \verb|0000010 0000000 0000000|
  \item Reorder chunks (least significant chunk first): \\
    \verb|0000010 0000000 0000000| $\rightarrow$ \verb|0000000 0000000 0000010|
  \item Add list constructor tags: \\
    \verb|0000000 0000000 0000010| $\rightarrow$ \verb|10000000 10000000 00000010|
\end{enumerate}

For \texttt{ByteString}s, \texttt{Char}s and \texttt{String}s we use a byte aligned
array of bytes (in the case of \texttt{Char} and \texttt{String} the bytes correspond
to the UTF-8 encoding of the text). The structure is pre-aligned to the byte boundary
by using the `0' bit as a filler and the `1' bit as the final bit. Following the filler
we have the number of bytes that the data uses, a number from 0 to 255 (1 byte),
followed by the bytes themselves, and a final 0 length block (the byte `0').

\subsection{Constants}

Constants are encoded as a combination of a tag indicating the kind of value that is serialized and the
value itself. Constants use 4 bits to encode the tags, so they allow for a
maximum of 16 constructors, of which 6 are used by the default set of builtin types.

\vspace{1cm}

\begin{minipage}{\linewidth}
\centering
\begin{tabular}{|l|c|l|}
  \hline
  \Strut
  \textrm{Name} & \textrm{Tag} & \textrm{Encoding} \\
  \hline
  Integer & 0 & ZigZag + Variable length \\[\sep]
  ByteString & 1 & Variable length \rule{0mm}{4mm} \\[\sep]
  String & 2 & UTF-8 \\[\sep]
  Char & 3 & UTF-8 \\[\sep]
  Unit & 4 & Empty \\[\sep]
  Bool & 5 & `1' is True, `0' is False \\[\sep]
  \hline
\end{tabular}
\captionof{figure}{Constant kinds}
\label{fig:serialisation-constants}
\end{minipage}

\subsection{Untyped terms}

Terms are encoded using 4 bit tags, which allows a total of 16 kinds of term, of which 8 are
currently used.

\vspace{1cm}

\begin{minipage}{\linewidth}
\centering
\begin{tabular}{|l|c|l|}
  \hline
  \Strut
  \textrm{Name} & \textrm{Tag} & \textrm{Arguments} \\
  \hline
  Variable & 0 & name \rule{0mm}{4mm} \\[\sep]
  Delay & 1 & term \\[\sep]
  Lambda abstraction & 2 & name, term \\[\sep]
  Application & 3 & term, term \\[\sep]
  Constant & 4 & constant \\[\sep]
  Force & 5 & term \\[\sep]
  Error & 6 & term \\[\sep]
  Builtin & 7 & builtin \\[\sep]
  \hline
\end{tabular}
\captionof{figure}{Untyped terms}
\label{fig:serialisation-terms}
\end{minipage}

\vspace{1cm}

\noindent Builtins use 5 bits for their tags, allowing for a maximum of 32 builtin
functions of which 25 are used.

\vspace{1cm}

\begin{minipage}{\linewidth}
\centering
\begin{tabular}{|c|c|c|c|c|c|}
  \hline
  \Strut
  \textrm{Name} & \textrm{Tag} & \textrm{Name} & \textrm{Tag} & \textrm{Name} & \textrm{Tag} \\
  \hline
   AddInteger & 0 & GreaterThanEqInteger & 8 & QuotientInteger & 17 \rule{0mm}{4mm} \\[\sep]
   SubtractInteger & 1 & EqInteger & 9 & ModInteger & 18 \\[\sep]
   MultiplyInteger & 2 & Concatenate & 10 & LtByteString & 19 \\[\sep]
   DivideInteger & 3 & TakeByteString & 11 & GtByteString & 20 \\[\sep]
   RemainderInteger & 4 & DropByteString & 12 & IfThenElse & 21 \\[\sep]
   LessThanInteger & 5 & SHA2 & 13 & CharToString & 22 \\[\sep]
   LessThanEqInteger & 6 & SHA3 & 14 & Append & 23 \\[\sep]
   GreaterThanInteger & 7 & VerifySignature & 15 & Trace & 24 \\[\sep]
   & & EqByteString & 16 & & \\[\sep]
   \hline
\end{tabular}
\captionof{figure}{Builtin tags}
\label{fig:serialisation-builtins}
\end{minipage}

\vspace{1cm}

\noindent Encoding primtive values:

\begin{itemize}
  \item Unsigned \texttt{Integer}s and \texttt{ByteString}s are encoded using the
    previously introduced encoding for variable length data types.
  \item Signed \texttt{Integer}s are first converted to an unsigned value using the
    \texttt{ZigZag}\footnote{The \texttt{ZigZag} encoding interleaves positive and negative numbers such that small
    negative numbers are stored using a small number of bytes.} encoding, then they are encoded as variable length data types.
  \item \texttt{String}s are encoded as lists of characters and use the \texttt{UTF-8}
    encoding.
  \item \texttt{Unit} and data structures which have only one constructor, are removed
    from the serialised data.
  \item Variable names are encoded using DeBruijn indices, which are unsigned, arbitrary
    length integers.
\end{itemize}

Possibly empty lists are encoded in the standard way, by the tag for the constructor,
`0' for `Nil' and `1' for `Cons'. The `Cons' constructor is followed by the serialised
element, and then, recursiveley another list.

Encoded values are aligned to byte/word boundary using a meaningless sequence of `0' bits
terminated with a `1' bit.

\subsection{Example}

We will serialise the program \verb|(program 11.22.33 (con integer 11))| compiled to untyped plutus core, using DeBruijn indices.

First, lets convert the program to the desired representation:

\begin{verbatim}
> stack exec plc -- convert --untyped --if plc --of flat -o program.flat <<EOF
> (program 11.22.33 (con integer 11))
> EOF
\end{verbatim}

Now, let's take a look at the output.

\begin{verbatim}
> xxd -b program.flat
> 00000000: 00001011 00010110 00100001 01001000 00000101 10000001  ..!H..
\end{verbatim}

\subsubsection{The program preamble.}

We define `Program` in the `PlutusCore.Core.Type` haskell module like this:

\begin{verbatim}
-- | A 'Program' is simply a 'Term' coupled with a 'Version' 
--   of the core language.
data Program tyname name uni fun ann = 
       Program ann (Version ann) (Term tyname name uni fun ann)
         deriving (Show, Functor, Generic, NFData, Hashable)
\end{verbatim}

Because the \verb|Program| data type has only one constructor we know that flat will not waste any space serialising it. `ann' will always be (for serialised ASTs) `()', which similarly to the \verb|Program| data type has only one constructor and flat will not serialise it.

Next, the `Version' is a tuple of 3 `Natural' numbers, which are encoded as variable length unsigned integers. Because all the version numbers can fit in a 7 bit word, we only need one byte to store each of them. Also, the first bit, which represents the non-empty list constructor will always be `0' (standing for `Last'), resulting in:

\begin{verbatim}
0 (*Last*) 000 (*Unused*) 1011   (*11 in binary*)
0 (*Last*) 00  (*Unused*) 10110  (*22 in binary*)
0 (*Last*) 0   (*Unused*) 100001 (*33 in binary*)
\end{verbatim}

\subsubsection{The integer constant `1`.}

Let's take a quick look at how we defined untyped plutus terms, in Figure~\ref{fig:serialisation-terms}.

We need to encode the `Constant', signed integer value `11'. Terms are encoded using 4 bits, and the `Constant' term has tag 4. This results in:

\begin{verbatim}
0 (*Unused*) 100 (*4 in binary*)
\end{verbatim}

For the `Default' universe we have the constant tags defined from Figure~\ref{fig:serialisation-constants}, wrapped in a list, followed by the encoding for the constant's value.

So we see how, for the integer type we care about the type is encoded as a list containing the id `0'. We know that we are using 3 bits to store the type of constant, so the encoding will be:

\begin{verbatim}
1 (*Cons*) 0000 (The `0` tag using 4 bits for storage) 0 (*Nil*)
\end{verbatim}

The annotation will not be serialised, and we are left with the constant itself. Because it is an variable length signed integer, we first need to find out it's value after conversion to the `ZigZag' format.

\begin{verbatim}
> stack repl plutus-core:exe:plc
> ghci> import Data.ZigZag
> ghci> zigZag (11 :: Integer)
> 22
\end{verbatim}

Next, we need to encode the variable length unsigned integer `22'. We only need one byte (as it fits in the available 7 bits), so we end up with the following:

\begin{verbatim}
0 (*Last*) 00 (*Unused*) 10110 (*22 in binary*) 000001 (*Padding to byte size*)
\end{verbatim}

\subsubsection{Note}

You may notice how in the rest of the codebase we use the `CBOR' format to serialise
everything.

So why did we choose to switch to `Flat' for on-chain serialisation?

`CBOR' pays a price for being a self-describing format. The size of the serialised
terms is consistently larger than a format that is not self-describing. Running the
`flat' benchmarks will show flat consistently out-performing `CBOR' by about 35\%
without using compression.

\begin{verbatim}
> stack bench plutus-benchmark:flat
> cat plutus-benchmark/flat-sizes.md

** Contract: crowdfunding-indices **
Codec            Size    Of minimum   Of maximum
flat             8148    2.240308     0.62652826
cbor             13005   3.5757492    1.0

** Contract: escrow-indices **
Codec            Size    Of minimum   Of maximum
flat             8529    2.2004645    0.6302838 
cbor             13532   3.491228     1.0

** Contract: future-indices **
Codec            Size    Of minimum   Of maximum
flat             17654   2.19141      0.6628619 
cbor             26633   3.305983     1.0

** Contract: game-indices **
Codec            Size   Of minimum   Of maximum
flat             5158   2.2290406    0.6254395 
cbor             8247   3.5639584    1.0

** Contract: vesting-indices **
Codec            Size    Of minimum   Of maximum
flat             8367    2.2288227    0.6273525 
cbor             13337   3.5527437    1.0
\end{verbatim}

\end{appendices}

\bibliographystyle{plainnat} %% ... or whatever
\bibliography{plutus-core-specification}

\end{document}
