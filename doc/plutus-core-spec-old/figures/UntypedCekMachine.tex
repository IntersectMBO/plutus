\documentclass[../plutus-core-specification.tex]{subfiles}

\begin{document}


\begin{figure}[H]
\begin{subfigure}[c]{\linewidth}
    \centering
    \[\begin{array}{lrclr}
        \textrm{Stack} & s      & ::= & \cdot \enspace | \enspace s,f\\
 \blue{ \textrm{Environment}} & \rho & ::= & [] \enspace | \enspace \rho[x \mapsto C] \\
        \blue{ \textrm{Closure}} & C  & ::= & (V,\rho) \\
        \textrm{State} & \sigma & ::= & s;\rho \compute M \enspace | \enspace s;\rho \return V  \enspace | \enspace \ckerror{} \enspace | \enspace \square (V,\rho)\\
    \end{array}\]

    \caption{Grammar of CEK machine states for type-erased Plutus Core}
    \label{fig:untyped-cek-frames}
\end{subfigure}
\begin{subfigure}[c]{\linewidth}
    \centering
    \[\begin{array}{lrclr}
        \textrm{Frame} & f  & ::=   & \inAppLeftFrame{M}                  & \textrm{left application}\\
                       &   &     & \inAppRightFrame{C}                    & \textrm{right application}\\
                       &   &     & (\inBuiltinU{bn}{C^*}{\_}{M^*}, \rho)  & \textrm{builtin application}\\
                       &   &     & \inForceFrame                          & \textrm{force}
    \end{array}\]
    \caption{Grammar of reduction frames for type-erased Plutus Core}
    \label{fig:untyped-cek-reduction-frames}
\end{subfigure}
\caption{A CEK machine for type-erased Plutus Core}
\end{figure}

% Allow page break for (slightly) better placement
\begin{figure}[H]
\ContinuedFloat
  \begin{subfigure}[c]{\linewidth}
    \judgmentdef{$\sigma \mapsto \sigma^{\prime}$}{Machine takes one step from state $\sigma$ to state $\sigma'$}

%\hspace{-1cm}
    \begin{minipage}{\linewidth}
\begin{alignat*}{2}
  \blue{s; \rho} & \bcompute \blue{x} &~\bmapsto~ &\blue{s;\rho^{\prime} \return V \enspace (\rho[x] = (V,\rho^{\prime}))}\\
  s;\rho & \compute \con{tn}{cn}                      &~\mapsto~& s;\rho \return \con{tn}{cn}\\
  s;\rho & \compute \lamU{x}{M}                       &~\mapsto~& s;\rho \return \lamU{x}{M}\\
  s;\rho & \compute \delay{M}                         &~\mapsto~& s;\rho \return \delay{M}\\
  s;\rho & \compute \force{M}                         &~\mapsto~& s,\inForceFrame{};\rho \compute M \\
  s;\rho & \compute \appU{M}{N}                       &~\mapsto~& s,\inAppLeftFrame{N};\rho \compute M\\
  s;\rho & \compute \builtinU{bn}{\!\!}               &~\mapsto~& s;\rho \compute M
                                                 \enspace (\textit{$bn$ computes to $M$}) \\
  s;\rho & \compute \builtinU{bn}{M \repetition{M}}   &~\mapsto~& s,(\inBuiltinU{bn}{}{\_}{\repetition{M}}, \rho);\rho \compute M\\
  s;\rho & \compute \errorU                           &~\mapsto~& \ckerror{}\\
  \\[-10pt] %% Put some vertical space between compute and return rules, but not a whole line
  \cdot & \return V                              &~\mapsto~& \square (V,\rho)\\
  s,\inAppLeftFrame{N};\rho& \return V               &~\mapsto~& s,\inAppRightFrame{V};\rho \compute N\\
  \blue{s,\inAppRightFrame{(\lam{x}{A}{M}, \rho^{\prime})}; \rho} &\breturn V &~\bmapsto~& \blue{s;\rho^{\prime}[x\mapsto (V,\rho)] \compute M}\\  %% Modified
  s,\inForceFrame{};\rho & \return \delay{M}          &~\mapsto~& s;\rho \compute M\\
  s,(\inBuiltinU{bn}{\repetition{C}}{\_}{M \repetition{M}}, \rho^{\prime});\rho & \return V
                                                 &~\mapsto~& s,(\inBuiltinU{bn}{\repetition{C}(V,\rho)}{\_}{\repetition{M}}, \rho^{\prime});\rho^{\prime}
                                                 \compute M\\
  s,(\inBuiltinU{bn}{\repetition{C}}{\_}{}, \rho^{\prime});\rho & \return V
                                                 &~\mapsto~&  s;\rho^{\prime} \compute M
                                                 \enspace (\textit{$bn$ computes on $\repetition{C}(V,\rho)$ to $M$})\\
    \end{alignat*}
\end{minipage}
    \caption{CEK machine transitions for type-erased Plutus Core}
    \label{fig:untyped-cek-transitions}
\end{subfigure}
\caption{A CEK machine for type-erased Plutus Core}
\label{fig:untyped-cek-machine}
\end{figure}

%% The rule for builtins looks a bit strange. For one thing, it gives
%% us two ways to write built-in constants: $\con{X}$ and
%% $\builtinU{X}{\!\!}$.  I think this was deliberately included
%% in Rebecca's original typed syntax to give us a way to
%% instantiate polymorphic constants.  Roman says it will probably
%% still be useful for his extensible builtins.

\end{document}
