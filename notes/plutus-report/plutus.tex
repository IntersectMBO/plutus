% The Plutus Platform Technical Report
% Note to contributors:
% - I am using a glossary package, as there is a *lot* of novel terminology in here.
%   Please use it as appropriate.
% - I am trying out a "precisely one sentence per line" policy (as opposed to reflowing with hard breaks) in an attempt to make diffs nicer.
%   Please also do so.
\documentclass{article}

\usepackage{amsmath, amssymb, stmaryrd, latexsym, mathtools}
\usepackage[hyperref]{ntheorem}

\usepackage[dvipsnames]{xcolor}
\usepackage{todonotes}
\usepackage{paralist}

% Smaller margins are nicer for this kind of content, I think.
\usepackage[margin=2.5cm]{geometry}

\usepackage{tikz}
\usetikzlibrary{arrows,shapes,snakes,automata,backgrounds,petri}

% TODO: better style
\usepackage[style=numeric]{biblatex}
\addbibresource{plutus.bib}

\usepackage{hyperref}
\hypersetup{
  breaklinks=true,
  colorlinks=true,
  urlcolor=NavyBlue,
  linkcolor=BrickRed,
  citecolor=Green
}

% Capitalize the "type" of the link, e.g. Section 2
\usepackage[capitalise]{cleveref}
% Automatically build the glossary index, and don't show the occurrence lists or add a dot at the end
\usepackage[automake=true, nonumberlist]{glossaries-extra}

\usepackage{fancyhdr}
\pagestyle{fancy}
\lhead{\color{red} DRAFT: DO NOT DISTRIBUTE}

\newcommand{\todompj}[1]{\todo[inline,color=yellow!40,author=Michael]{#1}}
% TODO: nicer code formatting inline and also in blocks
\newcommand{\code}[1]{\texttt{#1}}

% Break after header and don't italicise the body
\theoremstyle{break}
\theorembodyfont{\normalfont}

% An environment for requirements
\newtheorem{requirement}{Requirement}[section]

% cleveref "respects" your crefname and doesn't uppercase it even when
% capitalise is set, so you have to manually use their internal variable.
\makeatletter
\if@cref@capitalise
\crefname{requirement}{Requirement}{Requirements}
\else
\crefname{requirement}{requirement}{requirements}
\fi
\makeatother

% TODO: review the style
\setglossarystyle{altlist}
\makeglossaries
\setabbreviationstyle[acronym]{long-short-desc}
% General
\newglossaryentry{ada}{
  name=ada,
  description={
    The primary currency on the \gls{cardano} blockchain. Unlike other \glspl{currency}, ada cannot be forged.
  }
}
\newglossaryentry{cardano}
{
  name=Cardano,
  description={A third-generation blockchain developed by IOHK.}
}
\newglossaryentry{daedalus}
{
  name=Daedalus,
  description={The primary wallet frontend developed by IOHK for use with \gls{cardano}.}
}
\newglossaryentry{plutus-platform}
{
  name=Plutus Platform,
  description={
    The combined software support for writing \glspl{app}.
  }
}
\newglossaryentry{slot-leader}
{
  name=slot leader,
  description={
    The servers adding new blocks to the chain in the proof-of-stake Ouroboros protocol, corresponding to the miners of a proof-of-work protocol as employed in Bitcoin.
  }
}
\newglossaryentry{node}
{
  name=node,
  description={
    The user's local interface to the blockchain.
  }
}
\newglossaryentry{wallet-backend}
{
  name=wallet backend,
  description={
    The service that provides most of the wallet functionality, e.g. balance tracking, transaction submission, key management.
  }
}
\newglossaryentry{wallet-frontend}
{
  name=wallet frontend,
  description={
    A graphical user interface to a wallet, typically backed by a \gls{wallet-backend}, e.g. \gls{daedalus}.
  }
}

% Ledger
\newglossaryentry{address}{
  name=address,
  description={
    The address of an \gls{utxo} says where the output is ``going''.
    The address stipulates the conditions for unlocking the output.
    This can be a public key hash, or (in the \gls{eutxo-model}) a \gls{script} hash.
  }
}
\newglossaryentry{context}{
  name=context,
  description={
    A data structure containing a summary of the transaction being validated.
    See \cref{sec:eutxo}.
  }
}
\newglossaryentry{currency}{
  name=currency,
  plural=currencies,
  description={
    A class of \gls{token} whose \gls{forging} is controlled by a particular \gls{mps}.
  }
}
\newglossaryentry{currency-id}{
  name=currency id,
  description={
    An identifier for a \gls{currency}.
    This is the hash of the \gls{mps} that controls the \gls{forging} of the \gls{currency}.
  }
}
\newglossaryentry{data}{
  name=\textsf{Data},
  description={
    A generic type of structured data.
    See \cref{sec:data}.
  }
}
\newglossaryentry{datum}{
  name=datum,
  description={
    The data field on script outputs in the \gls{eutxo-model}.
    See \cref{sec:eutxo}.
  }
}
\newacronym[description={Our extended ledger model. See \cref{sec:eutxo}.}]{eutxo}{EUTXO}{Extended UTXO}
\newglossaryentry{eutxo-model}
{
  name=Extended UTXO Model,
  description={
    The ledger model which the \gls{plutus-platform} relies on.
    This is implemented in the Goguen release of \gls{cardano}.
    See \cref{sec:eutxo}.
  }
}
\newglossaryentry{forging}{
  name=forging,
  description={
    A transaction which forges \glspl{token} creates new \glspl{token}, providing that the
    corresponding \gls{mps} is satisfied.
    The amount forged can be negative, in which case the \glspl{token} will be destroyed instead of created.
    See \cref{sec:multicurrency}.
  }
}
\newglossaryentry{fungible}
{
  name=fungible,
  description={
    A \gls{token} is \emph{fungible} with another token if it can be used interchangeably with that token.
  }
}
\newglossaryentry{ledger}{
  name=ledger,
  description={
    A system for tracking ownership and transfers of assets.
    We will usually use this to mean ``distributed ledger''.
    See \cref{sec:ledger}.
  }
}
\newacronym[description={
  A \gls{script} which must be satisfied in order for a transaction to forge \glspl{token} of the corresponding \gls{currency}.
  See \cref{sec:multicurrency}.
}]{mps}{MPS}{Monetary Policy Script}
\newglossaryentry{multicurrency}
{
  name=multicurrency,
  description={
    A generic term for a ledger which supports multiple different currencies natively.
    See \cref{sec:multicurrency}.
  }
}
\newacronym[description={
  A unique token which is not \gls{fungible} with any other token.
}]{nft}{NFT}{Non-Fungible Token}
\newglossaryentry{one-shot}
{
  name=one-shot,
  description={
    A one-shot script is a script that requires that the pending transaction spend a specific existing transaction output.
    Since transaction outputs can only be spent once, this ensures that the script can only be (successfully) run once also.
    We may also refer to a one-shot transaction, which is a transaction that contains a one-shot script.
  }
}
\newglossaryentry{redeemer}
{
  name=redeemer,
  description={
    The argument to the validator \gls{script} which is provided by the transaction which spends a \gls{script-output}.
    See \cref{sec:eutxo}.
  }
}
\newglossaryentry{script}
{
  name=script,
  description={
    A generic term for an executable program used in the ledger.
    In the \gls{plutus-platform}, these are written in \gls{plutus-core}.
    See \cref{sec:eutxo}.
  }
}
\newglossaryentry{script-output}
{
  name=script output,
  description={
    A transaction output locked by a \gls{script}.
    See \cref{sec:eutxo}.
  }
}
\newglossaryentry{token}
{
  name=token,
  description={
    A generic term for a native tradeable asset in the ledger.
  }
}
\newglossaryentry{token-name}{
  name=token name,
  description={
    An identifier for a class of \glspl{token} within a \gls{currency}.
  }
}
\newacronym[description={
  A transaction output which has not been spent.
  Also refers to the traditional ledger model going back to Bitcoin.
}]{utxo}{UTXO}{Unspent Transaction Output}
\newglossaryentry{validator}
{
  name=validator script,
  description={
    The \gls{script} attached to a \gls{script-output} in the \gls{eutxo-model}.
    Must be run and return positively in order for the output to be spent.
    Determines the \gls{address} of the output.
    See \cref{sec:eutxo}.
  }
}
\newglossaryentry{value}
{
  name=\textsf{Value},
  description={
    A type representing a bundle of assets (\glspl{token}) in our \gls{multicurrency} system.
    See \cref{sec:value}.
  }
}

% Scripting
\newglossaryentry{cost-model}
{
  name=cost-model,
  description={
    A set of parameters which affect how the \gls{plutus-core} evaluator costs a program evaluation.
    See \cref{sec:costing}.
  }
}
\newglossaryentry{exunits}
{
  name=exunits,
  description={
    The units used to track execution costs.
    A pair of ``abstract time'' and ``abstract memory''.
    See \cref{sec:costing}.
  }
}
\newglossaryentry{space}
{
  name=Space,
  description={
    The abstract unit of memory.
    See \cref{sec:costing-units}.
  }
}
\newglossaryentry{time}
{
  name=Time,
  description={
    The abstract unit of time.
    See \cref{sec:costing-units}.
  }
}
\newglossaryentry{plutus-core}
{
  name=Plutus Core,
  description={
    The programming language used for \glspl{script} in the \gls{plutus-platform}.
    See \cref{sec:plutus-core}.
  }
}
\newglossaryentry{system-f}
{
  name={\ensuremath{\textrm{System}\ F}},
  description={
    A well-known simple programming language, often called the ``polymorphic lambda calculus''.
  }
}
\newglossaryentry{system-fomf}
{
  name={\ensuremath{\textrm{System}\ F_{\omega}^\mu}},
  description={
    An extension of \gls{system-f} with recursive types and higher-kinded types.
  }
}

% Applications
\newglossaryentry{app}
{
  name=Plutus application,
  description={
    TODO: better name?
    An application built using the \gls{plutus-platform}.
  }
}
\newglossaryentry{app-api}
{
  name=contract API,
  description={
    TODO: better name?
    The interface provided by the \gls{paf} which \glspl{app} must use.
  }
}
\newglossaryentry{app-inst}
{
  name=application instance,
  description={
    A particular instance of an \gls{app}.
  }
}
\newglossaryentry{app-exe}
{
  name=application executable,
  description={
    The compiled executable from an \gls{app}.
  }
}
\newglossaryentry{chain-index}
{
  name=chain index,
  description={
    A component of the \gls{pab} that tracks information from the chain, notably output \glspl{datum}.
  }
}
\newglossaryentry{marlowe}{
  name=Marlowe,
  description={A domain-specific language for writing financial contract applications.}
}
\newglossaryentry{off-chain}
{
  name=off-chain code,
  description={
    The part of an \gls{app}'s code which runs off the chain, usually on a user's computer.
  }
}
\newglossaryentry{on-chain}
{
  name=on-chain code,
  description={
    The part of an \gls{app}'s code which runs on the chain (i.e. as scripts).
  }
}
\newglossaryentry{pab-services}
{ name={\gls{pab} services},
  description={
    Services used by \glspl{app} that the \gls{pab} knows how to connect to.
    The services are \gls{chain-index}, \gls{wallet-backend} and \gls{node}.
  }
}
\newacronym[description={
  The component which manages \glspl{app} run on users' machines.
  See \cref{sec:pab}.
}]{pab}{PAB}{Plutus Application Backend}
\newacronym[description={
  The overall framework for writing and running \glspl{app}.
  See \cref{sec:paf}.
}]{paf}{PAF}{Plutus Application Framework}

% Authoring
\newglossaryentry{ghc-core}
{
  name=GHC Core,
  description={
    GHC's internal representation of Haskell.
    A variant of \gls{system-f}.
  }
}
\newglossaryentry{plutus-ir}
{
  name=Plutus IR,
  description={
    An intermediate language that compiles to \gls{plutus-core}.
    See \cref{sec:plutus-ir}.
  }
}
\newglossaryentry{plutus-playground}
{
  name=Plutus Playground,
  description={
    A web based environment for trying out the \gls{plutus-platform}.
    See \cref{sec:plutus-playground}.
  }
}
\newglossaryentry{plutus-sdk}
{
  name=Plutus Haskell SDK,
  description={
    The libraries and development tooling for writing \glspl{app} in Haskell.
    See \cref{sec:sdk}.
  }
}
\newglossaryentry{plutus-tx}
{
  name=Plutus Tx,
  description={
    The libraries and compiler for compiling Haskell into \gls{plutus-core} to form the on-chain part of an \gls{app}.
    See \cref{sec:plutus-tx}.
  }
}

\newacronym[description={
  \gls{constraint-emitting-machine}
}]{cem}{CEM}{Constraint-emitting machine}
\newglossaryentry{constraint-emitting-machine}{
  name={constraint-emitting machine},
  description={
    A state machine that produces constraints for the next transition on every step.
    See \cref{sec:application-design}.
  }
}


\begin{document}

\title{The Plutus Platform \\
  {\large \sc An IOHK technical report}}
\date{}
\author{
  Michael Peyton Jones \\ {\small \texttt{michael.peyton-jones@iohk.io}} \\
  \and Roman Kireev \\ {\small \texttt{roman.kireev@iohk.io}}
}

\maketitle

\tableofcontents

\section{The Plutus Platform}

The \gls{plutus-platform} is a platform for writing \emph{applications} that interact with a \emph{distributed ledger} featuring \emph{scripting capabilities}, in particular the \gls{cardano} blockchain.
A high-level architecture of the \gls{plutus-platform} on \gls{cardano} is shown in \cref{fig:platform-architecture}, with an emphasis on applications.

\paragraph{Ledgers.}
The \gls{plutus-platform} is designed to work with distributed ledgers (henceforth simply ``ledgers'').\footnote{
More specifically, it is designed to work with the \gls{cardano} blockchain, although it could function on other systems that implement the requisite ledger functionality.
}
While the design of the Platform supports writing very simple applications that do nothing but occasionally submit asset-transfer transactions to the ledger, the main focus is on the applications which are enabled by a ledger with scripting functionality.
Specifically, we look at \gls{utxo} ledgers which implement what we call the \gls{eutxo-model}.

The ledger model is described in \cref{sec:ledger}.

\paragraph{Scripting.}
Ledgers without scripting functionality can only support very simple applications.
Much of the interest in applications that interact with blockchains has come from the ability to put some part of the application code \emph{on} the chain itself (such code is generally referred to as a ``\gls{script}''), such that it is guaranteed to be executed correctly as part of the consensus protocol of the blockchain.
This enables applications to have small kernels of ``trusted'' code which ensures that critical safety conditions are met.

The \gls{plutus-platform} requires a particular scripting model from the ledger, which is described with the general ledger model in \cref{sec:ledger}.
This scripting model is agnostic about the scripting \emph{language} which is used for \glspl{script}, but the \gls{plutus-sdk} is designed to work with a particular scripting language, namely \gls{plutus-core}. \Gls{plutus-core} is described in \cref{sec:plutus-core}.

\paragraph{Applications.}
In the \gls{plutus-platform} a \gls{app} is simply a program that interacts with a distributed ledger.
We use the rather generic word ``application'' over the more usual ``smart contract'' or ``distributed application'' to emphasize the generic nature of the programs we are considering --- indeed, they need not scripting functionality or have distributed participants!

The point of the Platform is to enable such applications by providing support in the ledger, as well as support for authoring, distributing, and running them.
While the modifications to the ledger may seem like the most technically substantive part of the work, most of the application behaviour is in the part that runs off the chain, and this requires a commensurate amount of support.

The \gls{app} model is described in \cref{sec:paf}, and our support for authoring \glspl{app} in \cref{sec:sdk}.

\begin{figure}[t]
  \centering
  \includegraphics[width=\textwidth]{platform-architecture.png}
  \caption{Architecture of the \gls{plutus-platform}}
  \label{fig:platform-architecture}
\end{figure}

\section{The ledger: the Extended UTXO Model}
\label{sec:ledger}
\todompj{Maybe we should just move the content of the EUTXO doc into here?}

The \gls{plutus-platform} is designed to work with a specific kind of ledger, namely an extended form of the traditional \gls{utxo} ledger introduced by Bitcoin.
We refer to our ledger model as the \gls{eutxo} Model, and it is discussed in detail in \textcite{eutxo,utxoma,eutxoma}.

We give a high-level overview here, further details can be found in the above references.

\subsection{Requirements}
\begin{requirement}[Locality]
\label{req:ledger-locality}
\Gls{utxo} ledgers have the desirable property that all the information which is needed to validate a transaction is specified in the transaction.
In particular, there is no global state apart from the \gls{utxo} set.

This is in contrast to account-based ledgers, where account balances are global.\footnote{
  In fact Cardano has a small maount
}

We aim not to disturb this property, since it is helpful in a number of ways.
For example, it allows a greater degree of parallel processing.
In \textcite{chakravartyhydra} this property is used to enable fast optimistic settlement of transactions off-chain, \emph{including} support for the scripting model we use.
This would not be possible if we followed Ethereum's actor model and had contract instances with global state.
\end{requirement}

\begin{requirement}[Determinism]
\label{req:ledger-determinism}
We would like our ledger rules to be \emph{deterministic}, that is, the outcome of transaction validation is entirely determined by the transaction in question (and the current state of the ledger of course).
This means that the whole process can be simulated accurately by the user \emph{before} submitting a transaction: thus both the outcome of validation and the amount of resources consumed can be determined ahead of time.
This is helpful for systems that charge for \gls{script} execution, since users can reliably compute how much they will need to pay ahead of time.

A common way for systems to violate this property is by providing \glspl{script} with access to some piece of mutable information, such as the current time (in our system, the current slot has this role).
\Glspl{script} can then branch on this information, leading to non-deterministic behaviour.
\end{requirement}

\begin{requirement}[Inspectability]
\label{req:ledger-inspectability}
As far as possible, we would like the data which we put on the ledger to be \emph{inspectable}.
Hence the \gls{off-chain} needs to be able to inspect the \gls{datum} and interpret it as a value it can work with.

This is primarily important for \gls{off-chain}.
For example, when implementing \glspl{app} which are state machines, we will use the \glspl{datum} of outputs to store the on-chain \emph{state} of the machine.
If the \gls{off-chain} wants to submit a transaction to advance such a state machine, it may need to read the current state from the chain (after all, the last transition might not have been performed by this agent).
\end{requirement}

\begin{requirement}[UTXO size]
\label{req:ledger-utxo-size}
\Glspl{slot-leader} are likely to be faced with resource constraints when doing transaction validation.
A key source of memory usage is that the whole set of current \gls{utxo} entries must be kept in memory at all times to check for double-spending.
Hence, it is important to keep the size of \gls{utxo} entries small.
\end{requirement}

\subsection{The \glsentryname{data} type}
\label{sec:data}

We will later require the ability to pass arguments to \glspl{script}.
However, this requires us to turn such arguments into values in our scripting language of the appropriate type.

We have opted to use a single, structured datatype for interchange with \glspl{script}.
We call this type \gls{data}, and it is described further in \textcite{eutxo}.\footnote{
  In practice it is very similar to other forms of structured data, such as JSON, but optimized to work with CBOR \autocite{cbor}.
}
This means that the \gls{redeemer} and \gls{datum} are of type \gls{data}, and the \gls{context} is encoded as \gls{data} before being passed to the \gls{validator}.

We also require a way to turn the \gls{data} values into values in our scripting language, but we can do this via the lifting machinery in \cref{sec:plutus-tx-lifting}.

On the user side, users are responsible for turning their specific types into \gls{data}, but this can be done in a standardized way, and we provide Template Haskell functions for generating the appropriate typeclass instances.

\subsection{UTXO ledgers}

A \gls{utxo} ledger represents a ledger as a series of \emph{transactions}, each of which has a set of transaction \emph{inputs} and transaction \emph{outputs}.

A transaction output carries some amount of \emph{value} and is \emph{locked} in some way that restricts how it can be spent.
The simplest form of locking is simply to attach a public key.

A transaction input is a reference to a transaction output.
For example, a pair of a transaction ID and an index into the (ordered) set of its outputs.

\subsubsection{UTXO validity}
\label{sec:utxo-valid}

For a transaction to be \emph{valid} against an existing \gls{utxo} ledger, it must be the case that:
\begin{itemize}
\item All inputs refer to outputs which exist in the ledger and have not been spent by another transaction in the ledger.
\item The transaction balances: the sum of the value on the spent outputs equals the sum of the value on the transaction outputs.
\item The transaction is authorized: for each spent output locked by a public key, the transaction contains a witness (a signature over the transaction body) from the corresponding private key.
\end{itemize}
\medskip

Hence ownership in a \gls{utxo} ledger consists in having the right to spend an output: one owns outputs rather than an aggregate amount of funds.

\subsection{EUTXO-1: scripting}
\label{sec:eutxo}

\todompj{This section is a mess and doesn't tell the story well.}

The first addition we make is to support scripting.
Our approach is based of existing models for \gls{utxo} ledgers with scripts (e.g. \textcite{Zahnentferner18-UTxO}), but we make a number of additions.

We make the following changes to the \gls{utxo} model:

\begin{itemize}
\item
  Every transaction has a \emph{validity interval} over slots.
  A \gls{slot-leader} will only process the transaction if the current slot number lies within the transaction's validity interval.

\item
  The ledger learns about a \emph{scripting language} and how to evaluate scripts in that language.\footnote{
    In the \gls{plutus-platform} this language is \gls{plutus-core}.
    }

\item
  Transaction outputs can be locked by a \gls{validator} (such outputs are called \glspl{script-output}).

\item
  \Glspl{script-output} have an attached \gls{datum}, which is a value of type \gls{data}.

\item
  When spending a \gls{script-output}, the spending transaction must provide a \gls{redeemer} of type \gls{data} for each such output.

\item
  Validator scripts are passed three arguments: the \gls{datum} of the output being spent, the \gls{redeemer} for the output, and a value called the \gls{context} which has a representation of the transaction being validated.
\end{itemize}

\todompj{I'm totally glossing over the details about which things are hashes and the whole data witnesses thing.}

\subsubsection{The contents of the context}
The \gls{context} type is a summary of the information contained in a transaction, made suitable for consumption in a \gls{script}.

The exact form is subject to change (see \textcite{eutxo} for the details), but it should contain all the information about the inputs, outputs, metadata, etc.

\subsubsection{Ensuring determinism}
The information provided by the \gls{context} is entirely determined by the transaction itself.
This ensures that we satisfy \cref{req:ledger-determinism}.

We handle the issue of time by providing only the validation interval of the transaction.
This serves to give the \gls{validator} a window within which the current time must lie: in practice this is enough to establish that the current time is ``definitely before'' or ``definitely after'' some particular time, which is the most common kind of constraint.

Some additional responsibility passes to the author of the transaction as a result of this.
If the \gls{validator} they are trying to satisfy requires that the transaction occur before some time $T$, then they must not only submit the transaction before $T$, they must ensure that the upper bound of the validity interval is below $T$.
This doesn't make the transaction less likely to validate (it would not have validated after $T$ regardless!), but it does require some more work from the \gls{off-chain}.

\subsubsection{EUTXO-1 validity}
\label{sec:eutxo-1-valid}

In addition to the conditions in \cref{sec:utxo-valid}, the following condition must hold:
\begin{itemize}
\item Scripts must succeed: if a transaction spends a \gls{script-output}, then the \gls{validator} on the output must be run with the \gls{redeemer}, \gls{datum}, and \gls{context} and return successfully.
\end{itemize}

\subsection{EUTXO-2: \glsentrytext{multicurrency}}
\label{sec:multicurrency}

\todompj{This section is also a mess and doesn't tell the story well.}

The other extension that we make to the ledger is to enable \emph{\gls{multicurrency}}, that is, to enable our ledger to represent a wider variety of assets than a single one.
\Gls{multicurrency} is a key part of the \gls{plutus-platform} not only because custom \glspl{currency} are an important use case for blockchain systems, but because we use lightweight custom \glspl{token} as integral parts of our application designs.

The key design innovation is that \glspl{currency} are identified with the hash of the \gls{mps} that controls them.
This means that to be allowed to forge \glspl{token} of a particular \gls{currency}, you must present the corresponding \gls{mps} with the transaction and it must pass.

This design allows our system to be both \emph{local} and \emph{lightweight}:
\begin{itemize}
\item
  We do not require any kind of central registration of \glspl{currency}: any hash can be used as a \gls{currency-id}, you just cannot forge any unless you can produce a \gls{script} with that hash.
  This ensures we do not violate \cref{req:ledger-locality}.
\item
  \Gls{forging} \glspl{token} is cheap: all that is required is to provide and run the script at the time of forging.
\item
  Transacting with custom \glspl{currency} is cheap: they are natively supported and work just like the native token.
\end{itemize}

\todompj{Say more, maybe write out some of the requirements separately.}

\subsubsection{The \glsentryname{value} type}
\label{sec:value}
The first change we make is to our notion of ``value'' on the ledger.
In traditional \gls{utxo} ledgers, value is just an integer representing a quantity of the (unique) asset tracked by the ledger.

We generalize this to support multiple asset classes as follows: we define a type \gls{value} which represents a \emph{bundle} of assets, indexed by a \emph{\gls{currency-id}} and then by a \emph{\gls{token-name}}.

\begin{quote}
\begin{verbatim}
type Value = Map CurrencyID (Map TokenName Quantity)
\end{verbatim}
\end{quote}

The \gls{currency-id} indicates the \emph{controller} of the currency, specifically it is the hash of the \gls{mps} that controls when tokens of that currency may be forged or destroyed (see \cref{sec:forging}).
The \gls{token-name} separates different \emph{classes} of \gls{token} within the currency. Only \glspl{token} of the same currency and \gls{token-name} are \gls{fungible} with each other.
Hence one could use a currency to track items in a game: different kinds of items would have different \glspl{token-name}, and an asset bundle could consist of e.g. 3 swords, 7 hats, and a cheesecake.

Operations on \gls{value} are defined as if it were a finitely-supported function.

The ledger rules do not need to change much in order to support this generalized \gls{value}: the only change is that we need to use the appropriate sum operation on \gls{value} in e.g. the balancing rules.

\subsubsection{\Glsentrytext{forging}}
\label{sec:forging}
\Gls{value} may also be created and destroyed, we call this \emph{\gls{forging}}.
To this end transactions are given two additional fields:
\begin{enumerate}
\item \code{forge} which contains \gls{value}
\item \code{forgeScripts} which contains \glspl{mps} for all the \glspl{currency} forged in \code{forge}
\end{enumerate}

\subsubsection{\glsentrylong{mps}s}
The \glspl{script} which control the forging of \glspl{token} are called \glsfirstplural{mps}.
Since these are able to see the \gls{context}, and hence the whole transaction being validated, they can enforce a wide range of properties on the transaction.

\subsubsection{\glsentrylong{nft}s}
A \gls{nft} is a unique \gls{token} which can be transferred to another user, but not duplicated.
\Glspl{nft} have proven useful in a number of blockchain applications (see \textcite{ERC-721}); for example, they can represent ownership of some object in a game, or shares in a company, or many other kinds of asset.
We can implement \glspl{nft} as token classes within a \gls{currency} whose supply is limited to a single \gls{token}.

Ensuring that the tokens are unique is not trivial, since uniqueness is a global property and transactions only have access to local information.
Two approaches are:
\begin{itemize}
\item
  Use a \gls{one-shot} \gls{mps}.
  This ensures that the \gls{mps} can only be run once, and then so long as you do not issue any duplicates there, you will have uniqueness.
\item
  Run a unique state machine (beginning with a \gls{one-shot} transaction to ensure uniqueness) which keeps all the issued tokens in its state.
  This effectively introduces global state that must be consulted before each forging, allowing uniqueness to be maintained.
\end{itemize}

\subsubsection{EUTXO-2 validity}
\label{sec:eutxo-2-valid}

In addition to the conditions in \cref{sec:utxo-valid} and \cref{sec:eutxo-1-valid}, the following condition must hold:
\begin{itemize}
\item Forging must be authorized: for each \gls{currency-id} of which a non-zero amount is forged in \code{forge}, the corresponding \gls{mps} must be run with the \gls{context} and return successfully.
\end{itemize}

\documentclass[plutus.tex]{subfiles}
\begin{document}
\section{Scripting: Plutus Core}
\label{sec:plutus-core}

\todompj{This section needs more justification.}

Our scripting language for the \gls{plutus-platform} is \gls{plutus-core} (strictly, type-erased \gls{plutus-core}, see \cref{sec:erasure}).
A formal specification for \gls{plutus-core} is available in \textcite{plutus-core-spec}.

We give a high-level overview of the language here, further details can be found in the above reference.

\subsection{Requirements}
\begin{requirement}[Design conservatism]
\label{req:script-lang-conservatism}
Designing a new programming language is hard.
It is very easy to make choices that come back to haunt you.

Moreover, whatever language we choose for our scripting language will be very hard to change, since it will be involved in transaction validation, and we want to be able to validate old transactions.
So we can release new versions, but we must support old versions forever.

This combination means that it is hard to get programming languages right, and it is hard for us to iterate on our scripting language.
This suggests a case for \emph{conservatism}: pick things that are tried and tested, and don't try to innovate too much.

Conservatism also makes requirements such as \cref{req:script-lang-formalization} easier to satisfy, as we can build on existing work.
\end{requirement}

\begin{requirement}[Minimalism]
\label{req:script-lang-minimalism}
The smaller our language, the less there is to go wrong, and the less there is to reason about.

Minimalism makes requirements such as \cref{req:script-lang-formalization} easier to satisfy, since there is less to formalize.

However, this is a tradeoff, as a simpler target language often means more complexity in the compilers that target that language.
But it is much easier for us to change the compilers than it is to change the scripting language, so this is worth it.
\end{requirement}

\begin{requirement}[Safety]
\label{req:source-lang-reasoning}
Once submitted as part of a blockchain transaction, scripts are immutable.
One must have absolute certainty as to what the code will do, otherwise there is risk that the value involved will be lost or stolen.
\end{requirement}

\begin{requirement}[Formalization]
\label{req:script-lang-formalization}
One part of reasoning about what our programming language does is to \emph{formalize} its semantics, so we can be sure that
\begin{inparaenum}
  \item it has a sensible semantics, and
  \item the implementation agrees with that semantics
\end{inparaenum}.
\end{requirement}

\begin{requirement}[Size]
\label{req:script-lang-size}
The representation of the scripting language on the chain must not be too large, since
\begin{inparaenum}
\item users will pay for the size of transactions, and
\item transaction size has a major effect on the throughput of the system.
\end{inparaenum}
\end{requirement}

\begin{requirement}[Multiple source languages]
\label{req:source-lang-multiple}
It would be nice if multiple source languages could be compiled to our scripting language.
This would potentially encourage usage by developers who are more comfortable with one or the other of the source languages.

However, this is somewhat speculative and actually implementing other source languages would be a lot of work.
But if we can cheaply make this easier then that is good.
\end{requirement}

\subsection{Designing Plutus Core}

How do we design our scripting language?
\Cref{req:script-lang-conservatism,req:script-lang-minimalism} suggest adhering to existing languages as much as possible, and picking a small, well-studied language.
We also want to use a statically-typed functional programming language, since we intend to use functional programming languages (starting with Haskell) as our source languages.

This suggests using some variant of the lambda calculus.
We decided to start with \emph{\gls{system-f}}, also known as the polymorphic lambda calculus \autocite{Girard-thesis}.
Haskell's internal language, GHC Core \parencite{jones1998transformation}, is also based on \gls{system-f} (no coincidence), but we make a couple of different decisions:
\begin{enumerate}
  \item We do not have primitive datatypes and case expressions, rather we base our language on \emph{\gls{system-fomf}}, which extends \gls{system-f} with recursive types and higher-kinded types.
  \item Our language is strict, rather than lazy, by default.
  \item We do not support most of the extensions that Haskell has pioneered, such as coercions.
\end{enumerate}

As a result, the formal specification of our language can be described in one line: it is exactly \gls{system-fomf} with appropriate primitive types and operations.

\todompj{Discuss builtins}

\subsection{Recursive types}

The one-line description above turns out not to be as unambiguous as one might hope. We have
to choose between equirecursive types and isorecursive types \autocite[chapter 21]{pierce2002types}.

There is a tradeoff here between simplicity of writing code in the language, and simplicity of the language's metatheory.
Since \gls{plutus-core} is a compilation target rather than a source language, we opted to go for isorecursive types, which have the simpler metatheory.
The complexity is handled by the \gls{plutus-ir} compiler.

This choice is discussed more in \textcite{plutus-core-spec, peytonjones2019unraveling}.

\subsection{Datatypes}
If we do not have primitive datatypes, how \emph{do} we deal with datatypes?
The answer is that it is up to the compiler to encode them --- another example of the tradeoff discussed in \cref{req:script-lang-minimalism}.

In our case \gls{plutus-ir} does have datatypes, so this is handled by the \gls{plutus-ir} compiler. See \cref{sec:plutus-ir} for more details.

\subsection{Recursive values}

While \gls{plutus-core} has support for recursive types, it does not have any (direct) support for recursive \emph{values}.
It turns out that recursive types are sufficient to implement the usual array of fixpoint combinators, and so encode recursive values \autocite{harper2012practical}.
Again, this encoding is handled by the \gls{plutus-ir} compiler, see \cref{sec:plutus-ir}.

Doing this in full generality turns out to be surprisingly tricky, see \textcite{peytonjones2019unraveling}.

\subsection{Erasure}
\label{sec:erasure}

Originally we planned to use typed \gls{plutus-core} as the actual scripting language.
However, we discovered that the explicit types made up a large proportion of the overall size of the code ($\approx 80\%$).
Given that we care about size (\cref{req:script-lang-size}), this was too compelling an improvement to pass by.

Hence we decided to instead use \emph{type-erased} \gls{plutus-core} as our scripting language instead, with typed \gls{plutus-core} as a (useful) intermediary language.

We were initially concerned that if we did not typecheck the application of the \gls{validator} to its arguments (which we cannot do if we've erased the types of the \gls{validator}!), then that might allow a malicious attacker to pass a script ill-typed arguments, potentially causing unexpected behaviour.
However, given the current design where all the arguments to the \gls{validator} are of type \gls{data}, and are constructed by the validating node (which is a trusted party), it is no longer possible for the arguments to be ill-typed.

Erased \gls{plutus-core} is much closer to the untyped lambda calculus, and as such is also an easier compilation target (e.g. for a dependently-typed language, or a non-statically-typed language), hence this also helps with \cref{req:source-lang-multiple}.

\subsection{Formalization}

As discussed in \cref{req:script-lang-formalization}, we would like to formalize \gls{plutus-core}.
We have done so in Agda: the resulting formalization is partially published in \textcite{chapman2019system}, and the living version is available in \textcite{plutus-repo}.

Our formalization includes the type system and semantics, proofs of progress and preservation, and an evaluator which provably implements the semantics.
This evaluator can be extracted to a Haskell executable, which we use to cross-test against the Haskell interpreter.

\subsection{Costing}

\todompj{Write this.}

\end{document}

\section{Resources: Costing Plutus Core}
\label{sec:costing}

A key, unusual, requirement of the scripting language we use is that it must have a built-in notion of resource tracking.
We are going to run untrusted code on many machines during transaction validation and diffusion - it is important that we carefully control its resource usage.

This is a familiar problem in other systems such as Ethereum (which tracks resource usage using ``gas''), but we are focussing on a somewhat different set of requirements.

\subsection{Requirements}
\begin{requirement}[Profitable fees]
\label{req:costing-profitability}
\Glspl{slot-leader} are compensated for the work that they do validating transactions by being given the transaction fees.
This ensures that it is economical (or even profitable) to run a stake pool even just based on transaction fee income.
We therefore need to ensure that the fees for script execution don't change this dynamic, e.g. by making it un-economical to run a pool.

If transactions with scripts are un-economical to process, then slot leaders may simply choose not to include them, which would compromise the usability of the system.
\end{requirement}

\begin{requirement}[DoS security]
\label{req:costing-dos}
Scripts allow the user who submits a transaction to force every node on the network to perform some computation of the user's choice.
This provides an obvious angle for DoS attacks: simply flood the network with pointless transactions that require a lot of computation, thus preventing the network from doing anything else.
Of course, any such attack must pay the script execution fees for these transactions.
So we need to make the fees high enough that such an attack is un-economical.
\end{requirement}

\begin{requirement}[Usable fees]
\label{req:costing-usable}
We do want users of \gls{cardano} to actually use scripts.
So the cost to run scripts can't be too high, otherwise nobody will use them.
The simplest way to encourage this is to make the evaluator faster, which benefits legitimate users without compromising the security of the system.
\end{requirement}

\begin{requirement}[Determinism]
\label{req:costing-determinism}
Script execution cost affects whether a transaction validates, and as such if the costs are non-deterministic then it is non-deterministic whether a transaction with scripts will validate at all.
Predictable costs ensure that users can submit transactions and be sure that they will validate (or if they don't, it will be for other reasons).

See also \cref{req:ledger-determinism}.
\end{requirement}

\begin{requirement}[Evaluator abort]
\label{req:costing-abort}
The evaluator must be able to run scripts with limited resources and stop them when they exceed it, so we actually enforce budgets.
\end{requirement}

\begin{requirement}[Intuitive costs]
\label{req:costing-intuitive}
Programmers will want to reason about costs when writing their programs.
So ideally costs should track those intuitions.
That way, if a user does something that they expect to reduce the cost of the program, then it probably does do so.
\end{requirement}

\begin{requirement}[Specifiablity]
\label{req:costing-specifiable}
We want to formally specify how the costs for a script should be calculated, which means that the method for computing them must be relatively simple and manageable to specify.
\end{requirement}

\begin{requirement}[Low overhead]
\label{req:costing-overhead}
Tracking costs during script execution should not impose too much overhead.
We don't want the process of preventing scripts from running too slow to itself slow them down significantly!
\end{requirement}

\begin{requirement}[Parameterizable models]
\label{req:costing-parameters}
We are unlikely to get all the numeric choices about how to assign costs correct first time.
Or, they might be invalidated by hardware or software changes.
Hence we should parameterize the model so that the ledger can change the parameters (e.g. with a protocol parameter update).
\end{requirement}

\begin{requirement}[Language-agnostic ledger support]
\label{req:costing-language-agnostic}
The ledger will support many different languages, which may think about costing in quite different ways.
We want to make this as easy as possible, ideally by abstracting away some of the details.
\end{requirement}

\begin{requirement}[Block budgets]
\label{req:costing-block-budget}
We have some constraints on how long block validation as a whole can take, in order to ensure that blocks can propagate across the network fast enough.
In addition, we have limits on how much peak memory we can use in a block, based on our expectations of typical hardware.

We may therefore need hard limits on resource usage per-block, in addition to setting prices for resource usage.
\end{requirement}

\subsection{Non-Requirements}
There are some things which we could have designed for, but we have chosen not to.
A few notable examples are given here.

\paragraph{Units of cost should be persistent across transactions}
On Ethereum, gas which a contract does not spend in a single transaction is persistent - it can be used later.
We don't intend to replicate this: because execution costs are predictable, it is no problem to require users to provide exactly the amount that they need.
Moreover, persistent cost units potentially allows undesirable arbitrage where a user can ``buy'' computation when it is cheap and ``spend'' it later when it is expensive, which is not what the network wants.

\paragraph{Users should be able to calculate script execution costs for any input to that script}
Users often care about the ``total cost of operation'' of an application.
But this may consist of running the same script with many different inputs across time, depending on contingent facts which will not be known until later.
To give this information to users precisely we would need to be able to predict, for any input, what the execution cost would be.
This is something that has been studied, but we believe it would be a lot of work to apply to our situation.
\footnote{
  There is research in doing this for languages with explicit recursion and datatypes.
  We would probably want to do this analysis on Plutus IR, and then try and translate the analysis down to Plutus Core.
  This seems like a substantial amount of work, probably a PhD thesis or so.
}
Users will still be able to get some idea of total cost of operation by running their script with a selection of indicative inputs.

\paragraph{Slot leaders should be compensated for the opportunity cost of memory usage}
We want to enforce a peak memory limit to avoid crashes, but we don't bother compensating slot leaders for the opportunity cost of using their memory.

\subsection{High-level approach}
This section gives a very high-level overview of the approach we take to the problem.
Further details on each of these topics are given later.

\paragraph{Abstract resources}
Rather than computing costs directly in Ada or in real resource units (e.g. seconds), we instead compute them in \emph{abstract resource units}, which do not have a definitional link to real resources.
The reasons for this are that:
\begin{itemize}
\item Costs must be deterministic and specifiable (\cref{req:costing-determinism,req:costing-specifiable}), so we cannot use \emph{real} resource units which will differ by machine.
\item The decision for how to convert these resource units into Ada (``pricing'') can be the ledger's responsibility alone.
\item Language-specific costing solutions don't need to worry about the (potentially strategic) concerns that affect pricing (\cref{req:costing-language-agnostic}).
\end{itemize}

\paragraph{Limiting execution}
The evaluator can run in a mode (``restricting'') where it is given a budget of abstract resources, and terminates if the execution exceeds that (\cref{req:costing-abort}).
We also provide another mode (``counting'') where we do not take a budget, and instead return the minimum budget that the script requires.

\subsubsection{Cost model}
We call the system which tells the evaluator how much an script should cost a ``cost model''.

\paragraph{Cost model strategy}
The cost model for \gls{plutus-core} is defined in a fairly simple, compositional way by giving costs for individual operations, and then accumulating them over the course of the evaluation.

This cost model naturally has a lot of parameters: the numbers that influence the costs for the individual operations.
We allow all of these to be changed, so the evaluator accepts a bag of parameters which gives all of these values.
These parameters will be put into the protocol parameters, so we should have a large amount of flexibility if we need to tweak the cost model later (\cref{req:costing-parameters}).

\paragraph{Choosing the cost model parameters}
The decoupling of the abstract resource units from the pricing units allows us to use a fairly simple rubric for picking cost model parameters: try and actually follow (be correlated with) real costs!

This ensures that resource costs will follow programmers' intuitions (\cref{req:costing-intuitive}), while giving plenty of flexibility in the final pricing of those resources.

If we later change our minds about the cost model parameters (e.g. because what we consider ``typical hardware'' changes), we can re-calibrate our parameters and submit a protocol parameter change.

However, we do want our choices to generally be an \emph{over-approximation}, because anywhere we under-approximate is a potential source of attacks (\cref{sec:costing-security}).

\subsubsection{Pricing model}
The conversion from abstract resources to Ada is done by the ledger.
We call the system which tells us how to do this conversion the ``pricing model''.

\paragraph{Pricing model strategy}
The current proposed pricing model simply consists of an Ada price for each of the abstract resources.
So a budget in terms of \gls{space} and \gls{time} will be turned into a price in terms of Ada by multiplying the components of the limit by their respective prices.

These prices also obviously function as parameters for the pricing model.
Hence they will also be put in the protocol parameters, so that pricing can be changed with only a protocol parameter change.

\paragraph{Choosing the pricing model parameters}
Choosing the pricing model parameters is a complex problem that may have strategic considerations.
At minimum, the prices will need to be high enough to avoid DoS attacks (\cref{req:costing-dos}).

\subsubsection{Validating scripts}
Putting the pieces together, what happens when a script is validated is:
\begin{itemize}
\item
  The ledger uses the pricing model parameters to compute the price for the stated resource budget in Ada, and ensures that the transaction has sufficient fees to cover this, and that it does not cause us to exceed the per-block budget (\cref{req:costing-block-budget}).
\item
  The ledger runs the script, passing in the stated budget and the cost model parameters.
\item
  The evaluator either terminates normally, with success or an error depending on the program, or stops early if it runs out of budget.
\end{itemize}

\subsection{Abstract resource units and resource budgets}
\label{sec:costing-units}
We use two abstract resource units:
\begin{itemize}
\item Abstract time/CPU usage (``\gls{time}'')
\item Abstract peak memory usage (``\gls{space}'')
\end{itemize}

Script execution is limited in both \gls{time} and \gls{space}.
The overall limit (budget) is therefore a pair of some amount of \gls{time} and \gls{space}.

Crucially, the interpretation of the \gls{space} limit is that it is a limit on (our over-approximation of) \emph{peak} space usage.
This is because the limit is mainly to prevent crashes due to exceeding a machine's available RAM, not to compensate the slot leader for RAM usage.

The reason we need to track memory at all is because our evaluators can use unbounded memory.
This is not true in e.g. Ethereum where there is a (small) stack limit, so programs can be assumed to fit within a constant memory budget.
Plutus Core has both unbounded integers and unbounded recursion, so it is not hard to use a large amount of memory.\footnote{
Most programs that use lots of memory will also use lots of time, and we could price them punitively to ensure that the time budget was always a binding constraint.
But this requires some subtle reasoning: it is easier to simply talk about memory if that's what we care about.
}

\subsection{Building the cost model}
The high-level description of the cost model said that we would give costs for ``individual operations''.
This means that the model is tied to exactly what kind of evaluator we use.
At the moment we use a fairly standard CEK machine.

\todompj{Ref into PLC section}

Our goal in producing a cost model is to make the computed costs be well-correlated with the real costs.
Since the pricing model can be used to control how these are paid for, we can stick to trying to track reality.
Hence we start by benchmarking individual parts of execution, and using statistical methods to infer model parameters from that.

\subsubsection{Mapping between real and abstract units}
Any benchmarks we run will give results in actual units (e.g. microseconds), not our abstract units.
So in order to actually use them we need to do one of two things:
\begin{enumerate}
\item
  Set a target mapping for how real units should correspond to abstract units, e.g. 1 \gls{time} = 1 microsecond.
  We can then use this to interpret our benchmarks as benchmarks of ideal abstract unit usage.
\item
  Pick one of the parameters as a baseline and relativize all the others to that.
  For example, we could declare that 1 machine step will take 1 \gls{time}, and then divide the times for other operations by the time for a machine step to get their abstract \gls{time} usage.
  Then the abstract units would indicate ``resource usage relative to the chosen baseline operation''.
\end{enumerate}

Option 1 has two advantages:
\begin{enumerate}
\item It gives us a way to calibrate abstract resource usage across multiple scripting languages, which is important if we want to use (and price) the same units for all of them.
\item It gives us an obvious way to say how many real units correspond to an abstract unit, which is useful for constructing the pricing model.
\end{enumerate}

So we adopt option 1.

The specific targets we are using at the moment are:
\begin{itemize}
\item \gls{time}: 1 \gls{time} = 1 microsecond
\item \gls{space}: 1 \gls{space} = 1 machine word (8 bytes on a 64 bit machine)
\end{itemize}

\subsubsection{Simplifying assumptions about memory usage in Haskell}
Memory usage in Haskell is quite complicated.
However, we can simplify the situation by making some assumptions.

\paragraph{No memory is ever garbage-collected or freed}
This is a pessimistic assumption, but being pessimistic is fine since we're aiming for an overestimate.
It simplifies our calculations, since it means that we can just track all \emph{allocations} of memory, and don't have to worry about the details of when things are de-allocated.

\paragraph{All references to the AST are shared}
This is an optimistic assumption, but justified by our knowledge about how Haskell works.
This means that we can work out how much memory the AST takes up, and then we can assume that references to parts of the AST (e.g. in variable environments) take no memory for the AST part.
That simplifies our accounting significantly, since the new allocations will be for small things like entries in variable environment mappings.

\subsubsection{\gls{space} for the AST}
We need to compute \gls{space} usage for the AST in several places:
\begin{itemize}
\item At the start of execution to account for loading it into memory and to handle references (see the above discussion of sharing).
\item When evaluating a builtin operation (see below) the cost may depend on the size of  the arguments.
\end{itemize}

For the AST itself we follow a fairly simple scheme where we make some assumptions about how much space the nodes themselves take up, as well as their children.
For constants, we simply think about the underlying type: how much space do we think an integer, say, takes up?

\subsubsection{\gls{space} and \gls{time} for builtin operations}
Builtin operations are in some ways the trickiest thing to cost, since they are very heterogeneous, and are implemented by external code.
Moreover, their performance characteristics often depend on the inputs to the function: adding two 1000000 digit numbers is much slower than adding two 10 digit numbers.

\paragraph{Size of builtin arguments}
We use the \gls{space} usage of an argument (typically a constant term) as a proxy for its size.
This is not perfectly precise, but it is reasonably correlated with the measures of size that e.g. addition is likely to care about.

\paragraph{Builtin accounting happens before execution}
Builtin operations can in principle use a lot of resources just in the operation itself.
For example, multiplying two sufficiently large numbers can require a large amount of memory to hold the result, much larger than that required to hold the arguments.
This leads to a conflict: builtin operations are atomic (in that we can't interrupt them), but we want to interrupt evaluation as soon as we exceed our resource budget.

Our solution is to compute the costs for a builtin operation before we run it.
Since we are not actually measuring anything, but rather computing the cost from our model, we are able to do this.
That way we can terminate evaluation if we would exceed the budget after running the operation, without actually running the operation.

\paragraph{\gls{time} for builtin operations}
The approach we take for \gls{time} is very empirical:
\begin{enumerate}
\item
  Micro-benchmark the builtin with arguments of varying sizes.
\item
  Look at the resulting data and try to pick an appropriate linear model for the running time based on the argument sizes (e.g. ``linear in both arguments'').\footnote{
  This is an imprecise approach, but choice of statistical models is always more of an art than a science.
  }
\item
  Use linear regression to infer the parameters for the model, these are the cost model parameters for this builtin.
\end{enumerate}

\paragraph{\gls{space} for builtin operations}
The approach we take for \gls{space} is more deductive.
Generally it is hard to measure memory allocation, especially when it occurs in a foreign library.
So we largely rely on reasoning about or inspecting the libraries in question.
Many perform no allocation, or a predictable amount of allocation.

\subsubsection{\gls{space} and \gls{time} for machine steps}
The CEK machine itself has many steps that it goes through, which correspond to handling e.g. application of user-defined functions.
These also need to have costs associated: naive profiling suggests that the operation of the machine itself (as opposed to builtin functions) is usually more than 50\% of evaluation time.

\paragraph{\gls{time} for machine steps}
The approach we initially took for \gls{time} is empirical, matching the approach for builtins:
\begin{enumerate}
\item Assume that the execution time is linear in all the kinds of machine steps.
\item Benchmark execution of a large number of varied programs that use all the execution steps.
\item Use linear regression to infer the parameters for the model, these are the cost parameters for the machine steps.
\end{enumerate}

However, when we actually did this, we found that the number of machine steps of different kinds are almost perfectly correlated, such that there is little point trying to infer coefficients for them individually.
So instead we fit a single-parameter linear model based only on the total number of steps.\footnote{
The argument from correlation holds even if the different steps do actually have very different execution costs!
It's possible that they are all indistinguishably similar, but even if they were quite distinct, if they're very well correlated a single-parameter model will still do a better job.
}

\paragraph{\gls{space} for machine steps}
The approach we take for \gls{space} is more deductive.
Generally we assume that machine steps incur memory usage in certain specific ways, such as creating machine frames or creating mappings in variable environments, and we handle these specifically.

\subsubsection{Summary of overall calculation}
To sum up, the cost for a program execution is:
\begin{itemize}
\item The initial \gls{space} cost of the AST.
\item The initial \gls{time} cost for starting the machine (the intercept of the linear model for the machine steps).
\item The \gls{time} and \gls{space} costs for each machine step that is taken (as determined by the coefficients of the linear model for the machine steps).
\item The \gls{time} and \gls{space} costs for each builtin call (as determined by the linear models for each builtin operation).
\end{itemize}

\subsection{Building the pricing model}
The pricing model is out of our control - it is the ledger's responsibility, and as we discussed earlier, the process of choosing the model is likely to be complex and influenced by many non-technical factors.
However, anyone making a decision about how to set prices does need at least some input from us, because prices relate to real costs (real CPU seconds, real memory usage), and so we need to know how our abstract units correspond (in reality, at the current time) to real units.

Fortunately, we have this information easily to hand, since we are targeting specific relationships between abstract units and real units when we build our models.
So we can assume that those relationships hold (e.g. 1 \gls{time} = 1 microsecond).

\subsection{Security}
\label{sec:costing-security}

The top priority of the model is to ensure the security of the ledger against attacks, particularly DoS attacks, but also economic attacks that sap profit from the system.

\paragraph{Basic structure of an attack}
The basic kind of attack we are worried about is where we have set one of our cost parameters too low, effectively under-costing a particular operation.
For example, maybe addition is too cheap, or a certain kind of machine step.

An attacker could then construct a synthetic program that disproportionately uses the under-costed operation.
Such a program would then be more expensive to execute in reality than the model predicts, allowing the attacker to force node operators to do more work than they are paying for.

How dangerous this is depends on how badly we under-estimate the parameter.
If we under-estimate it by only 10\%, then an attacker is only getting a 10\% ``discount'' on computation: probably not enough to make it worthwhile to mount an attack.
If we under-estimate it by an order of magnitude or two, then we could be in trouble.

We might think that such malicious programs would be large, and so transaction size limits would help us.
But we have loops and recursion in Plutus Core, so it is likely that an attacker could make a relatively small program that does a very large number of the problematic runtime operations.

\paragraph{Simple prevention approaches}
Our attack prevention approach is fairly simple.
We try and make the cost model give an \emph{over}-estimate of reality, and then we try to make the pricing model over-price the resources.

In order to keep this working, we need to:
\begin{enumerate}
\item Ensure that we take great care when updating the cost model, especially if we make things cheaper (say, on the basis of the evaluator getting faster)
\item Ensure that we adjust prices as appropriate when the cost of hardware changes.
\end{enumerate}

\paragraph{Tools for node operators to check accuracy of models on their hardware}
Our benchmarks are going to be run on a reference machine.
Of course, not all machines are alike, and it's possible that our hardware may be unusual, or certain node operators' hardware is unusual in such a way that allows an attack.

For example, perhaps addition is unusually slow on operator O's machine: then a program which does lots of addition might be costed cheaply, run fine on our reference machine, but overload O's machine.

A simple way to mitigate this risk is to provide tools for node operators to run the benchmarks on their own hardware.
If they get results that indicate that costs should be higher, then that indicates a potential attack and we may need to raise costs.

\section{Applications}
\label{sec:applications}

\UTXOma\ is able to support a large number of standard use cases for multi-asset ledgers, as well as some novel ones.
In this section we give a selection of examples.
There are some common themes: (1) Tokens as resources can be used to reify many non-obvious things, which makes them first-class tradeable items; (2) cheap tokens allow us to solve many small problems with \emph{more tokens}; and (3) the power of the scripting language affects what examples can be implemented.

\subsection{Simple single token issuance}
%
To create a simple currency $\mathsf{SimpleCoin}$ with a fixed supply of $\texttt{s = 1000 SimpleCoins}$ tokens, we might try to use the simple policy script $\texttt{Forges(s)}$ with a single forging transaction. Unfortunately, this is not sufficient as somebody else could submit another transaction forging another $\texttt{1000 SimpleCoins}$.

In other words, we need to ensure that there can only ever be a single transaction on the ledger that successfully forges $\mathsf{SimpleCoin}$. We can achieve that by requiring that the forging transaction consumes a specific \UTXO. As \UTXO{}s are guaranteed to be (1) unique and (2) only be spent once, we are being guaranteed that the forging policy can only be used once to forge tokens. We can use the script:

\begin{alltt}
  simple_policy(o, v) = SpendsOutput(o) && Forges(v)
\end{alltt}

\noindent where $\texttt{o}$ is an output that we create specifically for this purpose in a preceding setup transaction, and $\texttt{v = s}$.

\subsection{Reflections of off-ledger assets}
\label{sec:asset-tokens}

Many tokens are used to represent (be backed by) off-ledger assets on the ledger. An
important example of this is \emph{backed stablecoins}. Other noteworthy
examples of such assets include video game tokens, as well as service
tokens (which represent service provider obligations).

A typical design for such a system is that a trusted party (the ``issuer'') is responsible for creation and destruction of the asset tokens on the ledger.
The issuer is trusted to hold one of the backing off-ledger assets for every token that exists on the ledger, so the only role that the on-chain policy can play is to verify that the forging of the
token is signed by the trusted issuer.
This can be implemented with a forging policy that enforces
an $m$-out-of-$n$ multi-signature scheme, and no additional clauses:
\begin{alltt}
  trusted_issuer(msig) = JustMSig(msig)
\end{alltt}

\subsection{Vesting}

A common desire is to release a supply of some asset on some schedule.
Examples include vesting schemes for shares, and staged releases of newly minted tokens.
This seems tricky in our simple model: how is the forging policy supposed to know which tranches have already been released without some kind of global state which tracks them?
However, this is a problem that we can solve with more tokens. We start building
this policy by following the single issuer scheme, but we need to express more.

Given a specific output \code{o}, and two tranches of tokens \code{tr1} and \code{tr2} which should be released after \code{tick1} and \code{tick2}, we can write a forging policy such as:
\begin{alltt}
  vesting = SpendsOutput(o) && Forges(\cL"tr1" \mapsTo 1, "tr2" \mapsTo 1\cR)
         || TickAfter(tick1) && Forges(tr1bundle) && Burns(\cL"tr1" \mapsTo 1\cR)
         || TickAfter(tick2) && Forges(tr2bundle) && Burns(\cL"tr2" \mapsTo 1\cR)
\end{alltt}
%
This disjunction has three clauses:
\begin{itemize}
\item
  Once only, you may forge two unique tokens \code{tranche1} and \code{tranche2}.
\item
  If you spend and burn \code{tr1} and it is after \code{tick1}, then you may forge all the tokens in \code{tr1bundle}.
\item
  If you spend and burn \code{tr2} and it is after \code{tick2}, then you may forge all the tokens in \code{tr2bundle}.
\end{itemize}
%
By reifying the tranches as tokens, we ensure that they are unique and can be used precisely once.
As a bonus, the tranche tokens are themselves tradeable.

\subsection{Inventory tracker: tokens as state}

We can use tokens to carry some data for us, or to represent state.
A simple example is inventory tracking, where the inventory listing can only be modified by a set of trusted parties.
To track inventory on-chain, we want to have a single output containing all of the tokens of an ``inventory tracking'' asset.
If the trusted keys are represented by the multi-signature $\texttt{msig}$, the inventory tracker tokens should always be kept in a \UTXO\ entry with the following output:
\begin{alltt}
  (hash(msig) , \cL{}hash(msig) \mapsTo \cL{}hats \mapsTo 3, swords \mapsTo 1, owls \mapsTo 2\cR\cR)
\end{alltt}

The inventory tracker is
an example of an asset that should indefinitely be controlled by a specific script
(which ensures only authorized users can update the inventory), and
we enforce this condition in the forging script itself:

\begin{alltt}
  inventory_tracker(msig) = JustMSig(msig) && AssetToAddress(_)
\end{alltt}

In this case, $\texttt{inventory\_tracker(msig)}$ is both the forging
script and the output-locking script. The blank value supplied as the
argument means that the policy ID (and also the address) are both
assumed to be the hash of the $\texttt{inventory\_tracker(msig)}$
script.
Defined this way, our script is run at initial forge time, and any time
the inventory is updated. Each time
it only validates if all the inventory tracker tokens in the transaction's
outputs are always locked by this exact output script.

\subsection{Non-fungible tokens}

A common case is to want an asset group where \emph{all} the tokens are non-fungible.
A simple way to do this is to simply have a different asset policy for each token, each of which can only be run once by requiring a specific \UTXO\ to be spent. However, this is clumsy, and typically we want to have a set of non-fungible tokens all controlled by the same policy. We can do this with the \texttt{FreshTokens} clause.
If the policy always asserts that the token names are hashes of data unique to the transaction and token, then the tokens will always be distinct.

\subsection{Revocable permission}

An example where we employ this dual-purpose nature of scripts is revocable permission.
We will express permissions as a \emph{credential token}.

The list of users (as a list of hashes of their public keys) in a credential token is
composed by some central accreditation authority. Users usually trust that this authority
has verified some
real-life data, e.g. that a KYC accreditation authority has checked off-chain
that those it accredits meet some standard.\footnote{
KYC stands for ``know your customer'', which
is the process of verifying a customer's identity before allowing the customer
to use a company's service.
}
Note here that we significantly
simplify the function of KYC credentials for brevity of our example.

For example, suppose that
exchanges are only willing to transfer funds to those that have proved that
they are KYC-accredited.

In this case, the accreditation authority could issue an asset that looks like
\begin{alltt}
  \cL{}KYC_accr_authority \mapsTo \cL{}accr_key_1 \mapsTo 1, accr_key_2 \mapsTo 1, accr_key_3 \mapsTo 1\cR\cR
\end{alltt}

\noindent where the token names are the public keys of the accredited users.
We would like to make sure that

\begin{itemize}
  \item only the
authority has the power to ever forge or burn tokens controlled by this policy,
and it can do so at any time,
  \item all the users with listed keys are able to spend this asset
  as on-chain proof that they are KYC-accredited, and
  \item once a user is able to prove they have the credentials, they should be allowed
 to receive funds from an exchange.
\end{itemize}

We achieve this with a script of the following form:
\begin{alltt}
  credential_token(msig) = JustMSig(msig) && DoForge
                        || AssetToAddress(_) && Not DoForge && SignedByPIDToken(_)
\end{alltt}

Here, forges (i.e. updates to credential tokens) can only be done by the
$\texttt{msig}$ authority,
but every user whose key hash is included in the token names can spend from
this script, provided they return the asset to the same script.
To make a script that only allows spending from it if the user doing so
is on the list of key hashes in the credential token made by $\texttt{msig}$, we write

\begin{alltt}
  must_be_on_list(msig) = SpendsCur(credential_token(msig))
\end{alltt}

In our definition of the credential token, we have used all the strategies
we discussed above to extend the expressivity of an FPS language.
We are not yet using the \UTXO\ model to its full potential, as we are just
using the \UTXO\ to store some information that cannot be traded. However, we
could consider updating
our credential token use policy to associate spending it with another action,
such as adding a pay-per-use clause. Such a change really relies on the
\UTXO\ model.

\section{Authoring: the Plutus Haskell SDK}
\label{sec:sdk}

Just like web applications, \glspl{app} have two separate components which are deployed in separate execution environments:
\begin{itemize}
\item \gls{on-chain} which is stored on the blockchain and executed during the validation of new transactions (similar to the server component of a web application), and
\item \gls{off-chain} code which is deployed to and executed on the client machine of a blockchain user with access to the user’s wallet (similar to the client portion of a web application running in a user’s web browser\footnote{
In fact, in other systems off-chain code often executes in a web browser as well!
})
\end{itemize}

Why is this decomposition necessary? The on-chain code contains the \gls{app}'s enforceable components.
It needs to enforce that only transactions that meet the enforceable obligations are successfully validated and added to the chain.
In other words, the integrity of an \gls{app} depends on the integrity of the on-chain code: thus we need to store it on the cryptographically immutable blockchain to prevent tampering.
Moreover, \glspl{slot-leader} need to execute on-chain code to check that it is in fact satisfied.

Conversely, the \gls{off-chain}, which submits new transactions to the chain for validation and inclusion, necessarily needs to run in close association with a contract user’s wallet.
After all, each transaction needs to be paid for with transaction fees, and a wallet is the only place where the necessary credentials are held (anything else would compromise the security of the funds).

Existing blockchains and their smart contract and dapp frameworks use separate languages for the \gls{on-chain} and \gls{off-chain} (in Ethereum, Solidity and JavaScript), and they tend to invent new languages for the on-chain component (e.g., Solidity).
This comes with the same disadvantages as using different languages for the client and server component of web apps.
However, when new languages are invented the situation is even worse because of the enormous overhead involved in creating a new language, compilers and other tools, libraries, teaching material, and generally growing a new language community. We would like to avoid this (\cref{req:source-lang-conservatism}).

We overcome these problems by using Haskell for both \gls{on-chain} and \gls{off-chain} code.
This enables us to build on the existing Haskell ecosystem and to share datatypes and code between the two.
However, a downside of this approach is that GHC forms part of our compilation toolchain, which we do not control and which may change unexpectedly (this can make it hard to satisfy \cref{req:compilation-stability,req:compilation-reproducibility}, for example).

The \gls{plutus-sdk} is our library and tooling support for writing \glspl{app} --- both their \gls{on-chain} and their \gls{off-chain} together.

\subsection{Requirements}
\begin{requirement}[Conservatism]
\label{req:source-lang-conservatism}
Designing a new programming language is hard.
Designing a new \emph{source} programming language (one written by users directly) is even worse.
One must worry about syntax, tooling, build systems, libraries and their ecosystem, etc.

Ideally, we would like to avoid all this by reusing existing languages as much as possible.
\end{requirement}

\begin{requirement}[Lifting values at runtime]
\label{req:runtime-args}
The programs which we put on the chain cannot be entirely static (i.e. determined at \gls{off-chain} compile time).
It must be possible to parameterize them or partially generate them at runtime.

One reason for this is that we may want to configure our code.
For example, a crowdfunding \gls{app} might want to parameterize its \gls{on-chain} by the crowdfunding target, or the beneficiary of the funds.\footnote{
In some instances one can get away with putting this information in the \gls{datum}.
The crowdfunding example is interesting because many people pay to the \gls{address} spontaneously, and the owner cannot control what those people put in the \glspl{datum}.
But they can control the \gls{address} people send to, and hence the \gls{script}.
So it is necessary to bake the parameters into the \gls{script} in this instance.
}
How can we parameterize a \gls{plutus-core} program?
One way is to write the program as a function, compile that function statically, then construct the argument at runtime and apply the program to the argument.
In pseudo-Haskell:
\begin{quote}
\begin{verbatim}
compile (\parameter -> ...) `apply` lift argument
\end{verbatim}
\end{quote}
\noindent This requires a way to ``lift'' a runtime value into an appropriate \gls{plutus-core} term so we can actually apply our compiled program to it.

This is just generally a very handy thing to be able to do and we want to be able to do it.
\end{requirement}

\begin{requirement}[Stability]
\label{req:compilation-stability}
When the \gls{plutus-core} program that makes up the \gls{on-chain} of a \gls{app} changes, so does its hash, and hence its \gls{address}.
This can be a big problem for applications: you cannot spend a \gls{script-output} without presenting a \gls{script} with \emph{exactly} that hash.
If your tooling won't produce such a \gls{script} any more, then you can't get the money!

We therefore want our tooling to be as stable and deterministic as possible, so we don't change the output unnecessarily.

At the very least, we must not be sensitive to:
\begin{itemize}
\item The platform we are working on (linux/macos etc.)
\item Any random or mutable conditions
\end{itemize}
\end{requirement}

\begin{requirement}[Compilation reproducibility]
\label{req:compilation-reproducibility}
As discussed in \cref{req:compilation-stability}, if the output of compilation changes unexpectedly, then that can be a big problem.
But if the user changes their source or their tooling (e.g. their version of GHC), then that may just genuinely change the input to our compiler.

In this instance there is not a great deal we can do at a technical level, except help people to build their applications reproducibly, so that
they can at least revert to previous states reliably.
\end{requirement}

\subsection{Haskell}

We need a source language for users to write \glspl{app} in.
We don't want to write our own one (\cref{req:source-lang-conservatism}), so we would really like to reuse an existing one.
We decided to use Haskell for a number of reasons:
\begin{itemize}
\item
  It is a powerful functional programming language.
\item
  It has an industrial-grade compiler, adequate tooling, and a good community and ecosystem.
\item
  It has good metaprogramming facilities.
\item
  We are familiar with it as a team and a company.
\end{itemize}

However, this means we need a way to compile (as subsection of) a Haskell program into \gls{plutus-core}.
Our solution to this is \gls{plutus-tx}.

\subsection{Plutus Tx}
\label{sec:plutus-tx}

The \gls{plutus-tx} compiler compiles \gls{ghc-core} into \gls{plutus-core}.
That is, it takes Haskell after it has been desugared from its source representation into GHC's internal representation (\gls{ghc-core}), and compiles that further.
This approach allows us to support all of source Haskell while only having to deal with the much smaller \gls{ghc-core} language.

\subsubsection{Plugins for custom compilation}

GHC core-to-core plugins enable us to inject our own code including the \gls{plutus-tx} compiler into the GHC pipeline.
The \gls{plutus-tx} plugin,
\begin{enumerate}
\item locates \gls{ghc-core} fragments representing to \gls{on-chain},
\item compiles them to \gls{plutus-core}, and
\item replaces each \gls{ghc-core} AST subtree representing \gls{on-chain} code with an AST representing a serialised version of the generated \gls{plutus-core}.
\end{enumerate}

Overall, we end up with compiled \gls{off-chain} that embeds blobs of \gls{on-chain} in its serialised \gls{plutus-core} representation, ready to be submitted to the blockchain attached to transactions generated by the \gls{off-chain}.

There just seem to be two problems:
\begin{inparaenum}
\item how does the plugin identify on-chain code and
\item how do we ensure that the type of the serialised on-chain code lines up with the source code?\footnote{
\Gls{ghc-core} is a typed intermediate language; hence, any code transformation needs to be type-preserving.
}
\end{inparaenum}
We achieve this using a trick that to the best of our knowledge was first used in the \code{inline-java} package embedding Java into Haskell.
This packages uses GHC plugins to extract type information at a Template Haskell splice point \autocite{inline-java-blog-post}.
The idea is to wrap the target Haskell code inside a splice of a Template Haskell function that inserts a marker around that AST fragment.

This function does not actually compile the AST of the target program fragment.
Instead, it inserts a marker function that is picked up by the \gls{plutus-tx} compiler injected with the plugin.
When the plugin runs, it finds the marker, compiles the code, and inserts the serialized form back into the program.
By taking a little care over the types of our marker function, we can ensure that the expression in question remains well-typed at each stage of this process.

\subsubsection{Compiling GHC Core to Plutus Core}

Both \gls{ghc-core} and \gls{plutus-core} are extensions of \gls{system-f}.
\Gls{ghc-core} is a much more generous extension.
It adds mutually-recursive binding groups, algebraic data types, case expressions, coercions, and more.
In contrast, \gls{plutus-core}, as discussed in \cref{sec:plutus-core}, is much more minimal.

How do we deal with the extra features of \gls{ghc-core}?
First, we split the problem in half, by defining an intermediate language, \gls{plutus-ir} (see \cref{sec:plutus-ir}), which is much closer to \gls{ghc-core}.
Most of the theoretical complexity is therefore moved to the \gls{plutus-ir} compiler.

The remaining work of the \gls{plutus-tx} compiler is then to \emph{lower} the \gls{ghc-core} terms and types into their corresponding \gls{plutus-ir} variants, emitting errors as appropriate if we encounter features we do not support.

\subsubsection{Supporting Haskell's features}

As alluded to in the previous section, we do not support the entirety of Haskell.
Thanks to the design of GHC, we get a great deal for free, as we compile programs after they have been converted to \gls{ghc-core}, which means that most of the complex source-level features of Haskell have already been desugared into a smaller set of simpler features.

While we support most ``standard'' Haskell, there are quite a few things we do not support. A non-exhaustive list of features that we do not support is:
\begin{itemize}
\item Not implemented yet
  \begin{itemize}
  \item Mutually recursive datatypes (should be done by release)
  \end{itemize}
\item Incompatible with the design of \gls{plutus-core}
  \begin{itemize}
    \item \code{PolyKinds}, \code{DataKinds}, anything that moves towards ``Dependent Haskell''
  \end{itemize}
\item Technically difficult
  \begin{itemize}
  \item Literal patterns
  \end{itemize}
\item Requires access to function definitions (might be fixed with some GHC work)
  \begin{itemize}
  \item Function usage without \code{INLINEABLE} or \code{-fexpose-all-unfoldings}
  \item Typeclass dictionaries
  \end{itemize}
\item Use of coercions required
  \begin{itemize}
  \item GADTs
  \item \code{Data.Coerce}
  \item \code{DerivingVia}, \code{GeneralizedNewtypeDeriving}, etc.
  \end{itemize}
\item Assumes ``normal'' codegen
  \begin{itemize}
  \item FFI
  \item Numeric types other than integers
  \item Unlifted/\code{MagicHash} types
  \item Machine words, C strings, etc.
  \end{itemize}
\end{itemize}

\subsubsection{Strictness}
\label{sec:plutus-tx-strictness}

Haskell is a lazy language and \gls{plutus-core} is a strict language.
How can we compile a lazy language into a strict language efficiently?

The answer is that we handle this partially.
We generally compile Haskell as though it were strict, but the key exception is for non-value let-bindings.
That is, if we see a let-binding whose right-hand-side is not a value (i.e. may evaluate further), then we compile it as a non-strict let-binding (see \cref{sec:pir-non-strict}).

Unfortunately we have no proof that this approach is sound, which is an area for future work.

\subsubsection{Lifting values at runtime}
\label{sec:plutus-tx-lifting}

We are going to great lengths to compile \gls{on-chain} \glspl{validator} at \gls{off-chain} compile time.
However, we may also need to create some \gls{plutus-core} programs from \emph{runtime} values (\cref{req:runtime-args}).

Unfortunately we cannot use the main \gls{plutus-tx} compiler for this: the \gls{plutus-tx} compiler turns Haskell \emph{programs} represented as \gls{ghc-core} into \gls{plutus-core}.
It cannot do anything with the \emph{runtime} representation of a Haskell value!

We therefore need to replicate what we \emph{would} do with the \gls{plutus-tx} compiler, but at runtime.
Fortunately, we can reuse the \gls{plutus-ir} compiler, which helps a lot.
We define a pair of typeclasses inspired by the Haskell typeclasses for ``lifting'' runtime values into metaprograms: \code{Lift} and \code{Typeable}.
Our typeclasses look something like this (\code{Term} and \code{Type} are the types for \gls{plutus-ir} terms and types; class constraints on the methods are omitted for simplicity):

\begin{quote}
\begin{verbatim}
class Lift a where
    lift :: (...) => a -> m (Term TyName Name ())

class Typeable a where
    typeRep :: (...) => Proxy a -> m (Type TyName ())
\end{verbatim}
\end{quote}

With some effort, we are able to generate instances for these with Template Haskell, so the burden on users is minimal.

Why do we output \gls{plutus-ir} here, rather than running the \gls{plutus-ir} compiler each time and just producing \gls{plutus-core}?
The reason is that the \gls{plutus-ir} compiler has some support for \emph{sharing} definitions, and it is important that programs generated from multiple calls to \code{lift} (e.g. from one implementation calling another, as is common) share the same definition of their shared types.

\subsection{Plutus IR}
\label{sec:plutus-ir}

\Gls{plutus-ir} is an intermediate language that sits between \gls{ghc-core} and \gls{plutus-core}.
Many of the core ideas are published in \textcite{peytonjones2019unraveling}, including the complex parts of compilation and the typesystem.

We give a high-level overview of the language here, further details can be found in the above reference.

\Gls{plutus-ir} is essentially \gls{plutus-core}, but with the addition of:
\begin{itemize}
\item Datatypes, including recursive and mutually recursive datatypes
\item Let terms, including recursive and mutually recursive bindings
\end{itemize}

Compiling recursive datatypes and recursive values are the two trickiest compilation problems, and are covered in \textcite{peytonjones2019unraveling}.

\subsubsection{Non-strict let bindings}
\label{sec:pir-non-strict}

\Gls{plutus-ir} has an additional feature which isn't discussed in \textcite{peytonjones2019unraveling}: non-strict let-bindings.
By default (and in the paper), let-bindings are \emph{strict}, meaning that the right-hand-side of the binding is evaluated before the body of the term.

However, it is useful to support \emph{non-strict} let-bindings, particularly because these correspond more closely to the semantics of Haskell (see \cref{sec:plutus-tx-strictness}).
We can desugar these into strict let-bindings simply by inserting a \code{delay} on the binding right-hand-side and a \code{force} at every use site.

\subsubsection{Optimization}

We do a small amount of optimization in the \gls{plutus-ir} compilation pipeline.
We don't want to do too much in case we make the generated code too unstable (\cref{req:compilation-stability}).

\paragraph{Dead code elimination}
\label{para:dead-code}

Dead code elimination is a straightforward optimization and close to a clear win:
\begin{inparaenum}
\item it reduces code size,
\item it makes the code easier to read, and
\item it has no effect on the semantics.
\end{inparaenum}

It is particularly helpful as the \gls{plutus-tx} compiler introduces definitions for all the builtins, some of which will be unused.

\subsubsection{Compilation}

The \gls{plutus-ir} compiler works via a series of small passes that eliminate individual features of \gls{plutus-ir} in turn, until the remaining program is pure \gls{plutus-core} and can simply be lowered into that AST type.

The passes are:
\begin{itemize}
\item Non-strict let-bindings into strict let-bindings by inserting thunks
\item Type bindings and datatypes into simple type and lambda abstractions
\item Recursive term bindings into non-recursive term bindings
  \begin{itemize}
  \item We do another dead code elimination pass (\cref{para:dead-code}) as this can introduce dead bindings.
  \end{itemize}
\item Non-recursive term bindings into lambda abstractions
\end{itemize}

\subsection{Cross-compilation}
\label{sec:cross-compilation}

To support \cref{req:app-dist}, we want to be able to compile \glspl{app} into easily redistributable \glspl{app-exe}.

The current approach is to target Javascript or WebAssembly as our format for distribution, and leverage cross-compilation of Haskell to actually produce the executables.

\subsubsection{Cross-compilers}

At the moment IOHK is working on two cross-compilation efforts:
\begin{description}
  \item[GHCJS] GHCJS is a Haskell cross-compiler which targets Javascript \autocite{ghcjs-repo}.
  \item[Asterius] Asterius is a Haskell cross-compiler which targets WebAssembly \autocite{asterius-repo}.
\end{description}

\noindent We may use either or both of these in the end.

\subsubsection{haskell.nix}

Cross-compilation of Haskell projects is not easy.
Neither of the major Haskell build tools (\code{cabal} and \code{stack}) support cross-compilation well.

To address this issue, IOHK has developed the \code{haskell.nix} framework for building Haskell projects using Nix \autocite{haskell-nix-repo}.
In addition to supporting cross-compilation well, Nix is well-suited to ensuring that builds are reproducible, which helps with \cref{req:compilation-reproducibility}.

\subsection{Developer tooling}
\label{sec:tooling}

Since the \gls{plutus-sdk} uses Haskell for development, there is much less need to create specialized development tooling, since generic Haskell tooling will work perfectly well.\footnote{
It is true that Haskell development tooling is generally considered to not be very good.
However, it is improving rapidly, and while it might be sensible for IOHK to contribute to the community's efforts on this front, that will be significantly less work than developing completely new tools.
}

It is possible that we may want to develop some tools, particularly for testing and visualization, but this has not been decided yet.


% Print the glossary
\printglossaries
\glsaddall

% Print the bibliography and include it in the TOC
\printbibliography[heading=bibintoc]

\end{document}
