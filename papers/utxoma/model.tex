\section{Formal ledger rules}
\label{sec:model}

\begin{ruledfigure}{t}
  \begin{displaymath}
    \begin{array}{rll}
      \B{} && \mbox{the type of Booleans}\\
      \N{} && \mbox{the type of natural numbers}\\
      \Z{} && \mbox{the type of integers}\\
      \H{} && \mbox{the type of bytestrings: } \bigcup_{n=0}^{\infty}\{0,1\}^{8n}\\
      (\phi_1 : T_1, \ldots, \phi_n : T_n) && \mbox{a record type with fields $\phi_1, \ldots, \phi_n$ of types $T_1, \ldots, T_n$}\\
      t.\phi && \mbox{the value of $\phi$ for $t$, where $t$ has type $T$ and $\phi$ is a field of $T$}\\
      \Set{T} && \mbox{the type of (finite) sets over $T$}\\
      \List{T} && \mbox{the type of lists over $T$, with $\_[\_]$ as indexing and $|\_|$ as length}\\
      h::t && \mbox{the list with head $h$ and tail $t$}\\
      x \mapsto f(x) && \mbox{an anonymous function}\\
      \hash{c} && \mbox{a cryptographic collision-resistant hash of $c$}\\
      \Interval{A} && \mbox{the type of intervals over a totally-ordered set $A$}\\
      \FinSup{K}{M} && \mbox{the type of finitely supported functions from a type $K$ to a monoid $M$}
    \end{array}
  \end{displaymath}
  \caption{Basic types and notation}
  \label{fig:basic-types}
\end{ruledfigure}
%
Our formal ledger model follows the style of the UTXO-with-scripts model from~\cite{Zahnentferner18-UTxO} adopting the notation from~\cite{eutxo-1-paper} with basic types defined as in Figure~\ref{fig:basic-types}.

\paragraph{Finitely-supported functions.}
\label{sec:fsfs}
%
We model token bundles as finitely-supported functions.
If $K$ is any type and $M$ is a monoid with identity element $0$, then a function $f: K \rightarrow M$ is \textit{finitely supported} if $f(k) \ne 0$ for only finitely many $k \in K$.
More precisely, for $f: K \rightarrow M$ we define the \textit{support} of $f$ to be
%%
$\supp(f) = \{k \in K : f(k) \ne 0\}$
%%
and
%%
$\FinSup{K}{M} = \{f : K \rightarrow M : \left|\supp(f)\right| < \infty \}$.
%%

If $(M,+,0)$ is a monoid then $\FinSup{K}{M}$ also becomes a monoid if we define addition pointwise (i.e., $(f+g)(k) = f(k) + g(k)$), with the identity element being the zero map.
Furthermore, if $M$ is an abelian group then $\FinSup{K}{M}$ is also an abelian group under this construction, with $(-f)(k) = -f(k)$.
Similarly, if $M$ is partially ordered, then so is $\FinSup{K}{M}$ with comparison defined pointwise: $f \leq g$ if and only if $f(k) \leq g(k)$ for all $k \in K$.

It follows that if $M$ is a (partially ordered) monoid or abelian group then so is $\FinSup{K}{\FinSup{L}{M}}$ for any two sets of keys $K$ and $L$.
We will make use of this fact in the validation rules presented later in the paper (see Figure~\ref{fig:validity}).
Finitely-supported functions are easily implemented as finite maps, with a failed map lookup corresponding to returning 0.

\subsection{Ledger types}

%
\begin{ruledfigure}{htb}
  \begin{displaymath}
    \begin{array}{rll}
      \multicolumn{3}{l}{\textsc{Ledger primitives}}\\
      \Quantity && \mbox{an amount of currency, forming an abelian group (typically \Z{})}\\
      \Asset && \mbox{a type consisting of identifiers for individual asset classes}\\
      \Tick && \mbox{a tick}\\
      \Address && \mbox{an ``address'' in the blockchain}\\
      \TxId && \mbox{the identifier of a transaction}\\
      \txId : \utxotx \rightarrow \TxId && \mbox{a function computing the identifier of a transaction}\\
      \lookupTx : \Ledger \times \TxId \rightarrow \utxotx && \mbox{retrieve the unique transaction with a given identifier}\\
      \verify : \PublicKey\times\H\times\H \rightarrow \B && \mbox{signature verification}\\
      \FPScript && \mbox{forging policy scripts}\\
      \scriptAddr : \Script \rightarrow \Address && \mbox{the address of a script}\\
      \applyScript{\_} : \Script \to (\Address \times \utxotx \times \Set{\Output}) \to
      \B && \mbox{apply script inside brackets to its arguments}\\
    \\
    \multicolumn{3}{l}{\textsc{Ledger types}} \\
    \Policy &=& \Address \qquad \mbox{(an identifier for a custom currency)}\\
    \Signature &=& \H\\
    \\
    \Quantities   &=& \FinSup{\Policy}{\FinSup{\Asset}{\Quantity}}\\
    \\
    \Output &=& (\addr: \Address, \val: \Quantities)\\
    \\
    \OutputRef &=& (\txrefid: \TxId, \idx: \s{Int})\\
    \\
    \Input &=& ( \outputref : \OutputRef\\
             & &\ \validator: \Script)\\
    \\
    \utxotx &=&(\inputs: \Set{\Input},\\
               & &\ \outputs: \List{\Output},\\
               & &\ \validityInterval: \Interval{\Tick},\\
               & &\ \mint: \Quantities\\
               & &\ \scripts: \Set{\FPScript},\\
               & &\ \sigs: \Set{\Signature})\\
    \\
    \Ledger &=&\!\List{\utxotx}\\
    \end{array}
  \end{displaymath}
  \caption{Ledger primitives and basic types}
  \label{fig:ledger-types}
\end{ruledfigure}
%
Figure~\ref{fig:ledger-types} defines the ledger primitives and types that we need to define the \UTXOma\ model.
All outputs use a pay-to-script-hash scheme, where an output is locked with the hash of a script. We use a single scripting language for forging policies and to define output locking scripts. Just as in Bitcoin, this is a restricted domain-specific language (and not a general-purpose language); the details follow in Section~\ref{sec:fps-language}.
We assume that each transaction has a unique identifier derived from its value by a hash function. This is the basis of the $\lookupTx$ function to look up a transaction, given its unique identifier.

\paragraph{Token bundles.}

We generalise per-output transferred quantities from a plain \Quantity\ to a bundle of \Quantities.
A \Quantities{} represents a token bundle: it is a mapping from a policy and an \emph{asset}, which defines the asset class, to a \Quantity{} of that asset.\footnote{
  We have chosen to represent \Quantities{} as a finitely-supported function whose values are themselves finitely-supported functions (in an implementation, this would be a nested map).
  We did this to make the definition of the rules simpler (in particular Rule~\ref{rule:forging}).
  However, it could equally well be defined as a finitely-supported function from tuples of \Policy{}s and \Asset{}s to \Quantity{}s.
}
Since a \Quantities\ is indexed in this way, it can represent any combination of tokens from any assets (hence why we call it a token \emph{bundle}).

\paragraph{Asset groups and forging policy scripts.}

A key concept is the \emph{asset group}.
An asset group is identified by the hash of special script that controls the creation and destruction of asset tokens of that asset group.
We call this script the \emph{forging policy script}.

\paragraph{Forging.}

Each transaction gets a $\mint$ field, which simply modifies the required balance of the transaction by the $\Quantities$ inside it: thus a positive $\mint$ field indicates the creation of new tokens.
In contrast to outputs, $\Quantities$ in mint fields can
also be negative, which effectively burns existing tokens.\footnote{
The restriction on outputs is enforced by Rule~\ref{rule:all-outputs-are-non-negative}.  We simply do not impose such a restriction on the $\mint$ field: this lets us define rules in a simpler way, with cleaner notation. }

Additionally, transactions get a $\scripts$ field holding a set of forging policy scripts: \(\Set{\FPScript}\).
This provides the forging policy scripts that are required as part of validation when tokens are minted or destroyed (see Rule~\ref{rule:forging} in Figure~\ref{fig:validity}). The forging scripts of the assets being minted are
executed and the transaction is only considered valid if the execution of the script returns $\true$.
A forging policy script is executed in a context that provides access to the main components of the forging transaction, the UTXOs it spends, and the policy ID.
The passing of the context provides a crucial piece of the puzzle regarding self-identification: it includes the script's own $\Policy$, which avoids the problem of trying to include the hash of a script inside itself.

\paragraph{Validity intervals.}
\label{para:validity-intervals}

A transaction's \emph{validity interval} field contains an interval of ticks (monotonically increasing units of ``time'', from~\cite{eutxo-1-paper}).
The validity interval states that the transaction must only be validated if the current tick is within the interval. The validity interval, rather than the actual current chain tick value, must be used
for script validation. In an otherwise valid transaction, passing the current
tick to the evaluator
could result in different script validation outcomes at different ticks, which
would be problematic.

\paragraph{Language clauses.}

In our choice of the set of predicates $\texttt{p1, ..., pn}$ to include in the
scripting language definition, we adhere to the following heuristic: we only
admit predicates with quantification over finite structures
passed to the evaluator in the transaction-specific data, i.e. sets,
maps, and lists. The computations we allow in the predicates themselves are well-known
computable functions, such as hashing, signature checking, arithmetic operations, comparisons,
etc.

The gamut of policies expressible in the model we propose here is
fully determined by the collection of predicates,
assembled into a single script by logical connectives
\texttt{\&\&}, \texttt{||}, and \texttt{Not}.
Despite being made up of only hard-coded predicates and connectives,
the resulting policies can be quite
expressive, as we will demonstrate in the upcoming applications section.
When specifying forging predicates, we use $\texttt{tx.\_}$ notation to access
the fields of a transaction.

\subsection{Transaction validity}
\label{sec:validity}

\begin{ruledfigure}{t}
  \begin{displaymath}
  \begin{array}{lll}
  \multicolumn{3}{l}{\txunspent : \utxotx \rightarrow \Set{\s{OutputRef}}}\\
  \txunspent(t) &=& \{(\txId(t),1), \ldots, (\txId(id),\left|t.outputs\right|)\}\\
  \\
  \multicolumn{3}{l}{\unspent : \s{Ledger} \rightarrow \Set{\s{OutputRef}}}\\
  \unspent([]) &=& \emptymap \\
  \unspent(t::l) &=& (\unspent(l) \setminus t.\inputs) \cup \txunspent(t)\\
  \\
  \multicolumn{3}{l}{\getSpent : \s{Input} \times \s{Ledger} \rightarrow \s{Output}}\\
  \getSpent(i,l) &=& \lookupTx(l, i.\outputref.\id).\outputs[i.\outputref.\idx]
  \end{array}
  \end{displaymath}
  \caption{Auxiliary validation functions}
  \label{fig:validation-functions}
\end{ruledfigure}
%
\begin{ruledfigure}{t}
\begin{enumerate}

\item
  \label{rule:slot-in-range}
  \textbf{The current tick is within the validity interval}
  \begin{displaymath}
    \msf{currentTick} \in t.\i{validityInterval}
  \end{displaymath}

\item
  \label{rule:all-outputs-are-non-negative}
  \textbf{All outputs have non-negative values}
  \begin{displaymath}
    \textrm{For all } o \in t.\outputs,\ o.\val \geq 0
  \end{displaymath}

\item
  \label{rule:all-inputs-refer-to-unspent-outputs}
  \textbf{All inputs refer to unspent outputs}
  \begin{displaymath}
    \{i.\outputref: i \in t.\inputs \} \subseteq \unspent(l).
  \end{displaymath}

\item
  \label{rule:value-is-preserved}
  \textbf{Value is preserved}
  \begin{displaymath}
    t.\mint + \sum_{i \in t.\inputs} \getSpent(i, l) = \sum_{o \in t.\outputs} o.\val
  \end{displaymath}

\item
  \label{rule:no-double-spending}
  \textbf{No output is double spent}
  \begin{displaymath}
    \textrm{If } i_1, i \in t.\inputs \textrm{ and }  i_1.\outputref = i.\outputref
    \textrm{ then } i_1 = i.
  \end{displaymath}

\item
  \label{rule:all-inputs-validate}
  \textbf{All inputs validate}
  \begin{displaymath}
    \textrm{For all } i \in t.\inputs,\
    \applyScript{i.\validator}(\scriptAddr(i.\validator), t,
    \{\getSpent(i, l) ~\vert~ i~\in~ t.\inputs\}) = \true
  \end{displaymath}

\item
  \label{rule:validator-scripts-hash}
  \textbf{Validator scripts match output addresses}
  \begin{displaymath}
    \textrm{For all } i \in t.\inputs,\ \scriptAddr(i.\validator) = \getSpent(i, l).\addr
  \end{displaymath}

\item
  \label{rule:forging}
  \textbf{Forging}\\
  A transaction with a non-zero \mint{} field is only
  valid if either:
  \begin{enumerate}
  \item the ledger $l$ is empty (that is, if it is the initial transaction).
  \item \label{rule:custom-mint}
    for every key $h \in \supp(t.\mint)$, there
    exists $s \in t.\scripts$ with
    $h = \scriptAddr(s)$.
  \end{enumerate}
  \medskip
  % ^ There's no space between these items without this, but all the other items have space due to \displaymath

\item
  \label{rule:all-mps-validate}
  \textbf{All scripts validate}
  \begin{displaymath}
    \textrm{For all } s \in t.\scripts,\ \applyScript{s}(\scriptAddr(s), t,
    \{\getSpent(i, l) ~\vert~ i~\in~ t.\inputs\}) = \true
  \end{displaymath}

\end{enumerate}
\caption{Validity of a transaction $t$ in a ledger $l$}
\label{fig:validity}
\end{ruledfigure}
%
Figure~\ref{fig:validity} defines what it means for a transaction $t$ to be valid for a valid ledger $l$ during the tick \currentTick, using some auxiliary functions from Figure~\ref{fig:validation-functions}. A ledger $l$ is \textit{valid} if either $l$ is empty or $l$ is of the form $t::l^{\prime}$ with $l^{\prime}$ valid and $t$ valid for $l^{\prime}$.

The rules follow the usual structure for an UTXO ledger, with a number of modifications and additions.
The new \textbf{Forging} rule (Rule~\ref{rule:forging}) implements the support for forging policies by requiring that the currency's forging policy is included in the transaction --- along with Rule~\ref{rule:all-mps-validate} which ensures that they are actually run!
The arguments that a script is applied to are the ones discussed earlier.

When forging policy scripts are run, they are provided with the appropriate transaction data, which allows them to enforce conditions on it.
In particular, they can inspect the $\mint$ field on the transaction, and so a forging policy script can identify how much of its own currency was minted, which is typically a key consideration in whether to allow the transaction.

We also need to be careful to ensure that transactions in our new system preserve value correctly.
There are two aspects to consider:
\begin{itemize}
\item
  We generalise the type of value to \Quantities.
  However, since \Quantities\ is a monoid (see Section~\ref{sec:model}), Rule~\ref{rule:value-is-preserved} is (almost) identical to the one in the original UTXO model, simply with a different monoid.
  Concretely, this amounts to preserving the quantities of each of the individual token classes in the transaction.
\item
  We allow forging of new tokens by including the mint field into the balance in Rule~\ref{rule:value-is-preserved}.
\end{itemize}
