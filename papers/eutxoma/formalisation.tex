\section{Meta-theoretical Properties of \EUTXOma}
\label{sec:formalization}

In~\cite{eutxo-1-paper}, we characterised the expressiveness of EUTXO
ledgers by showing that we can encode a class of state machines ---
\textit{Constraint Emitting Machines} (CEMs) --- into the ledger via a series
of transactions.  Formally, we showed that there is a weak
bisimulation between CEM transitions and their encoded
transitions in the ledger.

However, as we have seen from the example in \S\ref{sec:cem-example} above,
this result is not sufficient to allow us to reason about the properties
of the state machine.  Even if we know that each individual step is valid,
there may be some properties that are only guaranteed to hold if we started
from a proper initial state.
Specifically, we will not be able to establish any inductive or temporal property
unless we cover the base case corresponding to initial states.
Since our earlier model did not prevent
us from starting the ledger in an arbitrary (and perhaps non-valid) state, such
properties could not be carried over from the state machines to the
ledger.

In this section we extend our formalisation to map CEMs into threaded
ledgers.  By formalising some of the properties of \EUTXOma, we can
guarantee that a ledger was started from a valid initial state, and
thus carry over inductive and temporal properties from state machines.  All
the results in this section have been mechanised in Agda,
extending the formalisation from~\cite{eutxo-1-paper}.

\subsection{Token provenance}

Given a token in a \EUTXOma{} ledger, we can ask ``where did this token
\emph{come from}?''  Since tokens are always created in specific
forging operations, we can always trace them back through their
transaction graph to their \textit{origin}.

We call such a path through the transaction graph a \emph{trace}, and
define the \emph{provenance} of an asset at some particular output
to be a trace leading from the current transaction to an origin which
forges a value containing the traced tokens.

In the case of fungible tokens, there may be multiple possible traces:
since such tokens are interchangeable, if two tokens of some asset are forged separately, and then
later mingled in the same output, then we cannot say which one came from which forging.

Let $\nft = (p , a)$ denote a particular $\Asset\ a$ controlled by a $\Policy\ p$,
and $v^\nft = v(p)(a)$ the quantity of $\nft$ tokens present in value $v$.
$\Trace(l,o,\nft,n)$ is the type of sequences of transactions
$t_0,\dots, t_i, t_{i +1},\dots, t_k$ drawn from a ledger $l$ such that:
\begin{itemize}
\item the origin transaction $t_0$ forges at least $n$ $\nft$
  tokens: $t_0.\forge^\nft \ge n$
\item every pair $(t_i,t_{i + 1})$ has an input/output connection in
  the ledger that transfers at least $n$ $\nft$ tokens:
  $\exists o' \in t_{i + 1}.\inputs.\outputref. (t_i.\outputs[o'.\idx].\val^\nft \ge n)$
\item the last transaction $t_k$ carries the traced tokens in output $o$
\end{itemize}
We define $\Provenance(l,o,\nft)$ to be a
set of traces $\dots \Trace(l,o,\nft,n_i)\dots$, such that $\sum n_i \ge o.\val^\nft$.

To construct an output's provenance, with respect to
a particular policy and asset, we aggregate all possible
traces containing a token of this asset from the transaction's inputs as well
as from the value that is currently being forged.
Thus our tracing procedure will construct a sound over-approximation of the actual provenance.

\begin{displaymath}
\infer[\textsc{Provenance}]
  {\provenance(l,o,\nft) : \Provenance(l,o,\nft)}
  {o \in \{ t.\outputs \ | \ t \in l \}}
\end{displaymath}

\subsection{Forging policies are enforced}

One particular meta-theoretical property we want to check is
that each token is created properly: there is an origin transaction which forges
it and is checked against the currency's forging policy.

\begin{proposition}[Non-empty Provenance]
Given any token-carrying output of a valid ledger,
its computed provenance will be inhabited by at least one trace.

\begin{displaymath}
\infer[\textsc{Provenance}^+]
  {|\provenance(l, o, \nft)| > 0}
  {%
    o \in \{ t.\outputs \ | \ t \in l \}
  & o.\val^\nft > 0
  }
\end{displaymath}
\end{proposition}
%
However, this is not enough to establish what we want: due to
fungibility, we could identify a single origin as the provenance for
two tokens, even if the origin only forges a single token!

To remedy this, we complement (non-empty) token provenance with a
proof of \textit{global preservation}; a generalisation of the
local validity condition of \textit{value preservation} from
Rule~\ref{rule:value-is-preserved} in \S\ref{app:model}.

\begin{proposition}[Global Preservation]
Given a valid ledger $l$:
\begin{displaymath}
\sum_{t \in l} t.\forge = \sum_{o \in \unspent(l)} o.\val
\end{displaymath}
\end{proposition}

The combination of non-empty token provenance and global preservation
assures us that each token is created properly and there is globally
no more currency than is forged.

To prove these properties we require transactions to be unique
throughout the ledger, which is not implied by the validity rules of
Figure~\ref{fig:validity} in \S\ref{app:model}.  In practice though, one could derive
uniqueness from any ledger that does not have more than one
transaction with no inputs, since a transaction would always have to
consume some previously unspent output to pay its fees.
For our purposes, it suffices to assume there is always a single
\textit{genesis} transaction without any inputs:
\begin{displaymath}
\infer[\textsc{Genesis}]
  {|\{ t \in l\ | \ t.inputs = \emptyset \}| = 1}
  {}
\end{displaymath}
%
Then we can derive the uniqueness of transactions and unspent outputs.
%% \begin{minipage}{.45\linewidth}
%% \begin{displaymath}
%% \infer[\textsc{Unique-tx}]
%%   {\forall t_1, t_2 \in l.\ t_1 \neq t_2 }
%%   {\textsc{Genesis}}
%% \end{displaymath}
%% \end{minipage}
%% \begin{minipage}{.45\linewidth}
%% \begin{displaymath}
%% \infer[\textsc{Unique-utxo}]
%%   {\forall o_1, o_2 \in \unspent(l).\ o_1 \neq o_2 }
%%   {\textsc{Genesis}}
%% \end{displaymath}
%% \end{minipage}

\subsection{Provenance for non-fungible tokens}

However, our approach to threading crucially relies on the idea that a
non-fungible token can pick out a \emph{unique} path through the
transaction graph.

First, we define a token $\nft$ to be \textit{non-fungible} whenever
it has been forged at most once across all transactions of an existing
valid ledger $l$.
\ifootnote{%
  This really is a property of a particular ledger: a ``non-fungible'' token can become fungible in a longer ledger if other tokens of its asset are forged.
  Whether or not this is possible will depend on external reasoning about the forging policy.%
}
Then we can prove that non-fungible tokens have a singular provenance;
they really do pick out a unique trace through the transaction graph.

\begin{proposition}[Provenance for Non-fungible Tokens]
Given a valid ledger $l$ and an unspent output carrying a non-fungible
token $\nft$:

\begin{displaymath}
\infer[\textsc{NF-Provenance}]
  {|\provenance(l, o, \nft)| = 1}
  {%
    o \in \{ t.\outputs \ | \ t \in l \}
  & o.\val^\nft > 0
  & \sum_{t \in l} t.\forge^\nft \leq 1
  }
\end{displaymath}
\end{proposition}

\subsection{Threaded state machines}

Armed with the provenance mechanism just described, we are now ready
to extend the previously established bisimulation between CEMs and
state machines in ledgers, so that it relates CEMs with support for
initial states to \textit{threaded} state machines in ledgers.  That
is, in addition to creating an appropriate validator as
in~\cite{eutxo-1-paper}, we also create a forging policy that forges
a unique \textit{thread token} which is then required for the
machine to progress (and is destroyed when it terminates).  Crucially,
the forging policy also checks that the machine is starting in an
initial state.

For the sake of brevity, we refrain from repeating the entirety of
the CEM model definitions from~\cite{eutxo-1-paper}.
Instead, we focus only on the relevant modifications and refer
the reader to the mechanised model for more details.

First, we omit the notion of final states altogether, i.e., there is
no $\s{final}$ predicate in the definition of a CEM.
\ifootnote{%
  Note that this also simplifies the bisimulation propositions proven
  in~\cite{eutxo-1-paper}, since we no longer need to consider the
  special case of final states. Other than that, the statements stay
  exactly the same.%
}
We follow Robin Milner who observed, ``What matters about a sequence of
actions is not whether it drives the automaton into an accepting state
but whether the automaton is able to perform the sequence
interactively.''~\cite{milner-pibook} Formally, this corresponds to
prefix closure: if an automaton accepts a string $s$ then it accepts
any initial part of $s$~(Definition 2.6 \cite{milner-pibook}). Our
state machines are not classical state machines which accept or
reject a string --- rather they are \emph{interactive processes} which
respond to user input.
However, it might be useful to re-introduce final states in the future if we support burning of the thread token.

On the other hand, we now include the notion of initial state in the
predicate function $\initial : \s{S} \ra \B$, which characterises
the states in which a machine can start from. This enables us to ensure
that multiple copies of the machine with the same thread token cannot
run at once and the machine cannot be started in an intermediate state.
State machines whose execution must start from an initial state are also
referred to as \emph{rooted} state transition systems~\cite{TLSS}.

To enforce non-fungibility of the forged token, we require
that the forging transaction spends a specific output, fixed for a
particular CEM instance $\mathcal{C}$ and supplied along with its definition as a
field \s{origin}.  Therefore, since no output can be spent twice, we
guarantee that the token cannot be forged again in the future.

The CEM's forging policy checks that we only forge a single thread
token, spend the supplied origin, and that the state propagated in the
outputs is initial:

\begin{displaymath}
\polC(\mi{txInfo}, c) = \left\{
  \begin{array}{lll}
  \true  & \mi{if} \ \mi{txInfo}.\forge^\nft = 1 \\
         & \mi{and} \ \s{origin} \in \mi{txInfo}.\outputrefs \\
         & \mi{and} \ \initial(\mi{txInfo}.\mi{outputs}^\nft) \\
  \false & \mi{otherwise}
  \end{array}
\right.
\end{displaymath}
%
where $\nft = \{\hash{\valC} \mapsto \{ \hash{\polC} \mapsto 1 \}\}$
is the thread token and $\mi{txInfo}.outputs^\nft$ looks up the output
which carries the newly-forged thread token and is locked by the same
machine's validator, returning the datum decoded as a value of the
state type $\s{S}$.

We extend the previous definition of a CEM's validator (displayed in
grey) to also check that the thread token is attached to the source
state $s$ and propagated to the target state $s'$:

\begin{displaymath}
\color{gray}
\valC(s, i, \mi{txInfo}) = \left\{
  \begin{array}{ll}
  \true  & \mi{if}  \ \CStep{s} \\
         & \mi{and} \ \satisfies(\mi{txInfo}, \txeq) \\
         & \mi{and} \ \checkOutputs(s', \mi{txInfo}) \\
         & \begingroup\color{black}
           \mi{and} \ \s{propagates}(\mi{txInfo}, \nft, s, s')
           \endgroup \\
  \false & \mi{otherwise}
  \end{array}
\right.
\end{displaymath}

This is sufficient to let us prove a key result: given a threaded ledger
state (i.e., one that includes an appropriate thread token), that state
\emph{must} have evolved from a valid initial state.  Since we know
that the thread token must have a singular provenance, we know that
there is a unique ledger trace that begins by forging that token
--- which is guaranteed to be an initial CEM state.

\begin{proposition}[Initiality]
Given a valid ledger $l$ and an unspent output carrying the thread
token $\nft$, we can always trace it back to a single origin, which
forges the token and abides by the forging policy:
\end{proposition}
%
\begin{displaymath}
\infer[\textsc{Initiality}]
  { \exists tr. \ \s{provenance}(l, o, \nft) = \{ tr \}
  \ \land \
    \polC(\mkMpsContext(tr_0, \hash{\valC}, l^{tr_0})) = \true
  }
  {%
    o \in \{ t.\outputs \ | \ t \in l \}
  & o.\val^\nft > 0
  }
\end{displaymath}
%
where $tr_0$ denotes the origin of trace $tr$,
and $l^t$ is the prefix of the ledger up to transaction $t$.
The proof essentially relies on the fact that the thread token is
non-fungible, as its forging requires spending a unique output.
By \textsc{NF-Provenance}, we then get the unique trace back to a
\textit{valid} forging transaction, which is validated against the
machine's forging policy.

\subsection{Property preservation}

Establishing correct initialisation is the final piece we need to be
able to show that properties of abstract CEMs carry over to their
ledger equivalents.  It is no longer possible to reach any state that
cannot be reached in the abstract state machine, and so any properties
that hold over traces of the state machine also hold over traces of
the ledger.

Intuitively, looking at the ledger trace we immediately get from
\textsc{Initiality}, we can discern an underlying CEM trace
that throws away all irrelevant ledger information and only keeps
the corresponding CEM steps.
After having defined the proper notions of property preservation for
CEM traces, we will be able to transfer those for on-chain
traces by virtue of this extraction procedure.

A simple example of a property of a machine is an invariant. We
consider predicates on states of the machine that we call
state-predicates.
\begin{definition}
A state-predicate $P$ is an \emph{invariant} if it holds for all reachable
states of the machine. i.e., if it holds for the initial state, and
for any $s$, $i$, $s'$ and $\txeq$ such that $\CStep{s}$, if
it holds for $s$ then it holds for $s'$.
\end{definition}

\subsubsection{Traces of state machine execution.}
%
We have traced the path of a token through the ledger. We can also
trace the execution of a state machine. We consider rooted traces that
start in the initial state, and refer to them as just \emph{traces}. A
trace records the history of the states of the machine over time and
also the inputs that drive the machine from state to state.
\begin{definition}
A trace is an inductively defined relation on states:
\[
\infer[\mathsf{root}]
      {s \transTo s}
      {\initial(s) = \true}
\qquad
\infer[\mathsf{snoc}]
      {s \transTo s''}
      { s \transTo s'
      & \CStep{s'}
      }
\]
\end{definition}
%
This gives us a convenient notion to reason about temporal
properties of the machine, such as those which hold at all times, at
some time, and properties that hold until another one does. In this
paper we restrict our attention to properties that always hold.

\subsubsection{Properties of execution traces.}
%
Consider a predicate $P$ on a single step of execution.
A \textit{step-predicate} $P(s,i,s',\txeq)$ refers to the incoming state $s$, the input
$i$, the outgoing state $s'$ and the constraints $\txeq$. A
particularly important lifting of predicates on steps to predicates on
traces is the transformer that ensures the predicate holds for every
step in the trace.
\begin{definition}
Given a step-predicate $P$, a trace $tr$ satisfies $\All(P,tr)$
whenever every step $\CStep{s}$ satisfies $P(s,i,s',\txeq)$.
\end{definition}
A predicate transformer lifting state-predicates to a predicate that
holds everywhere in a trace can be defined as a special case:
$\AllS(P,tr) = \All(\lambda s\ i\ s'\ \txeq.\ P(s) \land P(s'),\ tr)$.

We are not just interested in properties of CEM traces in
isolation, but more crucially in whether these
properties hold when a state machine is compiled to a contract to be
executed on-chain.

We observe that the on-chain execution traces precisely follow the
progress of the thread token. We use the same notion of token
provenance to characterise on-chain execution traces. To facilitate
reasoning about these traces, we can \emph{extract} the corresponding
state machine execution traces and reason about those. This allows us
to confine our reasoning to the simpler setting of state machines.

\begin{proposition}[Extraction]
\label{prop:extraction}
Given a valid ledger $l$ and a singular provenance of the
\textup{(}non-fungible\textup{)} thread token $\nft$, we can extract a rooted state
machine trace.
\end{proposition}

\begin{displaymath}
\infer[\textsc{Extraction}]
  {tr.source \transTo tr.destination}
  {\provenance(l, o, \nft) = \{ tr \}}
\end{displaymath}
where $tr.source, tr.destination$ are the states associated with the endpoints of the trace $tr$.
\begin{proof}
  It is straightforward to show this holds, as a corollary of
  \textsc{Initiality}. For the base case, knowing that the origin of the
  trace abides by the forging policy ensures that it forges the thread
  token and outputs an initial state (\s{root}). In the inductive
  step, knowing that the validator runs successfully guarantees that
  there is a corresponding CEM step (\s{snoc}).
  \qed
\end{proof}

\begin{corollary}
\label{all-cor}
Any predicate that always holds for a CEM trace also holds for the one
extracted from the on-chain trace.
\end{corollary}
%
We will write $\extract(tr)$ whenever we want to extract the CEM trace from the on-chain trace $tr$.

\begin{example}
Consider a CEM representing a simple counter that counts up from
zero:
\[
\begin{array}{l}
  (\mathbb {Z}, \{\mathsf{inc}\}, \step, \initial)\quad
  \mathbf{where}\quad
  \step(i,\mathsf{inc}) = \mathsf{just}(i + 1);\quad
  \initial(0) = \true\\
\end{array}
\]
\end{example}
%
A simple property that we want to hold is that the state of the
counter is never negative.

\begin{property}
\label{prop:non-negative}
The counter state $c$ is non-negative, i.e., $c \geq 0$.
\end{property}

\begin{lemma}
\label{counter-inv}
Property~\ref{prop:non-negative} is an invariant, i.e., it holds for all reachable states:
  \begin{enumerate}
  \item $\forall c.\ \initial(c) \rightarrow c \geq 0$
  \item $\forall c\ c'.\ \CStep{c} \rightarrow c\geq 0 \rightarrow c' \geq 0$
  \end{enumerate}
\end{lemma}

\begin{proposition}
In all reachable states, both off-chain and on-chain, the counter is non-negative.
\end{proposition}
\begin{enumerate}
\item $\forall c\ c'\ (tr : c \transTo c').\ \AllS(\lambda x.\ x \geq 0,\ tr)$
\item $\forall l\ o\ tr.\ {\provenance(l, o, \nft) = \{ tr \}} \rightarrow \AllS(\lambda x.\ x \geq 0, \extract(tr))$
\end{enumerate}
\begin{proof}
  (1) follows from Lemma~\ref{counter-inv}. (2) follows from (1) and Corollary~\ref{all-cor}.
  \qed
\end{proof}

\begin{example}
We return to the $n$--of--$m$ multi-signature contract of
\S\ref{sec:EUTXOma}. We pass as parameters to the machine a
threshold number $n$ of signatures required and a list of $m$ owner
public keys \textit{signatories}. The states of the machine are given by
$\{\Holding, \msf{Collecting}\}$ and the
inputs are given by $\{\Pay, \Cancel, \msf{Add}, \msf{Propose}\}$,
along with their respective arguments.
The only initial state is $\Holding$ and we omit the definition of
$\step$ which can be read off from the picture in
Figure~\ref{fig:multisig-machine}.

First and foremost, the previous limitation of starting in non-initial
states has now been overcome, as proven for all state machines in
Proposition~\ref{prop:extraction}.
Specifically, this holds at any output in the ledger carrying the
$\msf{Collecting}$ state,
therefore it is no longer possible to circumvent the checks performed
by the $\msf{Add}$ input.

\end{example}
\begin{property}
It is only possible to cancel after the deadline.
\end{property}
%
We define a suitable step predicate $Q$ on inputs and constraints for a step. If the input is
$\Cancel$ then the constraints must determine that the
transaction can appear on the chain only after the deadline. If the
input is not $\Cancel$ then the predicate is trivially
satisfied.
\begin{displaymath}
Q(s,i,\_,\txeq) = \left\{
  \begin{array}{ll}
  \false & \mi{if}  \ i = \Cancel \ \mi{and}\ s = \Propose{\_\ }{\_\ }{d} \ \mi{and}\ \txeq.\s{range} \neq d\dots\!+\!\infty \\
  \true  & \mi{otherwise}
  \end{array}
\right.
\end{displaymath}
%
Note that we could extend $Q$ to include cases for correctness
properties of other inputs such as ensuring that only valid signatures
are added and that payments are sent to the correct recipient.
\begin{lemma}
\label{lem:msig-correct}
$Q$ holds everywhere in any trace. i.e. $\forall s\ s'\ (tr : s \transTo s').\ \All(Q,\ tr)$.
\end{lemma}
\begin{proof}
  By induction on the trace $tr$.
  \qed
\end{proof}
\begin{proposition}
  For any trace, all cancellations occur after the deadline.
\end{proposition}
\begin{proof}
  Follows from Lemma~\ref{lem:msig-correct} and the fact that the validator ensures all constraints are satisfied.
  \qed
\end{proof}

\paragraph{Beyond safety properties.}
All the properties presented here are \textit{safety} properties, i.e., they hold in every state and we express them as state predicates.
However, a large class of properties we would like to prove are \textit{temporal}, as in relating different states (also known as \textit{liveness} or \textit{progress} properties~\cite{manna-pnueli,baier-katoen}).

Scilla, for instance, provides the \texttt{since\_as\_long} temporal operator for such purposes~\cite{scilla},
which we have also encoded in our framework using a straightforward inductive definition.
However, one may need to move to infinitary semantics to encode the entirety of \textit{Linear Temporal Logic} (LTL)
or \textit{Computational Tree Logic} (CTL), in contrast to reasoning only about finite traces.
In our setting, we would need to provide a \textit{coinductive} definition of traces
to encode the temporal operators of the aforementioned logics,
as done in \cite{infinite-trees} in the constructive setting of the Coq proof assistant.
We leave this exploration for future work.
