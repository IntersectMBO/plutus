\section{Expressiveness of \EUTXO{}}
\label{sec:expressiveness}

In this section, we introduce a class of state machines that can admit
a straightforward modelling of smart contracts running on an \EUTXO{}
ledger. The class we choose corresponds closely to Mealy
machines~\cite{mealy} (deterministic state transducers). The transition
function in a Mealy machine produces a value as well as a new
state. We use this value to model the emission of constraints which
apply to the current transaction in the ledger. We do not claim that
this class captures the full power of the ledger: instead we choose it
for its simplicity, which is sufficient to capture a wide variety of
use cases.

We demonstrate how one can represent a smart contracts using Mealy
machines and formalise a \textit{weak bisimulation} between the
machine model and the ledger model.  Furthermore, we have mechanised
our results in Agda\site{
https://github.com/\GitUser/formal-utxo/tree/\AgdaCommit/Bisimulation.agda
}, based on an executable
specification of the model described in
Section~\ref{sec:formal-model}.

\subsection{Constraint Emitting Machines}
We introduce Constraint Emitting Machines (\CEM{}) which are based on
Mealy machines. A \CEM{} consists of its type of states \s{S} and inputs
\s{I}, a predicate function $\s{final} : \s{S} \rightarrow \s{Bool}$
indicating which states are final and a valid set of transitions,
given as a function $\s{step} : \s{S} \rightarrow \s{I} \rightarrow
\s{Maybe}\ (\s{S} \times \s{TxConstraints})$\footnote{
The result may be \s{Nothing}, in case no valid transitions exist from a given state/input.
}
from source state and input symbol to target state and constraints and
denoted $\CStep{s}{i}{\txeq}$.

The class of state machines we are concerned with here diverge from
the typical textbook description of Mealy Machines in the following aspects:
\begin{itemize}
\item The set of states can be infinite.

\item There is no notion of \textbf{initial state}, since we would not
  be able to enforce it on the blockchain level. Therefore, each
  contract should first establish some initial trust to bootstrap the
  process.
  One possible avenue for overcoming this limitation is to build a notion of \textit{trace simulation}
  on top of the current relation between single states,
  thus guaranteeing that only valid sequences starting from initial states appear on the ledger.
  For instance, this could be used to establish inductive properties of a state machine and
  carry them over to the ledger; we plan to investigate such concerns in future work.

\item While \textbf{final states} traditionally indicate that the
  machine \textit{may} halt at a given point, allowing this
  possibility would cause potentially stale states to clutter the \UTXO{}
  set in the ledger. Thus, a \CEM{} final state indicates that the
  machine \textit{must} halt. It will have no continuing transitions
  from this point onward and the final state will not appear in the
  \UTXO{} set. This corresponds to the notion of a \emph{stopped}
  process~\cite{sangiorgi} which cannot make any transitions.

\item The set of output values is fixed to \emph{constraints} which
  impose a certain structure on the transaction that will implement
  the transition.  Our current formalisation considers a limited set
  of first-order constraints, but these can easily be extended without
  too many changes in the accompanying proofs.
\end{itemize}

\subsection{Transitions-as-transactions}

We want to compile a smart contract $\mathcal{C}$ defined as a \CEM{} into
a smart contract that runs on the chain. The idea is to derive a
validator script from the step function, using the datum to hold the
state of the machine, and the redeemer to provide the transition signal.
A valid transition in a \CEM{}
will correspond to a single valid transaction on the chain. The
validator is used to determine whether a transition is valid and the
state machine can advance to the next state. More specifically, this
validator should ensure that we are transitioning to a valid target
state, the corresponding transaction satisfies the emitted constraints
and that there are no outputs in case the target state is final:

\[
\mkValidator{\mathcal{C}}(s, i, \mi{txInfo}) = \left\{
  \begin{array}{lll}
  \true  & \mi{if} \ \CStep{s}{i}{\txeq} \\
         & \mi{and} \ \satisfies(\mi{txInfo}, \txeq) \\
         & \mi{and} \ \checkOutputs(s', \mi{txInfo}) \\
  \false & \mi{otherwise}
  \end{array}
\right.
\]

\noindent
Note that unlike the step function which returns the new state, the
validator only returns a boolean. On the chain the next state is
provided with the transaction output that ``continues'' the state
machine (if it continues), and the validator simply validates that
the correct next state was provided.\footnote{
A user can run the step function locally to determine the correct next state off-chain.
}

\subsection{Behavioural Equivalence}
We have explained how to compile state machines to smart contracts but
how do we convince ourselves that these smart contracts will behave as
intended? We would like to show
\begin{inparaenum}[(1)]
\item that any valid transition in a \CEM{} corresponds to a valid transaction
  on the chain, and
\item that any valid transaction on the chain corresponds to a valid transition.
\end{inparaenum}
We refer to these two properties as soundness and completeness below.

While state machines correspond to automata, the automata theoretic
notion of equivalence --- trace equivalence --- is too coarse when we
consider state machines as running processes. Instead we use
bisimulation, which was developed in concurrency theory for exactly
this purpose, to capture when processes behave the
same~\cite{sangiorgi}. We consider both the state machine and the
ledger itself to be running processes.

If the state machine was the only user of the ledger then we could
consider so-called strong bisimulation where we expect transitions in
one process to correspond to transitions in the other and
vice-versa. But, as we expect there to be other unrelated transactions
occurring on the ledger we instead consider weak bisimulation where
the ledger is allowed to make additional so-called \emph{internal}
transitions that are unrelated to the behaviour we are interested in
observing.

The bisimulation proof relates steps of the \CEM{} to new transaction
submissions on the blockchain.  Note that we have a \textit{weak}
bisimulation, since it may be the case that a ledger step does not
correspond to a \CEM{} step.

\begin{definition}[Process relation]
A \CEM{} state $s$ corresponds to a ledger $l$ whenever $s$ appears
in the current \UTXO{} set,
locked by the validator derived from this \CEM{}:
\[
\Sim{l}{s}
\]
\end{definition}

\begin{definition}[Ledger step]
Concerning the blockchain transactions, we only consider valid
ledgers.\footnote{ In our formal development, we enforce validity
  statically at compile time.  }  Therefore, a valid step in the
ledger consists of submitting a new transaction $tx$,
valid w.r.t. to the current ledger $l$,
resulting in an extended ledger $l'$:
\[
\LStep{l}{tx}
\]
\end{definition}

\begin{proposition}[Soundness]
Given a valid \CEM{} transition $\CStep{s}{i}{\txeq}$ and a valid ledger
$l$ corresponding to source state $s$, we can construct a valid
transaction submission to get a new, extended ledger $l'$ that
corresponds to target state $s'$:
\[
\infer[\textsc{sound}]
  {\exists tx\ l'\ .\ \LStep{l}{tx}\ \wedge \Sim{l'}{s'}}
  {%
    \CStep{s}{i}{\txeq}
  & \Sim{l}{s}
  }
\]
\end{proposition}

\paragraph{Note.}
We also require that the omitted constraints are satisfiable in the
current ledger and the target state is not a final one, since there
would be no corresponding output in the ledger to witness
$\Sim{l'}{s'}$.  We could instead augment the definition of
correspondence to account for final states, but we have refrained from
doing so for the sake of simplicity.

\begin{proposition}[Completeness]
Given a valid ledger transition $\LStep{l}{tx}$ and a \CEM{} state $s$
that corresponds to $l$, either $tx$ is irrelevant to the current \CEM{}
and we show that the extended ledger $l'$ still corresponds to source
state $s$, or $tx$ is relevant and we exhibit the corresponding \CEM{}
transition $\CStep{s}{i}{\txeq}$\footnote{ We cannot provide a
  correspondence proof in case the target state is final, as explained
  in the previous note.  }:
\[
\infer[\textsc{complete}]
  { \Sim{l'}{s}\ \vee\ \exists i\ s'\ \txeq\ .\ \CStep{s}{i}{\txeq} }
  { \LStep{l}{tx}
  & \Sim{l}{s}
  }
\]
\end{proposition}

Together, soundness and completeness finally give us weak bisimulation.
Note, however, that our notion of bisimulation differs from the textbook one (e.g. in Sangiorgi~\cite{sangiorgi}),
due to the additional hypotheses that concern our special treatment of constraints and final states.
