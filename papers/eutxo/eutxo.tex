\documentclass[runningheads]{llncs}

% 4th Workshop on Trusted Smart Contracts: https://fc20.ifca.ai/wtsc/cfp.html
%
%   Abstract Registration (preferably)	December 9, 2019
%   Paper Submission Deadline	December 12, 2019
%   Early Author Notification	January 4, 2020
%   Late Abstract Registration	January 5, 2020
%   Late Submission Deadline	January 7, 2020
%   Late Author Notification	January 20, 2020
%
% LNCS: 15 pages including references and appendices

\usepackage{url}

\usepackage{graphicx}

% If you use the hyperref package, please uncomment the following line
% to display URLs in blue roman font according to Springer's eBook style:
%% kwxm: I did as it said, but then had to add \usepackage{url}, which
%% wasn't there before (but \url worked anyway).
\renewcommand\UrlFont{\color{blue}\rmfamily}

\usepackage[dvipsnames]{xcolor}
\usepackage{verbatim}
\usepackage{alltt}
\usepackage{etoolbox}
\usepackage{paralist}

\usepackage{float}

\usepackage[cmex10]{amsmath}
\usepackage{amssymb}
\usepackage{stmaryrd}
%\usepackage{amsthm}
\usepackage{proof}

\newcommand{\red}[1]{\textcolor{red}{#1}}

\usepackage{todonotes}
%\usepackage[disable]{todonotes}
\newcommand{\todochak}[1]{\todo[inline,color=purple!40,author=chak]{#1}}
\newcommand{\todompj}[1]{\todo[inline,color=yellow!40,author=Michael]{#1}}
\newcommand{\todokwxm}[1]{\todo[inline,color=blue!20,author=kwxm]{#1}}
\newcommand{\todojm}[1]{\todo[inline,color=purple!40,author=Jann]{#1}}
\newcommand{\todor}[1]{\todo[inline,color=orange!40,author=Orestis]{#1}}
\newcommand{\todojc}[1]{\todo[inline,color=gray!40,author=James]{#1}}

%% ... plus other authors.

\usepackage[colorlinks=true,linkcolor=MidnightBlue,citecolor=ForestGreen,urlcolor=Plum]{hyperref}
%% ^ To make the links a bit less garish.  Delete this to return to normal.

\newcommand\site[1]{\footnote{\url{#1}}}
%% ^ footnote URLs

%% A figure with rules above and below.
\newcommand\rfskip{7pt}
\newenvironment{ruledfigure}[1]{\begin{figure}[#1]\hrule\vspace{\rfskip}}{\vspace{\rfskip}\hrule\end{figure}}

\renewcommand{\i}{\textit}  % Just to speed up typing: replace these in the final version
\renewcommand{\t}{\texttt}  % Just to speed up typing: replace these in the final version
\newcommand{\s}{\textsf}  % Just to speed up typing: replace these in the final version
\newcommand{\msf}[1]{\ensuremath{\mathsf{#1}}}
\newcommand{\mi}[1]{\ensuremath{\mathit{#1}}}


%% Various text macros
\newcommand{\true}{\textsf{true}}
\newcommand{\false}{\textsf{false}}

\newcommand{\hash}[1]{\ensuremath{#1^{\#}}}

\newcommand{\List}[1]{\ensuremath{\s{List}[#1]}}
\newcommand{\Set}[1]{\ensuremath{\s{Set}[#1]}}
\newcommand{\Map}[2]{\ensuremath{\s{Map}[#1,#2]}}
\newcommand{\Interval}[1]{\ensuremath{\s{Interval}[#1]}}
\newcommand{\extended}[1]{#1^\updownarrow}

\newcommand{\script}{\ensuremath{\s{Script}}}
\newcommand{\scriptAddr}{\msf{scriptAddr}}
\newcommand{\ctx}{\ensuremath{\s{Context}}}
\newcommand{\toCtx}{\msf{toContext}}

\newcommand{\toData}{\msf{toData}}
\newcommand{\fromData}{\msf{fromData}}

% Macros for eutxo things.
\newcommand{\TxId}{\ensuremath{\s{TxId}}}
\newcommand{\txId}{\msf{txId}}
\newcommand{\txrefid}{\mi{id}}
\newcommand{\Address}{\ensuremath{\s{Address}}}
\newcommand{\DataHash}{\ensuremath{\s{DataHash}}}
\newcommand{\idx}{\mi{index}}
\newcommand{\inputs}{\mi{inputs}}
\newcommand{\outputs}{\mi{outputs}}
\newcommand{\forge}{\mi{forge}}
\newcommand{\fee}{\mi{fee}}
\newcommand{\addr}{\mi{addr}}
\newcommand{\val}{\mi{value}}  %% \value is already defined

\newcommand{\validator}{\mi{validator}}
\newcommand{\redeemer}{\mi{redeemer}}
\newcommand{\datum}{\mi{datum}}
\newcommand{\datumHsh}{\mi{datumHash}}
\newcommand{\datumWits}{\mi{datumWitnesses}}
\newcommand{\hashData}{\msf{dataHash}}
\newcommand{\validityInterval}{\mi{validityInterval}}
\newcommand{\Data}{\ensuremath{\s{Data}}}

\newcommand{\outputref}{\mi{outputRef}}
\newcommand{\txin}{\mi{in}}
\newcommand{\id}{\mi{id}}
\newcommand{\lookupTx}{\msf{lookupTx}}
\newcommand{\currentTick}{\msf{currentTick}}
\newcommand{\getSpent}{\msf{getSpentOutput}}

\newcommand{\tick}{\ensuremath{\s{Tick}}}
\newcommand{\spent}{\msf{spentOutputs}}
\newcommand{\unspent}{\msf{unspentOutputs}}
\newcommand{\txunspent}{\msf{unspentTxOutputs}}
\newcommand{\eutxotx}{\msf{Tx}}

\newcommand{\qty}{\ensuremath{\s{Quantity}}}
\newcommand{\token}{\ensuremath{\s{Token}}}
\newcommand{\currency}{\ensuremath{\s{CurrencyId}}}
\newcommand{\nativeCur}{\ensuremath{\mathrm{nativeC}}}
\newcommand{\nativeTok}{\ensuremath{\mathrm{nativeT}}}
\newcommand{\injectNative}{\ensuremath{\mathrm{inject}}}

\newcommand{\qtymap}{\ensuremath{\s{Quantities}}}

\newcommand\B{\ensuremath{\mathbb{B}}}
\newcommand\N{\ensuremath{\mathbb{N}}}
\newcommand\Z{\ensuremath{\mathbb{Z}}}
\renewcommand\H{\ensuremath{\mathbb{H}}}
%% \H is usually the Hungarian double acute accent
\newcommand{\emptyBs}{\ensuremath{\emptyset}}

\newcommand{\emptymap}{\ensuremath{\{\}}}

% multisig
\newcommand{\msc}{\mathrm{msc}}
\newcommand{\sig}{\mathit{sig}}
\newcommand{\sigs}{\mathit{sigs}}
\newcommand{\auth}{\mathrm{auth}}
\newcommand{\Holding}{\msf{Holding}}
\newcommand{\Collecting}[2]{\msf{Collecting}(#1, #2)}
\newcommand{\Propose}[1]{\msf{Propose}(#1)}
\newcommand{\Add}[1]{\msf{Add}(#1)}
\newcommand{\Cancel}{\msf{Cancel}}
\newcommand{\Pay}{\msf{Pay}}

% Agda code commit hash
\newcommand\AgdaCommit{a1574e6}

% For anonymisation
\newtoggle{anonymous}
\togglefalse{anonymous}
\iftoggle{anonymous}{
\newcommand{\Cardano}{CHAIN}
\newcommand{\Plutus}{LANG}
\newcommand{\GitUser}{anonymous-agda}
}{
\newcommand{\Cardano}{Cardano}
\newcommand{\Plutus}{Plutus Core}
\newcommand{\GitUser}{omelkonian}
}
\newcommand{\anonymize}[1]{\iftoggle{anonymous}{}{#1}}

% Names, for consistency
\newcommand{\UTXO}{UTXO}
\newcommand{\EUTXO}{E\UTXO{}}
\newcommand{\ExUTXO}{Extended \UTXO{}}
\newcommand{\CEM}{CEM}

% ------

\newcommand\isFinal{\msf{isFinal}}
\newcommand\step{\msf{step}}
\newcommand\satisfies{\msf{satisfies}}
\newcommand\checkOutputs{\msf{checkOutputs}}

\newcommand\mkValidator[1]{\msf{validator}_#1}
\newcommand\Sim[2]{\ensuremath{
#1 \sim #2
}}
\newcommand\CStep[3]{\ensuremath{
#1 \xrightarrow{\hspace{5pt} #2 \hspace{5pt}} (#1' , #3)
}}
\newcommand\LStep[2]{\ensuremath{
#1 \xrightarrow{\hspace{5pt} #2 \hspace{5pt}} #1'
}}
\newcommand\txeq{tx^\equiv}

%% ------------- Start of document ------------- %%

\sloppy
\begin{document}

\title{The \ExUTXO{} Model}

% \author{Anonymous\inst{1} \and
% Anonymous\inst{2,3}}

% First names are abbreviated in the running head.
% If there are more than two authors, 'et al.' is used.

\iftoggle{anonymous}{
\author{Anonymous}
\authorrunning{Anonymous et al.}
}{
% author order: alphabetical
\author{
  Manuel M.T. Chakravarty\inst{1}
  \and
  James Chapman\inst{1}
  \and
  Kenneth MacKenzie\inst{1}
  \and
  Orestis Melkonian\inst{1,2}
  \and
  Michael Peyton Jones\inst{1}
  \and
  Philip Wadler\inst{2}
}

\authorrunning{Chakravarty et al.}

\institute{
  IOHK,
  \email{\{manuel.chakravarty, james.chapman, kenneth.mackenzie, orestis.melkonian, michael.peyton-jones\}@iohk.io}
  \and
  University of Edinburgh, \email{orestis.melkonian@ed.ac.uk}, \email{wadler@inf.ed.ac.uk}
}
}

\maketitle

\begin{abstract}
  Bitcoin and Ethereum, hosting the two currently most valuable and
  popular cryptocurrencies, use two rather different ledger models,
  known as the \emph{\UTXO{} model} and the \emph{account model},
  respectively. At the same time, these two public blockchains differ
  strongly in the expressiveness of the smart contracts that they
  support. This is no coincidence. Ethereum chose the account model
  explicitly to facilitate more expressive smart contracts. On the
  other hand, Bitcoin chose \UTXO{} also for good reasons, including
  that its semantic model stays simple in a complex concurrent and
  distributed computing environment. This raises the question of
  whether it is possible to have expressive smart contracts, while
  keeping the semantic simplicity of the \UTXO{} model.

  In this paper, we answer this question affirmatively. We present
  \emph{\ExUTXO{}} (\emph{\EUTXO{}}), an extension to Bitcoin's \UTXO{} model
  that supports a substantially more
  expressive form of \emph{validation scripts}, including scripts that
  implement general state machines and enforce invariants across
  entire transaction chains.

  To demonstrate the power of this model, we also introduce a form of state
  machines suitable for execution on a ledger, based on Mealy machines
  and called Constraint Emitting Machines (\CEM{}).  We formalise \CEM{}s,
  show how to compile them to \EUTXO{}, and show a \emph{weak bisimulation}
  between the two systems. All of our work is formalised using the Agda
  proof assistant.
\end{abstract}

\keywords{blockchain \and \UTXO{} \and functional programming \and state machines.}

\section{The Plutus Platform}

The \gls{plutus-platform} is a platform for writing \emph{applications} that interact with a \emph{distributed ledger} featuring \emph{scripting capabilities}, in particular the \gls{cardano} blockchain.
A high-level architecture of the \gls{plutus-platform} on \gls{cardano} is shown in \cref{fig:platform-architecture}, with an emphasis on applications.

\paragraph{Ledgers.}
The \gls{plutus-platform} is designed to work with distributed ledgers (henceforth simply ``ledgers'').\footnote{
More specifically, it is designed to work with the \gls{cardano} blockchain, although it could function on other systems that implement the requisite ledger functionality.
}
While the design of the Platform supports writing very simple applications that do nothing but occasionally submit asset-transfer transactions to the ledger, the main focus is on the applications which are enabled by a ledger with scripting functionality.
Specifically, we look at \gls{utxo} ledgers which implement what we call the \gls{eutxo-model}.

The ledger model is described in \cref{sec:ledger}.

\paragraph{Scripting.}
Ledgers without scripting functionality can only support very simple applications.
Much of the interest in applications that interact with blockchains has come from the ability to put some part of the application code \emph{on} the chain itself (such code is generally referred to as a ``\gls{script}''), such that it is guaranteed to be executed correctly as part of the consensus protocol of the blockchain.
This enables applications to have small kernels of ``trusted'' code which ensures that critical safety conditions are met.

The \gls{plutus-platform} requires a particular scripting model from the ledger, which is described with the general ledger model in \cref{sec:ledger}.
This scripting model is agnostic about the scripting \emph{language} which is used for \glspl{script}, but the \gls{plutus-sdk} is designed to work with a particular scripting language, namely \gls{plutus-core}. \Gls{plutus-core} is described in \cref{sec:plutus-core}.

\paragraph{Applications.}
In the \gls{plutus-platform} a \gls{app} is simply a program that interacts with a distributed ledger.
We use the rather generic word ``application'' over the more usual ``smart contract'' or ``distributed application'' to emphasize the generic nature of the programs we are considering --- indeed, they need not scripting functionality or have distributed participants!

The point of the Platform is to enable such applications by providing support in the ledger, as well as support for authoring, distributing, and running them.
While the modifications to the ledger may seem like the most technically substantive part of the work, most of the application behaviour is in the part that runs off the chain, and this requires a commensurate amount of support.

The \gls{app} model is described in \cref{sec:paf}, and our support for authoring \glspl{app} in \cref{sec:sdk}.

\begin{figure}[t]
  \centering
  \includegraphics[width=\textwidth]{platform-architecture.png}
  \caption{Architecture of the \gls{plutus-platform}}
  \label{fig:platform-architecture}
\end{figure}

\section{Extending \UTXO}
\label{sec:informal-eutxo}

Various forms of state machines have been proposed to characterise smart contract functionality that goes beyond what is possible with the basic \UTXO{} model --- see, for example, \cite{fsolidm,scilla} using Ethereum's account-based model. However, we might wonder whether we can extend the basic \UTXO{} model in such a way as to support more expressive state machines without switching to an account-based model. 

Given that we can regard the individual transactions in a continuous chain of transactions as individual steps in the evolution of a state machine, we require two pieces of additional functionality from the \UTXO{} model: 
\begin{inparaenum}[(a)]
\item we need to be able to maintain the machine state, and 
\item we need to be able to enforce that the same contract code is used along the entire sequence of transactions --- we call this \emph{contract continuity}.
\end{inparaenum}

To maintain the machine state, we extend \UTXO{} outputs from being a
pair of a validator $\nu$ and a cryptocurrency value $\val$ to being a
triple \((\nu, \val, \delta)\) of validator, value, and a
\textit{datum} $\delta$, where $\delta$ contains arbitrary
contract-specific data. Furthermore, to enable validators to enforce
contract continuity, we pass the entirety of the transaction that
attempts to spend the output locked by a validator to the validator
invocation. Thus a validator can inspect the transaction that
attempts to spend its output and, in particular, it can ensure that the
contract output of that transaction uses validator code belonging to
the same contract --- often, this will be the same validator. Overall,
to check that an input with redeemer $\rho$ that is part of the
transaction $\mi{tx}$ is entitled to spend an output \((\nu, \val,
\delta)\), we check that \(\nu(\val, \delta, \rho, \mi{tx}) = \true\).

As we are allowing arbitrary data in $\delta$ and we enable the validator
$\nu$ to impose arbitrary validity constraints on the consuming
transaction $\mi{tx}$, the resulting \ExUTXO{} (\EUTXO{}) model goes
beyond enabling state machines. However, in this paper we restrict
ourselves to the implementation of state machines and leave the
investigation of further-reaching computational patterns to future
work.

\begin{figure}[t]
  \centering 
  \includegraphics[width=\textwidth]{EUTxO_MultiSig_States.pdf}
  \caption{Transition diagram for the multi-signature state machine; edges labelled with input from redeemer and transition constraints.}
  \label{fig:multisig-machine}
\end{figure}
%
As a simple example of a state machine contract consider an $n$--of--$m$
multi-signature contract. Specifically, we have a given amount
$\val_\msc$ of some cryptocurrency and we require the approval of at
least $n$ out of an a priori fixed set of $m \geq n$ owners to spend
$\val_\msc$. With plain \UTXO{} (e.g., on Bitcoin), a multi-signature
scheme requires out-of-band (off-chain) communication to collect all
$n$ signatures to spend $\val_\msc$. On Ethereum, and also in the
\EUTXO{} model, we can collect the signatures on-chain, without any
out-of-band communication. To do so, we use a state machine operating
according to the transition diagram in
Figure~\ref{fig:multisig-machine}, where we assume that the threshold
$n$ and authorised signatures $\sigs_\auth$ with \(|\sigs_\auth| = m\)
are baked into the contract code.

In its implementation in the \EUTXO{} model, we use a validator
function $\nu_\msc$ accompanied by the datum $\delta_\msc$ to lock
$\val_\msc$. The datum $\delta_\msc$ stores the machine state,
which is of the form \(\Holding\) when only holding the locked value
or \(\Collecting{(\val, \kappa, d)}{\sigs}\) when collecting
signatures $\sigs$ for a payment of $\val$ to $\kappa$ by the deadline
$d$. The initial output for the contract is \((\nu_\msc, \val_\msc,
\Holding)\).

The validator $\nu_\msc$ implements the state transition diagram from
Figure~\ref{fig:multisig-machine} by using the redeemer of the spending input to determine the transition that needs to be taken. That redeemer (state machine input) can take four forms: 
\begin{inparaenum}[(1)]
\item \(\Propose{\val, \kappa, d}\) to propose a payment of $\val$ to $\kappa$
  by the deadline $d$, 
\item \(\Add{\sig}\) to add a signature $\sig$ to a payment, 
\item $\Cancel$ to cancel a proposal after its deadline expired, and 
\item $\Pay$ to make a payment once all required signatures have been collected. 
\end{inparaenum}
It then validates that the spending transaction $\mi{tx}$ is a valid
representation of the newly reached machine state. This implies that
$\mi{tx}$ needs to keep $\val_\msc$ locked by $\nu_\msc$ and that the
state in the datum $\delta^{\prime}_\msc$ needs to be the successor state
of $\delta_\msc$ according to the transition diagram.

The increased expressiveness of the \EUTXO{} model goes far beyond
simple contracts such as this on-chain multi-signature contract. For
example, the complete functionality of the Marlowe domain-specific
language for financial contracts~\cite{marlowe} has been successfully
implemented as a state machine on the \EUTXO{} model.

\section{Formal model}
\label{sec:formal-model}

\subsection{Basic types and notation}
\label{sec:basic-notation}
Figure~\ref{fig:basic-types} defines some basic types and
notation used in the rest of the paper; we have generally followed the
notation established by Zahnentferner in~\cite{Zahnentferner18-UTxO}.

\begin{ruledfigure}{!ht}
  \begin{displaymath}
    \begin{array}{rll}
      \B{} && \mbox{the type of booleans}\\
      \N{} && \mbox {the type of natural numbers}\\
      \Z{} && \mbox {the type of integers}\\
      (\phi_1 : T_1, \ldots, \phi_n : T_n) && \mbox{a record type with fields $\phi_1, \ldots, \phi_n$ of types $T_1, \ldots, T_n$}\\
      t.\phi && \mbox{the value of $\phi$ for $t$, where $t$ has type $T$ and
                $\phi$ is a field of $T$}\\
      \Set{T} && \mbox{the type of (finite) sets over $T$}\\
      \List{T} && \mbox{the type of lists over $T$, with $\_[\_]$ as indexing
        and $|\_|$ as length}\\
      h::t && \mbox{the list with head $h$ and tail $t$}\\
      x \mapsto f(x) && \mbox{an anonymous function}\\
      \hash{c} && \mbox{a cryptographic collision-resistant hash of $c$}\\
      \Interval{A} && \mbox{the type of intervals over a totally-ordered set $A$}
    \end{array}
  \end{displaymath}
  \caption{Basic types and notation}
  \label{fig:basic-types}
\end{ruledfigure}

\subsubsection{The \Data{} type.}
\label{sec:data}
We will make particular use of a primitive type \Data{} which can be
used to pass information into scripts. This is intended to be any
relatively standard structured data format, for example JSON or
CBOR~\cite{cbor}.

The specific choice of type does not matter for this paper, so we have left it
abstract. The intention is that this should
be well supported by all the programming languages we are interested in,
including whatever language is used for scripts, and whatever languages
are used for off-chain code that interacts with the chain.

We assume that for every (non-function) type $T$ in the scripting
language we have corresponding \toData{} and \fromData{} functions.

\subsection{\EUTXO{}: Enhanced scripting}
\label{sec:eutxo}
Our first change to the standard UTXO model is that as well as the
validator we allow transaction outputs to carry a piece of data called
the \emph{datum} (or \emph{datum object}), which is passed in as an
additional argument during validation.  This allows a contract to
carry some state (the datum) without changing its ``code'' (the
validator). We will use this to carry the state of our state machines
(see Section~\ref{sec:informal-eutxo}).

The second change is that the validator receives some information
about the transaction that is being validated. This information, which
we call the \textit{context}, is passed in as an additional
argument of type \ctx{}. The information supplied in the
context enables the validator to enforce much stronger conditions than
is possible with a bare \UTXO{} model --- in particular, it can
inspect the \emph{outputs} of the current transaction, which is
essential for ensuring contract continuity (see
Section~\ref{sec:informal-eutxo}).

The third change is that we provide some access to time by adding a
\emph{validity interval} to transactions.
This is an interval of ticks (see Subsection~\ref{para:ticks})
during which a transaction can be processed (a generalisation of a ``time-to-live'').
Thus, any scripts which run during validation can assume that the current tick
is within that interval, but do not know the precise value of the current tick.

Finally, we represent all the arguments to the validator (redeemer, datum,
\ctx) as values of type \Data{}. Clients are therefore responsible for encoding
whatever types they would like to use into \Data{} (and decoding them inside the
validator script).

\subsection{A Formal Description of the \EUTXO{} Model}
\label{section:eutxo-spec}

In this section we give a formal description of the \EUTXO{} model.  The
description is given in a straightforward set-theoretic form, which
\begin{inparaenum}[(1)]
\item admits an almost direct translation into languages like Haskell for implementation, and
\item is easily amenable to mechanical formalisation.
\end{inparaenum}
We will make use of this in Section~\ref{sec:expressiveness}.

The definitions in this section are essentially the definitions of
\UTXO{}-based cryptocurrencies with scripts from
Zahnentferner~\cite{Zahnentferner18-UTxO}, except that we have made the changes
described above.

Figure~\ref{fig:eutxo-types} lists the types and operations used in the
the basic \EUTXO{} model. Some of these are defined here, the others must be provided by
the ledger (``ledger primitives'').
%%
\begin{ruledfigure}{!ht}
  \begin{displaymath}
  \begin{array}{rll}
    \multicolumn{3}{l}{\textsc{Ledger primitives}}\\
    \qty{} && \mbox{an amount of currency}\\
    \tick && \mbox{a tick}\\
    \Address && \mbox{an ``address'' in the blockchain}\\
    \Data && \mbox{a type of structured data}\\
    \DataHash && \mbox{the hash of a value of type \Data{}}\\
    \TxId && \mbox{the identifier of a transaction}\\
    \txId : \eutxotx \rightarrow \TxId && \mbox{a function computing the identifier of a transaction}\\
    \script && \mbox{the (opaque) type of scripts}\\
    \scriptAddr : \script \rightarrow \Address && \mbox{the address of a script}\\
    \hashData : \Data \rightarrow \DataHash && \mbox{the hash of an object of type\Data}\\
    \llbracket \_ \rrbracket : \script \rightarrow \Data \times \Data \times
    \Data \rightarrow \B && \mbox{applying a script to its arguments}\\
    \\
    \multicolumn{3}{l}{\textsc{Defined types}}\\
    \s{Output } &=&(\val: \qty,\\
                & &\ \addr: \Address,\\
                & &\ \datumHsh: \DataHash)\\
    \\
    \s{OutputRef } &=&(\txrefid: \TxId, \idx: \N)\\
    \\
    \s{Input } &=&(\outputref: \s{OutputRef},\\
               & &\ \validator: \script,\\
               & &\ \datum: \Data,\\
               & &\ \redeemer: \Data)\\
     \\
     \eutxotx\s{ } &=&(\inputs: \Set{\s{Input}},\\
                   & &\ \outputs: \List{\s{Output}},\\
                   & &\ \validityInterval: \Interval{\tick})\\
     \\
     \s{Ledger } &=&\!\List{\eutxotx}\\
  \end{array}
  \end{displaymath}
  \caption{Primitives and types for the \EUTXO{} model}
  \label{fig:eutxo-types}
\end{ruledfigure}

\paragraph{Addresses.}
We follow Bitcoin in referring to the targets of transaction outputs as
``addresses''. In this system, they refer only to \emph{script} addresses
(likely a hash of the script), but in a full system they would likely include
public-key addresses, and so on.

\paragraph{Ticks.}
\label{para:ticks}
A tick is a monotonically increasing unit of progress in the
ledger system. This corresponds to the ``block number''
or ``block height'' in most blockchain systems. We assume that there is some
notion of a ``current tick'' for a given ledger.

\paragraph{Inputs and outputs.} Transactions have a
\textsf{Set} of inputs but a \textsf{List} of outputs. There
are two reasons that we do not also have a \textsf{Set} of outputs although they
are conceptually symmetrical:
\begin{itemize}
\item We need a way to uniquely identify a transaction output, so
  that it can be referred to by a transaction input that spends it. The pair of
  a transaction id and an output index is sufficient for this, but other schemes
  are conceivable.
\item A \textsf{Set} requires a notion of equality. If we use the
  obvious structural equality on outputs, then if we had two outputs
  paying $X$ to address $A$, they would be equal. We need to
  distinguish these --- outputs must have an identity beyond
  just their address and value.
\end{itemize}

\paragraph{The location of validators and datum objects.} Validator scripts
and full datum objects are provided as parts of transaction \emph{inputs},
even though they are conceptually part of the output being spent. The
output instead specifies them by providing the corresponding address
or hash.\footnote{That these match up is enforced by
  Rules~\ref{rule:validator-scripts-hash} and
  \ref{rule:datums-hash} in Figure~\ref{fig:eutxo-validity}.}

This strategy reduces memory requirements, since
the \UTXO{} set must be kept in memory for rapid access while validating
transactions. Hence it is desirable to keep outputs small --- in
our system they are constant size.
Providing the much larger validator script only at the point where it is needed
is thus a helpful saving. The same considerations apply to datum objects.

An important question is how the person who spends an output \emph{knows} which
validator and datum to provide in order to match the hashes on the output.
This can always be accomplished via some off-chain mechanism, but we may
want to include some on-chain way of accomplishing this.\footnote{\Cardano{} will provide
a mechanism in this vein.} However, this is not directly relevant to this paper,
so we have omitted it.

\paragraph{Fees, forge, and additional metadata.}  Transactions will typically
have additional metadata, such as transaction fees or a ``forge''
field that allows value to be created or destroyed. These are
irrelevant to this paper, so have been omitted.\footnote{ Adding such
  fields might require amending Rule~\ref{rule:value-is-preserved}
  to ensure value preservation.  }

\paragraph{Ledger structure.} We model a ledger as a simple
list of transactions: a real blockchain ledger will be more complex
than this, but the only property that we really require is that
transactions in the ledger have some kind of address which allows them
to be uniquely identified and retrieved.

\subsection{The \ctx{} type}
\label{sec:validation-context}
Recall from the introduction to Section~\ref{sec:eutxo} that when a
transaction input is being validated, the validator script is supplied
with an object of type \ctx{} (encoded as \Data{}) which contains
information about the current transaction.  The \ctx{} type is defined
in Figure~\ref{fig:ptx-1-types}, along with some related types.

\begin{ruledfigure}{!ht}
  \begin{displaymath}
  \begin{array}{rll}
    \s{OutputInfo } &=&(\val: \qty,\\
                    & &\ \i{validatorHash}: \Address,\\
                    & &\ \datumHsh: \DataHash)\\
    \\
    \s{InputInfo } &=&(\outputref: \s{OutputRef},\\
                   & &\ \i{validatorHash}: \Address,\\
                   & &\ \i{datum}: \Data,\\
                   & &\ \i{redeemer}: \Data,\\
                   & &\ \val: \qty)\\
     \\
     \ctx\s{ } &=&(\i{inputInfo}: \Set{\s{InputInfo}},\\
               & &\ \i{outputInfo}: \List{\s{OutputInfo}},\\
               & &\ \i{validityInterval}: \Interval{\tick},\\
               & &\ \i{thisInput}: \N)\\
     \\
  \end{array}
  \end{displaymath}
  \caption{The \ctx{} type for the \EUTXO{} model}
  \label{fig:ptx-1-types}
\end{ruledfigure}

\paragraph{The contents of \ctx{}.}
The \ctx{} type is a summary of the information contained in the $\eutxotx$ type in
Figure~\ref{fig:eutxo-types}, situated in the context of a validating
transaction, and made suitable for consumption in a script. That results in the following changes:
\begin{enumerate}
\item The \s{InputInfo} type is augmented with information that comes
  from the output being spent, specifically the value attached to that output.
\item The \ctx{} type includes an index that indicates the input currently
  being validated. This allows scripts to identify their own address, for example.
\item Validators are included as their addresses, rather than as scripts. This
  allows easy equality comparisons without requiring script languages to be able
  to represent their own programs.
\end{enumerate}
\noindent We assume that there is a function $\toCtx: \eutxotx \times
  \s{Input} \times \s{Ledger} \rightarrow \ctx$ which summarises a
transaction in the context of an input and a ledger state.
%% kwxm: moved this out of the figure because adding the Ledger
%% parameter pushed everything too far to the right.

\paragraph{Determinism.}
The information provided by \ctx{} is entirely determined by the
transaction itself. This means that script execution during validation
is entirely deterministic, and can be simulated accurately by the user
\emph{before} submitting a transaction: thus both the outcome of
script execution and the amount of resources consumed can be
determined ahead of time. This is helpful for systems that charge for
script execution, since users can reliably compute how much they will
need to pay ahead of time.

A common way for systems to violate this property is by providing
access to some piece of mutable information, such as the current time
(in our system, the current tick has this role). Scripts can then
branch on this information, leading to non-deterministic behaviour. We
sidestep this issue with the validation interval mechanism (see the
introduction to Section~\ref{sec:eutxo}).

\begin{ruledfigure}{!ht}
  \begin{displaymath}
  \begin{array}{lll}
  \multicolumn{3}{l}{\lookupTx : \s{Ledger} \times \TxId \rightarrow \eutxotx{}}\\
  \lookupTx(l,id) &=& \textrm{the unique transaction in $l$ whose id is $id$}\\
  \\
  \multicolumn{3}{l}{\txunspent : \eutxotx \rightarrow \Set{\s{OutputRef}}}\\
  \txunspent(t) &=& \{(\txId(t),1), \ldots, (\txId(id),\left|t.outputs\right|)\}\\
  \\
  \multicolumn{3}{l}{\unspent : \s{Ledger} \rightarrow \Set{\s{OutputRef}}}\\
  \unspent([]) &=& \emptymap \\
  \unspent(t::l) &=& (\unspent(l) \setminus t.\inputs) \cup \txunspent(t)\\
  \\
  \multicolumn{3}{l}{\getSpent : \s{Input} \times \s{Ledger} \rightarrow \s{Output}}\\
  \getSpent(i,l) &=& \lookupTx(l, i.\outputref.\id).\outputs[i.\outputref.\idx]
  \end{array}
  \end{displaymath}
  \caption{Auxiliary functions for \EUTXO{} validation}
  \label{fig:validation-functions-1}
\end{ruledfigure}


\subsection{Validity of \EUTXO{} transactions}
\label{sec:eutxo-validity}

Figure~\ref{fig:eutxo-validity} defines what it means for a
transaction $t$ to be valid for a valid ledger $l$ during the tick
\currentTick, using some auxiliary functions from
Figure~\ref{fig:validation-functions-1}.  Our definition combines
Definitions 6 and 14 from Zahnentferner~\cite{Zahnentferner18-UTxO},
differing from the latter in Rule~\ref{rule:all-inputs-validate}. A
ledger $l$ is \textit{valid} if either $l$ is empty or $l$ is of the
form $t::l^{\prime}$ with $l^{\prime}$ valid and $t$ valid for
$l^{\prime}$.
%%
\vspace{-2mm}
%%
\begin{ruledfigure}{H}
\begin{enumerate}

\item
  \label{rule:tick-in-range}
  \textbf{The current tick is within the validity interval}
  \begin{displaymath}
    \currentTick \in t.\validityInterval
  \end{displaymath}

\item
  \label{rule:all-outputs-are-non-negative}
  \textbf{All outputs have non-negative values}
  \begin{displaymath}
    \textrm{For all } o \in t.\outputs,\ o.\val \geq 0
  \end{displaymath}

\item
  \label{rule:all-inputs-refer-to-unspent-outputs}
  \textbf{All inputs refer to unspent outputs}
  \begin{displaymath}
    \{i.\outputref : i \in t.\inputs \} \subseteq \unspent(l).
  \end{displaymath}

\item
  \label{rule:value-is-preserved}
  \textbf{Value is preserved}
  \begin{displaymath}
    \textrm{Unless $l$ is empty, } \sum_{i \in t.\inputs} \getSpent(i, l).\val = \sum_{o \in t.\outputs} o.\val
  \end{displaymath}

\item
  \label{rule:no-double-spending}
  \textbf{No output is double spent}
  \begin{displaymath}
    \textrm{If } i_1, i_2 \in t.\inputs \textrm{ and }  i_1.\outputref = i_2.\outputref
    \textrm{ then } i_1 = i_2.
  \end{displaymath}

\item
  \label{rule:all-inputs-validate}
  \textbf{All inputs validate}
  \begin{displaymath}
    \textrm{For all } i \in t.\inputs,\ \llbracket
    i.\validator\rrbracket (i.\datum,\, i.\redeemer,\,  \toData(\toCtx(t,i,l))) = \true.
  \end{displaymath}

\item
  \label{rule:validator-scripts-hash}
  \textbf{Validator scripts match output addresses}
  \begin{displaymath}
    \textrm{For all } i \in t.\inputs,\ \scriptAddr(i.\validator) = \getSpent(i, l).\addr
  \end{displaymath}

\item
  \label{rule:datums-hash}
  \textbf{Each datum matches its output hash}
  \begin{displaymath}
    \textrm{For all } i \in t.\inputs,\ \hashData(i.\datum) = \getSpent(i, l).\datumHsh
  \end{displaymath}

\end{enumerate}
\caption{Validity of a transaction $t$ in the \EUTXO{} model}
\label{fig:eutxo-validity}
\end{ruledfigure}
%%
%% kwxm: some dubious fiddling with vertical space to get the validity
%% figure in the right place and still stay within the page limit
%%
\vspace{-8mm}
\paragraph{Creating value.}
Most blockchain systems have special rules for creating or destroying value.
These are usually fairly idiosyncratic, and are not relevant to this paper, so
we have provided a simple genesis condition in
Rule~\ref{rule:value-is-preserved} which allows the initial transaction in the ledger
to create value.
\vspace{-1mm}
\paragraph{Lookup failures.}
The function $\getSpent$ calls $\lookupTx$, which looks up the unique
transaction in the ledger with a particular id and can of course
fail. However Rule~\ref{rule:all-inputs-refer-to-unspent-outputs}
ensures that during validation all of the transaction inputs refer to
existing unspent outputs, and in these circumstances $\lookupTx$ will
always succeed for the transactions of interest.

\section{Formally Verified Behaviours}
To follow.

\section{Related work}
\label{sec:related}

Bitcoin Covenants~\cite{moser2016bitcoin} allow Bitcoin transactions
to restrict how the transferred value can be used in the future,
including propagating themselves to ongoing outputs. This provides
contract continuity and allows the implementation of simple state
machines. Our work is inspired by Covenants, although our addition of
a datum is novel and simplifies the state passing.

The Bitcoin Modelling Language (BitML)~\cite{bitml} is an idealistic process calculus
that specifically targets smart contracts running on Bitcoin.
The semantics of BitML contracts essentially comprise a (labelled) \textit{transition system}, aka a state machine.
Nonetheless, due to the constrained nature of the plain \UTXO{} model without any extensions,
the construction is far from straightforward and requires quite a bit of off-chain communication to set everything up.
Most importantly, the BitML compilation scheme only concerns a restricted form of state machines,
while ours deals with a more generic form that admits any user-defined type of states and inputs.
BitML builds upon an abstract model of Bitcoin transactions by the same
authors~\cite{formal-model-of-bitcoin-transactions};
one of our main contributions is an extended version of such an abstract model,
which also accounts for the added functionality apparent in \Cardano{}.

Ethereum and its smart contract language, Solidity~\cite{Solidity}, are powerful
enough to implement state machines, due to their native support for
global contract instances and state. However, this approach has some major downsides,
notably that contract state is global, and must be kept indefinitely by all core nodes.
In the \EUTXO{} model, contract state is localised to where it is used, and
it is the responsibility of clients to manage it.

Scilla~\cite{scilla} is a intermediate-level language for writing smart
contracts as state machines. It compiles to Solidity and is amendable to formal verification.
Since Scilla supports the asynchronous messaging capabilities of Ethereum,
Scilla contracts correspond to a richer class of automata, called
\textit{Communicating State Transition Systems}~\cite{csta}.
In the future, we plan to formally compare this class of state machines with our own class of \CEM{}s,
which would also pave the way to a systematic comparison of Ethereum's account-based model against \Cardano{}'s \UTXO{}-based one.

Finally, there has been an attempt to model Bitcoin contracts using \textit{timed automata}~\cite{timed-btc},
which enables semi-automatic verification using the UPPAAL model checker~\cite{uppaal}.
While this provides a pragmatic way to verify temporal properties of concrete smart contracts,
there is no formal claim that this class of automata actually corresponds to the semantics of Bitcoin smart contracts.
In contrast, our bisimulation proof achieves the bridging of this semantic gap.


\bibliographystyle{splncs04}
\bibliography{eutxo}

\end{document}
