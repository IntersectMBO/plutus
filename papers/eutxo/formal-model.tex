\section{Formal model}
\label{sec:formal-model}

\subsection{Basic types and notation}
\label{sec:basic-notation}
Figure~\ref{fig:basic-types} defines some basic types and
notation used in the rest of the paper; we have generally followed the
notation established by Zahnentferner in~\cite{Zahnentferner18-UTxO}.

\begin{ruledfigure}{!ht}
  \begin{displaymath}
    \begin{array}{rll}
      \B{} && \mbox{the type of booleans}\\
      \N{} && \mbox {the type of natural numbers}\\
      \Z{} && \mbox {the type of integers}\\
      (\phi_1 : T_1, \ldots, \phi_n : T_n) && \mbox{a record type with fields $\phi_1, \ldots, \phi_n$ of types $T_1, \ldots, T_n$}\\
      t.\phi && \mbox{the value of $\phi$ for $t$, where $t$ has type $T$ and
                $\phi$ is a field of $T$}\\
      \Set{T} && \mbox{the type of (finite) sets over $T$}\\
      \List{T} && \mbox{the type of lists over $T$, with $\_[\_]$ as indexing
        and $|\_|$ as length}\\
      h::t && \mbox{the list with head $h$ and tail $t$}\\
      x \mapsto f(x) && \mbox{an anonymous function}\\
      \hash{c} && \mbox{a cryptographic collision-resistant hash of $c$}\\
      \Interval{A} && \mbox{the type of intervals over a totally-ordered set $A$}
    \end{array}
  \end{displaymath}
  \caption{Basic types and notation}
  \label{fig:basic-types}
\end{ruledfigure}

\subsubsection{The \Data{} type.}
\label{sec:data}
We will make particular use of a primitive type \Data{} which can be
used to pass information into scripts. This is intended to be any
relatively standard structured data format, for example JSON or
CBOR~\cite{cbor}.

The specific choice of type does not matter for this paper, so we have left it
abstract. The intention is that this should
be well supported by all the programming languages we are interested in,
including whatever language is used for scripts, and whatever languages
are used for off-chain code that interacts with the chain.

We assume that for every (non-function) type $T$ in the scripting
language we have corresponding \toData{} and \fromData{} functions.

\subsection{\EUTXO{}: Enhanced scripting}
\label{sec:eutxo}
Our first change to the standard UTXO model is that as well as the
validator we allow transaction outputs to carry a piece of data called
the \emph{datum} (or \emph{datum object}), which is passed in as an
additional argument during validation.  This allows a contract to
carry some state (the datum) without changing its ``code'' (the
validator). We will use this to carry the state of our state machines
(see Section~\ref{sec:informal-eutxo}).

The second change is that the validator receives some information
about the transaction that is being validated. This information, which
we call the \textit{context}, is passed in as an additional
argument of type \ctx{}. The information supplied in the 
context enables the validator to enforce much stronger conditions than
is possible with a bare \UTXO{} model --- in particular, it can
inspect the \emph{outputs} of the current transaction, which is
essential for ensuring contract continuity (see
Section~\ref{sec:informal-eutxo}).

The third change is that we provide some access to time by adding a
\emph{validity interval} to transactions.
This is an interval of ticks (see Subsection~\ref{para:ticks})
during which a transaction can be processed (a generalisation of a ``time-to-live'').
Thus, any scripts which run during validation can assume that the current tick
is within that interval, but do not know the precise value of the current tick.

Finally, we represent all the arguments to the validator (redeemer, datum,
\ctx) as values of type \Data{}. Clients are therefore responsible for encoding
whatever types they would like to use into \Data{} (and decoding them inside the
validator script).

\subsection{A Formal Description of the \EUTXO{} Model}
\label{section:eutxo-spec}

In this section we give a formal description of the \EUTXO{} model.  The
description is given in a straightforward set-theoretic form, which
\begin{inparaenum}[(1)]
\item admits an almost direct translation into languages like Haskell for implementation, and
\item is easily amenable to mechanical formalisation.
\end{inparaenum}
We will make use of this in Section~\ref{sec:expressiveness}.

The definitions in this section are essentially the definitions of
\UTXO{}-based cryptocurrencies with scripts from
Zahnentferner~\cite{Zahnentferner18-UTxO}, except that we have made the changes
described above.

Figure~\ref{fig:eutxo-types} lists the types and operations used in the
the basic \EUTXO{} model. Some of these are defined here, the others must be provided by
the ledger (``ledger primitives'').
%%
\begin{ruledfigure}{!ht}
  \begin{displaymath}
  \begin{array}{rll}
    \multicolumn{3}{l}{\textsc{Ledger primitives}}\\
    \qty{} && \mbox{an amount of currency}\\
    \tick && \mbox{a tick}\\
    \Address && \mbox{an ``address'' in the blockchain}\\
    \Data && \mbox{a type of structured data}\\
    \DataHash && \mbox{the hash of a value of type \Data{}}\\
    \TxId && \mbox{the identifier of a transaction}\\
    \txId : \eutxotx \rightarrow \TxId && \mbox{a function computing the identifier of a transaction}\\
    \script && \mbox{the (opaque) type of scripts}\\
    \scriptAddr : \script \rightarrow \Address && \mbox{the address of a script}\\
    \hashData : \Data \rightarrow \DataHash && \mbox{the hash of an object of type\Data}\\
    \llbracket \_ \rrbracket : \script \rightarrow \Data \times \Data \times
    \Data \rightarrow \B && \mbox{applying a script to its arguments}\\
    \\
    \multicolumn{3}{l}{\textsc{Defined types}}\\
    \s{Output } &=&(\val: \qty,\\
                & &\ \addr: \Address,\\
                & &\ \datumHsh: \DataHash)\\
    \\
    \s{OutputRef } &=&(\txrefid: \TxId, \idx: \N)\\
    \\
    \s{Input } &=&(\outputref: \s{OutputRef},\\
               & &\ \validator: \script,\\
               & &\ \datum: \Data,\\
               & &\ \redeemer: \Data)\\
     \\
     \eutxotx\s{ } &=&(\inputs: \Set{\s{Input}},\\
                   & &\ \outputs: \List{\s{Output}},\\
                   & &\ \validityInterval: \Interval{\tick})\\
     \\
     \s{Ledger } &=&\!\List{\eutxotx}\\
  \end{array}
  \end{displaymath}
  \caption{Primitives and types for the \EUTXO{} model}
  \label{fig:eutxo-types}
\end{ruledfigure}

\paragraph{Addresses.}
We follow Bitcoin in referring to the targets of transaction outputs as
``addresses''. In this system, they refer only to \emph{script} addresses
(likely a hash of the script), but in a full system they would likely include
public-key addresses, and so on.

\paragraph{Ticks.}
\label{para:ticks}
A tick is a monotonically increasing unit of progress in the
ledger system. This corresponds to the ``block number''
or ``block height'' in most blockchain systems. We assume that there is some
notion of a ``current tick'' for a given ledger.

\paragraph{Inputs and outputs.} Transactions have a
\textsf{Set} of inputs but a \textsf{List} of outputs. There
are two reasons that we do not also have a \textsf{Set} of outputs although they
are conceptually symmetrical:
\begin{itemize}
\item We need a way to uniquely identify a transaction output, so
  that it can be referred to by a transaction input that spends it. The pair of
  a transaction id and an output index is sufficient for this, but other schemes
  are conceivable.
\item A \textsf{Set} requires a notion of equality. If we use the
  obvious structural equality on outputs, then if we had two outputs
  paying $X$ to address $A$, they would be equal. We need to
  distinguish these --- outputs must have an identity beyond
  just their address and value.
\end{itemize}

\paragraph{The location of validators and datum objects.} Validator scripts
and full datum objects are provided as parts of transaction \emph{inputs},
even though they are conceptually part of the output being spent. The
output instead specifies them by providing the corresponding address
or hash.\footnote{That these match up is enforced by
  Rules~\ref{rule:validator-scripts-hash} and
  \ref{rule:datums-hash} in Figure~\ref{fig:eutxo-validity}.}

This strategy reduces memory requirements, since
the \UTXO{} set must be kept in memory for rapid access while validating
transactions. Hence it is desirable to keep outputs small --- in
our system they are constant size.
Providing the much larger validator script only at the point where it is needed
is thus a helpful saving. The same considerations apply to datum objects.

An important question is how the person who spends an output \emph{knows} which
validator and datum to provide in order to match the hashes on the output.
This can always be accomplished via some off-chain mechanism, but we may
want to include some on-chain way of accomplishing this.\footnote{\Cardano{} will provide
a mechanism in this vein.} However, this is not directly relevant to this paper,
so we have omitted it.

\paragraph{Fees, forge, and additional metadata.}  Transactions will typically
have additional metadata, such as transaction fees or a ``forge''
field that allows value to be created or destroyed. These are
irrelevant to this paper, so have been omitted.\footnote{ Adding such
  fields might require amending Rule~\ref{rule:value-is-preserved}
  below to ensure value preservation.  }

\paragraph{Ledger structure.} We model a ledger as a simple
list of transactions: a real blockchain ledger will be more complex
than this, but the only property that we really require is that
transactions in the ledger have some kind of address which allows them
to be uniquely identified and retrieved.

\subsection{The \ctx{} type}
\label{sec:validation-context}
Recall from the introduction to Section~\ref{sec:eutxo} that when a
transaction input is being validated, the validator script is supplied
with an object of type \ctx{} (encoded as \Data{}) which contains
information about the current transaction.  The \ctx{} type is defined
in Figure~\ref{fig:ptx-1-types}, along with some related types.

\begin{ruledfigure}{!ht}
  \begin{displaymath}
  \begin{array}{rll}
    \s{OutputInfo } &=&(\val: \qty,\\
                    & &\ \i{validatorHash}: \Address,\\
                    & &\ \datumHsh: \DataHash)\\
    \\
    \s{InputInfo } &=&(\outputref: \s{OutputRef},\\
                   & &\ \i{validatorHash}: \Address,\\
                   & &\ \i{datum}: \Data,\\
                   & &\ \i{redeemer}: \Data,\\
                   & &\ \val: \qty)\\
     \\
     \ctx\s{ } &=&(\i{inputInfo}: \Set{\s{InputInfo}},\\
               & &\ \i{outputInfo}: \List{\s{OutputInfo}},\\
               & &\ \i{validityInterval}: \Interval{\tick},\\
               & &\ \i{thisInput}: \N)\\
     \\
  \end{array}
  \end{displaymath}
  \caption{The \ctx{} type for the \EUTXO{} model}
  \label{fig:ptx-1-types}
\end{ruledfigure}

\paragraph{The contents of \ctx{}.}
The \ctx{} type is a summary of the information contained in the $\eutxotx$ type in
Figure~\ref{fig:eutxo-types}, situated in the context of a validating
transaction, and made suitable for consumption in a script. That results in the following changes:
\begin{enumerate}
\item The \s{InputInfo} type is augmented with information that comes
  from the output being spent, specifically the value attached to that output.
\item The \ctx{} type includes an index that indicates the input currently
  being validated. This allows scripts to identify their own address, for example.
\item Validators are included as their addresses, rather than as scripts. This
  allows easy equality comparisons without requiring script languages to be able
  to represent their own programs.
\end{enumerate}
\noindent We assume that there is a function $\toCtx: \eutxotx \times
  \s{Input} \times \s{Ledger} \rightarrow \ctx$ which summarises a
transaction in the context of an input and a ledger state.
%% kwxm: moved this out of the figure because adding the Ledger
%% parameter pushed everything too far to the right.

\paragraph{Determinism.}
The information provided by \ctx{} is entirely determined by the
transaction itself. This means that script execution during validation
is entirely deterministic, and can be simulated accurately by the user
\emph{before} submitting a transaction: thus both the outcome of
script execution and the amount of resources consumed can be
determined ahead of time. This is helpful for systems that charge for
script execution, since users can reliably compute how much they will
need to pay ahead of time.

A common way for systems to violate this property is by providing
access to some piece of mutable information, such as the current time
(in our system, the current tick has this role). Scripts can then
branch on this information, leading to non-deterministic behaviour. We
sidestep this issue with the validation interval mechanism (see the
introduction to Section~\ref{sec:eutxo}).

\begin{ruledfigure}{!ht}
  \begin{displaymath}
  \begin{array}{lll}
  \multicolumn{3}{l}{\lookupTx : \s{Ledger} \times \TxId \rightarrow \eutxotx{}}\\
  \lookupTx(l,id) &=& \textrm{the unique transaction in $l$ whose id is $id$}\\
  \\
  \multicolumn{3}{l}{\txunspent : \eutxotx \rightarrow \Set{\s{OutputRef}}}\\
  \txunspent(t) &=& \{(\txId(t),1), \ldots, (\txId(id),\left|t.outputs\right|)\}\\
  \\
  \multicolumn{3}{l}{\unspent : \s{Ledger} \rightarrow \Set{\s{OutputRef}}}\\
  \unspent([]) &=& \emptymap \\
  \unspent(t::l) &=& (\unspent(l) \setminus t.\inputs) \cup \txunspent(t)\\
  \\
  \multicolumn{3}{l}{\getSpent : \s{Input} \times \s{Ledger} \rightarrow \s{Output}}\\
  \getSpent(i,l) &=& \lookupTx(l, i.\outputref.\id).\outputs[i.\outputref.\idx]
  \end{array}
  \end{displaymath}
  \caption{Auxiliary functions for \EUTXO{} validation}
  \label{fig:validation-functions-1}
\end{ruledfigure}


\subsection{Validity of \EUTXO{} transactions}
\label{sec:eutxo-validity}

Figure~\ref{fig:eutxo-validity} defines what it means for a transaction $t$
to be valid for a ledger $l$ during the tick \currentTick, using some auxiliary functions from
Figure~\ref{fig:validation-functions-1}.
Our definition combines
Definitions 6 and 14 from Zahnentferner~\cite{Zahnentferner18-UTxO}, differing from
the latter in Rule~\ref{rule:all-inputs-validate}.

This definition extends to an entire ledger: a ledger
$l$ is \textit{valid} if either $l$ is empty or
$l$ is of the form $t::l^{\prime}$ with
$l^{\prime}$ valid and $t$ valid for $l^{\prime}$.
%%
\vspace{-2mm}
%%
\begin{ruledfigure}{H}
\begin{enumerate}

\item
  \label{rule:tick-in-range}
  \textbf{The current tick is within the validity interval}
  \begin{displaymath}
    \currentTick \in t.\validityInterval
  \end{displaymath}

\item
  \label{rule:all-outputs-are-non-negative}
  \textbf{All outputs have non-negative values}
  \begin{displaymath}
    \textrm{For all } o \in t.\outputs,\ o.\val \geq 0
  \end{displaymath}

\item
  \label{rule:all-inputs-refer-to-unspent-outputs}
  \textbf{All inputs refer to unspent outputs}
  \begin{displaymath}
    \{i.\outputref : i \in t.\inputs \} \subseteq \unspent(l).
  \end{displaymath}

\item
  \label{rule:value-is-preserved}
  \textbf{Value is preserved}
  \begin{displaymath}
    \textrm{Unless $l$ is empty, } \sum_{i \in t.\inputs} \getSpent(i, l).\val = \sum_{o \in t.\outputs} o.\val
  \end{displaymath}

\item
  \label{rule:no-double-spending}
  \textbf{No output is double spent}
  \begin{displaymath}
    \textrm{If } i_1, i_2 \in t.\inputs \textrm{ and }  i_1.\outputref = i_2.\outputref
    \textrm{ then } i_1 = i_2.
  \end{displaymath}

\item
  \label{rule:all-inputs-validate}
  \textbf{All inputs validate}
  \begin{displaymath}
    \textrm{For all } i \in t.\inputs,\ \llbracket
    i.\validator\rrbracket (i.\datum,\, i.\redeemer,\,  \toData(\toCtx(t,i,l))) = \true.
  \end{displaymath}

\item
  \label{rule:validator-scripts-hash}
  \textbf{Validator scripts match output addresses}
  \begin{displaymath}
    \textrm{For all } i \in t.\inputs,\ \scriptAddr(i.\validator) = \getSpent(i, l).\addr
  \end{displaymath}

\item
  \label{rule:datums-hash}
  \textbf{Each datum matches its output hash}
  \begin{displaymath}
    \textrm{For all } i \in t.\inputs,\ \hashData(i.\datum) = \getSpent(i, l).\datumHsh
  \end{displaymath}

\end{enumerate}
\caption{Validity of a transaction $t$ in the \EUTXO{} model}
\label{fig:eutxo-validity}
\end{ruledfigure}
%%
%% kwxm: some dubious fiddling with vertical space to get the validity
%% figure in the right place and still stay within the page limit
%%
\vspace{-8mm}
\paragraph{Creating value.}
Most blockchain systems have special rules for creating or destroying value.
These are usually fairly idiosyncratic, and are not relevant to this paper, so
we have provided a simple genesis condition in
Rule~\ref{rule:value-is-preserved} which allows the initial transaction in the ledger
to create value.
\vspace{-1mm}
\paragraph{Lookup failures.}
The function $\getSpent$ calls $\lookupTx$, which looks up the unique
transaction in the ledger with a particular id and can of course
fail. However Rule~\ref{rule:all-inputs-refer-to-unspent-outputs}
ensures that during validation all of the transaction inputs refer to
existing unspent outputs, and in these circumstances $\lookupTx$ will
always succeed for the transactions of interest.

